\pasCorrection{\nomPrenom*}

\begin{center}
  \chemfig{!\fructofuranoseHaw}
\end{center}

\question{
  Donner la formule brute de cette molécule.
}{
  \bruteCHO{6}{12}{6}
}[1]

\question{
  Calculer la masse molaire de cette molécule.
  \begin{donnees}[2]
    \item \masseMol{H} = \qty{1}{\g/\mole}
    \item \masseMol{C} = \qty{12}{\g/\mole}
    \item \masseMol{O} = \qty{16}{\g/\mole}
    \item[\vspace{\fill}]
  \end{donnees}
  \strut
}{
  \begin{equation*}
    M = 6\masseMol{C} + 12\masseMol{H} + 6\masseMol{O}
    = (\num{72} + \num{12} + \num{96})\unit{\g\per\mole}
    = \qty{180}{\g\per\mole}
  \end{equation*}
}[3]

\question{
  Entourer les groupes caractéristiques de la molécule et les nommer.
}{
  Hydroxyle et ether.
}[2]
