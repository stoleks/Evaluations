\pasCorrection{\nomPrenom*}

\begin{center}
  \chemfig{!\arginine}
  
  Molécule \important{d'arginine,} un des 20 acides $\alpha$-aminés protéinogène.
\end{center}

\question{
  Donner la formule brute de cette molécule.
}{
  \bruteCHO{6}{12}{6}
}[1]

\question{
  Calculer la masse molaire de cette molécule.
  \begin{donnees}[2]
    \item \masseMol{H} = \qty{1}{\g/\mole}
    \item \masseMol{C} = \qty{12}{\g/\mole}
    \item \masseMol{N} = \qty{14}{\g/\mole}
    \item \masseMol{O} = \qty{16}{\g/\mole}
  \end{donnees}
  \strut
}{
  \begin{equation*}
    M = 6\masseMol{C} + 14\masseMol{H} + 2\masseMol{O} + 4\masseMol{N}
    = (\num{72} + \num{14} + \num{32} + \num{56})\unit{\g\per\mole}
    = \qty{174}{\g\per\mole}
  \end{equation*}
}[2]

\question{
  Entourer les groupes caractéristiques de la molécule et les nommer.
}{
  Carboxyle et amine.
}[2]
