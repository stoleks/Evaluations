%%%% début de la page
\newpage
\enTete{Corps purs et solutions}{1}

%%%%
\nomPrenomClasse

%%%% evaluation
\vspace*{6pt}
\begin{tableauCompetences}
  \centering RCO --
  Restituer ses connaissances.
  & & & &
  \\ \hline
  %
  \centering APP --
  Extraire une information
  & & & &
  \\ \hline
  %
  \centering REA --
  Réaliser un calcul.
  & & & &
\end{tableauCompetences}


%%%% questions
%
\exo{Un ensemble d'entités chimiques identiques est :\competence{RCO}}{0}
\vspace*{-4pt}
\begin{qcm}
  \item une espèce chimique.
  \item un corps pur.
  \item un mélange.
\end{qcm}

%
\exo{Définir un mélange.\competence{RCO}}{1}

%
\exo{Citer deux liquides \textbf{miscibles}.\competence{RCO}}{1}

%
L'acier doux est un alliage composé de fer (\chemfig{Fe}) et de carbone (\chemfig{C}).
Cet alliage est constitué d'une seule phase.
Un échantillon d'acier de masse $m = 1000 \unit{g}$ comporte une masse de fer $m_{\chemfig{Fe}} = 998 \unit{g}$ et une masse de carbone $m_{\chemfig{C}} = 2,\!00 \unit{g}$

%
\exo{Cet alliage est-il un mélange homogène ou hétérogène ?\competence{APP, RCO}}{1}

%
\exo{Écrire $m_{\chemfig{Fe}}$ et $m_{\chemfig{C}}$ en notation scientifique.\competence{APP, REA}}{1}

%
\exo{Calculer la proportion massique de fer et de carbone dans l'alliage.\competence{APP, REA}}{2}