% \teteTermStssAlim
\newpage
\vAligne{-50pt}
\titre{Partie chimie}

\begin{boite}
  \centering
  Le candidat traite \important{AU CHOIX} 2 exercices sur les 3 proposés

  \begin{multicols}{3}
    \begin{boite}
      \begin{center}
        \important{Exercice 1 : Oxygénothérapie}
      \end{center}
      \vspace*{-12pt}
      \textbf{Mots-clés :} Loi des gaz parfaits, bilan de matière, débit.
    \end{boite}
    %
    \begin{boite}
      \begin{center}
        \important{Exercice 2 : Le rouge Ponceau, un colorant alimentaire}
      \end{center}
      \vspace*{-12pt}
      \textbf{Mots-clés :} DJA, dosage par étalonnage, concentration en masse.
    \end{boite}
    %
    \begin{boite}
      \begin{center}
        \important{Exercice 3 : Remplacer les sucres dans l'alimentation}
      \end{center}
      \vspace*{-12pt}
      \textbf{Mots-clés :} DJA, concentrations molaire et massique.
    \end{boite}
  \end{multicols}
\end{boite}

\textit{
  Le candidat \textbf{choisit obligatoirement deux exercices parmi les trois proposés} et indique clairement son choix au début de la copie.
}
  
\textit{
  Les exercices sont indépendants.
}
\vspace*{24pt}


%%%%
\vspace*{-4pt}
\exercice{Oxygénothérapie (10 points)}

\textbf{Mots-clés :} Loi des gaz parfaits, bilan de matière, débit.
\medskip


La drépanocytose entraîne des crises douloureuses qui peuvent être atténuées par des médicaments antalgiques et une hydratation par voie intraveineuse, mais si la douleur persiste, la médicamentation peut être complétée par une oxygénothérapie.
L’oxygénothérapie consiste en un apport supplémentaire de dioxygène à l’organisme.

\begin{doc}{Utilisation des bouteilles de dioxygène}{doc:BAC_bouteille_dioxygene}
  Le dioxygène est stocké à l’état gazeux comprimé à une pression initiale de \qty{200}{\bar}, dans
  des bouteilles spécialement conçues et de différents volumes selon leur utilisation. À la
  sortie des bouteilles, la pression du gaz est réduite par un manodétendeur pour la rendre
  acceptable par le patient. Au fur et à mesure que la bouteille se vide, la pression du gaz à
  l’intérieur diminue.
  Un débitmètre permet de régler le débit de dioxygène suivant la prescription médicale.
  D’après l’Association Nationale pour les Traitements à Domicile
\end{doc}

\begin{doc}{Durée d'autonomie d'une bouteille de dioxygène B2}{doc:BAC_autonomie}
  Les bouteilles B2 de volume 2,0 litres sont utilisées pour l’oxygénothérapie de déambulation.
  La masse totale moyenne de la bouteille pleine de dioxygène comprimé à \qty{200}{\bar} est de \qty{5,8}{\kg}.
  La durée d’autonomie d’une bouteille B2 est donnée dans le tableau ci-dessous, pour différentes valeurs de la pression initiale de la bouteille et du débit de dioxygène délivré par le manodétendeur.

  \centering
  \smallskip
  \begin{tblr}{
    colspec = {|c |c |c |c |c |}, hlines,
    row{1,2} = {couleurPrim!20}
  }
    \SetCell[r=2]{c} { Pression dans la \\ bouteille en bar } &
    \SetCell[c=4]{c} Débit de \chemfig{O_2} à la sortie du manodétendeur & & & \\
    & \qty{3}{\litre\per\minute}
    & \qty{6}{\litre\per\minute}
    & \qty{9}{\litre\per\minute}
    & \qty{15}{\litre\per\minute} \\
    200 & 2 h 15 min & 1 h 05 min & 0 h 45 min & 0 h 25 min \\
    150 & 1 h 40 min & 0 h 50 min & 0 h 30 min & 0 h 20 min \\
    100 & 1 h 05 min & 0 h 30 min & 0 h 20 min & 0 h 10 min \\
    50  & 0 h 30 min & 0 h 15 min & 0 h 10 min & < 10 min
  \end{tblr}

  \begin{flushright}
    \textit{D’après ansm.sante.fr pour les bouteilles d’Air liquide.}
  \end{flushright}
\end{doc}

\newpage
\vspace*{-36pt}
\begin{doc}{La loi des gaz parfaits}{doc:BAC_gaz_parfait}
  \begin{equation*}
    \Large
    P \times V = n \times R \times T
  \end{equation*}
      
  P : pression du gaz (\unit{\pascal}) \\
  V : volume occupé par le gaz (\unit{\m\cubed}) \\
  n : quantité de matière du gaz (\unit{\mole}) \\
  R : constante des gaz parfaits où R = \qty{8,31}{\pascal\m\cubed \per\mol\per\kelvin} \\
  T : température du gaz (\unit{\kelvin}).
  On rappelle que $T(\unit{\kelvin}) = T(\unit{\degreeCelsius}) + 273$
\end{doc}

\textbf{Données :}
\begin{itemize}
  \item \qty{1}{\litre} = \qty{e-3}{\m\cubed} et \qty{1}{\m\cubed} = \qty{e3}{\litre}.
  \item \qty{1}{\bar} = \qty{e5}{\pascal}.
  \item Pression atmosphérique normale : $P_{atm} = \qty{1,01e5}{\pascal}$.
  \item Masse molaire moléculaire du dioxygène \chemfig{O_2} : $M_{\chemfig{O_2}} = \qty{32,0}{\g\per\mole}$.
\end{itemize}


%%%%
\question{
  La pression du dioxygène à l’intérieur d’une bouteille B2 neuve est égale à \qty{200}{\bar}.
  Convertir cette valeur en pascal.
}{
  $\qty{200}{\bar} = \qty{200e5}{\pascal}$
  \points{1}
}{0}

\question{
  Montrer alors qu’à \qty{20}{\degreeCelsius}, la quantité de matière de dioxygène contenue dans la bouteille B2 neuve est voisine de $n_{\chemfig{O_2}} = \qty{16,4}{\mol}$.
}{
  On utilise la loi des gaz parfaits : 
  $n 
  = \dfrac{P\times V}{R \times T}
  = \dfrac{\qty{200e5}{\pascal} \times \qty{2e-3}{\m\cubed}} {\qty{8,31}{\pascal\m\cubed\per\mole\per\kelvin} \times \qty{293}{\kelvin}}
  = \qty{16,4}{\mole}$
  \points{1}
}{0}

\question{
  Calculer la masse $m_{\chemfig{O_2}}$ du dioxygène contenu à une pression de \qty{200}{\bar} dans la bouteille B2 neuve.
}{
  $m_{\chemfig{O_2}} 
  = M_{\chemfig{O_2}} \times n_{\chemfig{O_2}}
  = \qty{32,0}{\g\per\mole} \times \qty{16,4}{\mole}
  = \qty{526}{\g}$
  \points{1,5}
}{0}

\question{
  Montrer que la masse du gaz représente moins de \qty{10}{\percent} de la masse totale de la bouteille pleine.
}{
  \qty{5,8}{\kg} = \qty{5800}{\g}, donc la masse de dioxygène représente 
  $\dfrac{526}{5800} = \qty{9}{\percent} < \qty{10}{\percent}$ de la masse totale de la bouteille.
  \points{1}
}{0}

\question{
  Vérifier que le volume de dioxygène à la pression atmosphérique,
  libérable par la bouteille B2 neuve à la température de \qty{20}{\degreeCelsius} est d’environ \qty{0,4}{\m\cubed}.
}{
  Pour calculer le volume de dioxygène à pression atmosphérique, on utilise la loi des gaz parfaits 
  $V = \dfrac{n\times R\times T}{P} 
  = \dfrac{\qty{16,4}{\mole} \times \qty{8,31}{\pascal\m\cubed\per\mole\per\kelvin} \times \qty{293}{\kelvin}} {\qty{1,01e5}{\pascal}}
  = \qty{0,40}{\m\cubed}$
  \points{1,5}
}{0}

\question{
  La bouteille B2 est initialement à la pression de \qty{200}{\bar} et le manodétendeur est réglé pour
  délivrer un débit $D = \qty{3}{\litre\per\minute}$ de gaz à la pression atmosphérique.
  Vérifier que la durée d’autonomie est bien en accord avec celle indiquée dans le \textbf{document 2}.
  
  On rappelle que le débit $D$ d’écoulement d’un gaz ou d’un liquide est défini par :
  \begin{equation*}
    D = \dfrac{\text{volume écoulée}}{\text{durée de l'écoulement}} = \dfrac{V}{\Delta t}
  \end{equation*}
}{
  On peut calculer le temps que met la bouteille met à se vider à partir de la relation du débit
  \begin{equation*}
    \text{Durée} = \Delta t = \dfrac{V}{D} = \dfrac{\qty{400}{\litre}} {\qty{3}{\litre\per\minute}} = \qty{133}{\minute} = \text{2 h 13 min}
  \end{equation*}
  On trouve une durée cohérente.
  \points{2}
}{0}

\question{
  Justifier qualitativement l’évolution de la durée d’autonomie en fonction du débit du gaz.
}{
  Quand le débit augmente, la bouteille se vide plus rapidement et donc l'autonomie baisse.
  \points{1}
}{0}

\question{
  En exploitant la relation du \textbf{document 3},
  expliquer pourquoi la pression dans la bouteille diminue au fil de l’utilisation à température constante.
}{
  Le volume occupé et la température reste constante, mais la quantité de matière diminue, donc la pression diminue.
  \points{1}
}{0}


%%%%%%%%%%%%%%%%%%%%%%%%%%%%%%%%%%%%%%%%%%%%%%%%%%%%%%%
\vspace*{24pt}
\exercice{Le rouge Ponceau, un colorant alimentaire (10 points)}

\textbf{Mots-clés :} Dose journalière admissible, dosage par étalonnage, concentration en masse.
\medskip

\begin{doc}{La couleur des macarons}{doc:BAC_macaron}
  Les macarons sont des gâteaux individuels à l’amande dont les goûts peuvent être différents.
  Les macarons sont souvent colorés.
  Pour cela, certains professionnels n’hésitent pas à jouer la surenchère en ayant recours à un surdosage des colorants.
  Cependant, l’utilisation de ces substances dans les denrées alimentaires est rigoureusement encadrée par la réglementation sur les additifs. 
  
  \begin{flushright}
    \textit{Macarons, la ronde des couleurs | economie.gouv.fr }
  \end{flushright}
\end{doc}

\begin{doc}{Le colorant E124}{doc:BAC_colorant_E124}
  Le rouge Ponceau AR (E124) est un colorant azoïque de synthèse.
  C’est un additif alimentaire qui peut remplacer le rouge de cochenille (E120) car il est moins cher.
  En Europe, la dose journalière admissible (DJA) est de 0,7 milligramme par kilogramme de masse corporelle.
  En France, son usage doit s’accompagner de la mention « Peut avoir des effets indésirables sur l’activité et l’attention chez les enfants ». 

  \begin{flushright}
    \textit{Colorant-alimentaire.fr}
  \end{flushright}
\end{doc}

 
On souhaite déterminer la quantité en colorant E124 présente dans un macaron à l’aide d’un dosage par étalonnage avec un spectrophotomètre.  

Pour cela, on sèche puis on réduit en poudre un macaron de couleur rouge. On dissout cette poudre dans de l’eau.
Après filtration, on obtient une solution S de volume V = \qty{25}{\ml}.On considère que la totalité du rouge Ponceau AR (E124) contenu dans le macaron a été récupérée dans cette solution. 

\important{I \faMinus}
On réalise une courbe d’étalonnage représentée sur
\textbf{l’ANNEXE DE LA DERNIÈRE PAGE (À RENDRE AVEC LA COPIE DE CHIMIE)}
à partir de solutions étalons de concentrations connues en rouge Ponceau AR (E124).
Ces solutions sont obtenues par dilution d’une solution mère $S_0$
de concentration en masse \qty{100}{\mg\per\litre} en colorant E124.

\begin{doc}{}{doc:BAC_absorbance}
  On mesure l’absorbance des solutions
  \medskip
  
  \centering
  \begin{tblr}{
    colspec = {|c |c |c |c |c |}, hlines,
    row{1} = {couleurPrim!20}
  }
    Solutions étalons & $S_1$ & $S_2$ & $S_3$ & $S_4$ \\
    Concentration massique en \unit{\g\per\litre} & 50,0 & 25,0 & 12,5 & 5,0 \\
    Absorbance A sans unité & 1,56 & 0,82 & 0,37 & 0,16 \\
    Volume de la solution en \unit{\ml} & 20 & 20 & 20 & 20 \\
  \end{tblr}
\end{doc}
 
\question{
  Calculer le volume de solution mère $S_0$ à prélever pour réaliser la solution $S_2$.
}{
  On veut diviser la concentration par un facteur de dilution $F = \dfrac{\qty{100}{\mg\per\litre}}{\qty{25}{\mg\per\litre}} = 4$.
  Comme le volume de la solution $S_2$ final est $V_2 = \qty{20}{\ml}$, il faut prélever un volume $V_0 = \dfrac{V_2}{F} = \qty{5}{\ml}$.
  \points{1,5}
}{0}

\question{
  Indiquer le volume d’eau à rajouter au prélèvement pour réaliser la solution $S_2$.
}{
  Comme on a prélevé \qty{5}{\ml}, il faut rajouter \qty{15}{\ml} pour réaliser la solution $S_2$.
  \points{1}
}{0}

\question{
  Sur \textbf{l’ANNEXE (À RENDRE AVEC LA COPIE DE CHIMIE)}, compléter la deuxième ligne du tableau par les numéros (1 à 7) de façon à rendre compte de la chronologie des étapes à suivre pour réaliser la dilution.  
}{
  \points{1}
}{0}

\medskip
\important{II \faMinus} La mesure de l’absorbance A de la solution $S$ est de 0,94. 

\question{
  En utilisant la droite d’étalonnage de \textbf{l’ANNEXE (À RENDRE AVEC LA COPIE DE CHIMIE)}, déterminer la concentration en masse en colorant E124 de la solution $S$ et indiquer les traits de construction nécessaires sur l’annexe. 
}{
  Sur la courbe d'étalonnage, on peut lire qu'une absorbance de 0,94 correspond à une concentration en colorant $c = \qty{30}{\mg\per\litre}$.
  \points{1}
}{0}

\question{
  Montrer que la masse du colorant E124 contenu dans le macaron est d’environ 0,75 mg. 
}{
  Comme la totalité du colorant du macaron est passé dans la solution, il suffit de calculer la masse de colorant dans la solution
  $m = c \times V = \qty{30}{\mg\per\litre} \times \qty{25e-3}{\litre} = \qty{0,75}{\mg}$.
  \points{1,5}
}{0}

\question{
  Définir la dose journalière admissible (DJA). 
}{
  C'est la quantité maximale d'un produit que l'on peut avaler tous les jours sans conséquences négatives sur la santé.
  \points{1}
}{0}

\question{
  Indiquer si un enfant de 40 kg pourrait manger le contenu d’une boîte de 12 macarons rouges dans la journée sans dépasser la DJA du colorant E124. 
}{
  On multiplie la DJA et la masse de l'enfant pour déterminer la dose maximale à ne pas dépasser $\qty{40}{\kg} \times DJA = \qty{40}{\kg} \times \qty{0,7}{\mg\per\kg} = \qty{28}{\mg}$.

  Dans 12 macaron on a une masse de $12\times \qty{0,75}{\mg} = \qty{9}{\mg}$ de colorant, donc l'enfant peut manger 12 macarons sans danger.
  \points{2}
}{0}

\question{
  Indiquer si cela présente un autre risque pour sa santé. 
}{
  En mangeant 12 macarons, l'enfant a certainement eu un apport en sucre trop élevé.
  \points{1}
}{0}


%%%%%%%%%%%%%%%%%%%%%%%%%%%%%%%%%%%%%%%%%%%%%%%%%%%%%%%
\pasCorrection{\newpage}
\correction{\vspace*{24pt}}
\exercice{Remplacer les sucres dans l'alimentation (10 points)}

\textbf{Mots-clés :} Concentrations en masse et en quantité de matière, dose journalière
admissible (DJA).
\medskip

Les aliments riches en sucres favorisent l’apparition du diabète. Le diabète est déclaré si la
concentration en masse $C_m$ de sucres dans le sang à jeun est supérieure à \qty{1,26}{\g\per\litre}.
L’organisation mondiale de la santé (OMS) préconise de limiter l’apport en sucres à \qty{10}{\percent} de la ration énergétique totale qui s’élève en moyenne à \qty{104}{\kilo\joule} par jour pour l’adulte.
Certaines personnes choisissent de remplacer le sucre de leur alimentation par un édulcorant.

\begin{doc}{Le glucose}{doc:BAC_glucose}
  Une des molécules issue de la dégradation partielle du saccharose (sucre de table)
  dans l’organisme est le glucose dont la forme linéaire a pour formule partiellement développée :
  \begin{center}
    \chemfig{
      HO -CH_2 
      -C(-[-3] H) (-[3] OH)
      -C(-[-3] H) (-[3] OH)
      -C(-[-3] OH) (-[3] H)
      -C(-[-3] OH) (-[3] H)
      -C(-[-2] H) =[2] O
    }
  \end{center}
\end{doc}

\begin{doc}{La stévia}{doc:BAC_stevia}
  Le Rebaudioside A, extrait de la stévia, plante originaire du Paraguay, a un pouvoir sucrant tel
  qu’une sucrette contenant 20 mg de Rebaudioside A produit le même goût sucré qu’un morceau
  de sucre contenant l’équivalent de 5,0 g de glucose.
  Cependant l’agence européenne de sécurité des aliments (EFSA)
  a fixé la dose journalière admissible (DJA)
  pour le Rebaudioside A à 4,0 milligrammes par kilogramme de masse corporelle
  (DJA = \qty{4,0}{\milli\g\per\kg}).

  \begin{flushright}
    \textit{D’après www.efsa.europea.eu/}
  \end{flushright}
\end{doc}

\textbf{Données :}
\begin{itemize}
  \item Masse molaire moléculaire du glucose $M_\text{glucose} = \qty{180,0}{\g\per\mol}$.
  \item Le glucose a une valeur énergétique par unité de masse de \qty{15,6}{\kilo\joule\per\g}.
\end{itemize}

\question{
  Recopier la formule chimique du glucose.
  Entourer et nommer deux groupes fonctionnels différents de la molécule de glucose.
}{
  On a des groupes hydroxyle (alcool) tous le long de la chaîne et un groupe carbonyle (aldéhyde) en bout de chaîne à droite.
  \points{2}
}{0}

\question{
  Donner la formule brute du glucose.
}{
  \chemfig{C_{6} H_{12} O_{6}}
  \points{1}
}{0}

% \question{
%   Expliquer qualitativement pourquoi le glucose est soluble dans le sang considéré comme une
% solution aqueuse.
% }{
%   \points{1}
% }{0}


\question{
  L’analyse sanguine d’un patient à jeun indique une concentration en quantité de matière de glucose égale à \qty{7,8}{\milli\mole\per\litre}.
  Montrer que ce résultat confirme que ce patient souffre du diabète.
}{
  Pour avoir la concentration massique en glucose, on multiplie la concentration molaire par la masse molaire moléculaire du glucose $c_m = \qty{180,0}{\g\per\mole} \times \qty{7,8e-3}{\mol\per\litre} = \qty{1,40}{\g\per\litre}$.
  Comme $C_m$ est supérieur à \qty{1,26}{\g\per\litre}, le patient souffre de diabète.
  \points{2}
}{0}

\question{
  La consommation quotidienne en sucre de ce patient est équivalente à 75 g de glucose.
  Indiquer si cette consommation est conforme à celle préconisée par l’OMS.
}{
  L'énergie produite par le glucose ingéré est $E = \qty{75}{\g} \times \qty{15,6}{\kilo\joule\per\g} = \qty{1170}{\kilo\joule}$, ce qui est largement supérieur à la valeur préconisé par l'OMS (\qty{104}{\kilo\joule}) !
  \points{1}
}{0}

\question{
  Ce patient, qui pèse 68 kg, envisage de remplacer sa consommation de sucre par du Rebaudioside A.
  Calculer, à l’aide du document 2, la masse maximale de cet édulcorant qu'il peut consommer par jour.
}{
  Il peut ingérer au maximum $\qty{4,0}{\mg\per\kg} \times \qty{68}{\kg} = \qty{272}{\mg}$ par jour.
  \points{1}
}{0}

\question{
  En déduire le nombre de sucrettes qu’il peut consommer par jour.
}{
  Il peut consommer $\dfrac{272}{20} = 14$ sucrette par jour.
  \points{1}
}{0}

\question{
  Indiquer s’il peut substituer sa consommation quotidienne de sucre,
  équivalente à 75 g de glucose, par la consommation de Rebaudioside A.
}{
  Comme une sucrette équivaut à \qty{5}{g} de sucre, il peut consommer l'équivalent de $14\times \qty{5}{\g} = \qty{70}{\g}$ par jour, ce qui ne remplace pas sa consommation quotidienne.
  \points{2}
}{0}