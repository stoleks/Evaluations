% \teteTermStssAlim
\newpage
\vAligne{-70pt}
\titre{Partie chimie}

\begin{boite}
  \centering
  Le candidat traite \important{AU CHOIX} 2 exercices sur les 3 proposés

  \begin{multicols}{3}
    \begin{boite}
      \begin{center}
        \important{Exercice 1 : Curiethérapie}
      \end{center}
      \vspace*{-12pt}
      \textbf{Mots-clés :} Isotopes, période radioactive, activité.
    \end{boite}
    %
    \begin{boite}
      \begin{center}
        \important{Exercice 2 : Antiseptique préopératoire}
      \end{center}
      \vspace*{-12pt}
      \textbf{Mots-clés :} Dilution, dosage par étalonnage.
    \end{boite}
    %
    \begin{boite}
      \begin{center}
        \important{Exercice 3 : Imagerie médicale}
      \end{center}
      \vspace*{-12pt}
      \textbf{Mots-clés :} Radiographie, noyau atomique, radioactivité.
    \end{boite}
  \end{multicols}
\end{boite}

\textit{
  Le candidat \textbf{choisit obligatoirement deux exercices parmi les trois proposés} et indique clairement son choix au début de la copie.
}
  
\textit{
  Les exercices sont indépendants.
}
\vspace*{24pt}


%%%%
\vspace*{-4pt}
\exercice{Traitement d'un cancer par curiethérapie (10 points)}\label{exo:curietherapie}

\textbf{Mots-clés :} Isotopes, période radioactive, activité.
\medskip

Le tabac contribue à augmenter le risque du cancer de la prostate.
Ce cancer peut être soigné par curiethérapie.
Cette thérapie consiste à implanter, à travers le périnée, des capsules de la taille de grains de riz contenant de l'iode 125.
Ces implants restent à demeure. 
Les noyaux d'iode 125 (symbole \isotope{125}{53}{I}) sont radioactifs, ils émettent des particules de faible énergie et un rayonnement électromagnétique de longueur d'onde 
$\lambda_0 = \qty{0,034}{\nm}$.
Les particules sont absorbées par les parois de la capsule contenant l'iode.
L'irradiation des tissus entourant l'implant n'est due qu'au rayonnement électromagnétique.

\question{
  Les noyaux d'iode 125 et d'iode 123 sont des isotopes.
  Définir le terme « isotopes » et donner le symbole du noyau d'iode 123.
}{}{0}

\question{
  Donner la composition d'un noyau d'iode 125.
}{}{0}

\question{
  La réaction de désintégration d'un noyau d'iode 125, s'accompagne de l'émission d'électrons et d'un rayonnement électromagnétique.
  En exploitant le texte introductif, préciser ce qu'il advient des électrons et du rayonnement émis.
}{}{0}

\question{
  À l'aide du document 1, déterminer le domaine des ondes électromagnétiques émises lors de cette désintégration radioactive. Rappel : \qty{1}{\nm} = \qty{e-9}{\m}.
}{}{0}

\question{
  Définir la période radioactive d'un radioélément.
}{}{0}

\question{
  À l'aide d'une construction graphique réalisée sur l'ANNEXE (à rendre avec la copie de chimie), montrer que la période radioactive de l'iode 125 est voisine de 60 jours.
}{}{0}

\begin{doc}{Domaines spectraux des ondes électromagnétiques}{doc:BB2_domaines}
  \begin{center}
    \image{1}{stssTerminale/bac/domaine_onde_radio}
  \end{center}
\end{doc}

\question{
  Une capsule d'implant possède une activité initiale de 16 MBq. Calculer l'activité de cette capsule au bout de 120 jours.
}{}{0}

% \question{
%   Expliquer pourquoi il est recommandé aux patients traités par curiethérapie à l'iode 125 d'éviter des contacts prolongés avec des femmes enceintes ou avec de jeunes enfants pendant les 6 mois qui suivent la pause des implants.
% }{}{0}


\question{
  Dans certains cas, le radioélément utilisé n'est pas l'iode 125 mais le palladium 103 qui a une période radioactive de 17 jours.
  Indiquer les avantages que l'usage du palladium peut présenter.
}{}{0}


%%%%
\newpage
\exercice{Étude d'un antiseptique préopératoire (10 points)}\label{exo:antiseptique}

\textbf{Mots-clés :} Dilution, dosage par étalonnage, concentrations en masse et en quantité
de matière.
\medskip

L'implantation de capsules de curiethérapie nécessite une intervention chirurgicale.
La Bétadine® est un antiseptique local utilisé pour la désinfection préopératoire des patients.
Son principe actif est le diiode \chemfig{I_2} qui élimine les micro-organismes par son action oxydante.
Les solutions de diiode sont colorées en jaune allant jusqu'au brun selon leur concentration.
Dans la Bétadine®, le diiode est « emprisonné » dans un polymère appelé polyvidone.
Une mole de polyvidone iodée contient une mole de diiode.
\medskip

D'après la notice, la Bétadine® à \qty{10}{\percent} contient \qty{10}{\g} de polyvidone iodée dans \qty{100}{\ml}.

\textbf{Données :}
\begin{itemize}
  \item Masse molaire de la polyvidone iodée : $M_\text{polyvidone iodée} = \qty{2363}{\g\per\mole}$.
  \item Masse molaire moléculaire du diiode : $M_{I_2} = \qty{253,8}{\g\per\mole}$.
\end{itemize}

On souhaite déterminer la teneur en diiode de la Bétadine® à \qty{10}{\percent} à l'aide d'un dosage spectrophotométrique par étalonnage.
Pour cela, on procède à l'étalonnage d'une gamme de solutions de diiode de concentrations $C(I_2)$ en quantité de matière de $I_2$, connues.
La mesure de l'absorbance $A$ de chaque solution est réalisée avec un spectrophotomètre UV–visible.
\medskip 

On obtient la courbe d'étalonnage donnée en ANNEXE (à rendre avec la copie de
chimie), qui représente l'absorbance $A$ des solutions en fonction de leur concentration
en quantité de matière de $I_2$, $C(I_2)$.

\question{
  Justifier à l'aide du graphique donné en ANNEXE (à rendre avec la copie de chimie) que l'absorbance A de la solution de diiode est proportionnelle à la concentration $C(I_2)$ en quantité de matière de diiode.
}{}{0}

\question{
  Pour comparer la solution commerciale de Bétadine® à \qty{10}{\percent} avec cette gamme
d'étalonnage, il est ici nécessaire de la diluer dix fois.
  Parmi le matériel disponible ci-dessous, choisir, en justifiant, l'association pipette jaugée / fiole jaugée à utiliser pour préparer la solution diluée souhaitée.
  Liste du matériel disponible :
  \begin{itemize}
    \item pipettes jaugées 2,0 mL, 10,0 mL, 20,0 mL ; 25,0 mL ;
    \item fioles jaugées 100,0 mL, 250,0 mL, 500,0 mL.
  \end{itemize}
}{}{0}

\question{
  Rappeler le protocole de la dilution.
}{}{0}

\question{
  Sans modifier les réglages du spectrophotomètre, on mesure l'absorbance de la solution ainsi diluée. On trouve Asolution diluée= 0,9.
  Déterminer graphiquement, à l'aide de l'ANNEXE (à rendre avec la copie de chimie), la concentration en quantité de matière de diiode de la solution.
  On fera apparaître la construction sur le graphique.
}{}{0}

\question{
  En déduire que la concentration en quantité de matière de diiode dans la solution de Bétadine® à \qty{10}{\percent} est voisine de \qty{0,043}{\mol\per\litre}.
}{}{0}

\question{
  En déduire la concentration en masse de la polyvidone iodée dans la Bétadine® à
\qty{10}{\percent}.
}{}{0}

\question{
  Vérifier la cohérence de l'indication de la notice : « La Bétadine® à \qty{10}{\percent} contient 10 g de polyvidone iodée dans 100 mL».
}{}{0}

\question{
  Identifier une cause possible de l'écart constaté. 
}{}{0}


%%%%
%\newpage
\vspace*{2cm}
\exercice{Exploration pulmonaire par imagerie médicale (10 points)}\label{exo:imagerie}

\textbf{Mots-clés :} Radiographie, fréquence, longueur d'onde, noyau atomique, radioactivité.
\medskip

Un patient fumeur peut aussi souffrir de liaisons aux poumons et présenter des difficultés respiratoires. Le médecin peut prescrire des explorations par imagerie médicale pour mesurer les volumes pulmonaires.

\begin{multicols}{2}
  \begin{doc}{Composition des tissus corporels}{doc:BB_composition}
    Les principaux éléments constitutifs des tissus mous (peau, muscles, graisse, tendons, vaisseaux sanguins et nerfs) sont l'hydrogène, le carbone, l'azote et l'oxygène.
    Les os, tissus corporels durs, sont constitués des mêmes éléments que les tissus mous et de sels minéraux inorganiques tels que le calcium, le phosphore et le magnésium.
  \end{doc}
  
  \begin{doc}{Radiographie thoracique}{doc:BB_}
    \begin{center}
      \image{0.5}{stssTerminale/bac/radio}

      \url{https://www.infirmiers.com}
    \end{center}
  \end{doc}
\end{multicols}

\textbf{Données :}
  \begin{itemize}
    \item 
    Vitesse de la lumière dans le vide ou dans l'air : $c = \qty{3,00e8}{\m\per\s}$.
  \end{itemize}
  \vspace*{-20pt}
  \begin{tableau}{|c |c |c |c |c |c |c |c |}
    Élément & Hydrogène & Carbone & Azote & Oxygène & Magnésium & Phosphore & Calcium \\
    Symbole & H & C & N & O & Mg & P & Ca \\
    Numéro atomique Z & 1 & 6 & 7 & 8 & 12 & 15 & 20
  \end{tableau}


\question{
  Rappeler le principe de la radiographie en précisant la nature des ondes utilisées.
}{}{0}

\question{
  Citer un point commun et une différence entre radiographie et radiothérapie.
}{}{0}

\question{
  En utilisant l'échelle de longueurs d'ondes ci-dessous, indiquer à quel numéro correspond le domaine des rayons X, utilisés en radiographie.
  \begin{center}
    \image{1}{stssTerminale/bac/domaine_onde}
  \end{center}
}{}{0}

\question{
  Après avoir rappelé la relation entre fréquence et longueur d'onde ainsi que les unités associées, déterminer l'intervalle de fréquences correspondant aux rayons X en utilisant l'échelle présentée à la question 3.
}{}{0}

\question{
  Les rayons X peuvent traverser certains des tissus corporels.
  En identifiant dans le document 2 les tissus corporels visualisés sur la radiographie, indiquer ceux qui ont tendance à absorber le plus fortement les rayons X et proposer une explication.
}{}{0}

La scintigraphie est parfois utilisée dans le diagnostic d'un lymphome.
On utilise dans ce cas un marqueur radioactif contenant du molybdène-99, de symbole \isotope{99}{42}{Mo}.

\question{
  Donner la composition d'un noyau atomique de molybdène-99.
  
L'équation de désintégration du molybdène-99 est partiellement donnée ci-dessous.
Elle fait apparaître le rayonnement $\gamma$ utilisé en scintigraphie et une particule notée \isotope{...}{...}{A}, de nature à déterminer.
  
  \begin{center}
    \isotope{99}{42}{Mo} 
    $\rightarrow$
    \isotope{99}{43}{Tc} 
    + \isotope{...}{...}{A} + $\gamma$
  \end{center}
}{}{0}

\question{
  Compléter le symbole de la particule \isotope{...}{...}{A} en remplaçant les pointillés par les nombres appropriés.
}{}{0}

\question{
  Identifier la particule \isotope{...}{...}{A},
  est-ce un positron \isotope{0}{+1}{e}
  ou un électron \isotope{0}{-1}{e} ?
  Nommer le type de désintégration subie par le molybdène-99.
}{}{0}


%%%%
\newpage
\begin{boite}
  \centering
  \textbf{ANNEXE - À RENDRE AVEC LA COPIE DE CHIMIE}
\end{boite}

%%
\textbf{Exercice \ref{exo:curietherapie}}
\begin{center}
\image{0.9}{stssTerminale/bac/evolution_I2}
\end{center}

%%
\textbf{Exercice \ref{exo:antiseptique}}
\begin{center}
\image{0.9}{stssTerminale/bac/etalonnage_betadine}
\end{center}
%\includepdf{stssTerminale/bac/chimie-bio-physi-paho-humaine}