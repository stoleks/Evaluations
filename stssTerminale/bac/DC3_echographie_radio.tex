% \teteTermStssAlim
\newpage
\vAligne{-80pt}
\titre{Partie chimie}


Une personne âgée vient réaliser des examens de santé suite à des douleurs dans la poitrine et dans une de ses jambes.
L'enjeu est de déterminer si elle souffre d'une maladie cardiovasculaire et d'ostéoporose.

%%%%
\bigskip
\exercice{Maladie cardiovasculaire (12 points)}
\medskip

\begin{doc}{Maladie cardiovasculaire}{doc:E3_cardiovasculaire}
  \extrait{
     Les maladies cardiovasculaires sont dues à une accumulation de dépôts de graisses (cholestérol) sur les parois des artères. 
     Ces dépôts forment des plaques appelées \important{plaques d’athérome.} Les parois des artères se durcissent.
     On parle alors \important{d’athérosclérose.}
     
     L’athérosclérose ne provoque dans un premier temps aucun symptôme.
     Puis, le rétrécissement des artères s'aggrave et entraîne un \important{ralentissement de la circulation sanguine} et une moins bonne oxygénation des organes (cœur, cerveau, muscles des jambes...) Les symptômes de la maladie cardiovasculaire apparaissent.

     La formation d’un caillot peut interrompre brutalement la circulation sanguine et provoquer un accident cardiovasculaire (infarctus du myocarde, accident vasculaire cérébral...)
   }{Source : ameli.fr}

   Pour contrôler la présence d'athérosclérose, on utilise \important{l'échographie Doppler.}
\end{doc}

\begin{doc}{Principe de l'échographie Doppler}{doc:E3_doppler}
  Lorsqu'une onde sonore ou ultrasonore émise par un émetteur rencontre un obstacle fixe, la fréquence de l’onde réfléchie est identique à la fréquence de l’onde émise.
  Si l'obstacle se déplace, la fréquence de l’onde réfléchie $f_r$ est différente de la fréquence de l’onde émise $f_e$.
  C’est l’effet Doppler.
  L’écart de fréquences est noté $\Delta f$.
  Il permet de déterminer le sens et la vitesse d'écoulement du sang dans les vaisseaux.
\end{doc}

\begin{doc}{Le décalage Doppler $\Delta f$}{doc:E3_decalage_doppler}
  Dans l’examen considéré dans cet exercice, l’écart de fréquences dû à l’effet Doppler est donné par la relation suivante :
  \begin{equation*}
    \Delta f = \dfrac{2f_e \times v}{c}
  \end{equation*}
  \begin{listePoints}
    \item $\Delta f$ : écart de fréquence mesuré en hertz noté \unit{\hertz}
    \item $f_e$ fréquence de l’onde émise en hertz (\unit{\hertz})
    \item $v$ vitesse d’écoulement des globules rouges (\unit{\m\per\s})
    \item $c$ célérité moyenne des ultrasons dans le corps humain (\unit{\m\per\s})
  \end{listePoints}
\end{doc}

\question{
  Lors d’une échographie Doppler mesurant la vitesse d’écoulement sanguin,
  préciser quels sont les composants du sang qui réfléchissent les ondes ultrasonores.
}{}{0}

\question{
  Compléter la légende dans les cadres du schéma donné dans l’ANNEXE (à rendre avec la copie de chimie).
}{}{0}

\question{
  Exprimer la vitesse $v$ d'écoulement du sang en fonction de $\Delta f$ et des paramètres $c$ et $f_e$ à l'aide de la formule du document~\ref{doc:E3_decalage_doppler}.
}{}{0}

\question{
  En utilisant les données suivantes, montrer que la vitesse v d’écoulement du sang dans cette artère vaut environ
  \qty{0,36}{\m\per\s}.
  
  \important{Données :} $f_e = \qty{4,5e6}{\hertz}$ ; $\Delta f = \qty{2,1e3}{\hertz}$ ;  $c = \qty{1540}{\m\per\s}$.
}{}{0}

\question{
  La vitesse normale d'écoulement sanguin dans une artère est comprise entre 55 et \qty{90}{\cm\per\s}.
  Indiquer si l’écoulement dans l’artère considérée présente une athérosclérose.
}{}{0}

\question{
  Donner les conséquences d'une athérosclérose.
}{}{0}



%%%%
\newpage
\exercice{Ostéoporose liée à l'age (8 points)}
\medskip

\begin{doc}{Ostéoporose}{doc:E3_osteoporose}
  \begin{wrapfigure}[10]{r}{0.4\linewidth}
    \centering
    \vspace*{-20pt}
    \image{1}{stssTerminale/bac/radio_jambe_osteoporose}
  \end{wrapfigure}
  
  \extrait{
    L'ostéoporose est une maladie diffuse du squelette caractérisée par une diminution de la densité osseuse et des altérations de la micro-architecture des os.
    Ces altérations rendent l'os plus fragile et augmentent le risque de fracture.
  }{Source : ameli.fr}
  
  Pour remédier à l'ostéoporose, une une activité sportive régulière est nécessaires.
  L'ostéoporose entraine une perte de calcium des os, ce qui les rend moins absorbants aux rayons X.
  On utilise les rayons X pour réaliser la radiographie des jambes d’une personne âgée pour contrôler la qualité de ses os.
  La plaque utilisée en radiographie est blanche et s'assombrit quand elle absorbe des rayons X.
  On obtient les deux images ci-contre. 
\end{doc}

\question{
  Identifier la jambe atteinte d'ostéoporose en comparant les deux image du document~\ref{doc:E3_osteoporose}.
  Justifier votre choix.
}{}{0}

\question{
  Citer deux risques encourus suite à une exposition régulière aux rayonnement X.
}{}{0}

\question{
  Indiquer un moyen utilisé par les manipulateurs ou manipulatrice en radiographie pour se protéger des rayons X.
}{}{0}

\question{
  Expliquer pourquoi les os apparaissent blanc sur la radiographie.
}{}{0}


\vspace*{36pt}
\begin{boite}
  \centering
  \textbf{ANNEXE À RENDRE AVEC LA COPIE DE CHIMIE}
\end{boite}

Exercice 1, question 2

\begin{center}
 \image{1}{stssTerminale/bac/schema_doppler}
\end{center}