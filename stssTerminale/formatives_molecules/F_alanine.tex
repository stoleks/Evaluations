\teteTermStssOrga

\nomPrenom

\begin{center}
  \chemfig{
    HO- C (=[3] O) -CH (-[-3] NH_2) -CH_3
  }
  \\[8pt]
  \important{Alanine}, un acide aminé
\end{center}

\question{
  Donner le nom de la formule utilisée pour représenter la molécule d'alanine.
}{
  C'est la formule semi-developpée.
}{1}

\question{
  Donner la formule brute de l'alanine.
}{
  \chemfig{C_3 H_7 O_2 N}
}{1}

\question{
  Donner la formule développée de l'alanine.
}{
  \chemfig{
    H-O- C (=[3] O) -C (-[3] H) (-[-3] N (-[-5]H) (-[-1] H)) -C (-[3] H) (-[-3] H) -H
  }
}{6}
  
\numeroQuestion
Entourer deux groupe fonctionnels dans la molécule d'alanine et les nommer.