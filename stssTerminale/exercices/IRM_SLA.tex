%
\exercice{IRM et diagnostic différentiel de la SLA (10 points)} \label{exo:IRM_SLA}

\textbf{Mots-clés :} Produit de contraste pour l'IRM, spectre des ondes électromagnétiques, radiographie
\medskip

Le diagnostic de la Sclérose Latérale Amyotrophique (SLA) peut être complexe, en particulier au début.
Une IRM (Imagerie par Résonance Magnétique) repose sur les propriétés magnétiques des noyaux d'atomes d'hydrogène (protons).
L'IRM du cerveau ou de la moelle est souvent réalisée.
Les lésions de la SLA étant difficiles à observer, l'IRM a pour but d'exclure d'autres atteintes du système nerveux telles que la sclérose en plaques ou des tumeurs.

\begin{doc}{Produits de contraste}
  Dans certains cas, la réalisation d'une IRM nécessite l'injection d'un produit de contraste.
  
  L'ion gadolinium \chemfig{Gd^{3+}} est bien adapté pour concevoir des agents de contraste. Néanmoins, à cause de sa haute toxicité, il ne peut être utilisé sous sa forme ionique libre \chemfig{Gd^{3+}}. 
  Il est possible de masquer cette toxicité en associant l'ion \chemfig{Gd^{3+}} à certaines molécules appelées ligands.
  C'est le cas des deux agents de contraste Gd-DOTA$^{-}$ et Gd-HP-DO3A dont les 
  structures sont schématisées ci-dessous.
  Pour garantir l'innocuité de ces composés tout au long de leur séjour dans l'organisme, leur stabilité chimique est primordiale.
  Le choix d'un produit de contraste dépend de ses propriétés d'accès à des territoires 
  pathologiques précis.
  Il faut pour cela optimiser l'affinité et la sélectivité du produit pour la cible visée et s'assurer de la bonne tolérance par le patient.
  \separationBlocs{
    \centering
    Gd-DOTA$^{-}$

    \chemfig{!\ionChelate}
  }{
    \centering
    Gd-HP-DO3A

    \chemfig{!\chelateAlcool}
  }

  \begin{flushright}
    D'après: \url{http://culturesciences.chimie.ens.fr/}
  \end{flushright}
\end{doc}

\begin{doc}{Domaines de fréquences des ondes électromagnétiques}
  \centering
  \includegraphics[width=\linewidth]{images/bacST2S/domaine_spectraux_EM.png}
\end{doc}

\question{
  La fréquence d'une onde électromagnétique émise par l'appareil d'IRM est $\nu = \qty{64}{\mega\hertz}$. 
  À l'aide du document 2, repérer puis indiquer à quel domaine appartient cette onde 
électromagnétique. 
\important{Données :} $\qty{1}{\mega\hertz} = \qty{1,0e6}{\hertz}.$
}{}

\medskip
La radiographie est une autre technique d'imagerie médicale.

\question{
  Rappeler le domaine de fréquence auquel appartient le rayonnement électromagnétique utilisé en radiographie. 
}{}


\question{
  Expliquer le principe de la radiographie et relier les nuances claires ou foncées d’un cliché à l’absorption des rayons X par les différents tissus.
}{}


\question{
  Expliquer à quoi sert un produit de contraste en imagerie médicale.
}{}

\question{
  À partir de sa formule donnée dans le document 1, justifier la charge électrique portée par Gd-DOTA$^{-}$ et montrer que Gd-HP-DO3A est électriquement neutre.
}{}

\question{
  La durée d'élimination d'un produit de contraste est un critère de choix. Expliquer pourquoi.
  Indiquer d'autres critères de choix d'un produit de contraste.
}{}.