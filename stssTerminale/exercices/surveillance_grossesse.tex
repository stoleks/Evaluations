\exercice{Surveillance de la grossesse par échographie}\label{exo:surveillance_grossesse}

\textbf{Mots clés :} échographie, effet Doppler
\medskip

\begin{doc}{Décalage Doppler}
  \begin{wrapfigure}[4]{r}{0.4\linewidth}
    \vspace*{-32pt}
    \includegraphics[width=\linewidth]{images/bacST2S/suivi_grossesse_LCC}
    \legende{Cliché de l'échographie}
  \end{wrapfigure}
  Au cours de la grossesse, 3 échographies (aux $3^e$, $5^e$ et $8^e$) sont préconisées.
  Lors de la première échographie, la longueur cranio-caudale LCC (flèche blanche sur le cliché ci-contre) permet d'estimer l'âge de la grossesse entre 7 et 13 semaines d'aménorrhée.

  \smallskip
  \begin{tblr}{colspec = {cc}, hlines, vlines}
    Aménorrhée en semaines & LCC en mm \\
    10 & entre 42 et 43 mm \\
    11 & entre 44 et 56 mm \\
    12 & entre 58 et 69 mm \\
    13 & entre 70 et 84 mm
  \end{tblr}

  \smallskip
  \textit{D'après la formule de Robinson et Fleming}
\end{doc}

\begin{doc}{Expression du décalage Doppler}
  Le décalage $\Delta f$ entre la fréquence de l'onde ultrasonore émise par la sonde et la fréquence de l'onde réfléchie par le sang circulant dans un vaisseau sanguin est donnée par la relation suivante :
  \begin{equation*}
    \Delta f = \dfrac{2f_E \times v \times \cos(\theta)}{c}
  \end{equation*}
  \begin{listePoints}
    \item $f_E$ fréquence de l'onde émise par la sonde
    \item $v$ vitesse moyenne des globules rouges dans le vaisseau sanguin
    \item $c$ valeur de la vitesse moyenne des ultrasons dans le corps humain
    \item $\theta$ angle entre la direction de la vitesse d'écoulement du sang dans le vaisseau et la direction de propagation de l'onde émise par la sonde.
  \end{listePoints}
\end{doc}

\question{
  Expliquer le principe de l’échographie permettant d’obtenir un cliché tel que celui du 
document 1, en précisant la nature des ondes et le phénomène physique mis en jeu.
}{
}

\question{
  À l’aide du document 1, estimer l’âge de la grossesse (en semaines d’aménorrhée)
}{}

\question{
  La grossesse peut induire des problèmes circulatoires dans les membres inférieurs de la 
femme enceinte. La vitesse de circulation du sang dans les veines peut être mesurée par 
échographie Doppler. Avec l’appui d’un schéma, décrire le principe de l’échographie 
Doppler.
}{}

\question{
  À partir de l’expression donnée dans le document 2, déduire l’expression de la vitesse v
d’écoulement du sang en fonction de $\Delta f$, $f_E$, $c$ et $\cos(\theta)$. On précisera les unités des grandeurs.
}{}

\question{
  L’examen porte sur la petite veine saphène de la jambe où la formation d’une varice est 
redoutée. À partir des données mesurées ci-dessous, montrer que la vitesse v d’écoulement 
du sang dans la veine saphène de la patiente est voisine de \qty{0,15}{\m\per\s}.

\important{Données :} 
$f_E = \qty{1,0e7}{\hertz}$ ;
$\theta = \qty{40}{\degree}$ ;
$\cos(\qty{40}{\degree}) = \num{0.77}$ ;
$\Delta f = \qty{1,5}{\kilo\hertz}$ ;
$c = \qty{1540}{\m\per\s}.$
}{}

\question{
  En situation normale, la vitesse moyenne du sang dans la veine saphène est comprise entre 
10 et \qty{25}{\cm\per\s}.
Déduire de la question 5 si l’examen réalisé révèle une anomalie.
}{}