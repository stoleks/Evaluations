\exercice{Un sommeil réparateur pour se sentir mieux}\label{exo:gaba}

\motsClesGABA
\medskip

Le sommeil est indispensable pour récupérer de la fatigue accumulée par l'organisme.
Plusieurs acides aminés permettent d'assurer un sommeil de bonne qualité. Par exemple,
l'acide glutamique est un acide aminé précurseur du GABA (gamma aminobutyric acid) qui
est un neurotransmetteur ayant des propriétés sédatives.

\begin{doc}{Synthèse du GABA}{doc:synthese_GABA}
  L'énantiomère L de l'acide glutamique est un des 22 acides aminés protéinogènes. Le
  GABA est synthétisé à partir de l'acide glutamique selon une réaction d'équation :
  \begin{center}
      \chemname{HOOC-\chemfig{CH_2}-\chemfig{CH_2}-CH(\chemfig{NH_2})-COOH}{
        acide glutamique
      }
      \reaction
      \chemname{HOOC-\chemfig{CH_2}-\chemfig{CH_2}-\chemfig{CH_2}-\chemfig{NH_2}}{
        GABA
      } + \chemfig{CO_2}
  \end{center}
  Une enzyme favorise cette réaction en se liant à l'acide glutamique.
  Elle ne peut se lier qu'à son énantiomère L et non à l'énantiomère D,
  car son site de liaison présente une complémentarité de forme avec l'énantiomère L de l'acide glutamique.
  Le schéma ci-dessous illustre cette propriété.
  \begin{center}
    \image{0.75}{images/bacST2S/enantiomere}
  \end{center}
\end{doc}

\question{
  Sur la formule topologique de l'acide glutamique représentée sur l'ANNEXE à rendre avec la copie de chimie,
  entourer et nommer les deux groupes caractéristiques qui justifient que cette molécule appartient à la famille des acides aminés.
}{
  Acide glutamique possède deux groupe carboxyle \chemfig{COOH} à ses extrémités et un groupe amine \chemfig{NH_2}, c'est donc bien un acide aminé.
}

\question{
  Préciser, en justifiant, s'il s'agit d'un acide $\alpha$-aminé.
}{
  Il s'agit d'un acide $\alpha$-aminé, car l'acide glutamique possède un groupe amine et un groupe carboxyle sur le même carbone fonctionnel (carbone $\alpha)$.
}

\question{
  Indiquer si l'acide gamma-aminobutyrique dont la formule topologique est représentée
  sur l'ANNEXE à rendre avec la copie de chimie est aussi un acide aminé.
}{
  C'est aussi un acide aminé, car l'acide gamma-aminobutyrique possède un groupe carboxyle et un groupe amine.
  Par contre ce n'est pas un acide $\alpha$-aminé, les deux groupes n'étant pas reliés au même carbone fonctionnel.
}

\question{
  Définir ce que l'on appelle un « atome de carbone asymétrique »
  et indiquer la propriété qui découle de la présence d'un atome de carbone asymétrique dans une molécule.
}{
  C'est un atome de carbone relié à quatre chaînes différentes, une molécule qui possède un carbone asymétrique va être chirale, la molécule ne pourra pas être superposée avec son image miroir.
}

\question{
  Sur la formule topologique de l'acide glutamique représentée sur l'ANNEXE à rendre avec la copie de chimie,
  repérer la position de l'atome de carbone asymétrique par un astérisque (*).
}{
  L'atome de carbone asymétrique se trouve sur le carbone $\alpha$ (au dessus du groupe amine).
}

\question{
  Justifier que cette molécule possède deux énantiomères en précisant ce que cela signifie.
}{
  Comme c'est une molécule chirale, cette molécule possède deux formes miroir l'une de l'autre, mais qui ne sont pas superposable, on parle de forme énantiomère gauche et droite.
}

\question{
  Donner les représentations de Cram et Fischer des 2 énantiomères de l'acide glutamique.
}{
  \begin{center}
    \chemname{
    \chemfig{
      [:-90] COOH -(-[::90] H) (-[::-90] NH_2) -(-[::90] H) (-[::-90] H) -(-[::90] H) (-[::-90] H) -COOH
    }}{
      Forme gauche
    } \nobreak \qq{}
    \chemname{
    \chemfig{
      [:-90] COOH -(-[::90] NH_2) (-[::-90] H) -(-[::90] H) (-[::-90] H) -(-[::90] H) (-[::-90] H) -COOH
    }}{
      Forme droite
    }
  \end{center}
}

\question{
  Justifier que l'enzyme favorisant la synthèse du GABA, schématisée ci-dessous,
  ne peut se lier qu'à l'un des énantiomères de l'acide glutamique
  \begin{center}
    \image{0.7}{images/bacST2S/enzyme_gaba}
  \end{center}
}{
  Comme l'enzyme présente une symétrie axiale, elle ne pourra accueillir qu'une seule des deux formes énantiomère, en l’occurrence la forme gauche.
}