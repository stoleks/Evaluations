\exercice{Intoxication au monoxyde de carbone (10 points)}\label{exo:intoxication_CO}
\medskip

Le monoxyde de carbone est responsable de plus de la moitié des morts par intoxication dans le monde.
Issu d'une combustion incomplète, le monoxyde de carbone peut avoir des sources variées : chaudières au gaz défectueuse, poëles ou cheminés qui ne tirent pas assez, moteur thermique utilisé avec un mauvais carburant, départ de feu dans une maison, etc.

\begin{doc}{Détection du monoxyde de carbone}{doc:detection_monoxyde}
    Pour détecter le monoxyde de carbone, on peut acheter un détecteur spécialisé qui utilise une réaction d'oxydo-réduction avec les deux couples suivants :
    \begin{alignat*}{2}
      \dioxydeDeCarbone/\chemfig{CO} : \quad
      & 2\chemfig{CO} + 2\eau 
      && \reaction 2\dioxydeDeCarbone + 4\ionHydrogene + 4\electron \\
      %
      \dioxygene/\eau : \quad
      & \dioxygene + 4\ionHydrogene + 4\electron 
      && \reaction 2\eau
    \end{alignat*}
  Les électrons échangés aux cours de la réaction d'oxydo-réduction entre le monoxyde de carbone et le dioxygène génèrent un courant électrique qui alimente une alarme sonore au delà d'un certain taux de monoxyde de carbone.

  Il est vivement recommandé d'avoir un détecteur de monoxyde de carbone chez soit, ils sont peu couteux et peuvent éviter des tragédies.
\end{doc}

\question{
  Expliquer pourquoi on ne peut pas se baser sur notre vue ou notre odorat pour détecter le monoxyde de carbone.
}{}

\question{
  Écrire l'équation de réaction du monoxyde de carbone \chemfig{CO} avec le dioxygène \chemfig{O_2} en utilisant les deux demi-équations d'oxydo-réduction fournies.
}{}

\question{
  Justifier que le monoxyde de carbone est un réducteur.
}{}


\begin{doc}{Effets physiologiques du \chemfig{CO}}{doc:effets_physiologiques}
  \begin{wrapfigure}[16]{r}{0.35\linewidth}
    \centering
    \vspace*{-20pt}
    \chemfig[atom sep = 20pt]{!\hemeB}
    
    \legende{Formule topologique de la molécule d'hème.}
  \end{wrapfigure}
  Quand nous respirons, le dioxygène est transporté des poumons aux tissus grâce à une protéine : l'hémoglobine notée Hb, ce qui permet la respiration cellulaire.
  L'hémoglobine est composée de quatres sous-unités qui peuvent transporter chacune une molécule de dioxygène.
  
  Dans une sous-unité, la molécule de dioxygène est fixée sur un groupe prosthétique appelé hème (« \textit{le sang} » en grec), qui est une molécule avec un ion fer II \ionFerII, sur lequel une molécule de dioxygène \dioxygene peut se fixer.

  Le monoxyde de carbone \chemfig{CO} se lie 250 fois plus facilement à l'hème que le dioxygène.
  Quand le monoxyde de carbone se lie à un ou plusieurs hèmes d'une hémoglobine, on parle de carboxyhémoglobine notée COHb.

  Chez une personne normale, le taux de COHb est inférieur à \qty{1}{\percent} et peut atteindre \num{3} à \qty{8}{\percent} chez une personne fumeuse, voire \qty{15}{\percent} pour les fumeurs compulsifs.

  Si le taux de COHb est inférieur à \qty{10}{\percent}, il y a peu ou pas d'effets physiologiques observés.
  À \qty{15}{\percent} de COHb, une personne va commencer à avoir des maux de têtes.
  Entre \qty{20}{\percent} et \qty{30}{\percent}, les maux de têtes sont sévères et accompagnés de nausées, d'étourdissement, de confusion et de légères hallucinations visuelles.
  Entre \qty{30}{\percent} et \qty{50}{\percent} les symptômes neurologiques deviennent plus graves.
  Au delà de \qty{50}{\percent}, il y a perte de connaissance et un risque de tomber dans le coma, avec un risque d'insuffisance respiratoire.
  La mort arrive quand le taux de COHb dépasse les \qty{60}{\percent}.
\end{doc}

\question{
  Donner le nom d'un groupe fonctionnel présent dans la molécule d'hème.
}{}

\question{
  En vous basant sur la formule topologique de la molécule d'hème, expliquer pourquoi elle ne peut pas se mélanger avec le sang et doit être transportée par une protéine.
}{}

\question{
  Indiquer le nombre maximal de molécules de dioxygène transportables par une hémoglobine.
}{}

\begin{doc}{Traitement d'une intoxication au monoxyde de carbone}{doc:traitement_intoxication}
  Quand on suppose qu'une personne est intoxiquée au monoxyde de carbone, il faut l'extraire rapidement de la source de \chemfig{CO}, mais ce n'est pas suffisant.
  Dans une atmosphère normale, la demi-vie $t_{1/2}$ des carboxyhémoglobine COHb est en moyenne de 4 heures (il faut 4 heures pour diviser par 2 la population de COHb dans le sang), ce qui entraine une récupération parfois trop lente.
  
  Pour accélérer l'élimination du monoxyde de carbone, le traitement le plus efficace consiste à utiliser un masque respiratoire délivrant \qty{100}{\percent} de dioxygène à \qty{3}{\bar}, ce qui permet de redescendre à moins de \qty{1}{\percent} de COHb en quelques dizaines de minutes.
\end{doc}

\question{
  Calculer le temps nécessaire pour passer de \qty{50}{\percent} à moins de \qty{1}{\percent} de COHb sous une atmosphère normale.
  Conclure sur l'intérêt d'utiliser du dioxygène pur pressurisé.
}{}


\begin{doc}{Chaudière au gaz défectueuse}{doc:chaudiere_combustion}
  Pour se chauffer ou pour cuisiner, il est très courant d'utiliser des chaudières au gaz qui réalisent une combustion du méthane de formule brute \chemfig{CH_4}.
  Si la chaudière est défectueuse et que l'apport de dioxygène est insuffisant, la combustion peut être incomplète et produire du monoxyde de carbone \chemfig{CO}, en plus de la vapeur d'eau \chemfig{H_2O}.

  Le monoxyde de carbone est un gaz inodore, incolore et mortel, même en très petite quantité.

  \begin{donnees}[2]
    \item $P_{atm} = \qty{1,01e5}{\pascal}$ ;
    \item $T (\unit{\kelvin}) = T(\unit{\degreeCelsius}) + 273,15$ ;
    \item $R = \qty{8.314}{\pascal\m\cubed\per\mol\per\kelvin}$ ;
    \item $\qty{1}{\m\cubed} = \qty{1000}{\litre}$.
  \end{donnees}
\end{doc}

\numeroQuestion
Recopier et ajuster la réaction modélisant la combustion incomplète du méthane : 
\begin{center}
  2\chemfig{CH_4} + \texteTrou{3} \chemfig{O_2} \reaction \texteTrou{2}\chemfig{CO} + \texteTrou{4} \chemfig{H_2O} 
\end{center}

\question{
  Dans un appartement fermé de \qty{125}{\m\cubed} (~\qty{50}{\m\squared}), il y a \qty{20}{\m\cubed} de dioxygène.
  En utilisant la loi des gaz parfait
  \begin{equation*}
    P \times V = n \times R \times T
  \end{equation*}
  calculer la quantité de dioxygène présente initialement dans l'appartement à \qty{20}{\degreeCelsius}.
}{}

\question{
  Une chaudière au gaz utilise \qty{1,8}{\m\cubed} de méthane pour chauffer un appartement pendant une heure.
  En supposant que tous le méthane soit utilisé pour la combustion incomplète, calculer la quantité de matière de monoxyde de carbone produite au bout d'une heure.
  Comparer avec la quantité de dioxygène.
}{}