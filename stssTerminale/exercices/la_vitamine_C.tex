\exercice{La vitamine C (10 points)} \label{exo:la_vitamine_C}

\medskip
\motsClesLaVitamineC
\medskip

Les personnes sujettes aux calculs urinaires et celles souffrant de maladies comme la drépanocytose, induisant une accumulation du fer dans l’organisme, doivent s’abstenir de prendre de la vitamine C en excès ou de manière prolongée.

\begin{doc}{Le stockage des vitamines dans l’organisme}{doc:stockage_vitamine}
  Les vitamines sont des molécules organiques indispensables au bon fonctionnement de l'organisme.
  Dans la plupart des cas, notre organisme est incapable de les synthétiser et elles sont apportées par l'alimentation.
  Au nombre de treize, elles se répartissent en deux catégories :
  \begin{listePoints}
    \item les vitamines liposolubles qui peuvent être stockées par l’organisme.
    Ce sont les vitamines A, D, E et K.
    \item les vitamines hydrosolubles qui ne sont pas stockées de manière prolongée et qui, en excès, sont rejetées dans les urines.
    C’est le cas de la vitamine C et des vitamines du groupe B.
  \end{listePoints}
  
  \sourceExtrait{D’après https://www.caducee.net/}
\end{doc}

\begin{doc}{L’apport de la vitamine C dans l’alimentation}{doc:apport_vitamine_C}
  \begin{wrapfigure}{l}{0.4\linewidth}
    \centering
    \chemfig{!\acideAscorbique}
  \end{wrapfigure}
  
  La vitamine C se trouve essentiellement dans les végétaux frais, les fruits frais, particulièrement
  dans les agrumes et les légumes verts.
  Un bon apport alimentaire doit suffire à couvrir les besoins quotidiens.
  On peut les compléter soit avec des extraits de fruits (cynorhodon, acérola, kiwi), soit avec de la vitamine de synthèse de façon à atteindre un apport journalier moyen de 100 mg.
  Cette vitamine très instable est détruite par la chaleur et par l'exposition à l'air.
  
  \sourceExtrait{D’après https://www.caducee.net/}
\end{doc}

%\begin{doc}{La liaison hydrogène}{doc:liaison_hydrogene}
 % La liaison hydrogène pourrait être renommée la « liaison de la vie » tant elle joue un rôle fondamental dans les processus biologiques.
  %C'est une liaison chimique inter ou intramoléculaire. Comme son nom l'indique, cette liaison implique forcément un atome d’hydrogène.
  %Une liaison hydrogène peut s'établir entre un atome d'hydrogène lié par covalence à un atome très électronégatif (comme le fluor F, l'oxygène O ou l'azote N), et un atome aussi très électronégatif.
  
  %\sourceExtrait{D’après https://www.futura-sciences.com}
%\end{doc}

\titreSousSection{Solubilité de la vitamine C}

Les vitamines hydrosolubles et liposolubles.

\question{
  Définir les termes « liposolubles » et « hydrosolubles ».
}{
  Une molécule liposoluble a peu de liaisons polaires et ne se mélangera pas avec de l'eau.
  Une molécule hydrosoluble a plusieurs liaisons polaires qui pourront former des liaisons hydrogènes avec les molécules d'eau, ce qui assure un bon mélange.
\points{1}
}

\question{
  Préciser dans quel type de tissus sont stockées les vitamines liposolubles.
}{
  Les vitamines liposolubles sont stockées dans le foie et les tissus adipeux (les graisses).
\points{1}
}

% \question{
%   À l'aide du document 3, schématiser par un trait pointillé la liaison hydrogène qui
% peut s'établir entre deux molécules d'eau.
% }{}

\question{
  À partir de sa formule topologique, déduire le caractère hydrosoluble de la vitamine C.
}{
  La molécule de vitamine C comporte plusieurs liaisons polaires, les groupes hydroxyles \chemfig{OH}, elle est donc hydrosoluble. 
\points{1}
}

\titreSousSection{De la vitamine C dans un jus d'orange}

Afin de comparer la concentration en vitamine C d’un jus d’orange fraîchement pressé (noté F) et d'un jus pasteurisé (noté P), on effectue le dosage par titrage d'un même volume $V_J$ de jus d’orange à l’aide de DCPIP (2,6-dichlorophénol-indophénol).
Le DCPIP est un réactif de couleur rose qui réagit mole à mole avec la vitamine C.
Lors de cette réaction, les produits obtenus sont incolores.
À l'équivalence du titrage, on observe la persistance de la coloration rose dans la solution titrée.

\pasCorrection{\pagebreak}
\begin{donnees}
  \item Volume de jus titré : $V_J = \qty{5,0}{\ml}$
  \item Concentration en quantité de matière de DCPIP : $C_\text{DCPIP} = \qty{1,0e-3}{\mole\per\l}$
  \item Masse molaire de la vitamine C : $\masseMol{Vit C} = \qty{176}{\g\per\mole}$
  \item Volume de DCPIP versés à l’équivalence pour le jus frais F : $V_E = \qty{10,0}{\ml}$
\end{donnees}

\question{
  Sur l'ANNEXE À RENDRE AVEC LA COPIE, légender le schéma à l'aide du vocabulaire suivant : agitateur magnétique, barreau aimanté, erlenmeyer, burette graduée, réactif titré, réactif titrant. 
}{
  Cf. annexe.
\points{2}
}
  
% \question{
%   Justifier l'observation effectuée à l'équivalence du titrage.
% }{}
  
\question{
  Déterminer la concentration en quantité de matière de vitamine C notée $C_\text{Vit C, F}$, dans le jus frais F.
  En sachant que la relation vérifiée à l’équivalence du titrage par les concentrations en quantité de matière est :
  \begin{equation*}  
    C_\text{DCPIP} \times V_E = C_\text{Vit C, F} \times V_J
  \end{equation*}
  où $V_E$ est le volume de DCPIP versé à l’équivalence. 
}{
  À partir de la relation fournie 
  \begin{align*}
    C_\text{Vit C, F} \times V_J &= C_\text{DCPIP} \times V_E \\
    C_\text{Vit C, F} &= \dfrac{C_\text{DCPIP} \times V_E}{V_J} \\
    &= \dfrac{ \qty{1,0e-3}{\mole\per\litre} \times \qty{10,0}{\ml} }{ \qty{5,0}{\ml} }
    &= \qty{2,0e-3}{\mole\per\litre}
  \end{align*}
\points{1}
}
  
\question{
  Montrer que la concentration en masse $C_m(\text{Vit C, F})$ de vitamine C dans le jus frais est voisine de \qty{350}{\mg\per\l}.
}{
  Il faut multiplier la concentration molaire en vitamine C par la masse molaire de la vitamine C
  \begin{align*}
    c_m(\text{Vit C, F}) &= c_\text{Vit C, F} \times \masseMol{Vit C} \\
    &= \qty{2,0e-3}{\mole\per\litre} \times \qty{176}{\g\per\mole} \\
    &= \qty{352e-3}{\g\per\litre} 
    = \qty{352}{\milli\g\per\litre}
\points{1}
  \end{align*}
}
\bigskip
 
La pasteurisation est un procédé de conservation des aliments par chauffage, puis refroidissement.
La concentration en masse de vitamine C dans le jus pasteurisé a pour valeur
$C_m(\text{Vit C, P}) = \qty{56}{\mg\per\l}$. 

\question{
  Comparer cette valeur à celle calculée pour le jus frais et commenter l’effet de la pasteurisation sur le jus.
}{
  352/56 = 6,29 
  
  Pour un jus frais, on a $\approx 7$ fois plus de vitamine C que dans un jus pasteurisé.
  La pasteurisation détruit la vitamine C.
\points{1}
}
  
\question{
  Calculer le volume de jus d'orange pasteurisé nécessaire pour couvrir les besoins journaliers en vitamine C indiqués dans le document 2.
}{
  Pour atteindre \qty{100}{\mg}, il faut boire un volume 
  \begin{equation*}
    V = \dfrac{ \qty{100}{\mg} }{ \qty{56}{\mg\per\litre} } = \qty{1.8}{\litre}
\points{1}
  \end{equation*}
  de jus pasteurisé.
}
  
\question{
  Donner un argument en faveur de  la consommation de fruits frais pour l’apport de vitamine C.
}{
  Les fruits frais contiennent plus de vitamine C que les jus pasteurisés, qui permettent difficilement de respecter les apports journaliers sans dépasser ceux en sucre.
\points{1}
}