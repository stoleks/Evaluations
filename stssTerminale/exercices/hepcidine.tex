\exercice{L'hepcidine, une protéine régulatrice}\label{exo:hepcidine} \textbf{(10pts)}
\bigskip

\begin{wrapfigure}[4]{r}{0.35\linewidth}
  \centering
  \vspace*{-24pt}
  \chemfig{!\cysteineSemiDev}
\end{wrapfigure}

L’hepcidine est une protéine qui régule l’absorption intestinale du fer. L’hepcidine est un polypeptide de 25 acides aminés dont la séquence montre la présence de cystéine (Cys). La formule
semi-développée de la cystéine est donnée ci-contre.


\medskip
\important{Étude de la molécule de cystéine}
\medskip

\question{
  Recopier la formule semi-développée de la cystéine. Entourer et nommer deux groupes fonctionnels présents.
}{
  On a un groupe amine \chemfig{NH_2} et un groupe carboxyle \chemfig{COOH}.
}

\question{
  Justifier que la cystéine appartient à la famille des acides $\alpha$-aminés.
}{
  La cystéine a un groupe amine et un groupe carboxyle sur le même carbone fonctionnel, donc c'est bien un acide $\alpha$-aminé.
}

\question{
  Justifier que la cystéine est une molécule chirale. \points{2}
}{
  La cystéine contient un carbone asymétrique, car il est connecté à quatre groupes différents, donc c'est une molécule chirale. 
}

\question{
  Donner la représentation de Fischer de l’énantiomère L de la cystéine qui est produit dans le corps humain naturellement.
}{
  \begin{center}
    \chemfig{COOH (-[-3] (-[-6] NH_2) (-[-3] (-[-6] H) (-[-3] SH) -H) -H)}
  \end{center}
}


\medskip
\important{Étude de la séquence Cys-Gly de l’hepcidine}
\medskip

La séquence d’acides aminés constituant l’hepcidine est la suivante :
\begin{center}  
  D T H F P I C I F C C G C C H R S K \tcbox[tcbox raise base, nobeforeafter]{C G} M C C K T
\end{center}

La séquence CG (Cys-Gly) de la protéine peut être obtenue par réaction entre la cystéine (Cys) et la glycine (Gly).


\question{
  À partir d’un mélange de cystéine (Cys) et de glycine (Gly), déterminer le nombre de dipeptides différents qu’il est possible d’obtenir si aucune précaution particulière n’est appliquée. 
  Nommer les dipeptides obtenus en utilisant les abréviations Gly et Cys.
}{
  On a 4 dipeptides : Cys-Cys, Cys-Gly, Gly-Cys, Gly-Gly. 
}
\medskip


L’équation de la réaction conduisant à la séquence CG s’écrit :
\begin{center}
  \small
  \chemfig{!\cysteineSemiDev} + \chemfig{!\glycineSemiDev}
  \reaction
  \chemfig{HS !\cysteineSemiDevB N !\glycineSemiDevH OH} + A
\end{center}

\question{
  Donner le nom et la formule brute du composé A.
}{
  La molécule formée est une molécule d'eau \chemfig{H_20}.
}

\question{
  Écrire la formule semi-développée des dipeptides GC et CC aussi présents dans la séquence de l’hepcidine.
}{
  
}

\question{
  Sur les formules des dipeptides représentés à la question précédente, entourer la liaison peptidique.
}{}

\question{
  Donner le nom du groupe fonctionnel correspondant.
}{
  C'est un groupe amide. 
}