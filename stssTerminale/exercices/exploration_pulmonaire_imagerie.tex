\exercice{Exploration pulmonaire par imagerie médicale}\label{exo:pulmonaire_imagerie}

\motsClesImageriePulmonaireRadio
\medskip

Un patient fumeur peut aussi souffrir de liaisons aux poumons et présenter des difficultés respiratoires. Le médecin peut prescrire des explorations par imagerie médicale pour mesurer les volumes pulmonaires.

\begin{multicols}{2}
  \begin{doc}{Composition des tissus corporels}{doc:BB_composition}
    Les principaux éléments constitutifs des tissus mous (peau, muscles, graisse, tendons, vaisseaux sanguins et nerfs) sont l'hydrogène, le carbone, l'azote et l'oxygène.
    Les os, tissus corporels durs, sont constitués des mêmes éléments que les tissus mous et de sels minéraux inorganiques tels que le calcium, le phosphore et le magnésium.
  \end{doc}
  
  \begin{doc}{Radiographie thoracique}{doc:BB_}
    \begin{center}
      \image{0.5}{images/bacST2S/radio}

      \url{https://www.infirmiers.com}
    \end{center}
  \end{doc}
\end{multicols}

\textbf{Données :}
  \begin{itemize}
    \item 
    Vitesse de la lumière dans le vide ou dans l'air : $c = \qty{3,00e8}{\m\per\s}$.
  \end{itemize}
  \vspace*{-20pt}
  \begin{tableau}{|c |c |c |c |c |c |c |c |}
    Élément & Hydrogène & Carbone & Azote & Oxygène & Magnésium & Phosphore & Calcium \\
    Symbole & H & C & N & O & Mg & P & Ca \\
    Numéro atomique Z & 1 & 6 & 7 & 8 & 12 & 15 & 20
  \end{tableau}


\question{
  Rappeler le principe de la radiographie en précisant la nature des ondes utilisées.
}{
  Une radiographie consiste à irradier un objet avec de la lumière dans le domaine des rayons X pour former une image en négatif sur une plaque (comme un jeu d'ombre).
}

\question{
  Citer un point commun et une différence entre radiographie et radiothérapie.
}{
  Les deux méthodes utilisent les rayons X.
  La radiographie sert à faire de l'imagerie médicale (observation interne du corps),
  la radiothérapie sert à soigner du cancer en détruisant des cellules cancéreuses à l'aide de rayons X ciblés.
}

\question{
  En utilisant l'échelle de longueurs d'ondes ci-dessous, indiquer à quel numéro correspond le domaine des rayons X, utilisés en radiographie.
  \begin{center}
    \image{1}{images/bacST2S/domaine_onde}
  \end{center}
}{
  Le domaine des rayons X correspond au numéro 1.
  2 : ultraviolet.
  3 : infrarouge.
}

\question{
  Après avoir rappelé la relation entre fréquence et longueur d'onde ainsi que les unités associées, déterminer l'intervalle de fréquences correspondant aux rayons X en utilisant l'échelle présentée à la question 3.
}{
  $f = \dfrac{c}{\lambda}$, avec $c$ la vitesse de la lumière en \unit{\m\per\s}, $f$ la fréquence en \unit{\per\s} et $\lambda$ la longueur d'onde en \unit{\m}.

  La plus grande fréquence du domaine X sera donc 
  \begin{equation*}
    f_\text{max}
    = \dfrac{\qty{3,00e8}{\m\per\s}}{\qty{e-11}{\m}}
    \simeq \qty{e19}{\s}
  \end{equation*}
  et la plus petite
  \begin{equation*}
    f_\text{min}
    = \dfrac{\qty{3,00e8}{\m\per\s}}{\qty{e-8}{\m}}
    \simeq \qty{e16}{\s}
  \end{equation*}
}

\question{
  Les rayons X peuvent traverser certains des tissus corporels.
  En identifiant dans le document 2 les tissus corporels visualisés sur la radiographie, indiquer ceux qui ont tendance à absorber le plus fortement les rayons X et proposer une explication.
}{
  La plaque est blanche initialement et noircit en présence de rayon X.
  Donc les tissus osseux sont ceux qui absorbent le plus les rayons X, ce qui est lié à leur composition chimique :
  principalement du phosphore et du calcium, qui absorbent plus fortement les rayons X.
}

La scintigraphie est parfois utilisée dans le diagnostic d'un lymphome.
On utilise dans ce cas un marqueur radioactif contenant du molybdène-99, de symbole \isotope{99}{42}{Mo}.

\question{
  Donner la composition d'un noyau atomique de molybdène-99.
}{
  Un noyau de molybdène-99 contient \num{99} nucléons, avec \num{42} protons et \num{57} neutrons.
}
  
L'équation de désintégration du molybdène-99 est partiellement donnée ci-dessous.
Elle fait apparaître le rayonnement $\gamma$ utilisé en scintigraphie et une particule notée \isotope{...}{...}{A}, de nature à déterminer.
  
  \begin{center}
    \isotope{99}{42}{Mo} 
    $\rightarrow$
    \isotope{99}{43}{Tc} 
    + \isotope{...}{...}{A} + $\gamma$
  \end{center}

\question{
  Compléter le symbole de la particule \isotope{...}{...}{A} en remplaçant les pointillés par les nombres appropriés.
}{
  Le nombre de nucléons n'a pas changé au cours de la réaction ($99 \reaction 99$), par contre le nombre de charge a augmenté ($42 \reaction 43$), donc pour respecter les règles de conservations, on doit avoir \isotope{0}{-1}{A}.
}

\question{
  Identifier la particule \isotope{...}{...}{A},
  est-ce un positron \isotope{0}{+1}{e}
  ou un électron \isotope{0}{-1}{e} ?
  Nommer le type de désintégration subie par le molybdène-99.
}{
  La particule est un électron \isotope{0}{-1}{e}, on a donc une désintégration $\beta-$
}