\exercice{Exploration pulmonaire par imagerie médicale}\label{exo:imagerie}

\textbf{Mots-clés :} Radiographie, fréquence, longueur d'onde, noyau atomique, radioactivité.
\medskip

Un patient fumeur peut aussi souffrir de liaisons aux poumons et présenter des difficultés respiratoires. Le médecin peut prescrire des explorations par imagerie médicale pour mesurer les volumes pulmonaires.

\begin{multicols}{2}
  \begin{doc}{Composition des tissus corporels}{doc:BB_composition}
    Les principaux éléments constitutifs des tissus mous (peau, muscles, graisse, tendons, vaisseaux sanguins et nerfs) sont l'hydrogène, le carbone, l'azote et l'oxygène.
    Les os, tissus corporels durs, sont constitués des mêmes éléments que les tissus mous et de sels minéraux inorganiques tels que le calcium, le phosphore et le magnésium.
  \end{doc}
  
  \begin{doc}{Radiographie thoracique}{doc:BB_}
    \begin{center}
      \image{0.5}{images/bacST2S/radio}

      \url{https://www.infirmiers.com}
    \end{center}
  \end{doc}
\end{multicols}

\textbf{Données :}
  \begin{itemize}
    \item 
    Vitesse de la lumière dans le vide ou dans l'air : $c = \qty{3,00e8}{\m\per\s}$.
  \end{itemize}
  \vspace*{-20pt}
  \begin{tableau}{|c |c |c |c |c |c |c |c |}
    Élément & Hydrogène & Carbone & Azote & Oxygène & Magnésium & Phosphore & Calcium \\
    Symbole & H & C & N & O & Mg & P & Ca \\
    Numéro atomique Z & 1 & 6 & 7 & 8 & 12 & 15 & 20
  \end{tableau}


\question{
  Rappeler le principe de la radiographie en précisant la nature des ondes utilisées.
}{}

\question{
  Citer un point commun et une différence entre radiographie et radiothérapie.
}{}

\question{
  En utilisant l'échelle de longueurs d'ondes ci-dessous, indiquer à quel numéro correspond le domaine des rayons X, utilisés en radiographie.
  \begin{center}
    \image{1}{images/bacST2S/domaine_onde}
  \end{center}
}{}

\question{
  Après avoir rappelé la relation entre fréquence et longueur d'onde ainsi que les unités associées, déterminer l'intervalle de fréquences correspondant aux rayons X en utilisant l'échelle présentée à la question 3.
}{}

\question{
  Les rayons X peuvent traverser certains des tissus corporels.
  En identifiant dans le document 2 les tissus corporels visualisés sur la radiographie, indiquer ceux qui ont tendance à absorber le plus fortement les rayons X et proposer une explication.
}{}

La scintigraphie est parfois utilisée dans le diagnostic d'un lymphome.
On utilise dans ce cas un marqueur radioactif contenant du molybdène-99, de symbole \isotope{99}{42}{Mo}.

\question{
  Donner la composition d'un noyau atomique de molybdène-99.
  
L'équation de désintégration du molybdène-99 est partiellement donnée ci-dessous.
Elle fait apparaître le rayonnement $\gamma$ utilisé en scintigraphie et une particule notée \isotope{...}{...}{A}, de nature à déterminer.
  
  \begin{center}
    \isotope{99}{42}{Mo} 
    $\rightarrow$
    \isotope{99}{43}{Tc} 
    + \isotope{...}{...}{A} + $\gamma$
  \end{center}
}{}

\question{
  Compléter le symbole de la particule \isotope{...}{...}{A} en remplaçant les pointillés par les nombres appropriés.
}{}

\question{
  Identifier la particule \isotope{...}{...}{A},
  est-ce un positron \isotope{0}{+1}{e}
  ou un électron \isotope{0}{-1}{e} ?
  Nommer le type de désintégration subie par le molybdène-99.
}{}