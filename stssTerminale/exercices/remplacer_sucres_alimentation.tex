\exercice{Remplacer les sucres dans l'alimentation} \textbf{(10pts)}

\textbf{Mots-clés :} Concentrations en masse et en quantité de matière, dose journalière
admissible (DJA).
\medskip

Les aliments riches en sucres favorisent l'apparition du diabète. Le diabète est déclaré si la
concentration en masse $C_m$ de sucres dans le sang à jeun est supérieure à \qty{1,26}{\g\per\litre}.
L'organisation mondiale de la santé (OMS) préconise de limiter l'apport en sucres à \qty{10}{\percent} de la ration énergétique totale qui s'élève en moyenne à \qty{e4}{\kilo\joule} par jour pour l'adulte.
Certaines personnes choisissent de remplacer le sucre de leur alimentation par un édulcorant.

\begin{doc}{Le glucose}{doc:BAC_glucose}
  Une des molécules issue de la dégradation partielle du saccharose (sucre de table)
  dans l'organisme est le glucose dont la forme linéaire a pour formule partiellement développée :
  \begin{center}
    \chemfig{
      HO -CH_2 
      -C(-[-3] H) (-[3] OH)
      -C(-[-3] H) (-[3] OH)
      -C(-[-3] OH) (-[3] H)
      -C(-[-3] OH) (-[3] H)
      -C(-[-2] H) =[2] O
    }
  \end{center}
\end{doc}

\begin{doc}{La stévia}{doc:BAC_stevia}
  Le Rebaudioside A, extrait de la stévia, plante originaire du Paraguay, a un pouvoir sucrant tel
  qu'une sucrette contenant 20 mg de Rebaudioside A produit le même goût sucré qu'un morceau
  de sucre contenant l'équivalent de 5,0 g de glucose.
  Cependant l'agence européenne de sécurité des aliments (EFSA)
  a fixé la dose journalière admissible (DJA)
  pour le Rebaudioside A à 4,0 milligrammes par kilogramme de masse corporelle
  (DJA = \qty{4,0}{\milli\g\per\kg}).

  \begin{flushright}
    \textit{D'après www.efsa.europea.eu/}
  \end{flushright}
\end{doc}

\begin{donnees}
  \item Masse molaire moléculaire du glucose $M_\text{glucose} = \qty{180,0}{\g\per\mol}$.
  \item Le glucose a une valeur énergétique par unité de masse de \qty{15,6}{\kilo\joule\per\g}.
\end{donnees}

\question{
  Recopier la formule chimique du glucose.
  Entourer et nommer deux groupes fonctionnels différents de la molécule de glucose.
}{
  On a des groupes hydroxyle (alcool) tous le long de la chaîne et un groupe carbonyle (aldéhyde) en bout de chaîne à droite.
  \points{2}
}

\question{
  Donner la formule brute du glucose.
}{
  \chemfig{C_{6} H_{12} O_{6}}
  \points{1}
}

\question{
  Expliquer qualitativement pourquoi le glucose est soluble dans le sang considéré comme une
solution aqueuse.
}{
  \points{1}
}

\question{
  L'analyse sanguine d'un patient à jeun indique une concentration en quantité de matière de glucose égale à \qty{7,8}{\milli\mole\per\litre}.
  Montrer que ce résultat confirme que ce patient souffre du diabète.
}{
  Pour avoir la concentration massique en glucose, on multiplie la concentration molaire par la masse molaire moléculaire du glucose $c_m = \qty{180,0}{\g\per\mole} \times \qty{7,8e-3}{\mol\per\litre} = \qty{1,40}{\g\per\litre}$.
  Comme $C_m$ est supérieur à \qty{1,26}{\g\per\litre}, le patient souffre de diabète.
  \points{2}
}

\question{
  La consommation quotidienne en sucre de ce patient est équivalente à 75 g de glucose.
  Indiquer si cette consommation est conforme à celle préconisée par l'OMS.
}{
  L'énergie produite par le glucose ingéré est $E = \qty{75}{\g} \times \qty{15,6}{\kilo\joule\per\g} = \qty{1170}{\kilo\joule}$, ce qui est largement supérieur à la valeur préconisé par l'OMS (\qty{104}{\kilo\joule}) !
  \points{1}
}

\question{
  Ce patient, qui pèse 68 kg, envisage de remplacer sa consommation de sucre par du Rebaudioside A.
  Calculer, à l'aide du document 2, la masse maximale de cet édulcorant qu'il peut consommer par jour.
}{
  Il peut ingérer au maximum $\qty{4,0}{\mg\per\kg} \times \qty{68}{\kg} = \qty{272}{\mg}$ par jour.
  \points{1}
}

\question{
  En déduire le nombre de sucrettes qu'il peut consommer par jour.
}{
  Il peut consommer $\dfrac{272}{20} = 14$ sucrette par jour.
  \points{1}
}

\question{
  Indiquer s'il peut substituer sa consommation quotidienne de sucre,
  équivalente à 75 g de glucose, par la consommation de Rebaudioside A.
}{
  Comme une sucrette équivaut à \qty{5}{g} de sucre, il peut consommer l'équivalent de $14\times \qty{5}{\g} = \qty{70}{\g}$ par jour, ce qui ne remplace pas sa consommation quotidienne.
  \points{1}
}