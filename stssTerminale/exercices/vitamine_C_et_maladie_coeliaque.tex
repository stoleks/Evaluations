\exercice{Vitamine C et maladie cœliaque (10 points)} \label{exo:vitamine_C_maladie_coeliaque}

\medskip

Les patients atteints de maladie cœliaque présentent parfois des carences en vitamine C. Cette vitamine (ou acide ascorbique) est présente naturellement dans les fruits et agrumes.
Parmi les agrumes, le citron est celui qui en contient le plus.
La vitamine C est présente dans le jus de citron mais se dégrade rapidement à l’air libre.
On souhaite vérifier, par titrage colorimétrique à l’aide du diiode (\chemfig{I_2}), la teneur en vitamine C (\chemfig{C_6 H_8 O_6}) dans un jus de citron.

Lors de ce titrage, l’équation de la réaction support du dosage est :
\begin{equation*}
  \chemfig{C_6 H_8 O_6}\aq + \chemfig{I_2}\aq 
  \reaction
  \chemfig{C_6 H_6 O_6}\aq + 2 \chemfig{I^{-}}\aq + 2 \ionHydrogene\aq
\end{equation*}

\begin{donnees}
  
  \item Couples oxydant / réducteur : 
  \begin{itemize}
    \item \chemfig{I_2}\aq/\chemfig{I^{-}}\aq
    \item \chemfig{C_6 H_6 O_6}\aq / \chemfig{C_6 H_8 O_6}\aq
  \end{itemize}
  \item Masse molaire de la vitamine C (\chemfig{C_6 H_8 O_6}) : $M = \qty{176}{\g\per\mol}$
\end{donnees}

\begin{doc}{Protocole du titrage colorimétrique}{doc:protocole_colorimetrique}
  \begin{protocole}
    \item Prélever un volume $V_0 = \qty{10,0}{\ml}$ de jus de citron filtré.
    \item L’introduire dans un erlenmeyer et ajouter quelques gouttes d’un indicateur coloré, l’empois d’amidon, bleu foncé en présence de \chemfig{I_2} et incolore en son absence.
    \item Introduire dans la burette graduée la solution aqueuse de diiode de concentration en quantité de matière $C_1 = \qty{2,5e-3}{\mol\per\l}$.
    \item Verser progressivement la solution de diiode dans l’erlenmeyer jusqu’au changement de couleur de l’indicateur coloré.
  \end{protocole}
\end{doc}


\question{
  Préciser si la vitamine C ou acide ascorbique (\chemfig{C_6 H_8 O_6}) est un oxydant ou un réducteur dans le contexte du dosage.
}{}

\question{
  Légender le schéma du montage sur l'ANNEXE À RENDRE AVEC LA COPIE du titrage décrit dans le document 1 en indiquant le nom de l’espèce titrante et de l’espèce titrée.
}{}

\question{
  Définir l’équivalence d’un titrage.
}{}

\question{
  Indiquer par quel changement de couleur est repérée l’équivalence lors du titrage décrit dans le document 1.
}{}

\question{
  Le volume à l’équivalence est VE = 9,0 mL. Montrer que la quantité de matière n0 de vitamine C contenue dans 10,0 mL de jus de citron est de n0 = 2,3 × 10- 5 mol.
}{}

\question{
  En déduire la masse de vitamine C m0 contenue dans 10,0 mL de jus de citron.
}{}

\begin{doc}{Masse de quelques oligo-éléments et vitamine dans 100 mL de jus de citron, immédiatement après l’avoir pressé}{doc:oligo-element}
  \begin{tblr}{
    colspec = {X[c] X[c] X[c] X[c]},
    hlines, vlines, row{1} = {couleurSec-100},
  }
    Potassium & Calcium & Magnésium & Vitamine C \\
    \qty{138}{\mg} & \qty{26}{\mg} & \qty{8}{\mg} & \qty{53}{\mg}
  \end{tblr}
\end{doc}

\question{
  Comparer le résultat du dosage précédent avec la valeur indiquée dans le document 2. À l’aide des connaissances et des documents à disposition, proposer une explication au résultat obtenu.
}{}

Une supplémentation en vitamine C peut être conseillée aux patients. Des comprimés de vitamine C vendus dans le commerce contiennent une masse m de vitamine C égale à \qty{500}{\mg}.


\question{
  Calculer le volume de jus de citron fraîchement pressé (document 2) qu’il faudrait boire pour absorber 500 mg de vitamine C. Commenter le résultat.
}{}