\exercice{Étude d'un antiseptique préopératoire}\label{exo:antiseptique}

\textbf{Mots-clés :} Dilution, dosage par étalonnage, concentrations en masse et en quantité
de matière.
\medskip

L'implantation de capsules de curiethérapie nécessite une intervention chirurgicale.
La Bétadine® est un antiseptique local utilisé pour la désinfection préopératoire des patients.
Son principe actif est le diiode \chemfig{I_2} qui élimine les micro-organismes par son action oxydante.
Les solutions de diiode sont colorées en jaune allant jusqu'au brun selon leur concentration.
Dans la Bétadine®, le diiode est « emprisonné » dans un polymère appelé polyvidone.
Une mole de polyvidone iodée contient une mole de diiode.
\medskip

D'après la notice, la Bétadine® à \qty{10}{\percent} contient \qty{10}{\g} de polyvidone iodée dans \qty{100}{\ml}.

\textbf{Données :}
\begin{itemize}
  \item Masse molaire de la polyvidone iodée : $M_\text{polyvidone iodée} = \qty{2363}{\g\per\mole}$.
  \item Masse molaire moléculaire du diiode : $M_{I_2} = \qty{253,8}{\g\per\mole}$.
\end{itemize}

On souhaite déterminer la teneur en diiode de la Bétadine® à \qty{10}{\percent} à l'aide d'un dosage spectrophotométrique par étalonnage.
Pour cela, on procède à l'étalonnage d'une gamme de solutions de diiode de concentrations $C(I_2)$ en quantité de matière de $I_2$, connues.
La mesure de l'absorbance $A$ de chaque solution est réalisée avec un spectrophotomètre UV–visible.
\medskip 

On obtient la courbe d'étalonnage donnée en ANNEXE (à rendre avec la copie de
chimie), qui représente l'absorbance $A$ des solutions en fonction de leur concentration
en quantité de matière de $I_2$, $C(I_2)$.

\question{
  Justifier à l'aide du graphique donné en ANNEXE (à rendre avec la copie de chimie) que l'absorbance A de la solution de diiode est proportionnelle à la concentration $C(I_2)$ en quantité de matière de diiode.
}{
  La courbe fournit en annexe est une droite qui passe par l'origine, donc les deux grandeurs sont proportionnelles.
}

\question{
  Pour comparer la solution commerciale de Bétadine® à \qty{10}{\percent} avec cette gamme
d'étalonnage, il est ici nécessaire de la diluer dix fois.
  Parmi le matériel disponible ci-dessous, choisir, en justifiant, l'association pipette jaugée / fiole jaugée à utiliser pour préparer la solution diluée souhaitée.
  Liste du matériel disponible :
  \begin{itemize}
    \item pipettes jaugées 2,0 mL, 10,0 mL, 20,0 mL ; 25,0 mL ;
    \item fioles jaugées 100,0 mL, 250,0 mL, 500,0 mL.
  \end{itemize}
}{
  On veut diluer 10 fois, donc il faut un rapport 10 entre la fiole jaugée et la pipette jaugée, donc 10,0 mL et 100,0 mL.
}

\question{
  Rappeler le protocole de la dilution.
}{
  Il faut 
  \begin{protocole}
    \item Prélever la solution à diluer avec la pipette jaugée, en s'arrêtant sur le trait de jauge.
    \item Verser la solution prélevée dans la fiole jaugée.
    \item Ajouter de l'eau distillée jusqu'au 2/3 de la fiole jaugée et agiter.
    \item Compléter jusqu'au trait de jauge de la fiole, puis agiter.
  \end{protocole}
}

\question{
  Sans modifier les réglages du spectrophotomètre, on mesure l'absorbance de la solution ainsi diluée. On trouve $A_\text{solution diluée} = \num{0,9}$.
  Déterminer graphiquement, à l'aide de l'ANNEXE (à rendre avec la copie de chimie), la concentration en quantité de matière de diiode de la solution.
  On fera apparaître la construction sur le graphique.
}{
  Avec le graphique, on trouve une concentration de \qty{0.0043}{\mol\per\litre}.
}

\question{
  En déduire que la concentration en quantité de matière de diiode dans la solution de Bétadine® à \qty{10}{\percent} est voisine de \qty{0,043}{\mol\per\litre}.
}{
  Comme on a dilué dix fois, il faut multiplier par dix la concentration, donc \qty{0,043}{\g\per\litre}.
}

\question{
  En déduire la concentration en masse de la polyvidone iodée dans la Bétadine® à
\qty{10}{\percent}.
}{
  Comme une mole de polyvidone contient une mole de diiode, leurs concentrations molaires sont égales $c(I_2) = c(polyvidone)$
  Pour convertir la concentration molaire en concentration massique, on multiplie la concentration molaire par la masse molaire
  \begin{equation*}
    c_m(polyvidone) = C(I_2) M_\text{polyvidone} 
    = \qty{0,043}{\mol\per\litre} \times \qty{2363}{\g\per\mole} 
    = \qty{101,6}{\g\per\litre}
  \end{equation*}
}

\question{
  Vérifier la cohérence de l'indication de la notice : « La Bétadine® à \qty{10}{\percent} contient 10 g de polyvidone iodée dans 100 mL ».
}{
  On a trouvé une concentration de $\qty{101,6}{\g\per\litre}$, or la notice annonce une concentration de $\dfrac{\qty{10}{\g}}{\qty{100}{\mL}} = \dfrac{10}{0,1}\unit{\g\per\litre} = \qty{100}{\g\per\litre}$, donc la notice est cohérente avec la mesure réalisé avec un écart de \qty{1}{\percent}.
}

\question{
  Identifier une cause possible de l'écart constaté. 
}{
  Ça peut être lié à la lecture graphique.
}