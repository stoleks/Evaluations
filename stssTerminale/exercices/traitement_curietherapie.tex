\exercice{Traitement d'un cancer par curiethérapie (10 points)}\label{exo:curietherapie}

\textbf{Mots-clés :} Isotopes, période radioactive, activité.
\medskip

Le tabac contribue à augmenter le risque du cancer de la prostate.
Ce cancer peut être soigné par curiethérapie.
Cette thérapie consiste à implanter, à travers le périnée, des capsules de la taille de grains de riz contenant de l'iode 125.
Ces implants restent à demeure. 
Les noyaux d'iode 125 (symbole \isotope{125}{53}{I}) sont radioactifs, ils émettent des particules de faible énergie et un rayonnement électromagnétique de longueur d'onde 
$\lambda_0 = \qty{0,034}{\nm}$.
Les particules sont absorbées par les parois de la capsule contenant l'iode.
L'irradiation des tissus entourant l'implant n'est due qu'au rayonnement électromagnétique.

\question{
  Les noyaux d'iode 125 et d'iode 123 sont des isotopes.
  Définir le terme « isotopes » et donner le symbole du noyau d'iode 123.
}{}

\question{
  Donner la composition d'un noyau d'iode 125.
}{}

\question{
  La réaction de désintégration d'un noyau d'iode 125, s'accompagne de l'émission d'électrons et d'un rayonnement électromagnétique.
  En exploitant le texte introductif, préciser ce qu'il advient des électrons et du rayonnement émis.
}{}

\question{
  À l'aide du document 1, déterminer le domaine des ondes électromagnétiques émises lors de cette désintégration radioactive. Rappel : \qty{1}{\nm} = \qty{e-9}{\m}.
}{}

\question{
  Définir la période radioactive d'un radioélément.
}{}

\question{
  À l'aide d'une construction graphique réalisée sur l'ANNEXE (à rendre avec la copie de chimie), montrer que la période radioactive de l'iode 125 est voisine de 60 jours.
}{}

\begin{doc}{Domaines spectraux des ondes électromagnétiques}{doc:BB2_domaines}
  \begin{center}
    \image{1}{images/bacST2S/domaine_onde_radio}
  \end{center}
\end{doc}

\question{
  Une capsule d'implant possède une activité initiale de 16 MBq. Calculer l'activité de cette capsule au bout de 120 jours.
}{}

\question{
  Expliquer pourquoi il est recommandé aux patients traités par curiethérapie à l'iode 125 d'éviter des contacts prolongés avec des femmes enceintes ou avec de jeunes enfants pendant les 6 mois qui suivent la pause des implants.
}{}


\question{
  Dans certains cas, le radioélément utilisé n'est pas l'iode 125 mais le palladium 103 qui a une période radioactive de 17 jours.
  Indiquer les avantages que l'usage du palladium peut présenter.
}{}