\exercice{Albuminurie (10 points)}\label{exo:albuminurie}

\medskip
\motsClesAlbuminurie
\medskip

Les complications chroniques de la drépanocytose peuvent endommager différents organes comme les reins.
Le recours à l’urographie peut être envisagé pour confirmer le diagnostic de drépanocytose.

\titreSousSection{ANALYSE D'URINE : Dosage de l'albumine par la méthode de Biuret}

Les albumines sont des protéines solubles dans l'eau.
L’albumine produite par le foie est la protéine la plus abondante dans le sang ; elle est filtrée dans les reins.
En cas d’anomalie rénale une quantité importante d’albumine peut atteindre les urines.
\begin{listePoints}
  \item Le taux normal d'albumine dans les urines ne doit pas dépasser \qty{50}{\mg\per\l}
  \item Lorsque cette dernière est supérieure à \qty{150}{\mg\per\l}, elle est considérée comme pathologique, il convient de consulter un spécialiste du rein.
\end{listePoints}
Une patiente a déposé dans un laboratoire d'analyses biologiques des échantillons de ses urines de la journée afin de déterminer leur concentration en masse d'albumine.
Une technique possible de dosage est la méthode de Biuret.


\begin{doc}{Principe de la méthode de Biuret}{doc:methode_biuret}
La méthode de Biuret utilise le réactif de Gornall dont l’action sur l’albumine donne un
produit de couleur bleu-violet. Un dosage spectrophotométrique par étalonnage est par
conséquent envisageable pour la longueur d’onde de 540 nm.
D’après : http://www.chimiegenerale.com/
\end{doc}

\begin{doc}{Réalisation du dosage spectrophotométrique}{doc:dosage_spectro}
  \important{1. Préparation d'une gamme étalon}
  
  À partir d’une solution aqueuse d’albumine de concentration en masse connue 
  $C_0 = \qty{5,0}{\g\per\l}$,
  on prépare une gamme étalon de quatre solutions aqueuses filles notées S$_1$, S$_2$, S$_3$, S$_4$ de concentrations en masse d’albumine respectivement égales à
  \qty{1,0}{\g\per\l} ; \qty{2,0}{\g\per\l} ; \qty{3,0}{\g\per\l} et \qty{4,0}{\g\per\l}.
  Chacune est colorée avec le réactif de Gornall en excès, selon un même protocole.
  
  \important{2. Mesure des absorbances}
  
  On relève les valeurs de l’absorbance des solutions filles colorées, à l’aide d’un
  spectrophotomètre. On obtient le graphique suivant.
  
  \image{1}{images/bacST2S/concentration_albumine_absorbance}
\end{doc}

\question{
  Pour préparer les solutions étalons S$_1$ à S$_4$, on a effectué des dilutions. Calculer le volume de solution mère S$_0$ d'albumine à prélever pour préparer 50,0 mL de la solution S$_1$.
}{
  Pour passer de S$_0$ à S$_1$, on a un facteur de dilution $F = 5$ (division par 5 de la concentration), il faut donc prélever un volume $F$ fois plus petit 
  \begin{equation*}
    V = \dfrac{\qty{50,0}{\ml}}{5} = \qty{10,0}{\ml}
    \points{1}
  \end{equation*}
}

\question{
  Proposer un protocole expérimental de dilution pour l’obtention de la solution S$_1$.
}{
  On prélève avec une pipette jaugée \qty{10,0}{\ml} de solution S$_0$, en faisant attention aux traits de jauge.
  On verse ces \qty{10,0}{\ml} dans une fiole jaugée de \qty{50,0}{\ml}.
  On ajoute de l'eau jusqu'au 2/3 du trait de jauge, puis on agite.
  Finalement, on complète jusqu'au trait de jauge et on agite à nouveau.
\points{2}
}

\question{
  En exploitant le graphique du document 2, justifier que la teinte bleu-violet d’une solution est d'autant plus intense que sa concentration en masse $C_m$ d’albumine est plus élevée.
}{
  L'absorbance est proportionnelle à la concentration massique, donc plus la concentration augmente, plus la solution s'assombrit et la teinte devient plus intense.
\points{1}
}

\question{
  La mesure de l'absorbance de l'urine de la patiente (colorée avec le réactif de Gornall selon le même protocole que pour les solutions de la gamme étalon) est A = 0,14. 
  Indiquer si ce résultat correspond à une situation normale ou si la patiente doit consulter un spécialiste.
}{
  Pour cette valeur d'absorbance, la concentration massique est $c_m \simeq \qty{1,4}{\g\per\litre} = \qty{1500}{\mg\per\litre}$.
  Comme cette valeur est au dessus de \qty{150}{\mg\per\litre}, la dose est pathologique et la patiente va donc devoir consulter un spécialiste du rein.
\points{1}
}

\titreSousSection{L'UROGRAPHIE}

Le médecin souhaite étudier la morphologie des voies urinaires de la patiente par une méthode radiographique.
Il procède à l’injection d'un produit de contraste à base d'iode (I) par voie intraveineuse puis prend des clichés du système urinaire.

\begin{doc}{Urographie}{doc:urographie}
  \separationBlocs{
    \centering
    Cliché a :
    Examen réalisé sans injection de produit de contraste.
    \medskip
    
    \image{0.8}{images/bacST2S/radio_urographie.png}
  }{
    \centering
    Cliché b :
    Examen réalisé 40 minutes après l’injection intraveineuse d’un produit de contraste.
    \medskip
    
    \image{0.8}{images/bacST2S/radio_urographie_contraste.png}
  }

  \sourceExtrait{Source : https://ims-77.fr}
\end{doc}
  
\important{Données :}

Les os contiennent principalement les éléments phosphore P et calcium Ca.
Les organes (reins ; uretères ; vessie ; urètre...) contiennent principalement les éléments : oxygène O, azote N, carbone C et hydrogène H.

Numéros atomiques :

\begin{tblr}{
  colspec = {X[c]X[c]X[c]X[c]X[c]X[c]X[c]X[c]},
  hlines, vlines, column{1} = {couleurSec-100},
}
  Éléments & H & C & N & O & P  & Ca & I  \\
  Z        & 1 & 6 & 7 & 8 & 15 & 20 & 53
\end{tblr}
\smallskip

\question{
  Préciser la nature des ondes utilisées lors d’une radiographie.
}{
  Ce sont des ondes électromagnétique dans le domaine X.
\points{1}
}

\question{
  Expliquer l’utilité d’un produit de contraste et proposer deux critères de choix d’un
produit de contraste.
}{
  Un produit de contraste améliore la visualisation de certains organes.
  La durée d'élimination et la tolérance par le corps sont des critères de choix pour le produit de contraste.
\points{1}
}

\question{
  En utilisant les données, justifier le fait que les os apparaissent plus clairs que les
autres organes sur les clichés.
}{
  Les éléments chimiques avec des numéros atomiques absorbent plus les rayons X.
  Les os sont composés de phosphore et de calcium, avec des numéros atomiques plus élevé que le carbone, l'oxygène, l'azote et l'hydrogène qui composent les autres tissus et organes.
  Les os absorbent donc plus les rayons X et apparaissent blanc sur la plaque, qui noircit au contact des rayons X (comme une ombre).
\points{3}
}

\question{
  Expliquer l’apparition de zones blanches sur le cliché b du document 3.
}{
  Le produit de contraste injecté est éliminé par les reins, et il contient de l'iode avec un numéro atomique élevé, ce qui rend les reins opaques aux rayons X et permet de les faire apparaître blanc sur le cliché radiographique.
\points{1}
}