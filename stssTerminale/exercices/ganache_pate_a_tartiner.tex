%%%%
\exercice{Une ganache à base de pâte à tartiner}\label{exo:ganache_pate_tartiner}

\motsClesGanache
\medskip

\begin{doc}{Les oméga 3 et 6}{doc:DC1_omega_3_6}
  Les oméga-3 et oméga-6 constituent une famille d'acides gras essentielle au bon fonctionnement du corps humain.
  Dans le cadre d'une alimentation équilibrée, l'agence française de sécurité sanitaire des aliments (Afssa)
  recommande un apport, en masse, au maximum cinq fois plus élevé d'oméga-6 que d'oméga-3.
  Un ratio plus élevé pourrait favoriser l'obésité.
  Les régimes occidentaux favorisent une surconsommation d'oméga-6 au détriment des oméga-3.
  Ainsi, en France, le ratio moyen est de 18 et aux États-Unis il peut monter jusqu'à 40.

  \begin{flushright}
    futurasciences.com 
  \end{flushright}
\end{doc}

\begin{doc}{Accumulation de graisse dans le corps humain}{doc:DC1_accumulation_graisse}
  Le surpoids et l'obésité sont dus à une accumulation excessive de graisse dans le corps.
  Cette accumulation de graisse peut résulter d'un excès d'acides gras provenant de la digestion
  des triglycérides.
  L'huile de palme, en particulier, est riche en triglycérides. Le tableau suivant rassemble
  quelques acides gras constitutifs des triglycérides de l'huile de palme.

  \begin{tableau}{|c |c |c |}
    Noms des acides gras & Famille d'acide gras & Masse pour \qty{100}{\g} \\
    Acide myristique        &        & \qty{1}{\g}    \\
    Acide palmitique        &        & \qty{43,5}{\g} \\
    Acide stéarique         &        & \qty{4,3}{\g}  \\
    Acide oléique           &oméga-9 & \qty{36,6}{\g} \\
    Acide linoléique        &oméga-6 & \qty{9,3}{\g}  \\
    Acide alpha-linolénique &oméga-3 & \qty{0,2}{\g}
  \end{tableau}
  
  \begin{flushright}
    wikipedia.org
  \end{flushright}
\end{doc}


%%
L'oléine est un triglycéride.
Par hydrolyse, on obtient entre autres un acide gras : l'acide oléique.
L'équation de la réaction d'hydrolyse est présentée ci-dessous, A et B désignent deux molécules.
\begin{center}  
  \chemfig{
    !\oleineSemiDev
    % H C (!\teteAcideDev C_{17} H_{33}) 
    % (-[3,1.7,2,2] H_2C (!\teteAcideDev C_{17} H_{33}))
    % -[-3,1.7,2,2] H_2 C (!\teteAcideDev C_{17} H_{33})
  }
  + 3\chemfig{A} \reaction \chemfig{B} 
  + 3 \chemname{\chemfig{!\oleiqueSemiDev}}{Acide oléique}%H_{33} C_{17} - C!\carboxyle}}{Acide oléique}
\end{center}
  
\question{
  Donner la définition d'un acide gras et d'un triglycéride.
}{
  Un acide gras est une molécule avec un groupe carboxyle \chemfig{COOH} lié à une longue chaîne carbonée.\points{1}

  Un triglycéride est une molécule de glycérol estérifié avec trois acide gras.\points{1}
}

\question{
  Nommer les molécules désignées par A et B dans l'équation de la réaction d'hydrolyse de l'oléine et préciser leur formule chimique.
  Écrire la formule semi-développée de la molécule B.
}{
  A : molécule d'eau \chemfig{H_2O}. \points{0,5}
  
  B : molécule de glycérol \bruteCHO{3}{8}{3} \points{0,5}

  \begin{center}  
    Glycérol : \chemfig{CH_2 (-[3] OH) - CH (-[3] OH) -CH_2 (-[3] OH)} \points{1}
  \end{center}
}


%%
\medskip
L'acide oléique a pour formule topologique :
\begin{center}
  \chemfig{!\oleique}
\end{center}

\question{
  Citer le groupe caractéristique présent dans cette molécule.
}{
  C'est un groupe carboxyle. \points{1}
}

\question{
  Justifier que l'acide oléique est un acide gras insaturé.
}{
  Il y a une double liaison carbone-carbone dans la chaîne carbonée de l'acide gras, donc il est insaturé en hydrogène. \points{1}
}


%%
\medskip
On hydrolyse \qty{100}{\g} d'huile de palme contenant \qty{38,2}{\percent} en masse d'oléine.

\important{Données :}
M$_\text{oléine}$ = \qty{885,4}{\g\per\mole} ; 
M$_\text{acide oléique}$ = \qty{282,5}{\g\per\mole}.

\question{
  Déterminer la quantité de matière $n_\text{oléine}$ d'oléine présente dans \qty{100}{\g} d'huile de palme.
}{
  On a une masse d'oléine $m_\text{oléine} = \qty{38,2}{\percent} \times \qty{100}{\g} = \qty{38,2}{\g}$, \points{1}
  donc la quantité de matière vaut
  \begin{equation*}
    n_\text{oléine} = \dfrac{m_\text{oléine}}{M_\text{oléine}}
    = \dfrac{\qty{38,2}{\g}}{\qty{885,4}{\g\per\mole}}
    = \qty{4,31e-2}{\mole} \points{1,5}
  \end{equation*}
}

\question{
  À partir de l'équation de la réaction d'hydrolyse supposée totale, 
  calculer la masse d'acide oléique dans l'huile de palme.
  Comparer avec celle mentionnée dans le tableau du document 2.
}{
  Comme la réaction est totale, toute l'oléine s'est transformée en acide oléique, donc $n_\text{oléique} = 3\times n_\text{oléine}$. \points{1}

  Et 
  \begin{equation*}
    m_\text{oléique} = n_\text{oléique} \times M_\text{oléique}
    = 3\times \qty{4,31e-2}{\mole} \times \qty{282,5}{\g\per\mole}
    = \qty{36,6}{\g} \points{1,5}
  \end{equation*}

  On trouve la même valeur que dans le tableau du document 2, les deux résultats sont cohérents. \points{1}
}


%%
\medskip
Dans le cadre d'une alimentation équilibrée, il est conseillé de consommer quotidiennement
\qty{500}{\mg} d'oméga-3.

\question{
  Calculer la masse d'huile de palme qu'il faudrait manger pour respecter cet apport.
}{
  On a \qty{200}{\mg} d'oméga-3 pour \qty{100}{\g} d'huile de palme, soit $\dfrac{\qty{100}{\g}}{\qty{200}{\mg}} = \dfrac{\qty{1}{\g}}{\qty{2}{\mg}}$.

  En regardant les unités, on en déduit qu'il faut multiplier l'apport recommandé par cette fraction pour obtenir la masse d'huile de palme à consommer
  \begin{equation*}
    \qty{500}{\mg} \times \dfrac{\qty{1}{\g}}{\qty{2}{\mg}} 
    = \qty{250}{\g} \points{2}
  \end{equation*}
}