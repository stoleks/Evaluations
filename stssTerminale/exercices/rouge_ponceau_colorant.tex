\exercice{Le rouge Ponceau, un colorant alimentaire (10 points)}\label{exo:colorantAlimentaire}

\textbf{Mots-clés :} Dose journalière admissible, dosage par étalonnage, concentration en masse.
\medskip

\begin{doc}{La couleur des macarons}{doc:BAC_macaron}
  Les macarons sont des gâteaux individuels à l’amande dont les goûts peuvent être différents.
  Les macarons sont souvent colorés.
  Pour cela, certains professionnels n’hésitent pas à jouer la surenchère en ayant recours à un surdosage des colorants.
  Cependant, l’utilisation de ces substances dans les denrées alimentaires est rigoureusement encadrée par la réglementation sur les additifs. 
  
  \begin{flushright}
    \textit{Macarons, la ronde des couleurs | economie.gouv.fr }
  \end{flushright}
\end{doc}

\begin{doc}{Le colorant E124}{doc:BAC_colorant_E124}
  Le rouge Ponceau AR (E124) est un colorant azoïque de synthèse.
  C’est un additif alimentaire qui peut remplacer le rouge de cochenille (E120) car il est moins cher.
  En Europe, la dose journalière admissible (DJA) est de 0,7 milligramme par kilogramme de masse corporelle.
  En France, son usage doit s’accompagner de la mention « Peut avoir des effets indésirables sur l’activité et l’attention chez les enfants ». 

  \begin{flushright}
    \textit{Colorant-alimentaire.fr}
  \end{flushright}
\end{doc}

 
On souhaite déterminer la quantité en colorant E124 présente dans un macaron à l’aide d’un dosage par étalonnage avec un spectrophotomètre.  

Pour cela, on sèche puis on réduit en poudre un macaron de couleur rouge. On dissout cette poudre dans de l’eau.
Après filtration, on obtient une solution S de volume V = \qty{25}{\ml}.On considère que la totalité du rouge Ponceau AR (E124) contenu dans le macaron a été récupérée dans cette solution. 

\important{I \faMinus}
On réalise une courbe d’étalonnage représentée sur
\textbf{l’ANNEXE DE LA DERNIÈRE PAGE (À RENDRE AVEC LA COPIE DE CHIMIE)}
à partir de solutions étalons de concentrations connues en rouge Ponceau AR (E124).
Ces solutions sont obtenues par dilution d’une solution mère $S_0$
de concentration en masse \qty{100}{\mg\per\litre} en colorant E124.

\begin{doc}{}{doc:BAC_absorbance}
  On mesure l’absorbance des solutions
  \medskip
  
  \centering
  \begin{tblr}{
    colspec = {|c |c |c |c |c |}, hlines,
    row{1} = {couleurPrim!20}
  }
    Solutions étalons & $S_1$ & $S_2$ & $S_3$ & $S_4$ \\
    Concentration massique en \unit{\g\per\litre} & 50,0 & 25,0 & 12,5 & 5,0 \\
    Absorbance A sans unité & 1,56 & 0,82 & 0,37 & 0,16 \\
    Volume de la solution en \unit{\ml} & 20 & 20 & 20 & 20 \\
  \end{tblr}
\end{doc}
 
\question{
  Calculer le volume de solution mère $S_0$ à prélever pour réaliser la solution $S_2$.
}{
  On veut diviser la concentration par un facteur de dilution $F = \dfrac{\qty{100}{\mg\per\litre}}{\qty{25}{\mg\per\litre}} = 4$.
  Comme le volume de la solution $S_2$ final est $V_2 = \qty{20}{\ml}$, il faut prélever un volume $V_0 = \dfrac{V_2}{F} = \qty{5}{\ml}$.
  \points{1,5}
}

\question{
  Indiquer le volume d’eau à rajouter au prélèvement pour réaliser la solution $S_2$.
}{
  Comme on a prélevé \qty{5}{\ml}, il faut rajouter \qty{15}{\ml} pour réaliser la solution $S_2$.
  \points{1}
}

\question{
  Sur \textbf{l’ANNEXE (À RENDRE AVEC LA COPIE DE CHIMIE)}, compléter la deuxième ligne du tableau par les numéros (1 à 7) de façon à rendre compte de la chronologie des étapes à suivre pour réaliser la dilution.  
}{
  \points{1}
}

\medskip
\important{II \faMinus} La mesure de l’absorbance A de la solution $S$ est de 0,94. 

\question{
  En utilisant la droite d’étalonnage de \textbf{l’ANNEXE (À RENDRE AVEC LA COPIE DE CHIMIE)}, déterminer la concentration en masse en colorant E124 de la solution $S$ et indiquer les traits de construction nécessaires sur l’annexe. 
}{
  Sur la courbe d'étalonnage, on peut lire qu'une absorbance de 0,94 correspond à une concentration en colorant $c = \qty{30}{\mg\per\litre}$.
  \points{1}
}

\question{
  Montrer que la masse du colorant E124 contenu dans le macaron est d’environ 0,75 mg. 
}{
  Comme la totalité du colorant du macaron est passé dans la solution, il suffit de calculer la masse de colorant dans la solution
  $m = c \times V = \qty{30}{\mg\per\litre} \times \qty{25e-3}{\litre} = \qty{0,75}{\mg}$.
  \points{1,5}
}

\question{
  Définir la dose journalière admissible (DJA). 
}{
  C'est la quantité maximale d'un produit que l'on peut avaler tous les jours sans conséquences négatives sur la santé.
  \points{1}
}

\question{
  Indiquer si un enfant de 40 kg pourrait manger le contenu d’une boîte de 12 macarons rouges dans la journée sans dépasser la DJA du colorant E124. 
}{
  On multiplie la DJA et la masse de l'enfant pour déterminer la dose maximale à ne pas dépasser $\qty{40}{\kg} \times DJA = \qty{40}{\kg} \times \qty{0,7}{\mg\per\kg} = \qty{28}{\mg}$.

  Dans 12 macaron on a une masse de $12\times \qty{0,75}{\mg} = \qty{9}{\mg}$ de colorant, donc l'enfant peut manger 12 macarons sans danger.
  \points{2}
}

\question{
  Indiquer si cela présente un autre risque pour sa santé. 
}{
  En mangeant 12 macarons, l'enfant a certainement eu un apport en sucre trop élevé.
  \points{1}
}