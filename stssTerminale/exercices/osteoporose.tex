\exercice{Ostéoporose liée à l'age (8 points)}
\medskip

\begin{doc}{Ostéoporose}{doc:E3_osteoporose}
  \begin{wrapfigure}[10]{r}{0.4\linewidth}
    \centering
    \vspace*{-20pt}
    \image{1}{images/bacST2S/radio_jambe_osteoporose}
  \end{wrapfigure}
  
  \extrait{
    L'ostéoporose est une maladie diffuse du squelette caractérisée par une diminution de la densité osseuse et des altérations de la micro-architecture des os.
    Ces altérations rendent l'os plus fragile et augmentent le risque de fracture.
  }{Source : ameli.fr}
  
  Pour remédier à l'ostéoporose, une une activité sportive régulière est nécessaires.
  L'ostéoporose entraine une perte de calcium des os, ce qui les rend moins absorbants aux rayons X.
  On utilise les rayons X pour réaliser la radiographie des jambes d’une personne âgée pour contrôler la qualité de ses os.
  La plaque utilisée en radiographie est blanche et s'assombrit quand elle absorbe des rayons X.
  On obtient les deux images ci-contre. 
\end{doc}

\question{
  Identifier la jambe atteinte d'ostéoporose en comparant les deux image du document~\ref{doc:E3_osteoporose}.
  Justifier votre choix.
}{}

\question{
  Citer deux risques encourus suite à une exposition régulière aux rayonnement X.
}{}

\question{
  Indiquer un moyen utilisé par les manipulateurs ou manipulatrice en radiographie pour se protéger des rayons X.
}{}

\question{
  Expliquer pourquoi les os apparaissent blanc sur la radiographie.
}{}