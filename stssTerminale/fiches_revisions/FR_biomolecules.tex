\enTeteFiche{\termStssBiom}

\begin{tableauConnaissances}
  Je sais définir un acide $\alpha$-aminé.
  & & & \\
  % 
  Je sais reconnaître les groupes caractéristiques dans les formules de certains acides aminés.
  & & & \\
  % 
  Je sais repérer un carbone asymétrique dans une molécule et je peux en conclure si c'est une molécule chirale ou non.
  & & & \\
  % 
  Je connais la définition de molécules énantiomères et je peux identifier si deux molécules sont énantiomères à partir de leur représentation de Cram et de Fischer.
  & & & \\
  % 
  Je connais la nomenclature D et L d'un acide $\alpha$-aminé.
  & & & \\
  % 
  Je peux écrire la réaction de condensation entre deux acides $\alpha$-aminés et donner le nom du dipeptide formé.
  & & & \\
  % 
  Je sais repérer une liaison peptidique et retrouver les formules des acides aminés qui composent un peptide.
  & & & \\
  % 
  Je peux analyser des documents qui présentent le lien entre structure tridimensionnelle et action des protéines dans l'organisme.
  & & & \\
  % 
  Je sais distinguer des acides gras saturés et insaturés.
  & & & \\
  % 
  Je connais la définition d'un triglycéride.
  & & & \\
  % 
  Je connais la réaction d'hydrolyse et de saponification d'un triglycéride, je peux faire un bilan de matière et calculer le rendement d'une réaction.
  & & & \\
  %
  Je connais la définition de liposoluble et d'hydrosoluble.
  & & & \\
  % 
  Je peux analyser la structure du cholestérol pour en déduire ses propriétés de solubilités et comprendre son transport dans l'organisme.
  & & & \\
  % 
  Je sais comparer les structures des vitamines A, C et D et dire si elles sont liposolubles ou hydrosolubles.
  & & & \\
  % 
  Je peux relier le caractère liposoluble ou hydrosoluble d'une vitamine et les besoins journaliers associés à la vitamine.
  & & & \\
  %
  Je peux analyser des documents sur les colorants et les texturants alimentaires E.
  & & & \\
  %
  Je peux analyser des documents sur les arômes alimentaires naturels et de synthèse.
  & & & \\
  % 
\end{tableauConnaissances}

% \basDePageFicheReussite
% 
% \questionFicheReussite{3}