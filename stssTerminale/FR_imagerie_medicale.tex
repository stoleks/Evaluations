\enTeteFiche{\termStssImag}

\begin{tableauConnaissances}
  Je connais le principe d'une échographie et quel type d'onde est utilisé dans une échographie.
  & & & \\
  %
  Je peux utiliser la durée entre l'émission et la réception d'un ultrason pour calculer la distance parcourue.
  & & & \\
  % 
  Je peux réaliser une échographie simplifié à partir d'un protocole fourni.
  & & & \\
  % 
  Je connais l'effet Doppler et le principe d'une échographie Doppler.
  & & & \\
  % 
  Je peux analyser les résultats d'une échographie Doppler pour détecter une anomalie cardiaque.
  & & & \\
  %
  Je sais distinguer les caractéristiques d'une onde électromagnétique et d'une onde sonore.
  & & & \\
  %
  Je sais repérer le domaine des rayons X sur une échelle de longueur d'onde ou de fréquence.
  & & & \\
  %
  Je sais calculer une fréquence à partir de la vitesse de propagation et de la longueur d'onde.
  & & & \\
  %
  Je connais le principe d'une radiographie et je connais le lien entre numéro atomique et absorption des rayons X.
  Je connais les précautions liés à l'utilisation des rayons X.
  & & & \\
  %
  Je peux exploiter des documents pour comparer une radiographie et une radiothérapie.
  & & & \\
  %
  Je connais le principe d'une IRM et je sais quel type d'onde est utilisé pendant une IRM.
  & & & \\
  %
  Je sais qu'un produit de contraste améliore la visualisation d'un cliché d'imagerie médicale, je peux identifier les groupes fonctionnels d'un produit de contraste.
  & & & \\
  %
  Je connais la composition du noyau d'un atome et je peux identifier des isotopes d'un élément.
  & & & \\
  %
  Je peux identifier la nature d'une émission radioactive ($\alpha$, $\beta-$, $\beta+$, $\gamma$) à partir d'une équation de désintégration donnée.
  & & & \\
  %
  Je sais repérer le domaine des rayons $\gamma$ sur une échelle de longueur d'onde ou de fréquence.
  & & & \\
  %
  Je connais la définition de la période ou demi-vie radioactive d'un radio-isotope.
  Je peux la déterminer graphiquement.
  & & & \\
  %
  Je connais la définition de l'activité en becquerel \unit{\becquerel} et de la dose en sievert \unit{\sievert}.
  & & & \\
  %
  Je peux utiliser des documents pour comparer des marqueurs radioactifs utilisés en imagerie médicale.
  Je peux comparer les doses utilisées en médecine nucléaire et en radiothérapie nucléaire.
  & & & \\
  %
  Je connais les précautions liés à l'utilisation d'une source radioactive.
  & & & \\
  %
\end{tableauConnaissances}

% \basDePageFicheReussite
% 
% \questionFicheReussite{3}