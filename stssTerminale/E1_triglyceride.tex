\teteTermStssAlim

\vAligne{-50pt}
\titre{Partie chimie}

\textit{Les exercices sont indépendants.}
\medskip


%%%%
\exercice{Une ganache à base de pâte à tartiner (10 points)}
\medskip

\begin{doc}{Les oméga 3 et 6}{doc:E1_omega_3_6}
  Les oméga-3 et oméga-6 constituent une famille d'acides gras essentielle au bon fonctionnement du corps humain.
  Dans le cadre d'une alimentation équilibrée, l'agence française de sécurité sanitaire des aliments (Afssa)
  recommande un apport, en masse, au maximum cinq fois plus élevé d'oméga-6 que d'oméga-3.
  Un ratio plus élevé pourrait favoriser l'obésité.
  Les régimes occidentaux favorisent une surconsommation d'oméga-6 au détriment des oméga-3.
  Ainsi, en France, le ratio moyen est de 18 et aux États-Unis il peut monter jusqu'à 40.

  \begin{flushright}
    futurasciences.com 
  \end{flushright}
\end{doc}

\begin{doc}{Accumulation de graisse dans le corps humain}{doc:E1_accumulation_graisse}
  Le surpoids et l'obésité sont dus à une accumulation excessive de graisse dans le corps.
  Cette accumulation de graisse peut résulter d'un excès d'acides gras provenant de la digestion
  des triglycérides.
  L'huile de palme, en particulier, est riche en triglycérides. Le tableau suivant rassemble
  quelques acides gras constitutifs des triglycérides de l'huile de palme.

  \begin{tableau}{|c |c |c |}
    Noms des acides gras & Famille d'acide gras & Masse pour \qty{100}{\g} \\
    Acide laurique          &        & \qty{0,1}{\g}  \\
    Acide myristique        &        & \qty{1}{\g}    \\
    Acide palmitique        &        & \qty{43,5}{\g} \\
    Acide stéarique         &        & \qty{4,3}{\g}  \\
    Acide érucastique       &oméga-9 & \qty{0,1}{\g}  \\
    Acide oléique           &oméga-9 & \qty{36,6}{\g} \\
    Acide palmitoléique     &oméga-7 & \qty{0,3}{\g}  \\
    Acide linoléique        &oméga-6 & \qty{9,3}{\g}  \\
    Acide alpha-linolénique &oméga-3 & \qty{0,2}{\g}
  \end{tableau}
  
  \begin{flushright}
    wikipedia.org
  \end{flushright}
\end{doc}

\numeroQuestion
La palmitine est un triglycéride. Par hydrolyse, on obtient entre autres un acide gras :
l'acide palmitique.
L'équation de la réaction d'hydrolyse est présentée ci-dessous, A et B désignent deux molécules.
\begin{center}  
  \chemfig{
    H C (!\teteAcideDev C_{15} H_{31}) 
    (-[3,1.7,2,2] H_2C (!\teteAcideDev C_{15} H_{31}))
    -[-3,1.7,2,2] H_2 C (!\teteAcideDev C_{15} H_{31})
  }
  + 3\chemfig{A} \reaction \chemfig{B} 
  + 3 \chemname{\chemfig{H_{31} C_{15} - C!\carboxyle}}{Acide palmitique}
\end{center}
  
\numeroSousQuestion
Donner la définition d'un acide gras et d'un triglycéride.

\numeroSousQuestion
Nommer les molécules désignées par A et B dans l'équation de la réaction d'hydrolyse de la palmitine
et préciser leur formule chimique. Écrire la formule semi-développée de la molécule B.

\numeroQuestion L'acide palmitique a pour formule topologique :
\begin{center}
  \chemfig{HO -[1] !\palmitique}
\end{center}

\numeroSousQuestion
Citer le groupe caractéristique présent dans cette molécule.

\numeroSousQuestion
Justifier que l'acide palmitique est un acide gras saturé.

\numeroQuestion
On hydrolyse \qty{100}{\g} d'huile de palme contenant \qty{46}{\percent} en masse de palmitine.

\numeroSousQuestion
Déterminer la quantité de matière $n_\text{palmitine}$ de palmitine présente dans 100 g
d'huile de palme.
\textbf{Donnée :} M$_\text{palmitine}$ = \qty{807,3}{\g\per\mole}.

\numeroSousQuestion
À partir de l'équation de la réaction d'hydrolyse supposée totale,
comparer la teneur en masse en acide palmitique de cette huile de palme à celle mentionnée dans le tableau du document 2.
\textbf{Donnée :} M$_\text{acide palmitique}$ = \qty{256,0}{\g\per\mole}.

\numeroQuestion
L'huile de palme contient de l'acide linoléique et de l'acide alpha-linolénique qui
appartiennent respectivement à la famille des oméga-6 et oméga-3.

En s'appuyant sur l'ensemble des documents, indiquer si la pâte à tartiner contenant de
l'huile de palme dont la composition est donnée dans le document 2, peut être
considérée comme entrant dans le cadre d'une alimentation équilibrée.


%%%%
\newpage
\exercice{Autorégulation de l'apport en triglycérides (10 points)}
\medskip

Les triglycérides font partie, comme le cholestérol, des composés lipidiques de l'organisme.
Ils en constituent la principale réserve énergétique et sont indispensables au bon fonctionnement de l'organisme.

\begin{doc}{Résultat normaux de la concentration en masse des triglycérides dans le sang}{doc:E1_concentration_triglycerides}
  \begin{tableau}{|c |c |c |}
    Âge         & Femme \unit{\g\per\litre} & Homme \unit{\g\per\litre} \\
    0 – 4 ans   & \num{0,30} – \num{1,05}   & \num{0,30} – \num{1,00} \\
    4 – 10 ans  & \num{0,35} – \num{1,10}   & \num{0,30} – \num{1,05} \\
    10 – 15 ans & \num{0,35} – \num{1,35}   & \num{0,30} – \num{1,30} \\
    15 – 20 ans & \num{0,40} – \num{1,30}   & \num{0,35} – \num{1,50} \\
    Adultes     & \num{0,35} – \num{1,40}   & \num{0,45} – \num{1,75} \\
    > 70 ans    & \num{0,30} – \num{1,20}   & \num{0,45} – \num{1,50}
  \end{tableau}
\end{doc}

\begin{doc}{Comment faire baisser le taux sanguin de cholestérol ou de triglycérides ?}{doc:E1_taux_cholesterol}
  Pour faire baisser le taux sanguin de cholestérol ou de triglycérides, un changement de mode de vie est nécessaire (alimentation adaptée, activité physique...).
  Pour les personnes ayant un excès de cholestérol ou de triglycérides, il importe :
  \begin{listePoints}
    \item de réduire la consommation d'acide gras saturés d'origine animale
    (viande et produits carnés, fromage, beurre, etc.)
    ou végétale (huile de palme, coprah, etc.)
    et les acides gras trans issus de l'hydrogénation partielle des matières grasses
    (viennoiseries, pâtisseries, biscuits) ;
    \item de modérer les apports en cholestérol alimentaire (abats, foie, oeufs, etc.) ;
    \item de privilégier les acides gras insaturés d'origine animale (volaille) et végétale,
    qui sont sources d'acides gras oméga-9 (huile d'olive),
    oméga-6 et oméga-3 (huile de colza, soja, noix, margarines avec oméga-9,6 et 3) ;
    \item d'accroître la consommation des aliments sources de fibres alimentaires
    (céréales complètes et pain complet, légumes secs, fruits et légumes, etc.)
    et principalement de fibres solubles (avoine et orge).
  \end{listePoints}
  D'après \url{https://www.ameli.fr/}
\end{doc}

\numeroQuestion
La stéarine est le constituant principal de la graisse de boeuf.
Sa formule semi-développée est donnée ci-dessous :
\begin{center}
  \chemfig{
    H_2C (!\steraique)
    -[-3,1.7,2,2] H C (!\steraique)
    -[-3,1.7,2,2] H_2 C (!\steraique)
  }
\end{center}
Justifier que la stéarine est un triglycéride saturé.

\numeroQuestion
Écrire l'équation de la réaction d'hydrolyse de la stéarine et indiquer le nom ou le groupe
caractéristique des produits obtenus.

\numeroQuestion
Une patiente adulte a un taux de triglycéride de \qty{1,73}{\g\per\litre}.
Indiquer, en justifiant, si la patiente a un taux de triglycérides trop élevé.

\numeroQuestion
Elle envisage de supprimer les graisses de son alimentation.
Indiquer si un régime sans graisse peut être conseillé.

\numeroQuestion
Préciser le conseil qui pourrait lui être donné concernant la consommation de la viande de boeuf.

\numeroQuestion
L'acide alpha-linolénique de formule chimique \chemfig{C_{17}H_{29} - COOH}, est un acide gras oméga-3.
On le trouve dans les membranes des feuilles vertes des plantes et dans certaines graines.

\numeroSousQuestion
Recopier la formule chimique, donnée ci-dessus, de l'acide alpha-linolénique,
entourer et nommer le groupe caractéristique présent.

\numeroSousQuestion
La formule topologique de l'acide alpha-linolénique est représentée ci-dessous.
Indiquer s'il s'agit d'un acide gras saturé ou insaturé. Justifier la réponse.

\begin{center}
  \chemfig{H!\alphaLinolenique}
\end{center}

\numeroQuestion
Dans le cadre d'une alimentation équilibrée, il est conseillé de consommer quotidiennement
\qty{500}{\milli\g} d'oméga-3 que l'on trouve notamment dans les noix.
L'étiquette d'un sachet de noix en conserve indique \og \qty{250}{\g} de cerneaux de noix séchés \fg.
Les noix crues contiennent \qty{7,5}{\g} d'oméga-3 pour \qty{100}{\g} de noix séchées.
Calculer la masse de noix à consommer pour couvrir les besoins journaliers en oméga-3.

% \qty{500}{\milli\g} d'oméga-3 que l'on trouve notamment dans les poissons gras tels que le thon.
% Calculer la masse de thon à consommer pour couvrir les besoins journaliers en oméga-3.
% L'étiquette d'une boîte de thon en conserve indique «140 g de thon égoutté ».
% Ce thon au naturel contient 0,65 g d'oméga-3 pour 100 g de thon égoutté.
% Calculer la masse de thon à consommer pour couvrir les besoins journaliers en oméga-3.
