\teteTermStssOrga
\nomPrenom

\begin{center}
  \chemfig{
    O=[-1] C (-[1]OH) -[-3] C
    *6(=CH -CH =CH -CH =C (
        -[1] O -[-1] C (=[-3] O) -[1]CH_3
      )
      -
    )
  }
  \\[8pt]
  \important{Acide acétylsalicylique,} molécule composant l'aspirine.
\end{center}

\vspace*{-12pt}
\question{
  Donner le nom de la représentation de la molécule d'acide acétylsalicylique.
}{
  C'est la formule semi-développée.
}{1}

\numeroQuestion
Entourer les 2 groupes fonctionnels de l'acide acétylsalicylique.

\question{
  Donner le nom des 2 groupes et familles fonctionnelles dans l'acide acétylsalicylique.
}{
  Acide carboxylique (carboxyle), ester (ester).
}{2}

\question{
  Donner la formule brute de la molécule d'acide acétylsalicylique
}{
  \chemfig{C_9 H_8 O_4}
}{1}


\numeroQuestion
Écrire la formule topologique de l'acide acétylsalicylique.