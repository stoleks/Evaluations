\newpage
\titre{Révisions pour le bac}


%%%% Dosage par étalonnage
\exercice{Étude d'un antiseptique préopératoire}\label{exo:antiseptique}

\textbf{Mots-clés :} Dilution, dosage par étalonnage, concentrations en masse et en quantité
de matière.
\medskip

L'implantation de capsules de curiethérapie nécessite une intervention chirurgicale.
La Bétadine® est un antiseptique local utilisé pour la désinfection préopératoire des patients.
Son principe actif est le diiode \chemfig{I_2} qui élimine les micro-organismes par son action oxydante.
Les solutions de diiode sont colorées en jaune allant jusqu'au brun selon leur concentration.
Dans la Bétadine®, le diiode est « emprisonné » dans un polymère appelé polyvidone.
Une mole de polyvidone iodée contient une mole de diiode.
\medskip

D'après la notice, la Bétadine® à \qty{10}{\percent} contient \qty{10}{\g} de polyvidone iodée dans \qty{100}{\ml}.

\textbf{Données :}
\begin{itemize}
  \item Masse molaire de la polyvidone iodée : $M_\text{polyvidone iodée} = \qty{2363}{\g\per\mole}$.
  \item Masse molaire moléculaire du diiode : $M_{I_2} = \qty{253,8}{\g\per\mole}$.
\end{itemize}

On souhaite déterminer la teneur en diiode de la Bétadine® à \qty{10}{\percent} à l'aide d'un dosage spectrophotométrique par étalonnage.
Pour cela, on procède à l'étalonnage d'une gamme de solutions de diiode de concentrations $C(I_2)$ en quantité de matière de $I_2$, connues.
La mesure de l'absorbance $A$ de chaque solution est réalisée avec un spectrophotomètre UV–visible.
\medskip 

On obtient la courbe d'étalonnage donnée en ANNEXE (à rendre avec la copie de
chimie), qui représente l'absorbance $A$ des solutions en fonction de leur concentration
en quantité de matière de $I_2$, $C(I_2)$.

\question{
  Justifier à l'aide du graphique donné en ANNEXE (à rendre avec la copie de chimie) que l'absorbance A de la solution de diiode est proportionnelle à la concentration $C(I_2)$ en quantité de matière de diiode.
}{}{0}

\question{
  Pour comparer la solution commerciale de Bétadine® à \qty{10}{\percent} avec cette gamme
d'étalonnage, il est ici nécessaire de la diluer dix fois.
  Parmi le matériel disponible ci-dessous, choisir, en justifiant, l'association pipette jaugée / fiole jaugée à utiliser pour préparer la solution diluée souhaitée.
  Liste du matériel disponible :
  \begin{itemize}
    \item pipettes jaugées 2,0 mL, 10,0 mL, 20,0 mL ; 25,0 mL ;
    \item fioles jaugées 100,0 mL, 250,0 mL, 500,0 mL.
  \end{itemize}
}{}{0}

\question{
  Rappeler le protocole de la dilution.
}{}{0}

\question{
  Sans modifier les réglages du spectrophotomètre, on mesure l'absorbance de la solution ainsi diluée. On trouve Asolution diluée= 0,9.
  Déterminer graphiquement, à l'aide de l'ANNEXE (à rendre avec la copie de chimie), la concentration en quantité de matière de diiode de la solution.
  On fera apparaître la construction sur le graphique.
}{}{0}

\question{
  En déduire que la concentration en quantité de matière de diiode dans la solution de Bétadine® à \qty{10}{\percent} est voisine de \qty{0,043}{\mol\per\litre}.
}{}{0}

\question{
  En déduire la concentration en masse de la polyvidone iodée dans la Bétadine® à
\qty{10}{\percent}.
}{}{0}

\question{
  Vérifier la cohérence de l'indication de la notice : « La Bétadine® à \qty{10}{\percent} contient 10 g de polyvidone iodée dans 100 mL».
}{}{0}

\question{
  Identifier une cause possible de l'écart constaté. 
}{}{0}



%%%% Radiographie et radioactivité
\bigskip
\exercice{Exploration pulmonaire par imagerie médicale}\label{exo:imagerie}

\textbf{Mots-clés :} Radiographie, fréquence, longueur d'onde, noyau atomique, radioactivité.
\medskip

Un patient fumeur peut aussi souffrir de liaisons aux poumons et présenter des difficultés respiratoires. Le médecin peut prescrire des explorations par imagerie médicale pour mesurer les volumes pulmonaires.

\begin{multicols}{2}
  \begin{doc}{Composition des tissus corporels}{doc:BB_composition}
    Les principaux éléments constitutifs des tissus mous (peau, muscles, graisse, tendons, vaisseaux sanguins et nerfs) sont l'hydrogène, le carbone, l'azote et l'oxygène.
    Les os, tissus corporels durs, sont constitués des mêmes éléments que les tissus mous et de sels minéraux inorganiques tels que le calcium, le phosphore et le magnésium.
  \end{doc}
  
  \begin{doc}{Radiographie thoracique}{doc:BB_}
    \begin{center}
      \image{0.5}{stssTerminale/bac/radio}

      \url{https://www.infirmiers.com}
    \end{center}
  \end{doc}
\end{multicols}

\textbf{Données :}
  \begin{itemize}
    \item 
    Vitesse de la lumière dans le vide ou dans l'air : $c = \qty{3,00e8}{\m\per\s}$.
  \end{itemize}
  \vspace*{-20pt}
  \begin{tableau}{|c |c |c |c |c |c |c |c |}
    Élément & Hydrogène & Carbone & Azote & Oxygène & Magnésium & Phosphore & Calcium \\
    Symbole & H & C & N & O & Mg & P & Ca \\
    Numéro atomique Z & 1 & 6 & 7 & 8 & 12 & 15 & 20
  \end{tableau}


\question{
  Rappeler le principe de la radiographie en précisant la nature des ondes utilisées.
}{}{0}

\question{
  Citer un point commun et une différence entre radiographie et radiothérapie.
}{}{0}

\question{
  En utilisant l'échelle de longueurs d'ondes ci-dessous, indiquer à quel numéro correspond le domaine des rayons X, utilisés en radiographie.
  \begin{center}
    \image{1}{stssTerminale/bac/domaine_onde}
  \end{center}
}{}{0}

\question{
  Après avoir rappelé la relation entre fréquence et longueur d'onde ainsi que les unités associées, déterminer l'intervalle de fréquences correspondant aux rayons X en utilisant l'échelle présentée à la question 3.
}{}{0}

\question{
  Les rayons X peuvent traverser certains des tissus corporels.
  En identifiant dans le document 2 les tissus corporels visualisés sur la radiographie, indiquer ceux qui ont tendance à absorber le plus fortement les rayons X et proposer une explication.
}{}{0}

La scintigraphie est parfois utilisée dans le diagnostic d'un lymphome.
On utilise dans ce cas un marqueur radioactif contenant du molybdène-99, de symbole \isotope{99}{42}{Mo}.

\question{
  Donner la composition d'un noyau atomique de molybdène-99.
  
L'équation de désintégration du molybdène-99 est partiellement donnée ci-dessous.
Elle fait apparaître le rayonnement $\gamma$ utilisé en scintigraphie et une particule notée \isotope{...}{...}{A}, de nature à déterminer.
  
  \begin{center}
    \isotope{99}{42}{Mo} 
    $\rightarrow$
    \isotope{99}{43}{Tc} 
    + \isotope{...}{...}{A} + $\gamma$
  \end{center}
}{}{0}

\question{
  Compléter le symbole de la particule \isotope{...}{...}{A} en remplaçant les pointillés par les nombres appropriés.
}{}{0}

\question{
  Identifier la particule \isotope{...}{...}{A},
  est-ce un positron \isotope{0}{+1}{e}
  ou un électron \isotope{0}{-1}{e} ?
  Nommer le type de désintégration subie par le molybdène-99.
}{}{0}



%% Glucides, DJA, concentrations
%\bigskip
\newpage
\exercice{Remplacer les sucres dans l'alimentation}

\textbf{Mots-clés :} Concentrations en masse et en quantité de matière, dose journalière
admissible (DJA).
\medskip

Les aliments riches en sucres favorisent l'apparition du diabète. Le diabète est déclaré si la
concentration en masse $C_m$ de sucres dans le sang à jeun est supérieure à \qty{1,26}{\g\per\litre}.
L'organisation mondiale de la santé (OMS) préconise de limiter l'apport en sucres à \qty{10}{\percent} de la ration énergétique totale qui s'élève en moyenne à \qty{104}{\kilo\joule} par jour pour l'adulte.
Certaines personnes choisissent de remplacer le sucre de leur alimentation par un édulcorant.

\begin{doc}{Le glucose}{doc:BAC_glucose}
  Une des molécules issue de la dégradation partielle du saccharose (sucre de table)
  dans l'organisme est le glucose dont la forme linéaire a pour formule partiellement développée :
  \begin{center}
    \chemfig{
      HO -CH_2 
      -C(-[-3] H) (-[3] OH)
      -C(-[-3] H) (-[3] OH)
      -C(-[-3] OH) (-[3] H)
      -C(-[-3] OH) (-[3] H)
      -C(-[-2] H) =[2] O
    }
  \end{center}
\end{doc}

\begin{doc}{La stévia}{doc:BAC_stevia}
  Le Rebaudioside A, extrait de la stévia, plante originaire du Paraguay, a un pouvoir sucrant tel
  qu'une sucrette contenant 20 mg de Rebaudioside A produit le même goût sucré qu'un morceau
  de sucre contenant l'équivalent de 5,0 g de glucose.
  Cependant l'agence européenne de sécurité des aliments (EFSA)
  a fixé la dose journalière admissible (DJA)
  pour le Rebaudioside A à 4,0 milligrammes par kilogramme de masse corporelle
  (DJA = \qty{4,0}{\milli\g\per\kg}).

  \begin{flushright}
    \textit{D'après www.efsa.europea.eu/}
  \end{flushright}
\end{doc}

\textbf{Données :}
\begin{itemize}
  \item Masse molaire moléculaire du glucose $M_\text{glucose} = \qty{180,0}{\g\per\mol}$.
  \item Le glucose a une valeur énergétique par unité de masse de \qty{15,6}{\kilo\joule\per\g}.
\end{itemize}

\question{
  Recopier la formule chimique du glucose.
  Entourer et nommer deux groupes fonctionnels différents de la molécule de glucose.
}{
  On a des groupes hydroxyle (alcool) tous le long de la chaîne et un groupe carbonyle (aldéhyde) en bout de chaîne à droite.
  \points{2}
}{0}

\question{
  Donner la formule brute du glucose.
}{
  \chemfig{C_{6} H_{12} O_{6}}
  \points{1}
}{0}

\question{
  Expliquer qualitativement pourquoi le glucose est soluble dans le sang considéré comme une
solution aqueuse.
}{
  \points{1}
}{0}

\question{
  L'analyse sanguine d'un patient à jeun indique une concentration en quantité de matière de glucose égale à \qty{7,8}{\milli\mole\per\litre}.
  Montrer que ce résultat confirme que ce patient souffre du diabète.
}{
  Pour avoir la concentration massique en glucose, on multiplie la concentration molaire par la masse molaire moléculaire du glucose $c_m = \qty{180,0}{\g\per\mole} \times \qty{7,8e-3}{\mol\per\litre} = \qty{1,40}{\g\per\litre}$.
  Comme $C_m$ est supérieur à \qty{1,26}{\g\per\litre}, le patient souffre de diabète.
  \points{2}
}{0}

\question{
  La consommation quotidienne en sucre de ce patient est équivalente à 75 g de glucose.
  Indiquer si cette consommation est conforme à celle préconisée par l'OMS.
}{
  L'énergie produite par le glucose ingéré est $E = \qty{75}{\g} \times \qty{15,6}{\kilo\joule\per\g} = \qty{1170}{\kilo\joule}$, ce qui est largement supérieur à la valeur préconisé par l'OMS (\qty{104}{\kilo\joule}) !
  \points{1}
}{0}

\question{
  Ce patient, qui pèse 68 kg, envisage de remplacer sa consommation de sucre par du Rebaudioside A.
  Calculer, à l'aide du document 2, la masse maximale de cet édulcorant qu'il peut consommer par jour.
}{
  Il peut ingérer au maximum $\qty{4,0}{\mg\per\kg} \times \qty{68}{\kg} = \qty{272}{\mg}$ par jour.
  \points{1}
}{0}

\question{
  En déduire le nombre de sucrettes qu'il peut consommer par jour.
}{
  Il peut consommer $\dfrac{272}{20} = 14$ sucrette par jour.
  \points{1}
}{0}

\question{
  Indiquer s'il peut substituer sa consommation quotidienne de sucre,
  équivalente à 75 g de glucose, par la consommation de Rebaudioside A.
}{
  Comme une sucrette équivaut à \qty{5}{g} de sucre, il peut consommer l'équivalent de $14\times \qty{5}{\g} = \qty{70}{\g}$ par jour, ce qui ne remplace pas sa consommation quotidienne.
  \points{2}
}{0}



%%%% Loi des gaz parfait, bilan de matière
%\bigskip
\newpage
\exercice{Oxygénothérapie}

\textbf{Mots-clés :} Loi des gaz parfaits, bilan de matière, débit.
\medskip


La drépanocytose entraîne des crises douloureuses qui peuvent être atténuées par des médicaments antalgiques et une hydratation par voie intraveineuse, mais si la douleur persiste, la médicamentation peut être complétée par une oxygénothérapie.
L'oxygénothérapie consiste en un apport supplémentaire de dioxygène à l'organisme.

\begin{doc}{Utilisation des bouteilles de dioxygène}{doc:BAC_bouteille_dioxygene}
  Le dioxygène est stocké à l'état gazeux comprimé à une pression initiale de \qty{200}{\bar}, dans
  des bouteilles spécialement conçues et de différents volumes selon leur utilisation. À la
  sortie des bouteilles, la pression du gaz est réduite par un manodétendeur pour la rendre
  acceptable par le patient. Au fur et à mesure que la bouteille se vide, la pression du gaz à
  l'intérieur diminue.
  Un débitmètre permet de régler le débit de dioxygène suivant la prescription médicale.
  D'après l'Association Nationale pour les Traitements à Domicile
\end{doc}

\begin{doc}{Durée d'autonomie d'une bouteille de dioxygène B2}{doc:BAC_autonomie}
  Les bouteilles B2 de volume 2,0 litres sont utilisées pour l'oxygénothérapie de déambulation.
  La masse totale moyenne de la bouteille pleine de dioxygène comprimé à \qty{200}{\bar} est de \qty{5,8}{\kg}.
  La durée d'autonomie d'une bouteille B2 est donnée dans le tableau ci-dessous, pour différentes valeurs de la pression initiale de la bouteille et du débit de dioxygène délivré par le manodétendeur.

  \centering
  \smallskip
  \begin{tblr}{
    colspec = {|c |c |c |c |c |}, hlines,
    row{1,2} = {couleurPrim!20}
  }
    \SetCell[r=2]{c} { Pression dans la \\ bouteille en bar } &
    \SetCell[c=4]{c} Débit de \chemfig{O_2} à la sortie du manodétendeur & & & \\
    & \qty{3}{\litre\per\minute}
    & \qty{6}{\litre\per\minute}
    & \qty{9}{\litre\per\minute}
    & \qty{15}{\litre\per\minute} \\
    200 & 2 h 15 min & 1 h 05 min & 0 h 45 min & 0 h 25 min \\
    150 & 1 h 40 min & 0 h 50 min & 0 h 30 min & 0 h 20 min \\
    100 & 1 h 05 min & 0 h 30 min & 0 h 20 min & 0 h 10 min \\
    50  & 0 h 30 min & 0 h 15 min & 0 h 10 min & < 10 min
  \end{tblr}

  \begin{flushright}
    \textit{D'après ansm.sante.fr pour les bouteilles d'Air liquide.}
  \end{flushright}
\end{doc}

\begin{doc}{La loi des gaz parfaits}{doc:BAC_gaz_parfait}
  \begin{equation*}
    \Large
    P \times V = n \times R \times T
  \end{equation*}
      
  P : pression du gaz (\unit{\pascal}) \\
  V : volume occupé par le gaz (\unit{\m\cubed}) \\
  n : quantité de matière du gaz (\unit{\mole}) \\
  R : constante des gaz parfaits où R = \qty{8,31}{\pascal\m\cubed \per\mol\per\kelvin} \\
  T : température du gaz (\unit{\kelvin}).
  On rappelle que $T(\unit{\kelvin}) = T(\unit{\degreeCelsius}) + 273$
\end{doc}

\textbf{Données :}
\begin{itemize}
  \item \qty{1}{\litre} = \qty{e-3}{\m\cubed} et \qty{1}{\m\cubed} = \qty{e3}{\litre}.
  \item \qty{1}{\bar} = \qty{e5}{\pascal}.
  \item Pression atmosphérique normale : $P_{atm} = \qty{1,01e5}{\pascal}$.
  \item Masse molaire moléculaire du dioxygène \chemfig{O_2} : $M_{\chemfig{O_2}} = \qty{32,0}{\g\per\mole}$.
\end{itemize}

\question{
  La pression du dioxygène à l'intérieur d'une bouteille B2 neuve est égale à \qty{200}{\bar}.
  Convertir cette valeur en pascal.
}{
  $\qty{200}{\bar} = \qty{200e5}{\pascal}$
  \points{1}
}{0}

\question{
  Montrer alors qu'à \qty{20}{\degreeCelsius}, la quantité de matière de dioxygène contenue dans la bouteille B2 neuve est voisine de $n_{\chemfig{O_2}} = \qty{16,4}{\mol}$.
}{
  On utilise la loi des gaz parfaits : 
  $n 
  = \dfrac{P\times V}{R \times T}
  = \dfrac{\qty{200e5}{\pascal} \times \qty{2e-3}{\m\cubed}} {\qty{8,31}{\pascal\m\cubed\per\mole\per\kelvin} \times \qty{293}{\kelvin}}
  = \qty{16,4}{\mole}$
  \points{1}
}{0}

\question{
  Calculer la masse $m_{\chemfig{O_2}}$ du dioxygène contenu à une pression de \qty{200}{\bar} dans la bouteille B2 neuve.
}{
  $m_{\chemfig{O_2}} 
  = M_{\chemfig{O_2}} \times n_{\chemfig{O_2}}
  = \qty{32,0}{\g\per\mole} \times \qty{16,4}{\mole}
  = \qty{526}{\g}$
  \points{1,5}
}{0}

\question{
  Montrer que la masse du gaz représente moins de \qty{10}{\percent} de la masse totale de la bouteille pleine.
}{
  \qty{5,8}{\kg} = \qty{5800}{\g}, donc la masse de dioxygène représente 
  $\dfrac{526}{5800} = \qty{9}{\percent} < \qty{10}{\percent}$ de la masse totale de la bouteille.
  \points{1}
}{0}

\question{
  Vérifier que le volume de dioxygène à la pression atmosphérique,
  libérable par la bouteille B2 neuve à la température de \qty{20}{\degreeCelsius} est d'environ \qty{0,4}{\m\cubed}.
}{
  Pour calculer le volume de dioxygène à pression atmosphérique, on utilise la loi des gaz parfaits 
  $V = \dfrac{n\times R\times T}{P} 
  = \dfrac{\qty{16,4}{\mole} \times \qty{8,31}{\pascal\m\cubed\per\mole\per\kelvin} \times \qty{293}{\kelvin}} {\qty{1,01e5}{\pascal}}
  = \qty{0,40}{\m\cubed}$
  \points{1,5}
}{0}

\question{
  La bouteille B2 est initialement à la pression de \qty{200}{\bar} et le manodétendeur est réglé pour
  délivrer un débit $D = \qty{3}{\litre\per\minute}$ de gaz à la pression atmosphérique.
  Vérifier que la durée d'autonomie est bien en accord avec celle indiquée dans le \textbf{document 2}.
  
  On rappelle que le débit $D$ d'écoulement d'un gaz ou d'un liquide est défini par :
  \begin{equation*}
    D = \dfrac{\text{volume écoulée}}{\text{durée de l'écoulement}} = \dfrac{V}{\Delta t}
  \end{equation*}
}{
  On peut calculer le temps que met la bouteille met à se vider à partir de la relation du débit
  \begin{equation*}
    \text{Durée} = \Delta t = \dfrac{V}{D} = \dfrac{\qty{400}{\litre}} {\qty{3}{\litre\per\minute}} = \qty{133}{\minute} = \text{2 h 13 min}
  \end{equation*}
  On trouve une durée cohérente.
  \points{2}
}{0}

\question{
  Justifier qualitativement l'évolution de la durée d'autonomie en fonction du débit du gaz.
}{
  Quand le débit augmente, la bouteille se vide plus rapidement et donc l'autonomie baisse.
  \points{1}
}{0}

\question{
  En exploitant la relation du \textbf{document 3},
  expliquer pourquoi la pression dans la bouteille diminue au fil de l'utilisation à température constante.
}{
  Le volume occupé et la température reste constante, mais la quantité de matière diminue, donc la pression diminue.
  \points{1}
}{0}



%%%% Echographie doppler
\bigskip
\exercice{Maladie cardiovasculaire}

\textbf{Mots-clés :} Échographie doppler, maladie cardiovasculaire
\medskip

\begin{doc}{Maladie cardiovasculaire}{doc:E3_cardiovasculaire}
  \extrait{
     Les maladies cardiovasculaires sont dues à une accumulation de dépôts de graisses (cholestérol) sur les parois des artères. 
     Ces dépôts forment des plaques appelées \important{plaques d'athérome.} Les parois des artères se durcissent.
     On parle alors \important{d'athérosclérose.}
     
     L'athérosclérose ne provoque dans un premier temps aucun symptôme.
     Puis, le rétrécissement des artères s'aggrave et entraîne un \important{ralentissement de la circulation sanguine} et une moins bonne oxygénation des organes (cœur, cerveau, muscles des jambes...) Les symptômes de la maladie cardiovasculaire apparaissent.

     La formation d'un caillot peut interrompre brutalement la circulation sanguine et provoquer un accident cardiovasculaire (infarctus du myocarde, accident vasculaire cérébral...)
   }{Source : ameli.fr}

   Pour contrôler la présence d'athérosclérose, on utilise \important{l'échographie Doppler.}
\end{doc}

\begin{doc}{Principe de l'échographie Doppler}{doc:E3_doppler}
  Lorsqu'une onde sonore ou ultrasonore émise par un émetteur rencontre un obstacle fixe, la fréquence de l'onde réfléchie est identique à la fréquence de l'onde émise.
  Si l'obstacle se déplace, la fréquence de l'onde réfléchie $f_r$ est différente de la fréquence de l'onde émise $f_e$.
  C'est l'effet Doppler.
  L'écart de fréquences est noté $\Delta f$.
  Il permet de déterminer le sens et la vitesse d'écoulement du sang dans les vaisseaux.
\end{doc}

\begin{doc}{Le décalage Doppler $\Delta f$}{doc:E3_decalage_doppler}
  Dans l'examen considéré dans cet exercice, l'écart de fréquences dû à l'effet Doppler est donné par la relation suivante :
  \begin{equation*}
    \Delta f = \dfrac{2f_e \times v}{c}
  \end{equation*}
  \begin{listePoints}
    \item $\Delta f$ : écart de fréquence mesuré en hertz noté \unit{\hertz}
    \item $f_e$ fréquence de l'onde émise en hertz (\unit{\hertz})
    \item $v$ vitesse d'écoulement des globules rouges (\unit{\m\per\s})
    \item $c$ célérité moyenne des ultrasons dans le corps humain (\unit{\m\per\s})
  \end{listePoints}
\end{doc}

\question{
  Lors d'une échographie Doppler mesurant la vitesse d'écoulement sanguin,
  préciser quels sont les composants du sang qui réfléchissent les ondes ultrasonores.
}{}{0}

\question{
  Compléter la légende dans les cadres du schéma donné dans l'ANNEXE (à rendre avec la copie de chimie).
}{}{0}

\question{
  Exprimer la vitesse $v$ d'écoulement du sang en fonction de $\Delta f$ et des paramètres $c$ et $f_e$ à l'aide de la formule du document~\ref{doc:E3_decalage_doppler}.
}{}{0}

\question{
  En utilisant les données suivantes, montrer que la vitesse v d'écoulement du sang dans cette artère vaut environ
  \qty{0,36}{\m\per\s}.
  
  \important{Données :} $f_e = \qty{4,5e6}{\hertz}$ ; $\Delta f = \qty{2,1e3}{\hertz}$ ;  $c = \qty{1540}{\m\per\s}$.
}{}{0}

\question{
  La vitesse normale d'écoulement sanguin dans une artère est comprise entre 55 et \qty{90}{\cm\per\s}.
  Indiquer si l'écoulement dans l'artère considérée présente une athérosclérose.
}{}{0}

\question{
  Donner les conséquences d'une athérosclérose.
}{}{0}



%%%% Oxydoreduction, sécurité routière
\bigskip
\exercice{La chimie d'un airbag}

\textbf{Mots-clés :} Airbag, oxydo-réduction, bilan de matière.
\medskip

\begin{doc}{Fonctionnement d'un airbag}{doc:airbag}
  Appelés sur le lieu d'un accident de la route, des policiers constatent qu'une voiture a percuté frontalement un arbre et que le conducteur, qui était seul à bord, n'est blessé que légèrement. L'airbag qui s'est déclenché au moment du choc a très probablement sauvé la vie du chauffeur.
  
  L'airbag est un coussin gonflable de sécurité qui équipe toutes les automobiles. Suite à une collision, il se gonfle en quelques millisecondes grâce à du diazote produit lors de transformations chimiques. 
  
  Lors d'un choc violent, une étincelle déclenche la décomposition de l'azoture de sodium \azoture(s) présent dans l'airbag en sodium \chemfig{Na}(s) et en diazote \chemfig{N_2}(g) selon la réaction chimique d'équation : 
  
  \begin{equation}   
    2\azoture(s)
    \reaction 
    2\chemfig{Na}(s) + 3\chemfig{N_2}(g)
  \end{equation}
  
  Le sodium produit par la réaction (1) réagit immédiatement et complètement avec du nitrate de potassium 
  \chemfig{KNO_3}(s) également présent dans l'airbag pour former à nouveau du diazote
  \chemfig{N_2}(g) ainsi que de l'oxyde de sodium \chemfig{Na_2O}(s) et de l'oxyde de potassium \chemfig{K_2O}(s). 
  
  La réaction chimique modélisant cette deuxième transformation est la suivante : 
  
  \begin{equation}
    10 \chemfig{Na}(s) + 2 \chemfig{KNO_3}(s)
    \reaction
    \chemfig{N_2}(g) + 5 \chemfig{Na_2O}(s) + \chemfig{K_2O}(s)
  \end{equation}
  
  L'oxyde de sodium \chemfig{Na_2O}(s) et de l'oxyde de potassium \chemfig{K_2O}(s) réagissent à leur tour,
  selon l'équation (3), sur de la silice \chemfig{SiO_2}(s) pour former une poudre inoffensive,
  le silicate alcalin de sodium et de potassium \chemfig{K_2Na_2SiO_4}(s) : 
  
  \begin{equation}
    \chemfig{Na_2O}(s) + \chemfig{K_2O}s + \chemfig{SiO_2}(s) 
    \reaction
    \chemfig{K_2Na_2SiO_4}(s)
  \end{equation}
  
  Pour des raisons de sécurité, toutes les espèces chimiques produites lors des transformations successives sont des solides, sauf le diazote. 
\end{doc}
%%
\medskip
\textbf{Données :}

Masses molaires atomiques :
$M_{Na} = \qty{23,0}{\g\per\mole}$ ;
$M_{N}  = \qty{14,0}{\g\per\mole}$.

Volume molaire gazeux dans les conditions de pression et de température considérées :
$V_m = \qty{24,0}{\litre\per\mole}$.

\qty{1}{\litre} = \qty{1000}{\cm\cubed}
\medskip
%% 

\question{
  En s'appuyant sur la description du fonctionnement de l'airbag, et en considérant que tous les réactifs mis en jeu sont totalement consommés, identifier les deux espèces chimiques restantes à l'issue de la succession des trois transformations et indiquer celle qui provoque le gonflement de l'airbag. 
}{
  \azoture est consommé pendant la réaction (1), \chemfig{Na} et \chemfig{KNO_3} pendant la réaction (2), \chemfig{SiO_2}, \chemfig{K_2O} et \chemfig{Na_2O} pendant la réaction (3).
  Il ne reste donc que du \chemfig{N_2} et du \chemfig{K_2Na_2SiO_4}.
  \points{2}
  
  Le diazote \chemfig{N_2} est dans un état gazeux.
  \points{0,5}
  
  C'est donc le diazote qui gonfle le ballon.
  \points{1,5}
}{0}

\medskip
 La quantité de matière totale de diazote formée 
$n_T(\chemfig{N_2})$ après le choc est reliée à la quantité de matière d'azoture de sodium décomposée 
$n_d(\azoture)$, telle que :  
\begin{equation*}
  n_T(\chemfig{N_2}) = \num{1,6} \times n_d(\azoture)
\end{equation*}

\question{
  La masse d'azoture de sodium décomposée lors du déclenchement de l'airbag est égale à \qty{82,0}{\g}. Calculer la quantité de matière totale de diazote formée.
}{
  \begin{equation*}
    n_d(\azoture)
    = \dfrac{m(\azoture)}{M(\azoture)}
    = \dfrac{m(\azoture)}{M_{Na} + 3M_{N}}
    = \dfrac{\qty{82,0}{\g}}{(\num{23,0} + 3\times\num{14,0})\unit{\g\per\mole}}
    = \qty{1,3}{\mole}
    \points{1,5}
  \end{equation*}

  \begin{equation*}
    n_T (\chemfig{N_2})
    = 1,6\times n_d(\azoture)
    = 1,6 \times \qty{1,3}{\mole}
    = \qty{2,1}{\mole}
    \points{1,5}
  \end{equation*}
}{0}

\question{
  Calculer le volume de l'airbag lorsqu'il est gonflé par le diazote formé.
}{
  Le volume de l'airbag correspond au volume de gaz produit au cours de la réaction.
  \points{0,5}
  
  \begin{equation*}
    V = n_T(\chemfig{N_2}) \times V_m = \qty{2,1}{\mole} \times \qty{24}{\litre\per\mole} = \qty{50}{\litre}
    \points{1,5}
  \end{equation*}  
}{0}

\question{
  Comparer le résultat obtenu à la question 3 avec le volume approximatif de l'airbag dont les dimensions sont précisées dans le document.
}{
  Le volume de l'airbag est approximativement celui d'un pavé
  \begin{equation*}  
    V_\text{airbag} = 70\times 70\times \qty{10}{\cm\cubed} = \qty{49000}{\cm\cubed}
    \points{1}
  \end{equation*}
  
  Comme \qty{1000}{\cm\cubed} = \qty{1}{\litre}, $V_\text{airbag} = \qty{49}{\litre}$.
  \points{1}
  
  Le volume approximatif de l'airbag est à peu près égal au volume calculé à partir de la réaction chimique $\qty{49}{\litre} \approx \qty{50}{\litre}$.
  Les deux calculs sont donc cohérent.
  \points{1}
}{0}


%%%% acide gras, triglycéride
\bigskip
\exercice{Une ganache à base de pâte à tartiner}

\textbf{Mots clés :} acide gras, triglycéride, hydrolyse
\medskip

\begin{doc}{Les oméga 3 et 6}{doc:DC1_omega_3_6}
  Les oméga-3 et oméga-6 constituent une famille d'acides gras essentielle au bon fonctionnement du corps humain.
  Dans le cadre d'une alimentation équilibrée, l'agence française de sécurité sanitaire des aliments (Afssa)
  recommande un apport, en masse, au maximum cinq fois plus élevé d'oméga-6 que d'oméga-3.
  Un ratio plus élevé pourrait favoriser l'obésité.
  Les régimes occidentaux favorisent une surconsommation d'oméga-6 au détriment des oméga-3.
  Ainsi, en France, le ratio moyen est de 18 et aux États-Unis il peut monter jusqu'à 40.

  \begin{flushright}
    futurasciences.com 
  \end{flushright}
\end{doc}

\begin{doc}{Accumulation de graisse dans le corps humain}{doc:DC1_accumulation_graisse}
  Le surpoids et l'obésité sont dus à une accumulation excessive de graisse dans le corps.
  Cette accumulation de graisse peut résulter d'un excès d'acides gras provenant de la digestion
  des triglycérides.
  L'huile de palme, en particulier, est riche en triglycérides. Le tableau suivant rassemble
  quelques acides gras constitutifs des triglycérides de l'huile de palme.

  \begin{tableau}{|c |c |c |}
    Noms des acides gras & Famille d'acide gras & Masse pour \qty{100}{\g} \\
    Acide myristique        &        & \qty{1}{\g}    \\
    Acide palmitique        &        & \qty{43,5}{\g} \\
    Acide stéarique         &        & \qty{4,3}{\g}  \\
    Acide oléique           &oméga-9 & \qty{36,6}{\g} \\
    Acide linoléique        &oméga-6 & \qty{9,3}{\g}  \\
    Acide alpha-linolénique &oméga-3 & \qty{0,2}{\g}
  \end{tableau}
  
  \begin{flushright}
    wikipedia.org
  \end{flushright}
\end{doc}


%%
L'oléine est un triglycéride.
Par hydrolyse, on obtient entre autres un acide gras : l'acide oléique.
L'équation de la réaction d'hydrolyse est présentée ci-dessous, A et B désignent deux molécules.
\begin{center}  
  \chemfig{
    H C (!\teteAcideDev C_{17} H_{33}) 
    (-[3,1.7,2,2] H_2C (!\teteAcideDev C_{17} H_{33}))
    -[-3,1.7,2,2] H_2 C (!\teteAcideDev C_{17} H_{33})
  }
  + 3\chemfig{A} \reaction \chemfig{B} 
  + 3 \chemname{\chemfig{H_{33} C_{17} - C!\carboxyle}}{Acide oléique}
\end{center}
  
\question{
  Donner la définition d'un acide gras et d'un triglycéride.
}{
  Un acide gras est une molécule avec un groupe carboxyle \chemfig{COOH} lié à une longue chaîne carbonée.\points{1}

  Un triglycéride est une molécule de glycérol estérifié avec trois acide gras.\points{1}
}{0}

\question{
  Nommer les molécules désignées par A et B dans l'équation de la réaction d'hydrolyse de l'oléine et préciser leur formule chimique.
  Écrire la formule semi-développée de la molécule B.
}{
  A : molécule d'eau \chemfig{H_2O}. \points{0,5}
  
  B : molécule de glycérol \bruteCHO{3}{8}{3} \points{0,5}

  \begin{center}  
    \chemfig{CH_2 (-[3] OH) - CH (-[3] OH) -CH_2 (-[3] OH)}

    Glycérol \points{1}
  \end{center}
}{0}


%%
\medskip
L'acide oléique a pour formule topologique :
\begin{center}
  \chemfig{H !\oleique}
\end{center}

\question{
  Citer le groupe caractéristique présent dans cette molécule.
}{
  C'est un groupe carboxyle. \points{1}
}{0}

\question{
  Justifier que l'acide oléique est un acide gras insaturé.
}{
  Il y a une double liaison carbone-carbone dans la chaîne carbonée de l'acide gras, donc il est insaturé en hydrogène. \points{1}
}{0}


%%
\medskip
On hydrolyse \qty{100}{\g} d'huile de palme contenant \qty{38,2}{\percent} en masse d'oléine.

\textbf{Données :}
M$_\text{oléine}$ = \qty{885,4}{\g\per\mole}.
M$_\text{acide oléique}$ = \qty{282,5}{\g\per\mole}.

\question{
  Déterminer la quantité de matière $n_\text{oléine}$ d'oléine présente dans \qty{100}{\g} d'huile de palme.
}{
  On a une masse d'oléine $m_\text{oléine} = \qty{38,2}{\percent} \times \qty{100}{\g} = \qty{38,2}{\g}$, \points{1}
  donc la quantité de matière vaut
  \begin{equation*}
    n_\text{oléine} = \dfrac{m_\text{oléine}}{M_\text{oléine}}
    = \dfrac{\qty{38,2}{\g}}{\qty{885,4}{\g\per\mole}}
    = \qty{4,31e-2}{\mole} \points{1,5}
  \end{equation*}
}{0}

\question{
  À partir de l'équation de la réaction d'hydrolyse supposée totale, 
  calculer la masse d'acide oléique dans l'huile de palme.
  Comparer avec celle mentionnée dans le tableau du document 2.
}{
  Comme la réaction est totale, toute l'oléine s'est transformée en acide oléique, donc $n_\text{oléique} = 3\times n_\text{oléine}$. \points{1}

  Et 
  \begin{equation*}
    m_\text{oléique} = n_\text{oléique} \times M_\text{oléique}
    = 3\times \qty{4,31e-2}{\mole} \times \qty{282,5}{\g\per\mole}
    = \qty{36,6}{\g} \points{1,5}
  \end{equation*}

  On trouve la même valeur que dans le tableau du document 2, les deux résultats sont cohérents. \points{1}
}{0}


%%
\medskip
Dans le cadre d'une alimentation équilibrée, il est conseillé de consommer quotidiennement
\qty{500}{\mg} d'oméga-3.

\question{
  Calculer la masse d'huile de palme qu'il faudrait manger pour respecter cet apport.
}{
  On a \qty{200}{\mg} d'oméga-3 pour \qty{100}{\g} d'huile de palme, soit $\dfrac{\qty{100}{\g}}{\qty{200}{\mg}} = \dfrac{\qty{1}{\g}}{\qty{2}{\mg}}$.

  En regardant les unités, on en déduit qu'il faut multiplier l'apport recommandé par cette fraction pour obtenir la masse d'huile de palme à consommer
  \begin{equation*}
    \qty{500}{\mg} \times \dfrac{\qty{1}{\g}}{\qty{2}{\mg}} 
    = \qty{250}{\g} \points{2}
  \end{equation*}
}{0}


%%%% Acides aminé, enantiomère
%\bigskip
\newpage
\exercice{Un sommeil réparateur pour se sentir mieux}\label{exo:gaba}

\textbf{Mots-clés :} Acides aminés, groupes caractéristiques, carbone asymétrique, énantiomérie.
\medskip

Le sommeil est indispensable pour récupérer de la fatigue accumulée par l'organisme.
Plusieurs acides aminés permettent d'assurer un sommeil de bonne qualité. Par exemple,
l'acide glutamique est un acide aminé précurseur du GABA (gamma aminobutyric acid) qui
est un neurotransmetteur ayant des propriétés sédatives.

\begin{doc}{Synthèse du GABA}{doc:synthese_GABA}
  L'énantiomère L de l'acide glutamique est un des 22 acides aminés protéinogènes. Le
  GABA est synthétisé à partir de l'acide glutamique selon une réaction d'équation :
  \begin{center}
      \chemname{HOOC-\chemfig{CH_2}-\chemfig{CH_2}-CH(\chemfig{NH_2})-COOH}{
        acide glutamique
      }
      \reaction
      \chemname{HOOC-\chemfig{CH_2}-\chemfig{CH_2}-\chemfig{CH_2}-\chemfig{NH_2}}{
        GABA
      } + \chemfig{CO_2}
  \end{center}
  Une enzyme favorise cette réaction en se liant à l'acide glutamique.
  Elle ne peut se lier qu'à son énantiomère L et non à l'énantiomère D,
  car son site de liaison présente une complémentarité de forme avec l'énantiomère L de l'acide glutamique.
  Le schéma ci-dessous illustre cette propriété.
  \begin{center}
    \image{0.75}{stssTerminale/enantiomere}
  \end{center}
\end{doc}

\question{
  Sur la formule topologique de l'acide glutamique représentée sur l'ANNEXE à rendre avec la copie de chimie,
  entourer et nommer les deux groupes caractéristiques qui justifient que cette molécule appartient à la famille des acides aminés.
}{}{0}

\question{
  Préciser, en justifiant, s'il s'agit d'un acide $\alpha$-aminé.
}{}{0}

\question{
  Indiquer si l'acide gamma-aminobutyrique dont la formule topologique est représentée
  sur l'ANNEXE à rendre avec la copie de chimie est aussi un acide aminé.
}{}{0}

\question{
  Définir ce que l'on appelle un « atome de carbone asymétrique »
  et indiquer la propriété qui découle de la présence d'un atome de carbone asymétrique dans une molécule.
}{}{0}

\question{
  Sur la formule topologique de l'acide glutamique représentée sur l'ANNEXE à rendre avec la copie de chimie,
  repérer la position de l'atome de carbone asymétrique par un astérisque (*).
}{}{0}

\question{
  Justifier que cette molécule possède deux énantiomères en précisant ce que cela signifie.
}{}{0}

\question{
  Donner les représentations de Cram et Fisher des 2 énantiomères de l'acide glutamique.
}{}{0}

\question{
  Justifier que l'enzyme favorisant la synthèse du GABA, schématisée ci-dessous,
  ne peut se lier qu'à l'un des énantiomères de l'acide glutamique
  \begin{center}
    \image{0.7}{stssTerminale/enzyme_gaba}
  \end{center}
}{}{0}


\begin{boite}
  \centering
  \textbf{ANNEXE - À RENDRE AVEC LA COPIE DE CHIMIE}
\end{boite}

%%
\textbf{Exercice \ref{exo:antiseptique}}
\begin{center}
\image{0.9}{stssTerminale/bac/etalonnage_betadine}
\end{center}

%%
\textbf{Exercice~\ref{exo:gaba}}
\begin{center}
  \chemname{
    \chemfig{
      HO-[1] (=[3]O) -[-1]-[1]-[-1] 
      (-[-3]NH_2) -[1] (=[3]O) -[-1]OH
    }
  }{
    Acide glutamique
  }
  \qq{}
  \chemname{
    \chemfig{
      H_2N -[-1]-[1]-[-1]-[1] (=[3]O) -[-1]OH
    }
  }{
    Acide gamma-aminobutyrique
  }
\end{center}