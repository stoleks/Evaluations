% \teteTermStssAlim
\newpage
\vAligne{-50pt}
\titre{Partie chimie}

%%%%
\vspace*{-4pt}
\exercice{Une ganache à base de pâte à tartiner (20 points)}
\vspace*{-8pt}
\medskip

\begin{doc}{Les oméga 3 et 6}[\label{doc:DC1_oméga_3_6}]
  Les oméga-3 et oméga-6 constituent une famille d'acides gras essentielle au bon fonctionnement du corps humain.
  Dans le cadre d'une alimentation équilibrée, l'agence française de sécurité sanitaire des aliments (Afssa)
  recommande un apport, en masse, \important{au maximum cinq fois plus élevé d'oméga-6 que d'oméga-3.}
  Un ratio plus élevé pourrait favoriser l'obésité.
  Les régimes occidentaux favorisent une surconsommation d'oméga-6 au détriment des oméga-3.
  Ainsi, en France, le ratio moyen est de 18 et aux États-Unis il peut monter jusqu'à 40.

  \begin{flushright}
    futurasciences.com 
  \end{flushright}
\end{doc}

\begin{doc}{Accumulation de graisse dans le corps humain}[\label{doc:DC1_accumulation_graisse}]
  Le surpoids et l'obésité sont dus à une accumulation excessive de graisse dans le corps.
  Cette accumulation de graisse peut résulter d'un excès d'acides gras provenant de la digestion
  des triglycérides.
  L'huile de palme, en particulier, est riche en triglycérides. Le tableau suivant rassemble
  quelques acides gras constitutifs des triglycérides de l'huile de palme.

  \begin{tableau}{|c |c |c |}
    Noms des acides gras & Famille d'acide gras & Masse pour \qty{100}{\g} \\
    Acide laurique          &        & \qty{0,1}{\g}  \\
    Acide myristique        &        & \qty{1}{\g}    \\
    Acide palmitique        &        & \qty{43,5}{\g} \\
    Acide stéarique         &        & \qty{4,3}{\g}  \\
    Acide érucastique       &oméga-9 & \qty{0,1}{\g}  \\
    Acide oléique           &oméga-9 & \qty{36,6}{\g} \\
    Acide palmitoléique     &oméga-7 & \qty{0,3}{\g}  \\
    Acide linoléique        &oméga-6 & \qty{9,3}{\g}  \\
    Acide alpha-linolénique &oméga-3 & \qty{0,2}{\g}
  \end{tableau}
  
  \begin{flushright}
    wikipedia.org
  \end{flushright}
\end{doc}

% \begin{doc}{Dose journalière de manganèse}{doc:DC1_DJA}
%   « À faible dose, le manganèse est un bioélément reconnu du monde végétal et animal.
%   Cet oligoélément, à des doses de l'ordre du milligramme par jour, est essentiel pour les enzymes du corps.
%   Les dérivés du manganèse sont toxiques à fortes doses. »

%   \begin{flushright}
%     wikipedia.org
%   \end{flushright}

%   La DJA du manganèse a été fixé à \qty{0,06}{\mg\per\kg} par l'OMS en 2006.
% \end{doc}

\begin{donnees}
  \item $M_\text{C} = \qty{12}{\g\per\mole}$.
  \item $M_\text{H} = \qty{1}{\g\per\mole}$.
  \item $M_\text{O} = \qty{16}{\g\per\mole}$.
\end{donnees}



%%
% \newpage
L'oléine est un triglycéride.
Par hydrolyse, on obtient entre autres un acide gras : l'acide oléique.
L'équation de la réaction d'hydrolyse est présentée ci-dessous, A et B désignent deux molécules.
\begin{center}  
  \chemfig{!\oleineSemiDev}
  + 3\chemfig{H_2O} \reaction \chemfig{!\glycerolSemiDev} 
  + 3 \chemname{\chemfig{!\oleiqueSemiDev}} {Acide oléique}
\end{center}
  
\question{
  Rappeler la définition d'un acide gras.
}{
  Un acide gras est une molécule composée d'un groupe carboxyle \chemfig{COOH}, avec une longue chaîne carbonée.\points{1}
}

\question{
  \textit{Bonus :} Rappeler la définition d'un triglycéride.
}{
  C'est une molécule composée de trois acides gras et de un glycérol estérifié.\points{1}
}

%%
\medskip
L'acide oléique a pour formule topologique :
\begin{center}
  \chemfig{!\oleique}
\end{center}


\question{
  Citer et entourer le groupe caractéristique présent dans cette molécule.
}{
  C'est un groupe carboxyle.\points{1,5}
}

%%
\medskip
\question{
  Justifier que l'acide oléique est un acide gras insaturé.
}{
  Il y a une double liaison carbone-carbone dans la chaîne carbonée de l'acide gras, donc il est insaturé en hydrogène. \points{1}
}

%%
\medskip
\question{
  Écrire la formule brute de l'acide oléique.
}{ 
  \bruteCHO{18}{34}{2}\points{1}
}

%%
\medskip
\question{
  Montrer que la masse molaire de la molécule $M_\text{acide oléique}$ d'acide oléique est  M$_\text{acide oléique}$ = \qty{282}{\g\per\mole}.
}{
  \begin{align*}
    M_\text{acide oléique}
    &= 18\times M_C + 34\times M_H + 2\times M_O \\
    &= (18\times 12 + 34 \times 1 + 2 \times 16) \unit{\g\per\mole} \\
    &= \qty{282}{\g\per\mole}
  \end{align*}\points{2}
}

%%
\medskip 

%%
\medskip
L'oléine a pour formule semi-développée :
\begin{center}  
  \chemfig{!\oleineSemiDev}
\end{center}

\question{
  Citer et entourer le groupe caractéristique présent 3 fois dans cette molécule.
}{
  On a trois groupes esters \chemfig{O-C=O}. \points{1,5}
}

%\question{
%  Donner le nom de la molécule B dans la réaction d'hydrolyse.
%}{

%}

%%
\medskip

\question{
  Une célèbre marque de pâte à tartiner est composée majoritairement d'huile de palme. 
  Indiquer si sa consommation permet d'avoir une alimentation équilibrée en oméga-3 et oméga-6, en utilisant les documents fournis.
}{
  Pour une alimentation équilibrée, il faut avoir au plus 5 fois plus d'oméga-6 que d'oméga-3, d'après le premier document.
  Or dans l'huile de palme, il y a \qty{9.3}{\g} d'oméga-6 pour \qty{0.2}{\g} d'oméga-3, soit 1 pour 47.
  Il n'y a donc pas assez d'oméga-3 pour que cela soit équilibrée.\points{2}
}

