\teteTermStssOrga
\nomPrenom

\begin{center}
  \chemfig{
    [:150]
    *6(-=-=-=) % phenyl
    -[0]-[-2]
      (-[-4] (=[6]O) -[-2]O -[-4]) % ester
    -[0]NH -[-2,,1] (=[-4] O) % Amide
    -[0] (-[2] NH_2) % amine
    -[-2] -[0] (=[2] O) -[-2] OH % acide carboxylique
  }
  \\[8pt]
  \important{Aspartame}, molécule au gout sucré
\end{center}

\question{
  Donner le nom de la représentation de la molécule d'aspartame
}{
  C'est la formule topologique.
}{1}

\question{
  Donner la formule brute de l'aspartame.
}{
  \chemfig{C_{14} H_{18} N_{2} O_{5}}
}{1}

\numeroQuestion
Entourer les quatre groupes fonctionnels dans la molécule d'aspartame. 

\question{
  Donner le nom des groupes fonctionnels et les noms des familles organiques associées.
}{
  Ester (ester), amide (amide), amine (amine) et carboxyle (acide carboxylique)
}{4}