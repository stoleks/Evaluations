\teteTermStssOrga
\nomPrenom

\begin{center}
  \chemfig{
    HO -[1]C (=[3] O) -[-1]CH (-[-3] NH_2) -[1]CH_2 -[-1]C (-[-3] NH_2) =[1] O
  }
  \\[8pt]
  \important{Asparagine}, molécule qui est un des 22 acides aminés protéinogène.
\end{center}

\question{
  Donner le nom de la représentation de la molécule d'asparagine.
}{
  C'est la formule semi-développée.
}{1}

\numeroQuestion
Entourer les 3 groupes fonctionnels de l'asparagine.

\question{
  Donner le nom des 3 groupes et familles fonctionnelles dans l'asparagine.
}{
  Acide carboxylique (carboxyle), amine (amine), amide (amide).
}{3}

\question{
  Donner la formule brute de la molécule d'asparagine
}{
  \chemfig{C_4 H_8 O_2 N_2}
}{1}


\numeroQuestion
Écrire la formule topologique de l'asparagine.