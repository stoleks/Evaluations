\teteTermStssOrga
\nomPrenom

\begin{center}
  \chemfig{
    H_3C -[1]C (-[-1] CH_3) % pied
    =[3]CH -[5]H_2C -[3,,2,2]H_2C -[1]C (-[3] CH_3) % partie centrale
    =[-1]CH -[1]CH_2 -[-1] OH % bras droit
  }
  \\[8pt]
  \important{Géraniol}, molécule liée à l'odeur de rose.
\end{center}

\question{
  Donner le nom de la représentation de la molécule de géraniol.
}{
  C'est la formule semi-développée.
}{1}

\numeroQuestion
Entourer le groupes fonctionnels du géraniol.

\question{
  Donner le nom du groupe fonctionnel et de la famille présente dans le géraniol.
}{
  Hydroxyle, alcool.
}{2}

\numeroQuestion
Écrire la formule développée du géraniol.