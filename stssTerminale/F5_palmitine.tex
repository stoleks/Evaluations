\teteTermStssOrga
\nomPrenom

\begin{center}
  \chemfig{
    H_2C (!\caproique)
    -[-3,1.7,2,2] H C (!\caproique)
    -[-3,1.7,2,2] H_2 C (!\caproique)
  }
  \\[8pt]
  \important{Caproïne}, triester de glycérol présent dans l'huile de palme
\end{center}

\question{
  Donner le nom de la représentation de la molécule de caproïne.
}{
  C'est la formule semi-développée.
}{1}

\question{
  Donner la formule brute de la caproïne.
}{
  \chemfig{C_{14} H_{18} N_{2} O_{5}}
}{1}

\numeroQuestion
Entourer les trois groupes fonctionnels de la caproïne.

\question{
  Donner le nom des trois groupes fonctionnels et les noms des familles organiques associées.
}{
  Ester (ester).
}{2}

\question{
  La caproïne est-elle saturée ou insaturée ? Justifier.
}{
  Elle est saturée, elle ne comporte que des liaisons simples
}{3}