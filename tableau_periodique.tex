%%%% Pour afficher un élément dans le tableau périodique
\NewDocumentCommand{\elementText}{m m m o}
{
  \begin{minipage}{2.2cm}
    \begin{center}
      \IfValueTF{#4}{ \vAligne{-20pt} }{ \vAligne{-34pt} } % position du nom
      {\small #3} \\[2pt] % nom de l'élément
      {\ensuremath\footnotesize \textbf{#1}} \\[6pt] % nombre atomique
      \chemfig[atom style={scale = 1.8}]{#2} % symbole atomique
      % \element{#1}{#2} % element symbol and atomic number
      \IfValueT{#4}{
        \\ {\small \qty{#4}{\g/\mole}}
      }
    \end{center}
  \end{minipage}
}


%%%% Pour afficher un tableau périodique
\newcommand{\tableauPeriodique}[1]{
\begin{tikzpicture}[font=\sffamily, scale=0.75, transform shape]
%% Couleur de remplissage
  \tikzstyle{elementFill}     = [fill=yellow!30]
  \tikzstyle{alkaliFill}      = [fill=red!45]
  \tikzstyle{alkaliEarthFill} = [fill=red!30]
  \tikzstyle{metalFill}       = [fill=red!15]
  \tikzstyle{metalloidFill}   = [fill=yellow!15]
  \tikzstyle{nonmetalFill}    = [fill=orange!15]
  \tikzstyle{halogenFill}     = [fill=orange!30]
  \tikzstyle{nobleGasFill}    = [fill=orange!45]

%% Style des éléments
  \tikzstyle{Element} = [
    draw=black, elementFill,
    minimum width=2.85cm, % Largeur de la case
    minimum height=2.7cm, % Hauteur de la case
    node distance=2.85cm % Espace entre deux case
  ]
  \tikzstyle{Alkali}      = [Element, alkaliFill]
  \tikzstyle{AlkaliEarth} = [Element, alkaliEarthFill]
  \tikzstyle{Metal}       = [Element, metalFill]
  \tikzstyle{Metalloid}   = [Element, metalloidFill]
  \tikzstyle{Nonmetal}    = [Element, elementFill]
  \tikzstyle{Halogen}     = [Element, halogenFill]
  \tikzstyle{NobleGas}    = [Element, nobleGasFill]
  \tikzstyle{PeriodLabel} = [font={\sffamily\LARGE}, node distance=2cm]
  \tikzstyle{GroupLabel}  = [font={\sffamily\LARGE}, minimum width=2.5cm, node distance=2cm]
  \tikzstyle{TitleLabel}  = [font={\sffamily\Huge\bfseries}]

%% Place des éléments
  #1
\end{tikzpicture}
}


%%%% Pour faciliter l'utilisation du tableau périodique
\newcommand{\elementH} {\elementText{1} {H} {Hydrogène}[1,00]}
\newcommand{\elementHe}{\elementText{2} {He}{Hélium}   [4,00]}
\newcommand{\elementLi}{\elementText{3} {Li}{Lithium}  [6,94]}
\newcommand{\elementBe}{\elementText{4} {Be}{Béryllium}[9,01]}
\newcommand{\elementB} {\elementText{5} {B} {Bore}     [10,8]}
\newcommand{\elementC} {\elementText{6} {C} {Carbone}  [12,0]}
\newcommand{\elementN} {\elementText{7} {N} {Azote}    [14,0]}
\newcommand{\elementO} {\elementText{8} {O} {Oxygène}  [16,0]}
\newcommand{\elementF} {\elementText{9} {F} {Fluor}    [19,0]}
\newcommand{\elementNe}{\elementText{10}{Ne}{Néon}     [20,2]}
\newcommand{\elementNa}{\elementText{11}{Na}{Sodium}   [23,0]}
\newcommand{\elementMg}{\elementText{12}{Mg}{Magnésium}[24,3]}
\newcommand{\elementAl}{\elementText{13}{Al}{Aluminium}[27,0]}
\newcommand{\elementSi}{\elementText{14}{Si}{Silicium} [28,1]}
\newcommand{\elementP} {\elementText{15}{P} {Phosphore}[31,0]}
\newcommand{\elementS} {\elementText{16}{S} {Soufre}   [32,1]}
\newcommand{\elementCl}{\elementText{17}{Cl}{Chlore}   [35,5]}
\newcommand{\elementAr}{\elementText{18}{Ar}{Argon}    [39,9]}
\newcommand{\elementK} {\elementText{19}{K} {Potassium}[39,1]}
\newcommand{\elementCa}{\elementText{20}{Ca}{Calcium}  [40,0]}