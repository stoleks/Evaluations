%%%%%%%%%%%%%%%%%%%%%%%%%%%%%%%%%%%%%%%%%%%%%%%%%%%%%%%%%%%%%
%------ Style commun à plusieurs fichiers
\tikzset{
  rectArrondi/.style = {
    rounded corners = 8pt, inner ysep = 10pt,
    text width = 3.75cm, align = center, fill = #1,
  },
  fleche/.style = {->, line width = 6pt, color = #1}
}

%%%%%%%%%%%%%%%%%%%%%%%%%%%%%%%%%%%%%%%%%%%%%%%%%%%%%%%%%%%%%
%------ Cercles, points et vecteurs
\pgfkeys{
  /planete/.is family, /planete,
  defaut/.style = {
    rayon = 20 pt,
    orbite = 0 pt,
    remplissage = couleurSec-100,
    contour = couleurSec-100,
    centre = A,
    satellite = {}
  },% stockage des clefs
  rayon/.store in = \planeteRayon,
  orbite/.store in = \planeteOrbite,
  centre/.store in = \planeteCentre,
  satellite/.store in = \planeteSatellite,
  contour/.store in = \planeteContour,
  remplissage/.store in = \planeteRemplissage
}
%
\NewDocumentCommand{\tikzSchemaPlanete}{O{}}{%
  \pgfkeys{/planete, defaut, #1}  
  \filldraw [color = \planeteContour, very thick, fill = white] (0,0) circle (\planeteOrbite);
  \filldraw [\planeteRemplissage] (0,0) circle (\planeteRayon);
  \filldraw (0, 0) circle (2pt) node[above = 4pt] {\planeteCentre};
  \filldraw (\planeteOrbite, 0) circle (2pt) node[right = 4pt] {\planeteSatellite};
}

% \tikzLabel (x, y) {texte} (xtexte, ytexte), trace un point avec un label; * = pas de point
\NewDocumentCommand{\tikzLabel}{s r() m d()}{
  \IfBooleanTF{#1}{
    \node at (#2) {#3};
  }{
    \IfValueTF{#4}{
      \filldraw (#2) circle (2pt);
      \node at (#4) {#3};
    }{
     \filldraw (#2) circle (2pt) node[above] {#3};
    }
  }
}
% \tikzVecteur [couleur] (x, y) (lx, ly) {legende} [position legende];  * = <->
\NewDocumentCommand{\tikzVecteur}{s O{black} r() r() m O{right}}{
  \coordinate (A) at (#3);
  \coordinate (B) at (#4);
  \coordinate (AB) at ($(A)+(B)$);
  \IfBooleanTF{#1}{
    \draw[#2, <->, very thick] (A) -- (AB) node[#6] {#5};
  }{
    \draw[#2, ->, very thick] (A) -- (AB) node[#6] {#5};
  }
}

% Trace une petite barre de pourcentage partiellement remplie sur la ligne
\NewDocumentCommand{\barrePourcentage}{O{couleurSec} m}{%
  \begin{tikzpicture}
    \fill[color=#1]     (0.0,    0.0) rectangle (#2*8ex, 1.5ex);
    \fill[color=#1-100] (#2*8ex, 0.0) rectangle (8.0ex,  1.5ex);
  \end{tikzpicture}
}


%%%%%%%%%%%%%%%%%%%%%%%%%%%%%%%%%%%%%%%%%%%%%%%%%%%%%%%%%%%%%------ Pour évaluer une expression mathémtique
%------ Trace une flèche de progression pour les plans de travail
\pgfkeys{% définition de la famille de clefs
  /prog/.is family, /prog,
  defaut/.style = {
    boucles = 1,
    largeur = 17,
    coeff = 2.4
  },% stockage des clefs
  boucles/.store in = \progBoucles,
  largeur/.store in = \progLargeur,
  coeff/.store in = \progCoeff,
}
%
\newcommand{\nombreBoucle}{}
\newdimen\tailleBoucles
\NewDocumentCommand{\flecheProgression}{s m}{%
  \pgfkeys{/prog, defaut, #2}
  \strut\vspace*{8pt}
  \edef\nombreBoucle{\fpeval{2 * (\progBoucles - 1)}}
  \begin{center}
    \begin{tikzpicture}
      \tikzset{styleFleche/.style={line width = 20pt, rounded corners = 10mm, color = couleurSec}}
      % Premier bout pour l'alignement, les parenthèses sont nécessaires
      \draw[styleFleche] (0, {
        (\nombreBoucle)*\progCoeff}) -- (1.1, {(\nombreBoucle)*\progCoeff
      }); 
      % Partie centrale répétée
      \draw[styleFleche]
        \foreach \x in {0,2,...,\nombreBoucle}  {
          \ifnum \x < \nombreBoucle
            (1, {(\nombreBoucle - \x)  *\progCoeff}) --
            (\progLargeur,  {(\nombreBoucle - \x)  *\progCoeff})  --
            (\progLargeur, {(\nombreBoucle - \x - 1)*\progCoeff}) --
            (0, {(\nombreBoucle - \x - 1)*\progCoeff}) --
            (0, {(\nombreBoucle - \x - 2)*\progCoeff}) --
            (1.1, {(\nombreBoucle - \x - 2)*\progCoeff})
          \fi
        };
      % Flèche finale
      \draw[-{Triangle [width = 36pt, length = 16pt]}, styleFleche] (0.8, 0) -- (\progLargeur, 0);
    \end{tikzpicture}
  \end{center}
  % retire l'espace ajouté par les boucles si demandé
  \IfBooleanTF{#1}{
  }{
    \vspace*{\fpeval{-72 * (\nombreBoucle + 1)} pt}
  }
}


%%%%%%%%%%%%%%%%%%%%%%%%%%%%%%%%%%%%%%%%%%%%%%%%%%%%%%%%%%%%%
%------ plan de classe

% Pour tracer une rangée d'élève à 2 ou 3 colonnes (<eleves>) (<eleves>) (<eleves>)
\pgfkeys{
  /rangee/.is family, /rangee,
  defaut/.style = {
    largeur = 3,
    hauteur = 2
  },% stockage des clefs
  largeur/.store in = \rangeeX,
  hauteur/.store in = \rangeeY
}
%
\ExplSyntaxOn
\fp_new:N \l_rangPositionX_fp% position de la place
\fp_new:N \l_largeurPlaceX_fp% largeur d'une place
\fp_new:N \l_hauteurPlaceY_fp% hauteur d'une place
\NewDocumentCommand{\rang}{ O{} >{\SplitList{,}}r() >{\SplitList{,}}r() >{\SplitList{,}}d() }{%
  \pgfkeys{/rangee, defaut, #1}
  \begin{tikzpicture}
    % Initialisation de la position horizontale et de la taille des places
    \fp_set:Nn \l_rangPositionX_fp {0}
    \fp_set:Nn \l_largeurPlaceX_fp {\rangeeX}
    \fp_set:Nn \l_hauteurPlaceY_fp {\rangeeY}
    % Première rangée
    \ProcessList{#2}{\rangImpl}
    \fp_add:Nn \l_rangPositionX_fp {1}
    % Deuxième rangée
    \ProcessList{#3}{\rangImpl}
    % Troisième rangée
    \IfValueT{#4}{
      \fp_add:Nn \l_rangPositionX_fp {1}
      \ProcessList{#4}{\rangImpl}
    }
  \end{tikzpicture}
  \bigskip
}
% Commande interne qui appelle \texteRectangle pour chaque élève
\NewDocumentCommand{\rangImpl}{m}{%
  \fp_add:Nn \l_rangPositionX_fp {\l_largeurPlaceX_fp}
  \texteRectangle
    (\fp_use:N \l_rangPositionX_fp, 0) % Position
    (\fp_use:N \l_largeurPlaceX_fp, \fp_use:N \l_hauteurPlaceY_fp) % largeur et hauteur
    {#1} % élève
}
\ExplSyntaxOff

% \texteRectangle [couleur cadre] (x, y) (lx, ly) {texte}
\NewDocumentCommand{\texteRectangle}{O{black} r() r() m}{
  \coordinate (A) at (#2);
  \coordinate (B) at (#3);
  \filldraw [fill=white, draw=#1, ultra thick] (A) rectangle ($(A)+(B)$);
  \coordinate (C) at (0.5*#3/2); % divide by two lx and ly
  \node at ($(A)+(C)$) [font=\sffamily] {\textbf{#4}};
}


%%%%%%%%%%%%%%%%%%%%%%%%%%%%%%%%%%%%%%%%%%%%%%%%%%%%%%%%%%%%%
%------ tube à essais
\pgfkeys{
  /tube/.is family, /tube,
  defaut/.style = {
    largeur = 0.25cm,
    hauteur = 1.5cm,
    rayon = 0.1cm,
    phase = 0.25cm,
    couleur = red-300
  },% stockage des clefs
  largeur/.store in = \tubeX,
  hauteur/.store in = \tubeY,
  rayon/.store in = \tubeRayon,
  phase/.store in = \tubePhase,
  couleur/.store in = \tubeCouleur
}

% Tube à essais vide
\NewDocumentCommand{\tikzTubeEssais}{O{}}{
  \pgfkeys{/tube, defaut, #1}
  \draw[thick] (\tubeX, \tubeY)-- (\tubeX, 0) arc (0:-180:\tubeX)-- (-\tubeX, \tubeY);
  % note : le + 0 permet d'éviter que l'unité et "and" soit collé et donc non compris par pgf
  \draw[thick] (0, \tubeY) ellipse (\tubeX+0 and \tubeRayon);
}

% Liquide dans un tube à essais
\NewDocumentCommand{\tikzPhaseBasTubeEssais}{O{}}{
  \pgfkeys{/tube, defaut, #1}
  \fill[color=\tubeCouleur] (-\tubeX, 0)-- (\tubeX, 0) arc (0:-180:\tubeX); % bas
  \fill[color=\tubeCouleur] (\tubeX, \tubeY)-- (-\tubeX, \tubeY)-- (-\tubeX, -0.01)-- (\tubeX, -0.01);% centre
  \fill[color=\tubeCouleur!85!black] (0, \tubeY) ellipse (\tubeX+0 and \tubeRayon);% ellipse du haut
}

% Phase d'un liquide dans un tube à essais
\NewDocumentCommand{\tikzPhaseTubeEssais}{O{}}{
  \pgfkeys{/tube, defaut, #1}
  \fill[color=\tubeCouleur] (\tubeX, \tubePhase) -- (-\tubeX, \tubePhase) -- (-\tubeX, \tubeY) -- (\tubeX, \tubeY); % centre de la phase
  \fill[color=\tubeCouleur] (0, \tubeY) ellipse (\tubeX+0 and \tubeRayon); % bas de la phase
  \fill[color=\tubeCouleur!85!black] (0, \tubePhase) ellipse (\tubeX+0 and \tubeRayon);% haut de la phase
}

% tube à essai avec une seule solution
\newcommand{\tubeEssaisSolution}[1]{
  \begin{tikzpicture}
    \tikzPhaseBasTubeEssais [couleur = #1, largeur = 0.25cm, hauteur = 0.75cm]% contenu du tube
    \tikzTubeEssais% tube
  \end{tikzpicture}
}

%%%%%%%%%%%%%%%%%%%%%%%%%%%%%%%%%%%%%%%%%%%%%%%%%%%%%%%%%%%%%
%------ Pour les circuits
\NewDocumentCommand{\fixedvlen}{O{0.5cm} m m O{}}{% [semilength]{node}{label}[extra options]
  % get the center of the standard arrow
  \coordinate (#2-Vcenter) at ($(#2-Vfrom)!0.5!(#2-Vto)$);
  % draw an arrow of a fixed size around that center and on the same line
  \draw[-Triangle, #4] ($(#2-Vcenter)!#1!(#2-Vfrom)$) -- ($(#2-Vcenter)!#1!(#2-Vto)$);
  % position the label as in the normal voltages
  \node[anchor=\ctikzgetanchor{#2}{Vlab}, #4] at (#2-Vlab) {#3};
}

%%%%%%%%%%%%%%%%%%%%%%%%%%%%%%%%%%%%%%%%%%%%%%%%%%%%%%%%%%%%%
%------ Pour tracer un mini-emploi du temps
% Clés pour tracer un emploi du temps
\pgfkeys{
  /emploi du temps/.is family, /emploi du temps,
  defaut/.style = {
    largeur = \linewidth,
    hauteur = 0.5*\linewidth,
    hauteur jours = 1 cm,
    largeur heures = 2 cm,
    nombre jours = 5,
    nombre creneaux = 9,
    arrondi = 4pt,
    bordure = 2pt
  },% stockage des clefs
  hauteur/.store in = \edtHauteur,
  largeur/.store in = \edtLargeur,
  hauteur jours/.store in = \edtHauteurJours,
  largeur heures/.store in = \edtLargeurHeures,
  nombre jours/.store in = \edtNombreJours,
  nombre creneaux/.store in = \edtNombreCreneaux,
  arrondi/.store in = \edtArrondi,
  bordure/.store in = \edtBordure
}
% Le premier jeu de clés regroupe les paramètre globaux, celui-là regroupe ceux pour un événement
\pgfkeys{
  /evenement edt/.is family, /evenement edt,
  defaut/.style = {
    titre = Physique-chimie,
    lieu = A104,
    jour = 1,
    creneau = 2,
    duree = 2,
    couleur = green-200
  },% stockage des clefs
  titre/.store in = \edtTitre,
  lieu/.store in = \edtLieu,
  jour/.store in = \edtJour,
  creneau/.store in = \edtCreneau,
  duree/.store in = \edtDuree,
  couleur/.store in = \edtCouleur
}
\newdimen\edtHauteurCreneau
\newdimen\edtLargeurJour
\newdimen\edtX
\newdimen\edtY
\newcommand{\jours}{
  \important{Lundi}, \important{Mardi}, \important{Mercredi}, \important{Jeudi}, \important{Vendredi}
}
\newcommand{\heures}{
  8h30, 9h30, 10h40, 11h40, 12h40, 13h40, 14h40, 15h50, 16h50, 17h50
}

\NewDocumentCommand{\emploiDuTemps}{O{}}{
  % Paramètre par défaut
  \pgfkeys{/emploi du temps, defaut, #1}
  \draw [draw=black, rounded corners = 8pt] (0,0) rectangle (\edtLargeur, -\edtHauteur);
  % Calcul de la taille d'un creneau et de la largeur d'un jour
  \pgfmathsetlength{\edtHauteurCreneau}{(0.96*\edtHauteur - \edtHauteurJours) / \edtNombreCreneaux}
  \pgfmathsetlength{\edtLargeurJour}{(0.96*\edtLargeur - \edtLargeurHeures) / \edtNombreJours}
  % Trace les lignes pour les créneaux horaires
  \pgfmathsetlength{\edtX}{\edtLargeurHeures * 0.5}
  \foreach \h [count=\i] in \heures {
    \pgfmathsetlength{\edtY}{-(\i - 1)*\edtHauteurCreneau - \edtHauteurJours}
    \node at (\edtX, \edtY) {\footnotesize \h};
    \draw [dashed] (\edtLargeurHeures, \edtY) -- (0.96*\edtLargeur, \edtY);
  };
  % Trace les jours
  \pgfmathsetlength{\edtY}{0}
  \foreach \j [count=\i] in \jours {
    \pgfmathsetlength{\edtX}{\edtLargeurHeures + (\i - 1)*\edtLargeurJour}
    \draw [draw = none] 
      (\edtX , \edtY) rectangle (\edtX + \edtLargeurJour, \edtY - \edtHauteurJours) 
      node[pos=.5, text centered] {\j};
  };
}

\NewDocumentCommand{\evenement}{O{}}{
  \pgfkeys{/evenement edt, defaut, #1}
  % calcule la position du créneau
  \pgfmathsetlength{\edtX}{\edtLargeurHeures + (\edtJour - 1)*\edtLargeurJour}
  \pgfmathsetlength{\edtY}{-(\edtCreneau-1)*\edtHauteurCreneau - \edtHauteurJours}
  % dessine le créneau
  \filldraw [
    fill = \edtCouleur,
    draw = \edtCouleur!75!black,
    line width = \edtBordure,
    rounded corners = \edtArrondi
  ]
    (\edtX, \edtY) rectangle (\edtX + \edtLargeurJour, \edtY - \edtDuree*\edtHauteurCreneau)
    node[pos=.5, text width = \edtLargeurJour, align = center, font = \footnotesize] {
      \important{\edtTitre} \linebreak \edtLieu
    };
}