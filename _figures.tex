%%%%%%%%%%%%%%%%%%%%%%%%%%%%%%%%%%%%%%%%%%%%%%%%%%%%%%%%%%%%%
%%%% figures simples
\newcommand{\tkzRect}[4]{
  \fill[color=#1] (#2,#4) -- (-#2,#4) -- (-#2,#3) -- (#2,#3);
}
\newcommand{\tkzEllipse}[4]{
  \fill[color=#1] (0,#3) ellipse (#2 and #4);
}

%%%% Point et vecteurs
% \tikzCercle [couleur fond] (x, y) {rayon} [couleur ligne]
\NewDocumentCommand{\tikzCercle}{O{black} r() m o}{
  \IfValueTF{#4}{
    \filldraw [color = #4, fill = #1, very thick] (#2) circle (#3 pt);
  }{
    \filldraw [#1] (#2) circle (#3 pt);
  }
}
% \tikzPointLabel (x, y) [texte] (xtexte, ytexte), trace un point avec un label; * = pas de point
\NewDocumentCommand{\tikzLabel}{s r() m d()}{
  \IfBooleanTF{#1}{
    \node at (#2) {#3};
  }{
    \IfValueTF{#4}{
      \filldraw (#2) circle (2pt);
      \node at (#4) {#3};
    }{
     \filldraw (#2) circle (2pt) node[above] {#3};
    }
  }
}
% \tikzVecteur [couleur] (x, y) (lx, ly) {legende} [position legende];  * = <->
\NewDocumentCommand{\tikzVecteur}{s O{black} r() r() m O{right}}{
  \coordinate (A) at (#3);
  \coordinate (B) at (#4);
  \coordinate (AB) at ($(A)+(B)$);
  \IfBooleanTF{#1}{
    \draw[#2, <->, very thick] (A) -- (AB) node[#6] {#5};
  }{
    \draw[#2, ->, very thick] (A) -- (AB) node[#6] {#5};
  }
}

% \tikzLegende [couleur] (x, y) (lx, ly) {légende}; * = <-
% ajouter une * passe de la version gauche -> à la version droite <-
\NewDocumentCommand{\tikzLegende}{s O{black} r() r() m}{
  \coordinate (A) at (#3);
  \coordinate (B) at (#4);
  \coordinate (AB) at ($(A)+(B)$);
  \IfBooleanTF{#1}{
    \draw[#2, ->, very thick] (AB) node[right] {#5} -- (A);
  }{
    \draw[#2, ->, very thick] (A) node[left] {#5} -- (AB);
  }
}

%% Trace une petite barre de pourcentage partiellement remplie sur la ligne
\newcommand{\barrePourcentage}[1]{%
  \begin{tikzpicture}
    \fill[color=couleurSec]    (0.0,    0.0) rectangle (#1*8ex, 1.5ex);
    \fill[color=couleurSec!20] (#1*8ex, 0.0) rectangle (8.0ex,  1.5ex);
  \end{tikzpicture}
}


%%%%%%%%%%%%%%%%%%%%%%%%%%%%%%%%%%%%%%%%%%%%%%%%%%%%%%%%%%%%%
%% Trace une flèche de progression pour les plans de travail
% \flecheProgression {<nombre de boucles>} [<largeur>] [<espacement vertical>]
\NewDocumentCommand{\flecheProgression}{m O{17} O{2.4}}{%
  \strut\vspace*{8pt}
  \def\nombreBoucle{\numexpr((#1 - 1)*2)}
  \begin{center}
    \begin{tikzpicture}
      \tikzset{bluestyle/.style={line width = 20pt, rounded corners = 10mm, color = couleurSec}}
      % Premier bout pour l'alignement, les parenthèses sont nécessaires
      \draw[bluestyle] (0, {(\nombreBoucle)*#3}) -- (1, {(\nombreBoucle)*#3}); 
      % Partie centrale répétée
      \draw[bluestyle]
        \foreach \x in {0,2,...,\nombreBoucle}  {
          \ifnum \x < \nombreBoucle
            ( 1, {(\nombreBoucle - \x)  *#3}) -- (#2,  {(\nombreBoucle - \x)  *#3}) --
            (#2, {(\nombreBoucle - \x - 1)*#3}) -- ( 0,  {(\nombreBoucle - \x - 1)*#3}) --
            ( 0, {(\nombreBoucle - \x - 2)*#3}) -- (1.1, {(\nombreBoucle - \x - 2)*#3})
          \fi
        };
      % Flèche finale
      \draw[-{Triangle [width = 36pt, length = 16pt]}, bluestyle] (0.8, 0) -- (#2, 0);
    \end{tikzpicture}
  \end{center}
  %\vspace*{-{#1*118}pt} %% Trouver comment faire le calcul automatiquement...
}


%%%%%%%%%%%%%%%%%%%%%%%%%%%%%%%%%%%%%%%%%%%%%%%%%%%%%%%%%%%%%
%%%% plan de classe
%% Trace un texte centré dans un cadre (x, x+l) -- (y, y+h)
% #1 couleur cadre ; #2 Positionition x ;
% #3 largeur l ;     #4 Positionition y ;
% #5 hauteur h ;     #6 texte.
\NewDocumentCommand{\texteCadre}{O{black} r() O{2} r() O{2} m}{
  \filldraw [fill=white, draw=#1, ultra thick] (#2, #4) rectangle (#2 + #3, #4 + #5);
  \node at (#2 + #3/2, #4 + #5/2) [font=\sffamily] {\textbf{#6}};
}
%% place dans la classe
\NewDocumentCommand{\place}{r() m}{
  \texteCadre(#1)[3](0)[2] {#2}
}

%% Pour tracer une rangée d'élève avec 2 ou 3 colonnes
% \rang {<numero rangee>} {<eleves>} {<eleves>} [<eleves>]
\ExplSyntaxOn
% Position de la place horizontale
\int_new:N \l_rangPositionX_int
\NewDocumentCommand{\rang}{m >{\SplitList{,}} m >{\SplitList{,}} m >{\SplitList{,}} d[]}{
  \begin{tikzpicture}
    % Initialisation de la position horizontale
    \int_set:Nn \l_rangPositionX_int {0}
    % Première rangée
    \ProcessList{#2}{\rangImpl}
    \int_add:Nn \l_rangPositionX_int { 1 }
    % Deuxième rangée
    \ProcessList{#3}{\rangImpl}
    % Troisième rangée
    \IfValueT{#4}{
      \int_add:Nn \l_rangPositionX_int { 1 }
      \ProcessList{#4}{\rangImpl}
    }
  \end{tikzpicture}
  \bigskip
}
\NewDocumentCommand{\rangImpl}{m}{
  \int_add:Nn \l_rangPositionX_int { 3 }
  \place(\l_rangPositionX_int){#1} %
}
\ExplSyntaxOff


%%%%%%%%%%%%%%%%%%%%%%%%%%%%%%%%%%%%%%%%%%%%%%%%%%%%%%%%%%%%%
%%%% tube à essais
\newcommand{\tkzTubeEssais}[3]{
  \draw[thick] (#1,#2) -- (#1,0) arc (0:-180:#1) -- (-#1,#2);
  \draw[thick] (0,#2) ellipse (#1 and #3);
}
\newcommand{\tkzBasTubeEssais}[5]{
  \fill[color=#1] (-#2,#3) -- (#2,#3) arc (0:-180:#2);
  \tkzRect{#1}{#2}{#3 - 0.01}{#4}
  \tkzEllipse{#1!85!black}{#2}{#4}{#5}
}
\newcommand{\tkzPhaseTubeEssais}[5]{
  \tkzRect{#1}{#2}{#3}{#4}
  \tkzEllipse{#1}{#2}{#3}{#5}
  \tkzEllipse{#1!85!black}{#2}{#4}{#5}
}

%%%% tube à essai de sang
\newcommand{\tubeEssaisSolution}[1]{
  \begin{tikzpicture}
    \tkzBasTubeEssais{#1}{0.25}{0}{0.75}{0.1} % contenu du tube
    \tkzTubeEssais{0.25}{1.5}{0.1} % tube
  \end{tikzpicture}
}

\newcommand{\tubeEssaisSangCentrifuge}[3]{
  \begin{tikzpicture}
    % phases dans le tube à essai
    \tkzBasTubeEssais{rougeSombre!75!white} {0.35}{0}{#1}{0.1}
    \tkzPhaseTubeEssais{gray!10!white}      {0.35}{#1}{#2}{0.1}
    \tkzPhaseTubeEssais{jauneClair!75!white}{0.35}{#2}{#3}{0.1}
    \tkzTubeEssais{0.35}{#3 + 1}{0.1}
    % Légende
    \tkzLegende(0.4)(#3 - 0.1) [1]{Plasma}*
    \tkzLegende(0.4)(#2 - 0.08)[1]{Globules blancs}*
    \tkzLegende(0.4)(-0.1)     [1]{Globules rouges}*
  \end{tikzpicture}
}