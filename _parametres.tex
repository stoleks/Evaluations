%%%% Pour avoir les accents et autre caractère français
\usepackage[french]{babel}
\usepackage[T1]{fontenc}
\usepackage[utf8]{inputenc}

%%%% Paquets utilisés
%\usepackage{ghsystem} % pictogramme de sécurité
\usepackage{subcaption} % pour les légendes des figures
\usepackage[european, straightvoltages, RPvoltages]{circuitikz} % pour dessiner des circuits électrique
\usepackage{pdfpages} % pour inclure des fichiers pdf
\usepackage{geometry} % définition des marges
\usepackage[export]{adjustbox} % alignement vertical des images
%% Paquets persos
\usepackage{biomolecules} % pour dessiner des formules chimiques
\usepackage{profSciences} % mise en page et autre


%%%% Commandes prédéfinies
%%%%%%%%%%%%%%%%%%%%%%%%%%%%%%%%%%%%%%%%%%%%%%%%%%%%%%%%%%%%%
%% grandeurs récurrentes
% Physique
\newcommand{\ISS}{\text{ISS}}
\newcommand{\Terre}{\text{Terre}}
\newcommand{\inertie}{\text{inertie}}
\newcommand{\Tfus}{T_\text{f}}
\newcommand{\Teb}{T_\text{éb}}
% Chimie
\newcommand{\solute}{\text{soluté}}
\newcommand{\solution}{\text{solution}}
\newcommand{\espece}{\text{espèce}}
\newcommand{\avogadro}{\num{6,02e23}}
% ions
\newcommand{\ionFerII}      {Fer II      \chemfig{Fe^{2+}}   }
\newcommand{\ionFerIII}     {Fer III     \chemfig{Fe^{3+}}   }
\newcommand{\ionSodium}     {Sodium      \chemfig{Na^{+}}    }
\newcommand{\ionCuivreII}   {Cuivre II   \chemfig{Cu^{2+}}   }
\newcommand{\ionCalcium}    {Calcium     \chemfig{Ca^{2+}}   }
\newcommand{\ionSulfate}    {Sulfate     \chemfig{SO_4^{2-}} }
\newcommand{\ionNitrate}    {Nitrate     \chemfig{NO_3^{-}}  }
\newcommand{\ionChlorure}   {Chlorure    \chemfig{Cl^{-}}    }
\newcommand{\ionFluorure}   {Fluorure    \chemfig{F^{-}}     }
\newcommand{\ionMagnesium}  {Magnésium   \chemfig{Mg^{2+}}   }
\newcommand{\ionPotassium}  {Potassium   \chemfig{K^{+}}     }
\newcommand{\ionBicarbonate}{Bicarbonate \chemfig{CO_3^{2-}} }

%% vecteurs
\newcommand{\FBsurA}{F_{B/A}}
\newcommand{\FAsurB}{F_{A/B}}
\newcommand{\vvFAsurB}{\vv{F}_{A/B}}
\newcommand{\vvFBsurA}{\vv{F}_{B/A}}
%%%% Ce fichier sert à déclarer les titres des chapitres des différents niveaux

%% Commun
\newcommand{\methode} {Outils pratiques}

%% Seconde
%%%% Chapitre
\newcommand{\snd}{Seconde}
\newcommand{\sndCorp} {Corps purs et mélanges}
\newcommand{\sndSolu} {Solutions}
\newcommand{\sndMouv} {Mouvement et interactions}
\newcommand{\sndAtom} {Structure de l'atome}
\newcommand{\sndMole} {Des atomes à la matière}
\newcommand{\sndLumi} {Ondes lumineuses et optique}
\newcommand{\sndTran} {Transformations de la matière}
\newcommand{\sndChim} {Transformations chimiques}
\newcommand{\sndSign} {Signaux et capteurs}

%%%% en-tête correspondant
\newcommand{\teteSndAP}   {\enTete[\snd]{Accompagnement personnalisé}}
\newcommand{\teteSndMeth} {\enTete[\snd]{\chapitre{\methode}}}
\newcommand{\teteSndCorp} {\enTete[\snd]{\chapitre{\sndCorp}}[1]}
\newcommand{\teteSndSolu} {\enTete[\snd]{\chapitre{\sndSolu}}[2]}
\newcommand{\teteSndMouv} {\enTete[\snd]{\chapitre{\sndMouv}}[3]}
\newcommand{\teteSndAtom} {\enTete[\snd]{\chapitre{\sndAtom}}[4]}
\newcommand{\teteSndMole} {\enTete[\snd]{\chapitre{\sndMole}}[6]}
\newcommand{\teteSndLumi} {\enTete[\snd]{\chapitre{\sndLumi}}[3]}
\newcommand{\teteSndTran} {\enTete[\snd]{\chapitre{\sndTran}}[7]}
\newcommand{\teteSndChim} {\enTete[\snd]{\chapitre{\sndChim}}[8]}
\newcommand{\teteSndSign} {\enTete[\snd]{\chapitre{\sndSign}}[9]}


%% Première ST2S
%%%% Chapitres
\newcommand{\premStss}{Première ST2S}
\newcommand{\premStssChim} {Sécurité chimique dans l'habitat}
\newcommand{\premStssVisi} {Propagation de la lumière et vision}
\newcommand{\premStssRedo} {Antiseptique et désinfectant, oxydoréduction}
\newcommand{\premStssLumi} {Les infrarouges et leurs applications}
\newcommand{\premStssStru} {Molécules d'intérêt biologique}
\newcommand{\premStssBiom} {Biomolécules dans l’organisme}
\newcommand{\premStssRout} {Sécurité routière}
\newcommand{\premStssAlim} {Gestion des ressources naturelles et alimentation}
\newcommand{\premStssElec} {Sécurité électrique dans l'habitat}
\newcommand{\premStssPres} {Propriétés des fluides et pression sanguine}
\newcommand{\premStssSono} {Ondes sonores et audition}

%%%% en-tête
\newcommand{\tetePremStssMeth} {\enTete[\premStss]{\chapitre{\methode}}}
\newcommand{\tetePremStssChim} {\enTete[\premStss]{\chapitre{\premStssChim}}[1]}
\newcommand{\tetePremStssVisi} {\enTete[\premStss]{\chapitre{\premStssVisi}}[2]}
\newcommand{\tetePremStssRedo} {\enTete[\premStss]{\chapitre{\premStssRedo}}[3]}
\newcommand{\tetePremStssLumi} {\enTete[\premStss]{\chapitre{\premStssLumi}}[4]}
\newcommand{\tetePremStssStru} {\enTete[\premStss]{\chapitre{\premStssStru}}[5]}
\newcommand{\tetePremStssBiom} {\enTete[\premStss]{\chapitre{\premStssBiom}}[6]}
\newcommand{\tetePremStssRout} {\enTete[\premStss]{\chapitre{\premStssRout}}[7]}
\newcommand{\tetePremStssAlim} {\enTete[\premStss]{\chapitre{\premStssAlim}}[8]}
\newcommand{\tetePremStssElec} {\enTete[\premStss]{\chapitre{\premStssElec}}[9]}
\newcommand{\tetePremStssPres} {\enTete[\premStss]{\chapitre{\premStssPres}}[10]}
\newcommand{\tetePremStssSono} {\enTete[\premStss]{\chapitre{\premStssSono}}[11]}


%% Terminale ST2S
%%%% Chapitres
\newcommand{\termStss}{Terminale ST2S}
\newcommand{\termStssOrga} {Représentation des molécules organiques}
\newcommand{\termStssAlim} {Sécurité physico-chimique dans l'alimentation}
\newcommand{\termStssImag} {La physique de l'imagerie médicale}
\newcommand{\termStssBiom} {Biomolécules et alimentation}
\newcommand{\termStssMedi} {De la molécule aux médicaments}
\newcommand{\termStssEnvi} {Sécurité chimique dans l'environnement}
\newcommand{\termStssDosa} {Analyser la composition d'un milieu}
\newcommand{\termStssRout} {Sécurité routière}
\newcommand{\termStssCosm} {L'usage responsable des cosmétiques}

%%%% en-tête
\newcommand{\teteTermStssMeth} {\enTete[\termStss]{\chapitre{\methode}}}
\newcommand{\teteTermStssOrga} {\enTete[\termStss]{\chapitre{\termStssOrga}}[1]}
\newcommand{\teteTermStssRout} {\enTete[\termStss]{\chapitre{\termStssRout}}[8]}
\newcommand{\teteTermStssAlim} {\enTete[\termStss]{\chapitre{\termStssAlim}}[2]}
\newcommand{\teteTermStssEnvi} {\enTete[\termStss]{\chapitre{\termStssEnvi}}[6]}
\newcommand{\teteTermStssImag} {\enTete[\termStss]{\chapitre{\termStssImag}}[3]}
\newcommand{\teteTermStssDosa} {\enTete[\termStss]{\chapitre{\termStssDosa}}[7]}
\newcommand{\teteTermStssBiom} {\enTete[\termStss]{\chapitre{\termStssBiom}}[4]}
\newcommand{\teteTermStssMedi} {\enTete[\termStss]{\chapitre{\termStssMedi}}[5]}
\newcommand{\teteTermStssCosm} {\enTete[\termStss]{\chapitre{\termStssCosm}}[9]}

%%%% Clés utilisées dans le tableau périodique
\pgfkeys{% définition de la famille de clefs
  /periodique/.is family, /periodique,
  defaut/.style = {
    couleur = green-50,
    nom =,
    electronegativite = 0,
    masse = 0,
    charge = 0,
    echelle = 1
  },
  nom/.store in = \periodiqueNom,
  couleur/.store in = \periodiqueCouleur,
  symbole/.store in = \periodiqueSymbole,
  hauteur case/.store in = \periodiqueHauteur,
  largeur case/.store in = \periodiqueLargeur,
  electronegativite/.store in = \periodiqueElectroneg,
  masse/.store in = \periodiqueMasse,
  charge/.store in = \periodiqueCharge,
  echelle/.store in = \periodiqueEchelle
}

%%%% Pour afficher un tableau périodique
\NewDocumentEnvironment{tableauPeriodique}{O{} +m}{%
  \pgfkeys{/periodique, defaut, #1}
  %
  \begin{tikzpicture}[scale = \periodiqueEchelle, transform shape]
    #2
}{%
  \end{tikzpicture}
}

%%%% Pour afficher une case du tableau périodique
\NewDocumentCommand{\tkzElement}{O{} D(){}}{%
  \pgfkeys{/periodique, defaut, #1}
  %
  \couleurElectronegativite{}
  \node [
    node distance  = \periodiqueHauteur and \periodiqueLargeur,
    on grid,
    minimum width  = \periodiqueLargeur,
    minimum height = \periodiqueHauteur,
    name = \periodiqueSymbole,
    fill = \pgfkeysvalueof{/periodique/couleur},
    draw = cyan-800!50!black,
    align = center,
    #2
  ] {
    % nom de l'élément
    \IfValueT{\periodiqueNom}{
      {\scriptsize \periodiqueNom}%
      \compare {\periodiqueCharge > 0}{ \\[-1pt] }{ \\[2pt] }%
    }
    % nombre atomique
    \compareT {\periodiqueCharge > 0}{\important[black]{\periodiqueCharge}\\[2pt]}%
    % symbole atomique
    {\Large \important[black]{\periodiqueSymbole}}%
    % masse atomique
    \compareT {\periodiqueMasse > 0}{\\[-2pt]%
      {\footnotesize \num{\periodiqueMasse}}%
    }%
    % électronégativité
    \compareT {\periodiqueElectroneg > 0}{%
      \compare{\periodiqueMasse > 0}{\\}{\\[-2pt]}%
      {\footnotesize $\chi = \num{\periodiqueElectroneg}$}%
    }%
  };
}

%%%% Réglage des couleurs automatiques si l'électronégativité est réglé
\newcommand{\couleurElectronegativite}{%
  \compare {\periodiqueElectroneg > 3.5}{%T
    \pgfkeyssetvalue{/periodique/couleur}{red-400}
  }{%F
    \compare {\periodiqueElectroneg > 3.0}{% T
      \pgfkeyssetvalue{/periodique/couleur}{red-300}
    }{% F
      \compare {\periodiqueElectroneg > 2.5}{% T
        \pgfkeyssetvalue{/periodique/couleur}{red-200!95!black}
      }{% F
        \compare {\periodiqueElectroneg > 2.0}{% T
          \pgfkeyssetvalue{/periodique/couleur}{orange-200}
        }{% F
          \compare {\periodiqueElectroneg > 1.5}{% T
            \pgfkeyssetvalue{/periodique/couleur}{yellow-150}
          }{% F
            \compare {\periodiqueElectroneg > 1.0}{% T
              \pgfkeyssetvalue{/periodique/couleur}{green-100}
            }{% F
              \pgfkeyssetvalue{/periodique/couleur}{\periodiqueCouleur}
            }% pas d'électronégativité défaut
          }% 0.5 < chi < 1.0
        }% 1.0 < chi < 1.5
      }% 1.5 < chi < 2.0
    }% 2.0 < chi < 2.5
  }% 2.5 < chi < 3.0
}% 3.0 < chi < 3.5

%%%% Pour faciliter l'utilisation du tableau périodique
\NewDocumentCommand{\elementH}  {O{} D(){}} {\tkzElement[symbole = H,  charge = 1,  nom = Hydrogène, #1](#2)} % masse = 1.00, 
\NewDocumentCommand{\elementHe} {O{} D(){}} {\tkzElement[symbole = He, charge = 2,  nom = Hélium, #1](#2)}    % masse = 4.00, 
\NewDocumentCommand{\elementLi} {O{} D(){}} {\tkzElement[symbole = Li, charge = 3,  nom = Lithium, #1](#2)}   % masse = 6.94, 
\NewDocumentCommand{\elementBe} {O{} D(){}} {\tkzElement[symbole = Be, charge = 4,  nom = Béryllium, #1](#2)} % masse = 9.01, 
\NewDocumentCommand{\elementB}  {O{} D(){}} {\tkzElement[symbole = B,  charge = 5,  nom = Bore, #1](#2)}      % masse = 10.8, 
\NewDocumentCommand{\elementC}  {O{} D(){}} {\tkzElement[symbole = C,  charge = 6,  nom = Carbone, #1](#2)}   % masse = 12.0, 
\NewDocumentCommand{\elementN}  {O{} D(){}} {\tkzElement[symbole = N,  charge = 7,  nom = Azote, #1](#2)}     % masse = 14.0, 
\NewDocumentCommand{\elementO}  {O{} D(){}} {\tkzElement[symbole = O,  charge = 8,  nom = Oxygène, #1](#2)}   % masse = 16.0, 
\NewDocumentCommand{\elementF}  {O{} D(){}} {\tkzElement[symbole = F,  charge = 9,  nom = Fluor, #1](#2)}     % masse = 19.0, 
\NewDocumentCommand{\elementNe} {O{} D(){}} {\tkzElement[symbole = Ne, charge = 10, nom = Néon, #1](#2)}      % masse = 20.2, 
\NewDocumentCommand{\elementNa} {O{} D(){}} {\tkzElement[symbole = Na, charge = 11, nom = Sodium, #1](#2)}    % masse = 23.0, 
\NewDocumentCommand{\elementMg} {O{} D(){}} {\tkzElement[symbole = Mg, charge = 12, nom = Magnésium, #1](#2)} % masse = 24.3, 
\NewDocumentCommand{\elementAl} {O{} D(){}} {\tkzElement[symbole = Al, charge = 13, nom = Aluminium, #1](#2)} % masse = 27.0, 
\NewDocumentCommand{\elementSi} {O{} D(){}} {\tkzElement[symbole = Si, charge = 14, nom = Silicium, #1](#2)}  % masse = 28.1, 
\NewDocumentCommand{\elementP}  {O{} D(){}} {\tkzElement[symbole = P,  charge = 15, nom = Phosphore, #1](#2)} % masse = 31.0, 
\NewDocumentCommand{\elementS}  {O{} D(){}} {\tkzElement[symbole = S,  charge = 16, nom = Soufre, #1](#2)}    % masse = 32.1, 
\NewDocumentCommand{\elementCl} {O{} D(){}} {\tkzElement[symbole = Cl, charge = 17, nom = Chlore, #1](#2)}    % masse = 35.5, 
\NewDocumentCommand{\elementAr} {O{} D(){}} {\tkzElement[symbole = Ar, charge = 18, nom = Argon, #1](#2)}     % masse = 39.9, 
\NewDocumentCommand{\elementK}  {O{} D(){}} {\tkzElement[symbole = K,  charge = 19, nom = Potassium, #1](#2)} % masse = 39.1, 
\NewDocumentCommand{\elementCa} {O{} D(){}} {\tkzElement[symbole = Ca, charge = 20, nom = Calcium, #1](#2)}   % masse = 40.0, 


%%%% Réglages de la taille des indentations et des sauts de paragraphes
\setlength{\parskip}{0cm}
\setlength{\parindent}{0cm}
\renewcommand{\baselinestretch}{1}
% réglage du niveau (sous-section) ou s'arrête la table des matières
\setcounter{tocdepth}{2}


%%%% Réglage de la géométrie des pages
\geometry{
  a4paper, % format
  left=1.3cm, right=1.3cm, % marge horizontale
  top=2.2cm, bottom=2.1cm % marge verticale
}

%%%% Réglage des en-tête
\renewcommand{\headrulewidth}{0.4pt}
\setlength{\headheight}{22.50113pt}
