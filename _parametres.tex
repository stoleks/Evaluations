%%%% Pour avoir les accents et autre caractère français
\usepackage[french]{babel}
\usepackage[T1]{fontenc}
\usepackage[utf8]{inputenc}

%%%% Paquets utilisé
\usepackage{ifthen} % pour programmer avec des boucle et des conditions
%% Images/dessin
\usepackage{subcaption} % pour les légendes des figures
\usepackage{graphicx} % pour insérer des images
\usepackage[european, straightvoltages, RPvoltages]{circuitikz} % pour dessiner des circuits électrique
\usepackage{pdfpages} % pour inclure des fichiers pdf
\usepackage{wrapfig} % pour entourer les images par du texte 
\usepackage{chemfig} % pour dessiner des formules chimiques
\usepackage{fontawesome} % pour dessiner de jolies icônes
%% Mise en page
\usepackage{geometry} % définition des marges
\usepackage{dashundergaps} % pour avoir générer des textes à compléter
\usepackage{fancyhdr} % pour faire des en-tête
\usepackage[many]{tcolorbox} % pour faire de jolie boîtes colorée
\usepackage{enumitem} % pour pouvoir définir des listes personnalisées
\usepackage{hyperref} % pour insérer des liens
\usepackage{multicol} % pour avoir plusieurs colonnes côte-à-côte
\usepackage{listings} % pour insérer du code
\usepackage{marginnote} % pour insérer des notes sur le côté
%% Tableau
\usepackage{tabularray} % pour avoir de meilleurs tableaux
%% QR code
\usepackage{qrcode}
%% Math
\usepackage{amsmath} % symboles mathématiques
\usepackage{amssymb} % symboles mathématiques en gras
\usepackage{wasysym} % pour avoir des symbole d'intégrale
\usepackage{accents} % pour les notations mathématiques avec une barre
\usepackage{physics} % pour les dérivées, les bra, les kets, etc.
\usepackage{esvect} % pour faire de jolis vecteurs
\usepackage{siunitx} % pour avoir de jolie grandeurs avec des unités 


%%%% Commandes prédéfinies
%%%%%%%%%%%%%%%%%%%%%%%%%%%%%%%%%%%%%%%%%%%%%%%%%%%%%%%%%%%%%%%%%%%%%%%%%%
%% rectangle coloré
\NewDocumentCommand{\rectangle}{O{couleurPrim} m m}{%
  \shorthandoff{;}
  \tikz \node (rect) [draw, fill, color=#1,
              minimum width=#2,
              minimum height=#3] {};
  \shorthandon{;}
}

%%%%%%%%%%%%%%%%%%%%%%%%%%%%%%%%%%%%%%%%%%%%%%%%%%%%%%%%%%%%%%%%%%%%%%%%%%%
%%
\tcbset{
  boite cassable/.style = {
    breakable, enhanced jigsaw, % pour s'étendre sur plusieurs pages
  },
  %
  couleur fond/.style = {
    colback = #1, % fond blanc
    colbacktitle = #1, % fond pour le titre blanc
  },
  %
  titre sans separation/.style = {
    couleur fond = white,
    coltitle = black, % couleur du titre
    colframe = couleurSec-800, % couleur de la boite
    boxrule = #1, arc = 0.5mm, % largeur et arrondi des traits de la boite
    titlerule = 0mm, top = 0mm, % pour ne pas avoir de séparation titre/boite
    fonttitle = \bfseries\sffamily, % titre en gras et sans serif
  },
  %
  boite pleine/.style = {
    frame hidden, sharp corners, boxrule = 0mm, % pas de contours
    colback = #1, % fond
  },
  %
  titre sans boite/.style = {
    empty, % pas de boite automatique
    fonttitle = \bfseries\sffamily, coltitle = black, % paramètre du titre
    attach boxed title to top left = {yshift=-2.5mm}, % position
    boxed title style = {empty, size = small, top = 1mm, bottom = 0.5mm},
    title = #1,
  },
}

%% une simple boite vide
\newtcolorbox{boite}[1][]{
  boite cassable, colback = white, top = 4pt,
  #1
}

%%%% Boite colorée avec 
% \begin{boiteColoree}{couleur} contenu \end{boiteColoree}
\newtcolorbox{boiteColoree}[2][]{
  boite cassable, boite pleine = #2,
  #1
}

%%%% Boite pour les documents des activités, avec le format suivant
% \begin{doc}{titre}{label} contenu \end{doc}
\newcounter{documentNum}
\newtcolorbox{doc}[3][]{
  boite cassable,
  titre sans separation = 0.5mm,
  before title = {\refstepcounter{documentNum}},
  title = {Document \arabic{documentNum} -- #2\strut \label{#3}},
  #1
}

%%%% Boîtes pour les activités et TP pour une séquence en plan de travail
% Boite pour afficher la durée recommandée en bas à gauche
\newtcbox{\dureeActivite}[1][]{
  arc = 2mm, % courbes des bords
  colback = couleurTer, colframe = white, % couleurs boite
  coltext = white, % couleur texte
}
% Boite de base pour les activités ou TP, avec le format suivant 
% \begin{boiteActivite}{titre activité}{durée}{label}{compteur}{format titre}
%   contenu
% \end{boiteActivite}
\newtcolorbox{boiteActivite}[6][]{
  boite cassable,
  titre sans separation = 0.5mm,
  before title = {\refstepcounter{#5}},
  title = {#6 \arabic{section}.\arabic{#5} -- #2\strut \label{#4}},
  enlarge bottom by = 12pt,
  overlay= {
    \node at ([xshift = -36pt, yshift = -6pt] frame.south east) {
      \dureeActivite{
        {\small \important[white]{#3}}
      }
    };
  }, % affiche la durée de l'activité dans une petite boite
  remember as = #4,
  #1
}

% Boite pour les activités/TP en plan de travail.
% \begin{activite ou TP}{titre activité}[durée]{label} ; durée = 1h acti, 2h TP par défaut
%   contenu
% \end{activite ou TP}
\NewDocumentEnvironment{activite}{m O{1 h} m}{%
  \begin{boiteActivite}{#1}{#2}{#3}
    {activiteNum}{\documentaire* \hspace{-4pt} Activité}
}{
  \end{boiteActivite}
}
% Compteur pour les TP
\newcounter{TPNum} 
\NewDocumentEnvironment{TP}{m O{2 h} m}{%
  \begin{boiteActivite}{#1}{#2}{#3}
    {TPNum}{\mesure* \hspace{-4pt} TP}
}{
  \end{boiteActivite}
}
% Pour réferencer une activité
\newcommand{\reference}[1]{\arabic{section}.\ref{#1}}

% Boîte pour la tâche finale
\newtcolorbox{tacheFinale}[1][]{
  boite cassable,
  titre sans separation = 0.5mm,
  title = {Tâche finale}, % titre
  #1
}

% Boîte pour l'organisation des séances
\newcounter{seanceNum}
\newtcbox{\seance}[2][]{
  boite cassable,
  titre sans separation = 0.5mm,
  before title = {\refstepcounter{seanceNum}},
  title = {Séance \arabic{seanceNum} \hfill (#2)}, % titre,
  halign=center, valign=center, % pour centrer le contenu
  height = 0.13\textheight, % hauteur de la boite
  #1
}
% Pour afficher 3/2/1 séances dans la programmation
\NewDocumentEnvironment{programmeSeance}{O{3} D(){34}}{
  \begin{tcbraster}[
    raster columns = #1, 
    raster width center = (\linewidth - 3cm - 5cm*(3 - #1))
  ]
}{
  \end{tcbraster}
  \vspace*{#2 pt}
}


%%%% Passage important à connaître
\newtcolorbox{importants}[1][]{
  boite cassable,
  boite pleine = couleurPrim-50, % fond
  shadow = {-4pt}{0mm}{0mm}{couleurPrim}, % barre gauche
  #1
}

%%%% Boîte pour le contexte
\newtcolorbox{contexte}[1][]{
  boite cassable, 
  titre sans separation = 3pt,
  couleur fond = couleurPrim-50!50, 
  colframe = couleurPrim, sharp corners, %  contours
  title = {Contexte :}, % titre
  detach title, before upper={\vspace{2pt}\hspace{-8pt}\tcbtitle\;}, % titre "en ligne"
  #1
}

%%%% Pour les objectifs
\newenvironment{objectifs}{ %
  \begin{listeObjectifs}
} {\end{listeObjectifs}}
%
\tcolorboxenvironment{objectifs}{
  titre sans boite = {Objectifs :},
  frame code = { % tracé de la boite
    \path [draw=couleurPrim, line width = 3pt]
    (frame.west) |- ([xshift=6mm] title.north east)
    to[out=0, in=180] ([xshift=20mm] title.east) -| % définit la courbe
    (frame.east) |- (frame.south) -| cycle; % trace la boite
  },
}

%% Pour les pré-requis
\newenvironment{prerequis}{ %
  \begin{listeObjectifs}
} {\end{listeObjectifs}}
%
\tcolorboxenvironment{prerequis}{
  titre sans boite = {Prérequis :},
  frame code = { % tracé de la boite
    \path [draw=couleurPrim, line width = 3pt]
    (frame.south east) |- ([xshift=-34mm, yshift=-4mm] frame.south east)
    to[out=180, in=0] ([xshift=-47mm] frame.south east) -| % définit la courbe
    (frame.west) |- (title.north) -| cycle; % trace la boite
  },
}

%%%% Espace pour un coup de pouce
\newcounter{coupDePouceNum}
\newtcolorbox{coupDePouce}[1][]{
  boite cassable, 
  titre sans separation = 0.5mm,
  before title = {\refstepcounter{coupDePouceNum}},
  title = {
    \textcolor{couleurPrim}{\faThumbsUp}
    Coup de pouce \arabic{coupDePouceNum} :
    \flushright \vspace*{-24pt}\faSquareO
  },
  #1
}

%%%% Espace pour une appréciation
\newcommand{\appreciation}[1]{
  \pasCorrection{
    \begin{boite}
      \vspace*{-4pt}
      \important[black]{Appréciation et remarques}
      \vspace*{#1 cm} \phantom{b}
    \end{boite}
  }
}

%%%% Espace fiche TP
\newtcolorbox{boiteMateriel}[2][]{
  titre sans separation = 0.5mm,
  coltitle = white, colbacktitle = couleurPrim, % fond pour le titre blanc
  title = {\centering \large \phantom{À} #2 \phantom{g}}, % titre
  #1
}

%%%%%%%%%%%%%%%%%%%%%%%%%%%%%%%%%%%%%%%%%%%%%%%%%%%%%%%%%%%%%%%%%%%%%%%%%%
%%%% pagination et sections
\NewDocumentCommand{\titre}{O{black} m}{
  \begin{center}
    \textcolor{#1} {\textsf{\bfseries \Large #2}}
  \end{center}
}
\newcommand{\pasDePagination}{
  \thispagestyle{empty}
}
\newcommand{\feuilleBlanche}{
  \newpage
  \phantom{b}
  \pasDePagination
}
\NewDocumentCommand{\inclusActivite}{O{1} m}{
  \numeroActivite{#1}
  \input{#2}
}
    

%%%% activité ou TP
\newcounter{activiteNum}
\newcommand{\titreActi}[2]{
  \refstepcounter{activiteNum}
  \titre{#1 \arabic{section}.\arabic{activiteNum} -- #2}
}
\NewDocumentCommand{\titreTP}{s m}{
  \IfBooleanTF{#1}{ % Version *
    \titreActi{Activité expérimentale}{#2}
  }{
    \titreActi{TP}{#2}
  }
}
\NewDocumentCommand{\titreActivite}{s m O{Activité}}{
  \IfBooleanTF{#1}{ % Version * sans le numéro du chapitre
    \refstepcounter{activiteNum}
    \titre{#3 \arabic{activiteNum} -- #2}
  }{
    \titreActi{#3}{#2}
  }
}
\NewDocumentCommand{\titreEvaluation}{o m}{
  \IfNoValueTF{#1}{
    \titre{Évaluation \arabic{section} -- #2}
  }{
    \titre{Évaluation #1 -- #2}
  }
  % reinitialisation du numéro de page et d'exercices
  \reinitialiseCompteur
}
\newcounter{exerciceNum}
\newcommand{\exercice}[1]{
  \refstepcounter{exerciceNum}
  \important[black]{\large Exercice \arabic{exerciceNum} : #1}
  % reset des numéros de questions
  \setcounter{questionNum}{0}
  \setcounter{documentNum}{0}
}


%%%% chapitre, partie, section et sous-section
\newcommand{\chapitre}[1]{ % Réglage du titre du chapitre
  Chapitre \arabic{section} -- #1
}
\newcommand{\titreChapitre}[1]{ % Affichage du titre du chapitre
  \titre{Chapitre \arabic{section} -- #1}
}
\newcommand{\titrePartie}[1]{
  \vspace*{24pt}
  \refstepcounter{subsection}
  \rectangle{40pt}{1pt}
  \important[black]{\Large \Roman{subsection} -- #1}
  \rectangle{40pt}{1pt}
  \vspace*{10pt}
}
\newcounter{sousSectionNum}
\newcommand{\titreSection}[1]{
  \vspace*{16pt}
  \refstepcounter{subsubsection}
  \setcounter{sousSectionNum}{0}
  \rectangle{30pt}{4pt}
  \important[black]{\large \arabic{subsubsection} -- #1}
  \vspace*{4pt}
}
\newcommand{\titreSousSection}[1]{
  \vspace*{12pt}
  \refstepcounter{sousSectionNum}
  \important[black]{\Alph{sousSectionNum} -- #1}
  \vspace*{4pt}
}

%%%% fixe le numéro de l'activité
\newcommand{\reinitialiseCompteur}{
    % fixe les compteurs LaTeX
  \setcounter{subsection}{0}
  \setcounter{subsubsection}{0}
  \setcounter{figure}{0}
  % fixe les compteurs internes
  \setcounter{documentNum}{0}
  \setcounter{questionNum}{0}
  \setcounter{coupDePouceNum}{0}
  \setcounter{sousSectionNum}{0}
  \setcounter{seanceNum}{0}
}
\newcommand{\numeroActivite}[1]{
  \reinitialiseCompteur
  \setcounter{activiteNum}{#1 - 1}
}
% fixe le numéro de partie (#1) et le numéro de la page (#2)
\newcommand{\numeroPartieCours}[2]{
  \newpage
  \setcounter{subsection}{#1 - 1}
  \setcounter{page}{#2}
}

%%%% lignes
\newcommand{\ligne}{
  \par\noindent\rule{\textwidth}{0.4pt}
}
\NewDocumentCommand{\lignePointillee}{o}{
  \IfValueTF{#1}{
    \makebox[#1\linewidth]{\dotfill}
  }{
    \phantom{b}\hspace*{-12pt} \dotfill
  }
}


%%%%%%%%%%%%%%%%%%%%%%%%%%%%%%%%%%%%%%%%%%%%%%%%%%%%%%%%%%%%%%%%%%%%%%%%%%
%%%% Paramètre par défaut pour l'en-tête
\newcommand{\classe}{Réglez avec \textbackslash renewcommand\{\textbackslash classe\}\{Seconde\}}
\newcommand{\etablissement}{Réglez avec \textbackslash renewcommand\{\textbackslash etablissement\}\{Lycée\}}

%%%% en-tête
\newcommand{\teteGauche}[2]{
  \lhead{
    \textbf{\footnotesize #1}
    \newline
    \footnotesize #2
  }
}
\NewDocumentCommand{\teteDroite}{m o}{
  \rhead{
    \IfValueT{#2}{
      \hfill \textbf{\footnotesize #2}
    }
    \newline 
    \hfill \footnotesize #1
  }
}
%% \enTete [compteur] {titre} [numéro de section] ; * = version simplifié sans pagination
\NewDocumentCommand{\enTete}{s o m O{0}}{
  % reset des compteurs
  \newpage
  \reinitialiseCompteur
  \setcounter{section}{#4}
  
  \IfBooleanTF{#1}{ % affichage de l'entête version fiche réussite (*)
    \pasDePagination
    \phantom{b} \vspace*{-70pt}
    \titre{#3}
  }{ % affichage de l'entête version activité
    \pagestyle{fancy}
    \teteGauche{\etablissement{}}{#3} % left header
    \chead{} % central header
    \teteDroite{#2} % right header
  }
}


%%%%%%%%%%%%%%%%%%%%%%%%%%%%%%%%%%%%%%%%%%%%%%%%%%%%%%%%%%%%%%%%%%%%%%%%%%
%%%% exercice
% définit un booléen pour entrer ou sortir du mode correction
\newboolean{modeProf}
\setboolean{modeProf}{false}
\newcommand{\modeCorrection}{
  \setboolean{modeProf}{true}
  \TeacherModeOn
}

%% Affiche le numéro d'une question avec choix du compteur et de l'espacement
\NewDocumentCommand{\numeroQuestion}{s O{questionNum} O{16}}{
  \refstepcounter{#2}
  \setcounter{sousQuestionNum}{0}
  \vspace*{2pt}
  \IfBooleanTF{#1}{}{ % cas non étoilé
    \ifnum \thequestionNum > 9
      \hspace{6 pt}
    \else
      \hspace{#3 pt}
    \fi
  }
  \textcolor{couleurSec}{
    \textbf{\arabic{#2}} {\small\faMinus}
  }
}
%% Sous question sour la forme 1.2
\NewDocumentCommand{\numeroSousQuestion}{O{16}}{
  \refstepcounter{sousQuestionNum}
  \ifnum \thesousQuestionNum > 9
    \hspace{6 pt}
  \else
    \hspace{#1 pt}
  \fi
  \textcolor{couleurSec}{
    \textbf{\arabic{questionNum}.\arabic{sousQuestionNum}.}
  }
}


%% trace des lignes pointillées pour répondre aux questions
% \lignesDeReponse* complète la ligne actuelle par des pointillées
% \lignesDeReponse commence à la ligne suivante
\newcounter{ligneNum}
\NewDocumentCommand{\lignesDeReponse}{s m}{
  % Trace la fin de la ligne, ou pas
  \IfBooleanTF{#1}{ % Version *
    \espaceReponse \dotfill\phantom{bb}
    \ifnum #2 < 1
      \newline
    \fi
  }{}
  % Trace le bon nombre de lignes
  \setcounter{ligneNum}{-1}
  \loop
    \stepcounter{ligneNum}
    \ifnum \value{ligneNum} < #2
      \\[8pt] \lignePointillee
  \repeat
  \vspace*{1pt}
}


%% définit une commande pour afficher une question 
% \qestion {question} {réponse} [nombre de lignes] ; * = pas d'alinéa
\newcounter{questionNum}
\newcounter{sousQuestionNum}
\NewDocumentCommand{\question}{s +m +m O{0}}{
  \IfBooleanTF{#1}{ % étoile
    \numeroQuestion* \!#2
  }{ % 
    \numeroQuestion \!#2
  }
  % pointille ou correction
  \ifthenelse {\boolean{modeProf}} { % prof
    \begin{boiteColoree}{couleurPrim-50}
      #3
    \end{boiteColoree}
  }{ % eleve
    \lignesDeReponse{#4}
  }
}

% Affiche le contenu en mode correction
\newcommand{\correction}[1]{
  \ifthenelse {\boolean{modeProf}} { % correction
    #1
  }{}
}

% Affiche le contenu si on est pas en mode correction
\newcommand{\pasCorrection}[1]{
  \ifthenelse{\boolean{modeProf}} {}{ % pas correction
    #1
  }
}

% Point associé à une question
\newcommand{\points}[1]{
  \marginnote{#1}
}


% sous questions
\newcommand{\sousQuestion}[2]{
  \hspace{16pt}
  \textcolor{couleurSec}{\textbullet} #1
  
  \vspace*{8pt}
  \reponse{#2}
}

%%%% qcm
\newlist{QCM}{itemize}{2}
\setlist[QCM]{label = $\square$, leftmargin = 2cm}
%% #1 : question, 
%% #2 : réponses
\NewDocumentEnvironment{qcm}{+m +m}{
  \numeroQuestion #1
  \begin{QCM}
    #2
}{
  \end{QCM}
}

% À ajouter devant la bonne réponse dans un qcm
\newcommand{\reponseQCM}{
  \correction{
    \hspace*{-21pt}
    {\large\textcolor{couleurSec}{\faCheck}}
    \hspace*{-12pt}
  } % Note : trace une croix à la bonne position
}

%%%% Pour afficher les compétences
\newcommand{\competence}[1]{
  ~{\footnotesize\textit{(#1)}}
}

%%%% Espace pour indiquer nom, prénom et classe
\newcommand{\nomPrenomClasse}{
  \pasCorrection{
    \vspace*{-24pt}
    Nom : \lignePointillee[0.3]
    Prénom : \lignePointillee[0.3]
    Classe : \dotfill
  }
}
\newcommand{\nomPrenom}{
  \pasCorrection{
    \vspace*{-24pt}
    Nom : \lignePointillee[0.3]
    Prénom : \lignePointillee[0.3]
  }
}


%%%%%%%%%%%%%%%%%%%%%%%%%%%%%%%%%%%%%%%%%%%%%%%%%%%%%%%%%%%%%%%%%%%%%%%%%%
% texte à trou avec option pour régler la largeur
\NewDocumentCommand{\texteTrou}{s o +m}{
  \ifthenelse {\boolean{modeProf}}{ % prof
    \IfBooleanTF{#1}
    {#3}
    {\important[black]{#3}}
  }{ % élève
    \IfValueTF{#2}{ % Si la largeur est réglée, on utilise des lignes
      \espaceReponse
      \lignePointillee[#2]
      \hspace*{-12pt}
    }{ % Sinon on utilise dash undergap pour la version automatique
      \espaceReponse \hspace*{0.1pt}
      \gap{#3}
    }
  }
}

% texte à trou avec option pour laisser plusieurs lignes
\NewDocumentCommand{\texteTrouLignes}{O{0} +m}{
  \ifthenelse {\boolean{modeProf}} {% prof
    \important[black]{#2}
  }{% élève
    \lignesDeReponse*{#1}
  }
}

% espace vertical pour la réponse
\newcommand{\espaceReponse}{
  \phantom{$\dfrac{1}{1}$} % espace vertical
  \hspace*{-38pt} \phantom{b} % ajuste l'espace horizontal
}


%%%%%%%%%%%%%%%%%%%%%%%%%%%%%%%%%%%%%%%%%%%%%%%%%%%%%%%%%%%%%%%%%%%%%%%%%%
%%%% Pour choisir parmi deux sujets
\newboolean{sujetA}
\setboolean{sujetA}{true}
\newcommand{\sujetB}{
  \setboolean{sujetA}{false}
}
\newcommand{\sujetA}{
  \setboolean{sujetA}{true}
}

%%%% Pour faire plusieurs sujets en parallèle
\newcommand{\variationSujet}[2]{
  \hspace*{-6pt}
  \ifthenelse{\boolean{sujetA}}{#1}{#2}
  \hspace*{-6pt}
}


%%%%%%%%%%%%%%%%%%%%%%%%%%%%%%%%%%%%%%%%%%%%%%%%%%%%%%%%%%%%%%%%%%%%%%%%%%
%%%% Tableau générique avec la première ligne bleue
\NewDocumentEnvironment{tableau}{m}{
  \begin{center}
    \begin{tblr}{
      hlines,
      colspec = #1,
      row{1} = {couleurSec-100},
    }
}{
    \end{tblr}
  \end{center}
}

%%%% Tableau de competence
\newenvironment{tableauCompetences}{
  \begin{center}
    \begin{tblr}{
      colspec = {c X[l] c c c c},
      rows = {m}, hlines, vlines,
      row{1} = {c, couleurSec-100, font = \bfseries}
    }
      Comp. & Items & D & C & B & A \\
}{
    \end{tblr}
  \end{center}
}

%%%% Tableau de connaissances pour les fiches de révisions
\newenvironment{tableauConnaissances}{
  \begin{center}
  \begin{tblr}{
    colspec = {Q[t, wd=0.7\textwidth] c c c},
    rows = {m}, hlines, vlines,
    column{4} = {0.2},
    row{1} = {couleurSec-100, c}
  }
    \important{Connaissances et capacités exigibles} & \ok & \pasOk & \important{En classe} \\
}{ 
  \end{tblr}
  \end{center}
}

%%%% Tableau de mémorisation pour les fiches de mémorisation
% Question | | | Réponse
\newenvironment{tableauMemorisation}{
  \begin{center}
 \begin{tblr}{
    colspec = {
      Q[l, wd=0.3\textwidth] % Question
      X[c] X[c] % Auto-évaluation
      Q[l, wd=0.3\textwidth] % Réponse
      Q[c, wd=0.045\textwidth] Q[c, wd=0.045\textwidth] Q[c, wd=0.045\textwidth] % Répétitions
    }, rows = {m}, hlines, vlines,
    row{1,2} = {couleurSec-100, c}
  }
    \SetCell[r=2]{c} \important{Questions} & 
    \SetCell[c=2]{c} \important{Auto-évaluation} & &
    \SetCell[r=2]{c} \important{Réponses} & 
    \SetCell[r=2]{c} \important{J-7} &
    \SetCell[r=2]{c} \important{J-30} &
    \SetCell[r=2]{c} \important{J-180} \\
    %
    & % questions
    \ok & \pasOk & % auto evaluation
    & % réponses 
    & & \\ % répétitions
    %
}{ 
  \end{tblr}
  \end{center}
}


%%%% Alignement vertical dans un tableau
\newcommand{\vAligne}[1]{
  \strut \\ \vspace*{#1}
}


%%%%%%%%%%%%%%%%%%%%%%%%%%%%%%%%%%%%%%%%%%%%%%%%%%%%%%%%%%%%%%%%%%%%%%%%%%
%%%% symboles : chevron, flèche, attention, etc.
\NewDocumentCommand{\chevron}{O{couleurPrim}}{
  \textcolor{#1}{\small \faChevronRight}
}
%
\NewDocumentCommand{\fleche}{O{couleurPrim}}{
  \textcolor{#1}{\faCaretRight}
}
%
\NewDocumentCommand{\attention}{O{couleurPrim}}{
  \textcolor{#1}{\faExclamationTriangle}
}
%
\NewDocumentCommand{\flecheLongue}{O{couleurPrim}}{
  \textcolor{#1}{\faLongArrowRight}
}
%
\NewDocumentCommand{\ok}{O{couleurPrim}}{
  \textcolor{#1}{\faCheckCircle}
}
%
\NewDocumentCommand{\pasOk}{O{couleurPrim}}{
  \textcolor{#1}{\faTimesCircle}
}
%
\NewDocumentCommand{\pointCyan}{O{couleurPrim}}{
  \textcolor{#1}{\textbullet}
}

%% Commande générale pour colorer un texte avec une couleur par défaut
% L'intérêt de cette commande est d'éviter des répétion de O{couleurSec}
\NewDocumentCommand{\questionSpeciale}{O{couleurSec} m}{
  \textcolor{#1}{#2}
}
% Quelques questions spéciales avec couleur réglable et un espace horizontal
\NewDocumentCommand{\mesure}{s o}{
  \IfBooleanTF{#1}{}{ \hspace{7pt} }
  \questionSpeciale[#2]{\faFlask\hspace{1pt} \faWrench\!}
}
%
\NewDocumentCommand{\telechargement}{s o}{
  \IfBooleanTF{#1}{}{ \hspace{7pt} }
  \questionSpeciale[#2]{\faDownload\, \faMobile\;}
}
%
\NewDocumentCommand{\documentaire}{s o}{
  \IfBooleanTF{#1}{}{ \hspace{7pt} }
  \questionSpeciale[#2]{\faFileText\hspace{1pt} \faBook}
}

%% pictogramme de sécurité \picto {largeur} {pictogramme}
\newcommand{\picto}[2]{
  \image{#1}{images/pictogrammes/picto_#2}
}
% Nombre dans un cercle A REVOIR
\newcommand*\nombreCercle[1]{
  % \tikz[baseline=(char.base)]{
  %   \node [shape=circle, draw filled, inner sep=1.2pt, color=couleurSec-50] (char) {\textcolor{black}{#1};
  % }
  \important[couleurSec]{#1}
}
% Pour légender une image
\newcommand{\legende}[1]{
  \vspace*{4pt}
  \textcolor{couleurPrim}{\faArrowUp} \; #1
}

%%%%%%%%%%%%%%%%%%%%%%%%%%%%%%%%%%%%%%%%%%%%%%%%%%%%%%%%%%%%%%%%%%%%%%%%%%
%%%% points importants
\NewDocumentCommand{\important}{O{couleurSec-800} +m}{
  \!\textcolor{#1}{\textsf{\bfseries #2}}\!\!
}
%%%% Pour donner des exemples. 
% \exemple est au singulier, \exemple* est au pluriel.
\NewDocumentCommand{\exemple}{s}{
  \fleche
  \IfBooleanTF{#1}{
    \textit{Exemples :}
  }{
    \textit{Exemple :}
  }
}

%%%% Citation (#1) avec la source (#2)
\newcommand{\extrait}[2]{
  « #1 »
  
  \vspace*{-12pt}
  \begin{flushright}
    \textit{#2}
  \end{flushright}
  \vspace*{-12pt}
}

%%%% image avec la largeur réglée par rapport à celle de la ligne
\newcommand{\image}[2]{
  \includegraphics[width=#1\linewidth]{#2}
}

%%%% qr code en insert sur la droite
\NewDocumentCommand{\qrcodeCote}{o +m D(){-16pt} O{1.5cm}}{
  \IfNoValueTF{#1} {
    \begin{wrapfigure}{r}{0.1\linewidth}
      \vspace*{#3}
      \qrcode[height = #4]{#2}
    \end{wrapfigure}
  }{
    \begin{wrapfigure}[#1]{r}{0.1\linewidth}
      \vspace*{#3}
      \qrcode[height = #4]{#2}
    \end{wrapfigure}
  }
}


%%%%%%%%%%%%%%%%%%%%%%%%%%%%%%%%%%%%%%%%%%%%%%%%%%%%%%%%%%%%%%%%%%%%%%%%%%
%%%% protocole, avec plusieurs colonnes possibles
\NewDocumentEnvironment{protocole}{o +m}{
  \IfValueTF{#1}{
    \vspace*{-8pt}
    \begin{multicols}{#1}
  }{}
  \begin{itemize}[label = {\footnotesize \fleche[couleurSec]}]
    #2
}{
  \end{itemize}
  \IfValueTF{#1}{ \end{multicols} }{}
}

%%%% liste de points, avec plusieurs colonnes possibles
\NewDocumentEnvironment{listePoints}{o +m}{
  \IfValueTF{#1}{
    \vspace*{-8pt}
    \begin{multicols}{#1}
  }{}
  \begin{itemize}[label = \pointCyan]
    #2
}{
  \end{itemize}
  \IfValueTF{#1}{ \end{multicols} }{}
}

%%%% jeu de données, avec plusieurs colonnes possibles
\NewDocumentEnvironment{donnees}{o +m}{  
  \important{Données :}
  \IfValueTF{#1}{
    \vspace*{-8pt}
    \begin{multicols}{#1}
  }{}
  \begin{listeTirets}
    #2
}{
  \end{listeTirets}
  \IfValueTF{#1}{ \end{multicols} }{}
}

%%%% liste d'objectif
\newlist{listeObjectifs}{itemize}{2}
\setlist[listeObjectifs]{leftmargin=6pt, label=\fleche}

%%%% liste tirets
\newlist{listeTirets}{itemize}{2}
\setlist[listeTirets]{label = \textcolor{couleurPrim}{\small\faMinus}}

%%%% liste avec des flèches
\newlist{listeFleche}{itemize}{2}
\setlist[listeFleche]{label = \textbf{\flecheLongue}}

%%%% liste avec chiffre
\newlist{enumeration}{enumerate}{2}
\setlist[enumeration]{label = \textcolor{couleurSec}{\textbf{\arabic*.}} }

%%%% problématique 
\newcommand{\problematique}[1]{
  \hspace*{-12pt}
  \flecheLongue[couleurTer]
  \hspace*{-10pt}
  \important[couleurTer]{#1}
}


%%%%%%%%%%%%%%%%%%%%%%%%%%%%%%%%%%%%%%%%%%%%%%%%%%%%%%%%%%%%%%%%%%%%%%%%%%
%%%% Séparation de la page en trois blocs
\NewDocumentCommand{\separationTroisBlocs}{+m O{0.3} +m O{0.3} +m O{0.3}}{
  \begin{minipage}[T]{#2\linewidth}
    #1
  \end{minipage}
  ~
  \begin{minipage}[T]{#4\linewidth}
    #3
  \end{minipage}
  ~
  \begin{minipage}[T]{#6\linewidth}
    #5
  \end{minipage}
}
%%%% Separation en deux blocs
\NewDocumentCommand{\separationBlocs}{+m O{0.48} +m O{0.48}}{
  \begin{minipage}[T]{#2\linewidth}
    #1
  \end{minipage}
  \hfill
  \begin{minipage}[T]{#4\linewidth}
    #3
  \end{minipage}
}


%%%%%%%%%%%%%%%%%%%%%%%%%%%%%%%%%%%%%%%%%%%%%%%%%%%%%%%%%%%%%%%%%%%%%%%%%%
%% nombre algébrique et réaction chimique
\newcommand{\algebrique}[1]{
  \ensuremath{\overline{\mathrm{#1}}}
}
\newcommand{\reaction}{
  \;\text{\faLongArrowRight}\; % flèche courte et jolie
}

%% Pour simplifier l'écriture des formules brutes
% #1 : # carbone, #2 : # hydrogène, #3 : oxygène
\newcommand{\bruteCHO}[3]{
  \chemfig{C_{#1} H_{#2} O_{#3}}
}

%% pour les masse molaire et atomique (* -> en indice)
\NewDocumentCommand{\masseMol}{s m}{
  \IfBooleanTF{#1}{
    \ensuremath{M_{\chemfig{#2}}}
  }{
    \ensuremath{M(\chemfig{#2})}
  }
}
\NewDocumentCommand{\masseAtom}{s m}{
  \IfBooleanTF{#1}{
    \ensuremath{m_{\chemfig{#2}}}
  }{
    \ensuremath{m(\chemfig{#2})}
  }
}


%% Unités pour siunit
\DeclareSIUnit{\dioptre}{\text{$\delta$}}
\DeclareSIUnit{\dornic}{\text{\textdegree D}}
\DeclareSIUnit{\ppm}{\text{ppm}}
\DeclareSIUnit{\COeq}{\text{kgCO$_{2}$e}}
\DeclareSIUnit{\jour}{\text{jour}}
% \DeclareSIUnit{}{\text{}}


%%%% atome ou isotope 
%#1: Z, #2: A, #3: X
\makeatletter
\newcommand{\isotope}[3]{%
   \settowidth\@tempdimb{\ensuremath{\scriptstyle#1}}%
   \settowidth\@tempdimc{\ensuremath{\scriptstyle#2}}%
   \ifnum\@tempdimb>\@tempdimc%
       \setlength{\@tempdima}{\@tempdimb}%
   \else%
       \setlength{\@tempdima}{\@tempdimc}%
   \fi%
  \begingroup%
  \ensuremath{
    ^{\makebox[\@tempdima][r]{\ensuremath{\scriptstyle#1}}}
    _{\makebox[\@tempdima][r]{\ensuremath{\scriptstyle#2}}}
    \chemfig{#3}
  }%
  \endgroup%
}%
\makeatother

%% element chimique dans le tableau périodique
\makeatletter
\newcommand{\element}[2]{%
   \settowidth\@tempdimb{\ensuremath{\footnotesize #1}}%
  \begingroup%
  \ensuremath{
    _{\makebox[\@tempdimb][r]{\ensuremath{\small #1}}} 
    \chemfig[atom style={scale=1.3}]{#2}
  }%
  \endgroup%
}%
\makeatother

%% siècle
\newcommand{\siecle}[1]{
  \textsc{\romannumeral #1}\textsuperscript{e}~siècle%
}

%% texte avec une boite autour
\NewDocumentCommand{\texteEncadre}{m O{black}}{
  \textcolor{#2}{
    \frame{
      \vphantom{$\dfrac{1}{1}$} \textcolor{black}{\text{#1}}
    }
  }
}

%% case cochée
\newcommand{\caseCochee}{
  $\text{\rlap{$\checkmark$}}\square$
}


%%%%%%%%%%%%%%%%%%%%%%%%%%%%%%%%%%%%%%%%%%%%%%%%%%%%%%%%%%%%%%%%%%%%%%%%%%
%%%% Style python
\lstdefinestyle{codePython}{
  commentstyle=\color{magenta-500},
  keywordstyle=\color{green-500},
  stringstyle =\color{purple-500},
  numberstyle =\tiny\color{black!50},
  basicstyle  =\ttfamily\footnotesize,
  breakatwhitespace=false, breaklines=true, keepspaces=true,
  showspaces=false, showstringspaces=false, showtabs=false, tabsize=2,
  captionpos=b, numbers=left, numbersep=5pt,
  extendedchars=true,
  literate={é}{{\'e}}1 {è}{{\`e}}1 {à}{{\'a}}1 {°}{{\textdegree}}1 {²}{{$^2$}}1, 
}
\def\inline{\lstinline[style=codePython,language=python]}


%%%%%%%%%%%%%%%%%%%%%%%%%%%%%%%%%%%%%%%%%%%%%%%%%%%%%%%%%%%%%%%%%%%%%%%%%%
%%%% circuit tikz
\NewDocumentCommand{\fixedvlen}{O{0.5cm} m m O{}}{% [semilength]{node}{label}[extra options]
  % get the center of the standard arrow
  \coordinate (#2-Vcenter) at ($(#2-Vfrom)!0.5!(#2-Vto)$);
  % draw an arrow of a fixed size around that center and on the same line
  \draw[-Triangle, #4] ($(#2-Vcenter)!#1!(#2-Vfrom)$) -- ($(#2-Vcenter)!#1!(#2-Vto)$);
  % position the label as in the normal voltages
  \node[anchor=\ctikzgetanchor{#2}{Vlab}, #4] at (#2-Vlab) {#3};
}
%%%%%%%%%%%%%%%%%%%%%%%%%%%%%%%%%%%%%%%%%%%%%%%%%%%%%%%%%%%%%
%% grandeurs récurrentes
% Physique
\newcommand{\gISS}{g_\text{ISS}}
\newcommand{\MTerre}{M_\text{Terre}}
\newcommand{\RTerre}{R_\text{Terre}}
\newcommand{\inertie}{\text{inertie}}
\newcommand{\Tfus}{T_\text{f}}
\newcommand{\Teb}{T_\text{éb}}
% Chimie
\newcommand{\solute}{\text{soluté}}
\newcommand{\solution}{\text{solution}}
\newcommand{\espece}{\text{espèce}}
% ions
\newcommand{\ionFerII}      {Fer II      \chemfig{Fe^{2+}}   }
\newcommand{\ionFerIII}     {Fer III     \chemfig{Fe^{3+}}   }
\newcommand{\ionSodium}     {Sodium      \chemfig{Na^{+}}    }
\newcommand{\ionCuivreII}   {Cuivre II   \chemfig{Cu^{2+}}   }
\newcommand{\ionCalcium}    {Calcium     \chemfig{Ca^{2+}}   }
\newcommand{\ionSulfate}    {Sulfate     \chemfig{SO_4^{2-}} }
\newcommand{\ionNitrate}    {Nitrate     \chemfig{NO_3^{-}}  }
\newcommand{\ionChlorure}   {Chlorure    \chemfig{Cl^{-}}    }
\newcommand{\ionFluorure}   {Fluorure    \chemfig{F^{-}}     }
\newcommand{\ionMagnesium}  {Magnésium   \chemfig{Mg^{2+}}   }
\newcommand{\ionPotassium}  {Potassium   \chemfig{K^{+}}     }
\newcommand{\ionBicarbonate}{Bicarbonate \chemfig{CO_3^{2-}} }

%% vecteurs
\newcommand{\FBsurA}{F_{B/A}}
\newcommand{\FAsurB}{F_{A/B}}
\newcommand{\vvFAsurB}{\vv{F}_{A/B}}
\newcommand{\vvFBsurA}{\vv{F}_{B/A}}

%%%%%%%%%%%%%%%%%%%%%%%%%%%%%%%%%%%%%%%%%%%%%%%%%%%%%%%%%%%%%
%%%% figures simples
\newcommand{\tkzRect}[4]{
  \fill[color=#1] (#2,#4) -- (-#2,#4) -- (-#2,#3) -- (#2,#3);
}
\newcommand{\tkzEllipse}[4]{
  \fill[color=#1] (0,#3) ellipse (#2 and #4);
}

\newcommand{\tkzCercle}[4]{
  \filldraw [#3] (#1, #2) circle (#4pt);
}
\newcommand{\tkzCercleLigne}[5]{
  \filldraw [color = #4, fill = #3, very thick] (#1, #2) circle (#5pt);
}

%%%% tube à essais
\newcommand{\tkzTubeEssai}[3]{
  \draw[thick] (#1,#2) -- (#1,0) arc (0:-180:#1) -- (-#1,#2);
  \draw[thick] (0,#2) ellipse (#1 and #3);
}
\newcommand{\tkzBasTubeEssai}[5]{
  \fill[color=#1] (-#2,#3) -- (#2,#3) arc (0:-180:#2);
  \tkzRect{#1}{#2}{#3 - 0.01}{#4}
  \tkzEllipse{#1!85!black}{#2}{#4}{#5}
}
\newcommand{\tkzPhaseTubeEssai}[5]{
  \tkzRect{#1}{#2}{#3}{#4}
  \tkzEllipse{#1}{#2}{#3}{#5}
  \tkzEllipse{#1!85!black}{#2}{#4}{#5}
}

%%%% Point et vecteurs
\newcommand{\tkzLabel}[3]{
  \node at (#1, #2) {#3};
}
\newcommand{\tkzPointLabel}[3]{
  \filldraw (#1, #2) circle (2pt) node[above] {#3};
}
% \tkzVecteur [couleur] (x) [longueur x] (y) [longueur y] {legende} [position legende] 
% ajouter une * à la fin transforme la flèche en double flèche <->
\NewDocumentCommand{\tkzVecteur}{O{black} r() O{0} r() O{0} m O{right} s}{
  \IfBooleanTF{#8}{
    \draw[#1, <->, very thick] (#2, #4) -- (#2 + #3, #4 + #5) node[#7] {#6};
  }{
    \draw[#1, ->, very thick] (#2, #4) -- (#2 + #3, #4 + #5) node[#7] {#6};
  }
}
% \tkzLegende (x) (y) [longueur fleche] {légende} 
% ajouter une * passe de la version gauche -> à la version droite <-
\NewDocumentCommand{\tkzLegende}{O{black} r() r() O{1.25} m s}{
  \IfBooleanTF{#6}{
    \draw[#1, ->, very thick] (#2 + #4, #3) node[right] {#5} -- (#2, #3);
  }{
    \draw[#1, ->, very thick] (#2, #3) node[left] {#5} -- (#2 + #4, #3);
  }
}


%%%%%%%%%%%%%%%%%%%%%%%%%%%%%%%%%%%%%%%%%%%%%%%%%%%%%%%%%%%%%
%%%% plan de classe
\NewDocumentCommand{\texteCadre}{O{black} r() O{2} r() O{2} m}{
  \filldraw [fill=white, draw=#1, ultra thick] (#2, #4) rectangle (#2 + #3, #4 + #5);
  \node at (#2 + #3/2, #4 + #5/2) [font=\sffamily] {\textbf{#6}};
}

%% place dans la classe
\NewDocumentCommand{\place}{r() r() m}{
  \texteCadre(#1)[3](#2)[2] {#3}
}
\NewDocumentCommand{\places}{r()r() r[] d[] d[] d[]}{
  \place(#1)(#2) {#3}
  \IfValueT{#4}{ \place(#1 + 1*3)(#2) {#4} }
  \IfValueT{#5}{ \place(#1 + 2*3)(#2) {#5} }
  \IfValueT{#6}{ \place(#1 + 3*3)(#2) {#6} }
}

%% rangée de classe ou de TP
\NewDocumentCommand{\rangee}{m r[]r[] r()r()r()d() r[]r[]}{
  \places(0)(0 - 3*#1) [#2][#3]
  \IfValueTF{#7}{
    \places(7) (0 - 3*#1) [#4][#5][#6][#7]
    \places(20)(0 - 3*#1) [#8][#9]
  }{
    \places(8.5)(0 - 3*#1) [#4][#5][#6]
    \places(20) (0 - 3*#1) [#8][#9]
  }
}

\NewDocumentCommand{\rangeeTP}{m r[]r[]r[] r()r()r()d()}{
  \places(3)(0 - 3*#1) [#2][#3][#4]
  \IfValueTF{#8}{
    \places(14) (0 - 3*#1) [#5][#6][#7][#8]
  }{
    \places(14) (0 - 3*#1) [#5][#6][#7]
  }
}
%%%% Ce fichier sert à déclarer les titres des chapitres des différents niveaux

%% Seconde
%%%% Chapitre 
\newcommand{\sndMeth} {Outils pratiques}
\newcommand{\sndCorp} {Corps purs et mélanges}
\newcommand{\sndSolu} {Solutions}
\newcommand{\sndMouv} {Mouvement et interactions}
\newcommand{\sndAtom} {Structure de l'atome}
\newcommand{\sndMole} {Des atomes à la matière}
\newcommand{\sndLumi} {Ondes lumineuses et optique}
\newcommand{\sndTran} {Transformations de la matière et nucléaires}
\newcommand{\sndChim} {Transformations chimiques}
\newcommand{\sndSign} {Signaux et capteurs}

%%%% en-tête correspondant
\newcommand{\teteSndMeth} {\newpage \enTete{\sndMeth}{Seconde}{0} }
\newcommand{\teteSndCorp} {\newpage \enTete{\sndCorp}{Seconde}{1} }
\newcommand{\teteSndSolu} {\newpage \enTete{\sndSolu}{Seconde}{2} }
\newcommand{\teteSndMouv} {\newpage \enTete{\sndMouv}{Seconde}{3} }
\newcommand{\teteSndAtom} {\newpage \enTete{\sndAtom}{Seconde}{4} }
\newcommand{\teteSndMole} {\newpage \enTete{\sndMole}{Seconde}{5} }
\newcommand{\teteSndLumi} {\newpage \enTete{\sndLumi}{Seconde}{6} }
\newcommand{\teteSndTran} {\newpage \enTete{\sndTran}{Seconde}{7} }
\newcommand{\teteSndChim} {\newpage \enTete{\sndReac}{Seconde}{8} }
\newcommand{\teteSndSign} {\newpage \enTete{\sndSign}{Seconde}{9} }


%% Première ST2S
%%%% Chapitres
\newcommand{\premStssChim} {Sécurité chimique dans l'habitat}
\newcommand{\premStssElec} {Sécurité électrique dans l'habitat}
\newcommand{\premStssRout} {Sécurité routière}
\newcommand{\premStssSono} {Ondes sonores et audition}
\newcommand{\premStssVisi} {Propagation de la lumière et vision}
\newcommand{\premStssPres} {Propriétés des fluides et pression sanguine}
\newcommand{\premStssStru} {Structure des molécules d'intérêt biologique}
\newcommand{\premStssComp} {Contrôle de la composition des milieux biologiques}
\newcommand{\premStssEner} {Besoins énergétiques et alimentation}
\newcommand{\premStssBiom} {Biomolécules dans l’organisme}
\newcommand{\premStssAlim} {Gestion des ressources naturelles et alimentation}

%%%% en-tête
\newcommand{\tetePremStssChim} {\newpage \enTete{\premStssChim}{Première ST2S}{1} }
\newcommand{\tetePremStssElec} {\newpage \enTete{\premStssElec}{Première ST2S}{2} }
\newcommand{\tetePremStssRout} {\newpage \enTete{\premStssRout}{Première ST2S}{3} }
\newcommand{\tetePremStssSono} {\newpage \enTete{\premStssSono}{Première ST2S}{4} }
\newcommand{\tetePremStssVisi} {\newpage \enTete{\premStssVisi}{Première ST2S}{5} }
\newcommand{\tetePremStssPres} {\newpage \enTete{\premStssPres}{Première ST2S}{6} }
\newcommand{\tetePremStssStru} {\newpage \enTete{\premStssStru}{Première ST2S}{7} }
\newcommand{\tetePremStssComp} {\newpage \enTete{\premStssComp}{Première ST2S}{8} }
\newcommand{\tetePremStssEner} {\newpage \enTete{\premStssEner}{Première ST2S}{9} }
\newcommand{\tetePremStssBiom} {\newpage \enTete{\premStssBiom}{Première ST2S}{10} }
\newcommand{\tetePremStssAlim} {\newpage \enTete{\premStssAlim}{Première ST2S}{11} }


%% Terminale ST2S
%%%% Chapitres
\newcommand{\termStssOrga} {Représentation des molécules organiques}
\newcommand{\termStssRout} {Sécurité routière}
\newcommand{\termStssAlim} {Sécurité physico-chimique dans l'alimentation}
\newcommand{\termStssEnvi} {Sécurité chimique dans l'environnement}
\newcommand{\termStssImag} {La physique au service de l'imagerie médicale}
\newcommand{\termStssDosa} {Contrôle de la composition des milieux naturels}
\newcommand{\termStssBiom} {Biomolécules et alimentation}
\newcommand{\termStssMedi} {De la molécule au médicament}
\newcommand{\termStssCosm} {L'usage responsable des cosmétiques}

%%%% en-tête
\newcommand{\teteTermStssOrga} {\newpage \enTete{\termStssOrga}{Terminale ST2S}{0} }
\newcommand{\teteTermStssRout} {\newpage \enTete{\termStssRout}{Terminale ST2S}{1} }
\newcommand{\teteTermStssAlim} {\newpage \enTete{\termStssAlim}{Terminale ST2S}{2} }
\newcommand{\teteTermStssEnvi} {\newpage \enTete{\termStssEnvi}{Terminale ST2S}{3} }
\newcommand{\teteTermStssImag} {\newpage \enTete{\termStssImag}{Terminale ST2S}{4} }
\newcommand{\teteTermStssDosa} {\newpage \enTete{\termStssDosa}{Terminale ST2S}{5} }
\newcommand{\teteTermStssBiom} {\newpage \enTete{\termStssBiom}{Terminale ST2S}{6} }
\newcommand{\teteTermStssMedi} {\newpage \enTete{\termStssMedi}{Terminale ST2S}{7} }
\newcommand{\teteTermStssCosm} {\newpage \enTete{\termStssCosm}{Terminale ST2S}{8} }

%%%%%%%% Interro formative sur molécules organiques
% \nomPrenom

\begin{center}
  \chemfig{!\acideAscorbique} \\[8pt]
  \important{Acide ascorbique,} aussi appelé vitamine C.
  Il faut en renouveler les réserves fréquemment en mangeant des fruits et légumes frais.
\end{center}

\question{
  Entourer et nommer les familles organiques présentes dans cette molécule.
}{}[4]

\question{
  Donner la formule brute de cette molécule.
}{}[1]

\question{
  Indiquer, en justifiant, si cette molécule est hydrosoluble ou liposoluble.
}{}[4]
% \nomPrenom

\begin{center}
  \chemfig{!\alanineSemiDev} \\[8pt]
  \important{Alanine}, un acide aminé
\end{center}

\question{
  Donner le nom de la formule utilisée pour représenter la molécule d'alanine.
}{
  C'est la formule semi-developpée.
}[1]

\question{
  Donner la formule brute de l'alanine.
}{
  \chemfig{C_3 H_7 O_2 N}
}[1]

\question{
  Donner la formule développée de l'alanine.
}{
  \chemfig{
    H-O- C (=[3] O) -C (-[3] H) (-[-3] N (-[-5]H) (-[-1] H)) -C (-[3] H) (-[-3] H) -H
  }
}[6]
  
\question{
  Entourer deux groupe fonctionnels dans la molécule d'alanine et les nommer.
}{
  Carboxyle et amine.
}[4]
% \nomPrenom

\begin{center}
  \chemfig{!\asparagineSemiDev} \\[8pt]
  \important{Asparagine}, molécule qui est un des 22 acides aminés protéinogène.
\end{center}

\question{
  Donner le nom de la représentation de la molécule d'asparagine.
}{
  C'est la formule semi-développée.
}[1]

\numeroQuestion
Entourer les 3 groupes fonctionnels de l'asparagine.

\question{
  Donner le nom des 3 groupes et familles fonctionnelles dans l'asparagine.
}{
  Acide carboxylique (carboxyle), amine (amine), amide (amide).
}[3]

\question{
  Donner la formule brute de la molécule d'asparagine
}{
  \chemfig{C_4 H_8 O_2 N_2}
}[1]


\numeroQuestion
Écrire la formule topologique de l'asparagine.
% \nomPrenom

\begin{center}
  \chemfig{!\aspartame} \\[8pt]
  \important{Aspartame}, molécule au goût sucré, potentiellement cancérigène, mais utilisé massivement par l'industrie des sodas.
\end{center}

\question{
  Donner le nom de la représentation de la molécule d'aspartame
}{
  C'est la formule topologique.
}[1]

\question{
  Donner la formule brute de l'aspartame.
}{
  \chemfig{C_{14} H_{18} N_{2} O_{5}}
}[1]

\numeroQuestion
Entourer les quatre groupes fonctionnels dans la molécule d'aspartame. 

\question{
  Donner le nom des groupes fonctionnels et les noms des familles organiques associées.
}{
  Ester (ester), amide (amide), amine (amine) et carboxyle (acide carboxylique)
}[4]
% \nomPrenom

\begin{center}
  \chemfig{!\aspirineSemiDev} \\[16pt]
  \important{Acide acétylsalicylique,} molécule composant l'aspirine.
\end{center}

\vspace*{-12pt}
\question{
  Donner le nom de la représentation de la molécule d'acide acétylsalicylique.
}{
  C'est la formule semi-développée.
}[1]

\numeroQuestion
Entourer les 2 groupes fonctionnels de l'acide acétylsalicylique.

\question{
  Donner le nom des 2 groupes et familles fonctionnelles dans l'acide acétylsalicylique.
}{
  Acide carboxylique (carboxyle), ester (ester).
}[2]

\question{
  Donner la formule brute de la molécule d'acide acétylsalicylique
}{
  \chemfig{C_9 H_8 O_4}
}[1]


\numeroQuestion
Écrire la formule topologique de l'acide acétylsalicylique.
% \teteTermStssOrga
\nomPrenom

\begin{center}
  \chemfig{!\caproineSemiDev} \\[8pt]
  \important{Caproïne}, triester de glycérol présent dans l'huile de palme.
\end{center}

\question{
  Donner le nom de la représentation de la molécule de caproïne.
}{
  C'est la formule semi-développée.
}[1]

\question{
  Donner la formule brute de la caproïne.
}{
  \chemfig{C_{14} H_{18} N_{2} O_{5}}
}[1]

\numeroQuestion
Entourer les trois groupes fonctionnels de la caproïne.

\question{
  Donner le nom des trois groupes fonctionnels et des familles organiques associées.
}{
  Ester (ester).
}[3]

\question{
  La caproïne est-elle saturée ou insaturée ? Justifier.
}{
  Elle est saturée, elle ne comporte que des liaisons simples
}[3]

% \nomPrenom

\begin{center}
  \chemfig{!\retinol} \\[8pt]
  \important{Rétinol,} aussi appelé vitamine A.
\end{center}

\question{
  Entourer et nommer les familles organiques présentes dans cette molécule.
}{}[4]

\question{
  Indiquer, en justifiant, si cette molécule est hydrosoluble ou liposoluble.
}{}[4]

% \nomPrenom

\begin{center}
  \chemfig{!\acideNicotinique} \\[8pt]
  \important{Acide nicotinique,} aussi appelé vitamine B3.
\end{center}

\question{
  Entourer et nommer les familles organiques présentes dans cette molécule.
}{}[4]

\question{
  Donner la formule brute de cette molécule.
}{}[1]

\question{
  Calculer la masse molaire de cette molécule.

  \important{Données :} \masseMol{H} = \qty{1,0}{\g\per\mole},
  \masseMol{C} = \qty{12}{\g\per\mole},
  \masseMol{N} = \qty{14}{\g\per\mole},
  \masseMol{O} = \qty{16}{\g\per\mole},
}{}[2]

% \nomPrenom

\begin{center}
  \chemfig{!\paracetamolDev} \\[8pt]
  \important{Paracétamol,} un analgésique qui est le principe actif du doliprane.
\end{center}

\question{
  Entourer et nommer les familles organiques présentes dans cette molécule.
}{}[3]

\question{
  Donner la formule brute de cette molécule.
}{}[1]

\question{
  Calculer la masse molaire de cette molécule.

  \important{Données :} \masseMol{H} = \qty{1,0}{\g\per\mole},
  \masseMol{C} = \qty{12}{\g\per\mole},
  \masseMol{N} = \qty{14}{\g\per\mole},
  \masseMol{O} = \qty{16}{\g\per\mole},
}{}[2]

\question{
  Calculer la quantité de matière de paracétamol dans un cachet de \qty{0,5}{\g}.
}{}[2]

% \input{molecules/arginine}
% \input{molecules/histidine}
% \input{molecules/fructose}
% \nomPrenom

\begin{center}
  \chemfig{ !\cholesterol } \\[8pt]
  \important{Cholestérol}, lipide de la famille des stérols, qui jouent un rôle important dans notre corps.
\end{center}

\question{
  Donner le nom de la représentation de la molécule.
}{
  C'est la formule topologique.
}[1]

\question{
  Donner la formule brute du cholestérol.
}{
  \bruteCHO{27}{42}
}[1]

\question{
  Entourer et nommer le(s) groupe(s) fonctionnel(s) du cholestérol.
}{
  Alcool et alcène.
}[3]

%\vfill 
%Si on devait utiliser la nomenclature pour nommer la molécule de cholestérol, son nom serait

% \centering
% (3S, 8S, 9S, 10R, 13R, 14S, 17R)-10,13-diméthyl-17-[(2R)-6-méthylheptan-2-yl]-2,3,4,7,8,9,11,12,14,15,16,17-dodécahydro-1H-cyclopenta[a]phénanthren-3-ol
% \teteTermStssOrga
\nomPrenom

\begin{center}
  \chemfigHaworth{!\cocaineHaw} \\[8pt]
  \important{Cocaïne,} une molécule avec des effets psychoactifs, particulièrement addictive et mauvaise pour la santé.
\end{center}

\question{
  Entourer et nommer les groupes caractéristiques présents dans cette molécule.
}{}[6]
% \teteTermStssOrga
\nomPrenom

\begin{center}
  \chemfig{!\cortisol} \\[8pt]
  \important{Cortisol,} une hormone de la famille des stérols, responsable en partie du stress.
\end{center}

\question{
  Entourer et nommer les familles organiques présentes dans cette molécule.
}{}[5]

\question{
  Indiquer en justifiant, si cette molécule est hydrosoluble ou liposoluble.
}{}[4]
% \teteTermStssOrga
\nomPrenom

\begin{center}
  \chemfig{!\creatinine} \\[8pt]
  \important{Créatinine,} une molécule qui est un déchet de certaines réactions biologiques dans le corps humain.
\end{center}

\question{
  Entourer et nommer les groupes fonctionnels de cette molécule.
}{}[4]

\question{
  Donner la formule semi-développée de cette molécule.
}{}[5]

\question{
  Donner la formule brute de cette molécule.
}{}[1]
% \teteTermStssOrga
\nomPrenom

\begin{center}
  \chemfig{!\estradiol} \\[8pt]
  \important{Estradiol,} une hormone de la famille des stérols. C'est une des hormone sexuelle primaire.
\end{center}

\question{
  Entourer et nommer les familles organiques présentes dans la testostérone.
}{}[3]

\question{
  Indiquer en justifiant, si cette molécule est hydrosoluble ou liposoluble.
}{}[4]
% \nomPrenom

\begin{center}
  \chemfig{!\geraniolSemiDev} \\[8pt]
  \important{Géraniol}, molécule liée à l'odeur de rose.
\end{center}

\question{
  Donner le nom de la représentation de la molécule de géraniol.
}{
  C'est la formule semi-développée.
}[1]

\numeroQuestion
Entourer le groupes fonctionnels du géraniol.

\question{
  Donner le nom du groupe fonctionnel et de la famille présente dans le géraniol.
}{
  Hydroxyle, alcool.
}[2]

\numeroQuestion
Écrire la formule développée du géraniol.
% \nomPrenom

\begin{center}
  \chemfigHaworth{!\glucoseHaw}
  %\chemfig{!\glucoseCycle}
\end{center}

\question{
  Donner la formule brute de cette molécule.
}{
  \bruteCHO{6}{12}{6}
}[1]

\question{
  Donner le nom de cette molécule.
}{
  Le glucose
}[1]

\question{
  Entourer les groupes caractéristiques de la molécule et les nommer.
}{
  Alcool et ether.
}[4]
% \teteTermStssOrga
\nomPrenom

\begin{center}
  \chemfig{!\oxyphenylone} \\[8pt]
  \important{Oxyphenylone,} la molécule responsable de l'arôme des framboises.
\end{center}

\question{
  Donner le nom de la représentation de cette molécule.
}{}[1]

\question{
  Donner la formule brute de cette molécule.
}{}[1]

\question{
  Entourer et nommer les groupes fonctionnels présents dans cette molécule.
}{}[3]
% \teteTermStssOrga
\nomPrenom

\begin{center}
  {
    \small
    \chemfig[atom sep = 1.8em]{!\phosphatidylcholine}
  } \\[8pt]
  \important{Phosphatidylcholine,} une molécule de la famille des phosphoglycérides, qui composent les bicouches lipidique qui servent de membranes au cellules animales.
\end{center}

\question{
  Entourer et nommer les groupes fonctionnels présents dans cette molécule.
}{}[4]

\numeroQuestion
Entourer le glycérol estérifié présent dans cette molécule.

\question{
  Combien d'acides gras possède cette molécule ?
}{}[1]
% \nomPrenom

\begin{center}
  \chemfig{!\progesterone} \\[8pt]
  \important{Progestérone,} une molécule qui est liée au cycle menstruel et à la grossesse.
\end{center}

\question{
  Entourer et nommer les fonctions organiques présentes dans cette molécule.
}{}[3]

\question{
  Indiquer si cette molécule est liposoluble ou hydrosoluble. Justifier.
}{}[4]
% \nomPrenom

\begin{center}
  \chemfig{!\prostaglandine} \\[8pt]
  \important{Prostaglandine,} une famille de molécules qui sont des agents de signalisation de la douleur, notamment.
\end{center}

\question{
  Entourer et nommer les fonctions organiques présentes dans cette molécules.
}{}[4]

\question{
  Indiquer, en justifiant, si cette molécule est hydrosoluble.
}{}[4]
% \teteTermStssOrga
\nomPrenom

\begin{center}
  \chemfig{!\testosterone} \\[8pt]
  \important{Testostérone,} une hormone de la famille des stérols. C'est une des hormones sexuelle primaire.
\end{center}

\question{
  Entourer et nommer les familles organiques présentes dans la testostérone.
}{}[3]

\question{
  Indiquer en justifiant, si cette molécule est hydrosoluble ou liposoluble.
}{}[4]
% \teteTermStssOrga
\nomPrenom

\begin{center}
  \chemfig{!\THC} \\[8pt]
  \important{THC} ou \important{TétraHydroCannabinol}, molécule qui est le psychoactif majeur du cannabis.
\end{center}

\question{
  Donner le nombre de carbone et d'oxygène contenu dans la molécule.
}{
  \bruteCHO{21}{30}{2}
}[2]

\question{
  Entourer les groupes caractéristiques de la molécule et les nommer.
}{
  Alcool et éther.
}[3]

\question{
  Indiquer en justifiant si cette molécule est hydrosoluble ou liposoluble.
}{
  Elle est liposoluble, car elle ne contient qu'une liaison polaire.
}[4]

%%%% Pour afficher un élément dans le tableau périodique
\NewDocumentCommand{\elementText}{m m m o}
{
  \begin{minipage}{2.2cm}
    \begin{center}
      \IfValueTF{#4}{ \vAligne{-20pt} }{ \vAligne{-34pt} } % position du nom
      {\small #3} \\[2pt] % nom de l'élément
      {\ensuremath\footnotesize \textbf{#1}} \\[6pt] % nombre atomique
      \chemfig[atom style={scale = 1.8}]{#2} % symbole atomique
      % \element{#1}{#2} % element symbol and atomic number
      \IfValueT{#4}{
        \\ {\small \qty{#4}{\g/\mole}}
      }
    \end{center}
  \end{minipage}
}


%%%% Pour afficher un tableau périodique
\newcommand{\tableauPeriodique}[1]{
\begin{tikzpicture}[font=\sffamily, scale=0.75, transform shape]
%% Couleur de remplissage
  \tikzstyle{elementFill}     = [fill=yellow!30]
  \tikzstyle{alkaliFill}      = [fill=red!45]
  \tikzstyle{alkaliEarthFill} = [fill=red!30]
  \tikzstyle{metalFill}       = [fill=red!15]
  \tikzstyle{metalloidFill}   = [fill=yellow!15]
  \tikzstyle{nonmetalFill}    = [fill=orange!15]
  \tikzstyle{halogenFill}     = [fill=orange!30]
  \tikzstyle{nobleGasFill}    = [fill=orange!45]

%% Style des éléments
  \tikzstyle{Element} = [
    draw=black, elementFill,
    minimum width=2.85cm, % Largeur de la case
    minimum height=2.7cm, % Hauteur de la case
    node distance=2.85cm % Espace entre deux case
  ]
  \tikzstyle{Alkali}      = [Element, alkaliFill]
  \tikzstyle{AlkaliEarth} = [Element, alkaliEarthFill]
  \tikzstyle{Metal}       = [Element, metalFill]
  \tikzstyle{Metalloid}   = [Element, metalloidFill]
  \tikzstyle{Nonmetal}    = [Element, elementFill]
  \tikzstyle{Halogen}     = [Element, halogenFill]
  \tikzstyle{NobleGas}    = [Element, nobleGasFill]
  \tikzstyle{PeriodLabel} = [font={\sffamily\LARGE}, node distance=2cm]
  \tikzstyle{GroupLabel}  = [font={\sffamily\LARGE}, minimum width=2.5cm, node distance=2cm]
  \tikzstyle{TitleLabel}  = [font={\sffamily\Huge\bfseries}]

%% Place des éléments
  #1
\end{tikzpicture}
}


%%%% Pour faciliter l'utilisation du tableau périodique
\newcommand{\elementH} {\elementText{1} {H} {Hydrogène}[1,00]}
\newcommand{\elementHe}{\elementText{2} {He}{Hélium}   [4,00]}
\newcommand{\elementLi}{\elementText{3} {Li}{Lithium}  [6,94]}
\newcommand{\elementBe}{\elementText{4} {Be}{Béryllium}[9,01]}
\newcommand{\elementB} {\elementText{5} {B} {Bore}     [10,8]}
\newcommand{\elementC} {\elementText{6} {C} {Carbone}  [12,0]}
\newcommand{\elementN} {\elementText{7} {N} {Azote}    [14,0]}
\newcommand{\elementO} {\elementText{8} {O} {Oxygène}  [16,0]}
\newcommand{\elementF} {\elementText{9} {F} {Fluor}    [19,0]}
\newcommand{\elementNe}{\elementText{10}{Ne}{Néon}     [20,2]}
\newcommand{\elementNa}{\elementText{11}{Na}{Sodium}   [23,0]}
\newcommand{\elementMg}{\elementText{12}{Mg}{Magnésium}[24,3]}
\newcommand{\elementAl}{\elementText{13}{Al}{Aluminium}[27,0]}
\newcommand{\elementSi}{\elementText{14}{Si}{Silicium} [28,1]}
\newcommand{\elementP} {\elementText{15}{P} {Phosphore}[31,0]}
\newcommand{\elementS} {\elementText{16}{S} {Soufre}   [32,1]}
\newcommand{\elementCl}{\elementText{17}{Cl}{Chlore}   [35,5]}
\newcommand{\elementAr}{\elementText{18}{Ar}{Argon}    [39,9]}
\newcommand{\elementK} {\elementText{19}{K} {Potassium}[39,1]}
\newcommand{\elementCa}{\elementText{20}{Ca}{Calcium}  [40,0]}


%%%% Couleurs réglables
\colorlet{couleurPrim}{cyanSombre}
\colorlet{couleurSec}{bleuPale}
\colorlet{couleurTer}{vertSombre}
\colorlet{couleurQuat}{orangeSombre}


%%%% Réglages de la taille des indentations et des sauts de paragraphes
\setlength{\parskip}{0cm}
\setlength{\parindent}{0cm}
\renewcommand{\baselinestretch}{1}
% réglage du niveau (sous-section) ou s'arrête la table des matières
\setcounter{tocdepth}{2}


%%%% Réglage de la géométrie des pages
\geometry{
  a4paper, % format
  left=1.3cm, right=1.3cm, % marge horizontale
  top=2.2cm, bottom=2.3cm % marge verticale
}


%%%% Réglage de chemfig
\setchemfig{
  atom sep=24pt,
  bond style={line width=1pt},
  angle increment=30
}


%%% Apparence (couleur) des liens
\hypersetup{
  colorlinks=true,
  linkcolor=black, % lien type table des matière
  citecolor=black, % citation
  filecolor=black, 
  urlcolor=couleurPrim!10!black % lien internet
}


%%%% Réglage de tikz (flèche et caractères)
\usetikzlibrary{babel}
\tikzset{>=latex}


%%%% Réglage des en-tête
\renewcommand{\headrulewidth}{0.4pt}
\setlength{\headheight}{22.50113pt}


%%%% Réglage de dashundergaps pour avoir des points et pas de numération
\dashundergapssetup{
  gap-numbers = false,
  gap-format = dot,
  gap-widen,
  gap-extend-percent
}


%%%% Réglage de siunit
\sisetup{
  locale = FR, % français
	 group-minimum-digits = 4, % groupage des chiffres par millier
  inter-unit-product = \ensuremath { { } \cdot { } }, % point médian entre les unités,
  propagate-math-font = true, reset-math-version = false % pour avoir des versions bold
}
\AtBeginDocument{\RenewCommandCopy\qty\SI} % Pour "écraser" la commande \qty du package physics
