%%%% Pour avoir les accents et autre caractère français
\usepackage[french]{babel}
\usepackage[T1]{fontenc}
\usepackage[utf8]{inputenc}

%%%% Paquets utilisé
\usepackage{ifthen} % pour programmer avec des boucle et des conditions
%% Images/dessin
\usepackage{subcaption} % pour les légendes des figures
\usepackage{graphicx} % pour insérer des images
\usepackage[european, straightvoltages, RPvoltages]{circuitikz} % pour dessiner des circuits électrique
\usepackage{pdfpages} % pour inclure des fichiers pdf
\usepackage{wrapfig} % pour entourer les images par du texte 
\usepackage{chemfig} % pour dessiner des formules chimiques
\usepackage{fontawesome} % pour dessiner de jolies icônes
%% Mise en page
\usepackage{geometry} % définition des marges
\usepackage{dashundergaps} % pour avoir générer des textes à compléter
\usepackage{fancyhdr} % pour faire des en-tête
\usepackage[many]{tcolorbox} % pour faire de jolie boîtes colorée
\usepackage{enumitem} % pour pouvoir définir des listes personnalisées
\usepackage{hyperref} % pour insérer des liens
\usepackage{multicol} % pour avoir plusieurs colonnes côte-à-côte
\usepackage{listings} % pour insérer du code
\usepackage{marginnote} % pour insérer des notes sur le côté
%% Tableau
\usepackage{tabularray} % pour avoir de meilleurs tableaux
%% QR code
\usepackage{qrcode}
%% Math
\usepackage{amsmath} % symboles mathématiques
\usepackage{amssymb} % symboles mathématiques en gras
\usepackage{wasysym} % pour avoir des symbole d'intégrale
\usepackage{accents} % pour les notations mathématiques avec une barre
\usepackage{physics} % pour les dérivées, les bra, les kets, etc.
\usepackage{esvect} % pour faire de jolis vecteurs
\usepackage{siunitx} % pour avoir de jolie grandeurs avec des unités 


%%%% Commandes prédéfinies
%%%%%%%%%%%%%%%%%%%%%%%%%%%%%%%%%%%%%%%%%%%%%%%%%%%%%%%%%%%%%%%%%%%%%%%%%%
%%%% quelque couleurs
\definecolor{vertSombre}  {RGB} {  0,  95,  17}
%
\definecolor{jauneSombre}{RGB}  {213, 145,   2}
\definecolor{orangeSombre}{RGB} {174,  82,   0}
\definecolor{cyanSombre}  {RGB} {  0, 140, 128}
\definecolor{bleuPale}    {RGB} { 39,  76, 167}
%
\definecolor{jauneClair} {RGB} {218, 173,   0}
\definecolor{rougeSombre}{RGB} {148,  31,   0}
\definecolor{rougeClair} {RGB} {224,  39,  34}

%%% quelques couleurs dérivées des couleurs choisie
\newcommand{\couleurPrimSombre}{couleurPrim!60!black}

%%%% rectangle coloré
\NewDocumentCommand{\rectangle}{O{couleurPrim} m m}{%
  \shorthandoff{;}
  \tikz \node (rect) [draw, fill, color=#1,
              minimum width=#2,
              minimum height=#3] {};
  \shorthandon{;}
}


%%%%%%%%%%%%%%%%%%%%%%%%%%%%%%%%%%%%%%%%%%%%%%%%%%%%%%%%%%%%%%%%%%%%%%%%%%%
%%%% une simple boite vide
\newtcolorbox{boite}[1][]{
  breakable, enhanced jigsaw, % pour s'étendre sur plusieurs pages
  arc = 0mm, % les lignes de la boites sont droites
  colback = white, colframe = black, % fond blanc et traits noirs
  #1
}

%%%% document
\newcounter{documentNum}
\newtcolorbox{doc}[3][]{
  before title = {\refstepcounter{documentNum}},
  breakable, enhanced jigsaw, % pour s'étendre sur plusieurs pages
  colback = white, % fond blanc
  colframe = couleurPrim!25!black, % couleur de la boite
  coltitle = black, % couleur du titre
  boxrule = 0.5mm, arc = 0.5mm, % largeur et arrondi des traits de la boite
  titlerule = 0mm, top = 0mm, % pour ne pas avoir de séparation titre/boite
  colbacktitle = white, % fond pour le titre blanc
  fonttitle = \bfseries\sffamily,
  title = {Document \arabic{documentNum} -- #2\strut \label{#3}},
  #1
}

%%%% Passage important à connaître
\newtcolorbox{encart}[1][]{
  breakable, enhanced jigsaw, % pour s'étendre sur plusieurs pages
  frame hidden, sharp corners, boxrule = 0mm, % pas de contours
  colback = couleurPrim!10, % fond
  borderline west={4pt}{0pt}{couleurPrim}, % barre gauche
  #1
}

%%%% contexte
\newtcolorbox{contexte}[1][]{
  breakable, enhanced jigsaw, % pour s'étendre sur plusieurs pages
  boxrule = 3pt, sharp corners, % contours droits
  colframe = couleurPrim, % couleur des contours
  colback = couleurPrim!5, % fond
  #1
}

%%%% Pour les objectifs
\newtcolorbox{boiteObjectifs}[2][]{
  empty, % pas de boite automatique
  attach boxed title to top left = {yshift=-2.5mm}, % position
  boxed title style = {empty, size = small, top = 1mm, bottom = 0.5mm},
  frame code = { % tracé de la boite
    \path (title.east |- frame.north) coordinate (aux);
    \path [draw=couleurPrim, line width = 3pt]
    (frame.west) |- ([xshift=-4mm] title.north east)
    to[out=0, in=180] ([xshift=10mm] aux) -| % définit la courbe
    (frame.east) |- (frame.south) -| cycle; % trace la boite
  },
  coltitle = black, % couleur du titre
  fonttitle = \bfseries\sffamily, % police du titre
  title = {#2},
  #1 
}
\newenvironment{objectifs}{
  \begin{boiteObjectifs}{Objectifs de la séance :}
    \begin{listeObjectifs}
}{
    \end{listeObjectifs}
  \end{boiteObjectifs}
}

%%%% Espace pour un coup de pouce
\newcounter{coupDePouceNum}
\newtcolorbox{coupDePouce}[1][]{
  before title = {\refstepcounter{coupDePouceNum}},
  breakable, enhanced jigsaw, % pour s'étendre sur plusieurs pages
  arc = 0mm, % les lignes de la boites sont droites
  colback = white, colframe = black, % fond blanc et traits noirs
  fonttitle = \bfseries, coltitle = black, % couleur et police du titre
  titlerule = 0mm, top = 0mm, % pour ne pas avoir de séparation titre/boite
  colbacktitle = white, % fond pour le titre blanc
  title = {
    \textcolor{couleurPrim}{\faThumbsUp}
    Coup de pouce \arabic{coupDePouceNum} :
    \flushright \vspace*{-26pt}\faSquareO
  },
  #1
}

%%%% Espace pour une appréciation
\newcommand{\appreciation}[1]{
  \begin{boite}
    \vspace*{-4pt}
    \sousTitre{Appréciation et remarques}
    
    \vspace*{#1 cm}
    \phantom{b}
  \end{boite}
}


%%%%%%%%%%%%%%%%%%%%%%%%%%%%%%%%%%%%%%%%%%%%%%%%%%%%%%%%%%%%%%%%%%%%%%%%%%
%%%% pagination et sections
\newcommand{\titre}[1]{
  \begin{center}
    \textsf{\bfseries \Large #1}
  \end{center}
}
\newcommand{\sousTitre}[1]{
  \textsf{\bfseries #1}
}
\newcommand{\pasDePagination}{
  \thispagestyle{empty}
}
\newcommand{\feuilleBlanche}{
  \newpage
  \phantom{b}
  \pasDePagination
}


%%%% activité ou TP
\newcounter{activiteNum}
\newcommand{\titreActi}[2]{
  \refstepcounter{activiteNum}
  \titre{#1 \arabic{section}.\arabic{activiteNum} -- #2}
}
\newcommand{\titreTP}[1]{
  \titreActi{TP}{#1}
  % \titreActi{Activité expérimentale}{#1}
}
\newcommand{\titreActivite}[1]{
  \titreActi{Activité}{#1}
}
\newcommand{\titreEvaluation}[1]{
  \titre{Évaluation \arabic{section} -- #1}
  % reset du numéro de page et d'exercices
  \setcounter{page}{1}
  \numeroActivite{1}
}


%%%% chapitre, section et sous-section
\newcommand{\titreChapitre}[1]{
  \titre{Chapitre \arabic{section} : #1}
}
\newcommand{\titrePartie}[1]{
  \vspace*{24pt}
  \refstepcounter{subsection}
  \rectangle{60pt}{1pt}
  \sousTitre{\Large \Roman{subsection} -- #1}
  \rectangle{60pt}{1pt}
  \vspace*{10pt}
}
\newcounter{sousSectionNum}
\newcommand{\titreSection}[1]{
  \vspace*{16pt}
  \refstepcounter{subsubsection}
  \setcounter{sousSectionNum}{0}
  \rectangle{30pt}{4pt}
  \sousTitre{\large \arabic{subsubsection} -- #1}
  \vspace*{10pt}
}
\newcommand{\titreSousSection}[1]{
  \vspace*{12pt}
  \refstepcounter{sousSectionNum}
  \sousTitre{\Alph{sousSectionNum} -- #1}
  \vspace*{8pt}
}

%%%% fixe le numéro de l'activité
\newcommand{\numeroActivite}[1]{
  % fixe les compteurs LaTeX
  \setcounter{page}{1}
  \setcounter{subsection}{0}
  \setcounter{subsubsection}{0}
  \setcounter{figure}{0}
  % fixe les compteurs internes
  \setcounter{qcmNum}{0}
  \setcounter{documentNum}{0}
  \setcounter{questionNum}{0}
  \setcounter{coupDePouceNum}{0}
  \setcounter{sousSectionNum}{0}
  \setcounter{activiteNum}{#1 - 1}
}
% fixe le numéro de partie (#1) et le numéro de la page (#2)
\newcommand{\numeroPartieCours}[2]{
  \newpage
  \setcounter{subsection}{#1 - 1}
  \setcounter{page}{#2}
}

%%%% lignes
\newcommand{\ligne}{
  \par\noindent\rule{\textwidth}{0.4pt}
}
\newcommand{\lignePointillee}[1]{
  \makebox[#1\linewidth]{\dotfill}
}


%%%%%%%%%%%%%%%%%%%%%%%%%%%%%%%%%%%%%%%%%%%%%%%%%%%%%%%%%%%%%%%%%%%%%%%%%%
%%%% Paramètre par défaut pour l'en-tête
\newcommand{\annee}{Réglez avec \textbackslash renewcommand\{\textbackslash annee\}\{2023 -- 2024\}}
\newcommand{\classe}{Réglez avec \textbackslash renewcommand\{\textbackslash classe\}\{Seconde\}}
\newcommand{\etablissement}{Réglez avec \textbackslash renewcommand\{\textbackslash etablissement\}\{Lycée\}}

%%%% en-tête
\newcommand{\teteGauche}[2]{
  \lhead{
    \textbf{\footnotesize #1}
    \newline
    \footnotesize #2
  }
}
\newcommand{\teteDroite}[2]{
  \rhead{
    \hfill \textbf{\footnotesize #1}
    \newline \hfill
    \footnotesize #2
  }
}
\newcommand{\enTete}[3]{
  \pagestyle{fancy}
  \setcounter{section}{#3}
  \setcounter{subsection}{0}
  \setcounter{sousSectionNum}{0}
  \teteGauche{\etablissement{}}{Chapitre \arabic{section} -- #1} % left header
  \chead{} % central header
  \teteDroite{\annee{}}{#2} % right header
}
%
\newcommand{\enTeteFicheReussite}[1]{
  \newpage
  \setcounter{subsection}{0}
  \pasDePagination
  
  \phantom{b}
  \vspace*{-70pt}
  \titre{Fiche \og Réussir son évaluation \fg}
  \titre{#1}
  
  \vspace*{-6pt}
  % \titreSection{Ce que je dois savoir}
  
  Pour savoir quoi réviser, je lis les points clés du chapitre évalués :
  \begin{itemize}
    \item Si je pense maîtriser une notion, je coche la case \ok
    \item Si je pense que je dois la retravailler, je coche la case \pasOk
  \end{itemize}
  
  Pour travailler les notions qui ne sont pas maîtrisées, je reprend les activités associés.
}
\newcommand{\basDePageFicheReussite}{
  \titreSection{Ce qu'il me reste à faire}
}
\newcommand{\travailExerciceCorrige}{
  Pour être sûr-e d'obtenir une bonne note, je m'entraîne avec les exercices corrigés du manuel indiqués dans la colonne de droite.
}
\newcommand{\questionFicheReussite}[1]{
  S'il me reste des questions, je les note ici pour les poser au début de l'évaluation : 
  \lignesDeReponse{#1}
}
\newcommand{\coursFicheReussite}{
  Je prépare une fiche au format A4 avec toutes les notions, définitions ou grandeurs dont je pense avoir besoin pendant l'évaluation.
}


%%%%%%%%%%%%%%%%%%%%%%%%%%%%%%%%%%%%%%%%%%%%%%%%%%%%%%%%%%%%%%%%%%%%%%%%%%
%%%% exercice
% définit un booléen pour entrer ou sortir du mode correction
\newboolean{modeProf}
\setboolean{modeProf}{false}
\newcommand{\modeCorrection}{
  \setboolean{modeProf}{true}
  \TeacherModeOn
}

% Pour afficher le numéro d'une question avec choix du compteur
\NewDocumentCommand{\numeroQuestion}{O{questionNum} O{16}}{
  \refstepcounter{#1}
  \vspace*{2pt}
  \hspace{#2pt}
  \textcolor{couleurSec}{
    \textbf{\arabic{#1}} {\small\faMinus}
  }
}

% Pour homogénéiser les commandes avec du texte corrigé
\newcommand{\texteCorrection}[1]{
  \textbf{\textcolor{couleurTer}{#1}}
}

% Pour simplifier la commande question
\newcommand{\reponseCorrigee}[1]{
  \vspace*{2pt}
  \textcolor{couleurSec}{\faCaretRight}
  \hspace{1pt}
  \texteCorrection{#1}
}


% trace des lignes pointillées pour répondre aux questions
% \lignessDeReponse* complète la ligne actuelle par des pointillées
% \lignesDeReponse commence à la ligne suivante
\newcounter{ligneNum}
\NewDocumentCommand{\lignesDeReponse}{s m}{
  % Trace la fin de la ligne, ou pas
  \IfBooleanTF{#1}{ % Version *
    \espaceReponse \dotfill
    \ifnum #2 < 1
      \newline
    \fi
  }{}
  % Trace le bon nombre de lignes
  \setcounter{ligneNum}{-1}
  \loop
    \stepcounter{ligneNum}
    \ifnum \value{ligneNum} < #2
      \\[8pt] \lignePointillee{0.98}
  \repeat
  \vspace*{1pt}
}


% définit une commande pour afficher une question 
% #1 : question
% #2 : réponse
% #3 : nombres de lignes pour répondre
\newcounter{questionNum}
\newcommand{\question}[3]{
  \numeroQuestion \!#1
  % pointille ou correction
  \ifthenelse {\boolean{modeProf}} { % prof
    \\[1pt] \reponseCorrigee{#2}
  }{ % eleve
    \lignesDeReponse{#3}
  }
}

% Correction affichée si en mode prof
\newcommand{\correction}[1]{
  \ifthenelse {\boolean{modeProf}} { % prof
    #1
  }{ % eleve
  }
}

% sous questions
\newcommand{\sousQuestion}[2]{
  \hspace{16pt}
  \textcolor{couleurSec}{\textbullet} #1
  
  \vspace*{8pt}
  \reponse{#2}
}

% question QCM
\newcommand{\QCM}[2]{
  \numeroQuestion[qcmNum][0] #1
  \begin{qcm}
    #2
  \end{qcm}
}

% À ajouter devant la bonne réponse dans un qcm
\newcounter{qcmNum}
\newcommand{\reponseQCM}{
  \correction{
    \hspace*{-15pt}$\checkmark$\hspace*{-12pt}
  } % Note : trace une croix à la bonne position
}

%%%% Pour afficher les compétences
\newcommand{\competence}[1]{
  ~{\footnotesize\textit{(#1)}}
}

%%%% Espace pour indiquer nom, prénom et classe
\newcommand{\nomPrenomClasse}{
  \vspace*{-24pt}
  Nom : \lignePointillee{0.3}
  Prénom : \lignePointillee{0.3}
  Classe : \dotfill
}
\newcommand{\nomPrenom}{
  \vspace*{-24pt}
  Nom : \lignePointillee{0.3}
  Prénom : \lignePointillee{0.3}
}


%%%%%%%%%%%%%%%%%%%%%%%%%%%%%%%%%%%%%%%%%%%%%%%%%%%%%%%%%%%%%%%%%%%%%%%%%%
% texte à trou avec option pour régler la largeur
\NewDocumentCommand{\texteTrou}{o +m}{
  \IfValueTF{#1}{ % Si la largeur est réglée, on utilise des lignes
    \ifthenelse {\boolean{modeProf}} {% prof
      \texteCorrection{#2}
    }{% élève
      \espaceReponse
      \lignePointillee{#1}
      \hspace*{-12pt}
    }
  }{ % Sinon on utilise dash undergap pour la version automatique
    \ifthenelse{\boolean{modeProf}}{% prof
      \hspace*{-24pt}
    }{% élève
      \espaceReponse \hspace*{0.1pt}
    }
    \gap{\texteCorrection{#2}}
  }
}

% texte à trou avec option pour laisser plusieurs lignes
\NewDocumentCommand{\texteTrouLignes}{O{0} +m}{
  \ifthenelse {\boolean{modeProf}} {% prof
    \texteCorrection{#2}
  }{% élève
    \lignesDeReponse*{#1}
  }
}

% espace vertical pour la réponse
\newcommand{\espaceReponse}{
  \phantom{$\dfrac{1}{1}$} % espace vertical
  \hspace*{-38pt} \phantom{b} % ajuste l'espace horizontal
}


%%%%%%%%%%%%%%%%%%%%%%%%%%%%%%%%%%%%%%%%%%%%%%%%%%%%%%%%%%%%%%%%%%%%%%%%%%
%%%% Pour choisir parmi deux sujets
\newboolean{sujetA}
\setboolean{sujetA}{true}
\newcommand{\sujetB}{
  \setboolean{sujetA}{false}
}
\newcommand{\sujetA}{
  \setboolean{sujetA}{true}
}

%%%% Pour faire plusieurs sujets en parallèle
\newcommand{\variationSujet}[2]{
  \hspace*{-6pt}
  \ifthenelse {\boolean{sujetA}}{#1}{#2}
  \hspace*{-6pt}
}


%%%%%%%%%%%%%%%%%%%%%%%%%%%%%%%%%%%%%%%%%%%%%%%%%%%%%%%%%%%%%%%%%%%%%%%%%%
%%%% Tableau générique avec la première ligne bleue
\NewDocumentEnvironment{tableau}{m}{
  \begin{center}
  \begin{tblr}{
    hlines,
    colspec = #1,
    row{1} = {couleurPrim!20},
  }
}{
  \end{tblr}
  \end{center}
}

%%%% Tableau de competence
\newenvironment{tableauCompetences}{
  \centering
  \begin{tblr}{
    colspec = {| c | X[l] | c | c | c | c |},
    rows = {m}, hlines,
    row{1} = {couleurPrim!20}
  }
    \textbf{Compétences} & \centering \textbf{Items} 
    & \textbf{D} & \textbf{C} & \textbf{B} & \textbf{A} \\
}{
  \end{tblr}
}

%%%% Tableau de connaissances
\newenvironment{tableauConnaissances}{
  \centering
  \begin{tblr}{
    colspec = {| X[c] | c | c | c | c |},
    rows = {m}, hlines,
    column{4} = {0.22},
    column{5} = {0.21},
    row{1} = {couleurPrim!20}
  }
    \textbf{Points clés du chapitre} & \ok & \pasOk
    & \textbf{En classe} & \textbf{Exercices} \\
}{   
  \end{tblr}
}

%%%% Tableau de connaissances sans exercices
\newenvironment{tableauConnaissancesSansExercices}{
  \centering
  \begin{tblr}{
    colspec = {| X[l,m] | c | c | c |},
    rows = {m}, hlines,
    column{4} = {0.2},
    row{1} = {couleurPrim!20}
  }
    \centering \textbf{Points clés du chapitre} & \ok & \pasOk
    & \textbf{En classe} \\
}{ 
  \end{tblr}
}

%%%% Tableau de correction élève
\newcommand{\correctionEleve}[1]{
  \begin{tblr}{
    hlines,
    colspec = {| X[-1, c] | X[2, c] | X[2, c] | X[2, c] |},
    row{1} = {couleurPrim!20}
  }
    \textbf{Question} & 
    \textbf{L'erreur} &
    \textbf{Analyse de l'erreur} &
    \textbf{La correction} \\
    %
    \phantom{b} \vspace{#1 pt} & & & \\
    \phantom{b} \vspace{#1 pt} & & & \\
    \phantom{b} \vspace{#1 pt} & & & \\
    \phantom{b} \vspace{#1 pt} & & & \\
  \end{tblr}
}

%%%% Tableau bilan de la correction
\newcommand{\bilanCorrection}[1]{
  \begin{tblr}{
    hlines,
    colspec = {| X[c] | X[c] |},
    row{1} = {couleurPrim!20}
  }
    \textbf{Ce que je n'avais pas compris...} &
    \textbf{Ce que maintenant j'ai compris...} \\
    \phantom{b} \vspace{#1 pt} & \\
  \end{tblr}
}


%%%% Alignement dans un tableau
\newcommand{\vAligne}[1]{
  \strut \\ \vspace*{#1}
}


%%%%%%%%%%%%%%%%%%%%%%%%%%%%%%%%%%%%%%%%%%%%%%%%%%%%%%%%%%%%%%%%%%%%%%%%%%
%%%% symboles : chevron, flèche, attention, etc.
\NewDocumentCommand{\chevron}{O{couleurPrim}}{
  \textcolor{#1}{\small \faChevronRight}
}
%
\NewDocumentCommand{\fleche}{O{couleurPrim}}{
  \textcolor{#1}{\faCaretRight}
}
%
\NewDocumentCommand{\attention}{O{couleurPrim}}{
  \textcolor{#1}{\faExclamationTriangle}
}
%
\NewDocumentCommand{\flecheLongue}{O{couleurPrim}}{
  \textcolor{#1}{\faLongArrowRight}
}
%
\NewDocumentCommand{\ok}{O{couleurPrim}}{
  \textcolor{#1}{\faCheckCircle}
}
%
\NewDocumentCommand{\pasOk}{O{couleurPrim}}{
  \textcolor{#1}{\faTimesCircle}
}
%
\NewDocumentCommand{\pointCyan}{O{couleurPrim}}{
  \textcolor{#1}{\textbullet}
}
%
\NewDocumentCommand{\mesure}{O{couleurPrim}}{
  \hspace{15pt}
  \textcolor{couleurSec}{\faWrench\faFlask}
}
% pictogramme sécurité
\newcommand{\picto}[2]{
  \image{#1}{images/pictogrammes/picto_#2}
}


%%%%%%%%%%%%%%%%%%%%%%%%%%%%%%%%%%%%%%%%%%%%%%%%%%%%%%%%%%%%%%%%%%%%%%%%%%
%%%% emphase
\newcommand{\emphase}[1]{
  \textcolor{couleurSec}{\textsf{\bfseries \large #1}}
}
%
\newcommand{\important}[1]{
  \!\textcolor{couleurSec}{\textsf{\bfseries #1}}\!\!
}
%
\newcommand{\exemple}{
  \flecheLongue \textit{Exemple :}
}
\newcommand{\exemples}{
  \flecheLongue \textit{Exemples :}
}
%
\newcounter{compteAppelProf}
\newcommand{\appelProf}{
  \refstepcounter{compteAppelProf}
  \hspace{24pt} \faHandPaperO \hspace{2pt}
  \textbf{Appel n$^\circ$ \arabic{compteAppelProf} :}
}


%%%% image
\newcommand{\image}[2]{
  \includegraphics[width=#1\linewidth]{#2}
}


%%%%%%%%%%%%%%%%%%%%%%%%%%%%%%%%%%%%%%%%%%%%%%%%%%%%%%%%%%%%%%%%%%%%%%%%%%
%%%% qcm
\newlist{qcm}{itemize}{2}
\setlist[qcm]{label=$\square$, leftmargin=2cm}

%%%% liste d'objectif
\newlist{listeObjectifs}{itemize}{2}
\setlist[listeObjectifs]{label = \chevron}

%%%% protocole
\newlist{protocole}{itemize}{2}
\setlist[protocole]{label = {\footnotesize \fleche[couleurSec]}}

%%%% liste de points
\newlist{listePoints}{itemize}{2}
\setlist[listePoints]{label = \pointCyan}

%%%% liste tirets
\newlist{listeTirets}{itemize}{2}
\setlist[listeTirets]{label = \textcolor{couleurPrim}{\small\faMinus}}

%%%% liste avec des flèches
\newlist{listeFleche}{itemize}{2}
\setlist[listeFleche]{label = \textbf{\flecheLongue}}

%%%% jeu de données
\newenvironment{donnees}{
  
  \textbf{Données :}
  \vspace*{-8pt}
  \begin{multicols}{2}
    \begin{listeTirets}
}{
    \end{listeTirets}
  \end{multicols}
}

%%%% problematique
\newcommand{\problematique}[1]{
  \hspace{8pt}
  \flecheLongue
  \textbf{#1}
}

%%%% liste avec chiffre
\newlist{enumeration}{enumerate}{2}
\setlist[enumeration]{label = \textcolor{\couleurPrimSombre}{\textbf{\arabic*.}} }


%%%%%%%%%%%%%%%%%%%%%%%%%%%%%%%%%%%%%%%%%%%%%%%%%%%%%%%%%%%%%%%%%%%%%%%%%%
%%%% Séparation de la page en blocs
\newcommand{\separationTroisBlocs}[3]{
  \begin{minipage}[T]{0.3\linewidth}
    #1
  \end{minipage}
  ~
  \begin{minipage}[T]{0.3\linewidth}
    #2
  \end{minipage}
  ~
  \begin{minipage}[T]{0.3\linewidth}
    #3
  \end{minipage}
}
%%%% Separation en deux blocs
\NewDocumentCommand{\separationBlocs}{+m O{0.48} +m O{0.48}}{
  \begin{minipage}[T]{#2\linewidth}
    #1
  \end{minipage}
  \hfill
  \begin{minipage}[T]{#4\linewidth}
    #3
  \end{minipage}
}


%%%%%%%%%%%%%%%%%%%%%%%%%%%%%%%%%%%%%%%%%%%%%%%%%%%%%%%%%%%%%%%%%%%%%%%%%%
%% nombre algébrique, réaction
\newcommand{\algebrique}[1]{
  \overline{\mathrm{#1}}
}
\newcommand{\reaction}{
  \!\!\schemestart \arrow(.mid east--.mid west){->}[, 0.9, ultra thick] \schemestop\!\!
}

%% Pour simplifier l'écriture des formules brutes
\newcommand{\bruteCHO}[3]{
  \chemfig{C_{#1} H_{#2} O_{#3}}
}

%% pour les masse molaire et atomique
\newcommand{\masseMol}[1]{
  M(\chemfig{#1})
  % M_{\chemfig{#1}}
}
\newcommand{\masseAtom}[1]{
  m(\chemfig{#1})
  % m_{\chemfig{#1}}
}


%% atome ou isotope
\makeatletter
\newcommand{\isotope}[3]{%
   \settowidth\@tempdimb{\ensuremath{\scriptstyle#1}}%
   \settowidth\@tempdimc{\ensuremath{\scriptstyle#2}}%
   \ifnum\@tempdimb>\@tempdimc%
       \setlength{\@tempdima}{\@tempdimb}%
   \else%
       \setlength{\@tempdima}{\@tempdimc}%
   \fi%
  \begingroup%
  \ensuremath{
    ^{\makebox[\@tempdima][r]{\ensuremath{\scriptstyle#1}}}
    _{\makebox[\@tempdima][r]{\ensuremath{\scriptstyle#2}}}
    \chemfig{#3}
  }%
  \endgroup%
}%
\makeatother

%% element chimique dans le tableau périodique
\makeatletter
\newcommand{\element}[2]{%
   \settowidth\@tempdimb{\ensuremath{\footnotesize #1}}%
  \begingroup%
  \ensuremath{
    _{\makebox[\@tempdimb][r]{\ensuremath{\small #1}}} 
    \chemfig[atom style={scale=1.3}]{#2}
  }%
  \endgroup%
}%
\makeatother

%% siècle
\newcommand{\siecle}[1]{
  \textsc{\romannumeral #1}\textsuperscript{e}~siècle
}

%% texte avec une boite autour
\newcommand{\texteEncadre}[1]{
  \frame{
    \vphantom{$\dfrac{1}{10}$} \text{#1}
  }
}

%% case cochée
\newcommand{\caseCochee}{
  $\text{\rlap{$\checkmark$}}\square$
}


%%%%%%%%%%%%%%%%%%%%%%%%%%%%%%%%%%%%%%%%%%%%%%%%%%%%%%%%%%%%%%%%%%%%%%%%%%
%%%% Couleur pour le code
\definecolor{vertCode}  {rgb}{0.2,0.6,0}
\definecolor{grisCode}  {rgb}{0.5,0.5,0.5}
\definecolor{violetCode}{rgb}{0.58,0,0.82}
\definecolor{fondCode}  {rgb}{0.95,0.95,0.92}
%%%% Style python
\lstdefinestyle{codePython}{
  backgroundcolor=\color{fondCode},
  commentstyle=\color{magenta},
  keywordstyle=\color{vertCode},
  numberstyle=\tiny\color{grisCode},
  stringstyle=\color{violetCode},
  basicstyle=\ttfamily\footnotesize,
  breakatwhitespace=false,
  breaklines=true,
  captionpos=b,
  keepspaces=true,
  numbers=left,
  numbersep=5pt, 
  showspaces=false,
  showstringspaces=false,
  showtabs=false, 
  tabsize=2
}
\def\inline{\lstinline[style=codePython,language=python]}


%%%%%%%%%%%%%%%%%%%%%%%%%%%%%%%%%%%%%%%%%%%%%%%%%%%%%%%%%%%%%%%%%%%%%%%%%%
%%%% circuit tikz
\NewDocumentCommand{\fixedvlen}{O{0.5cm} m m O{}}{% [semilength]{node}{label}[extra options]
  % get the center of the standard arrow
  \coordinate (#2-Vcenter) at ($(#2-Vfrom)!0.5!(#2-Vto)$);
  % draw an arrow of a fixed size around that center and on the same line
  \draw[-Triangle, #4] ($(#2-Vcenter)!#1!(#2-Vfrom)$) -- ($(#2-Vcenter)!#1!(#2-Vto)$);
  % position the label as in the normal voltages
  \node[anchor=\ctikzgetanchor{#2}{Vlab}, #4] at (#2-Vlab) {#3};
}

%%%%%%%%%%%%%%%%%%%%%%%%%%%%%%%%%%%%%%%%%%%%%%%%%%%%%%%%%%%%%
%% grandeurs récurrentes
% Physique
\newcommand{\ISS}{\text{ISS}}
\newcommand{\Terre}{\text{Terre}}
\newcommand{\inertie}{\text{inertie}}
\newcommand{\Tfus}{T_\text{f}}
\newcommand{\Teb}{T_\text{éb}}
% Chimie
\newcommand{\solute}{\text{soluté}}
\newcommand{\solution}{\text{solution}}
\newcommand{\espece}{\text{espèce}}
\newcommand{\avogadro}{\num{6,02e23}}
% ions
\newcommand{\ionFerII}      {Fer II      \chemfig{Fe^{2+}}   }
\newcommand{\ionFerIII}     {Fer III     \chemfig{Fe^{3+}}   }
\newcommand{\ionSodium}     {Sodium      \chemfig{Na^{+}}    }
\newcommand{\ionCuivreII}   {Cuivre II   \chemfig{Cu^{2+}}   }
\newcommand{\ionCalcium}    {Calcium     \chemfig{Ca^{2+}}   }
\newcommand{\ionSulfate}    {Sulfate     \chemfig{SO_4^{2-}} }
\newcommand{\ionNitrate}    {Nitrate     \chemfig{NO_3^{-}}  }
\newcommand{\ionChlorure}   {Chlorure    \chemfig{Cl^{-}}    }
\newcommand{\ionFluorure}   {Fluorure    \chemfig{F^{-}}     }
\newcommand{\ionMagnesium}  {Magnésium   \chemfig{Mg^{2+}}   }
\newcommand{\ionPotassium}  {Potassium   \chemfig{K^{+}}     }
\newcommand{\ionBicarbonate}{Bicarbonate \chemfig{CO_3^{2-}} }

%% vecteurs
\newcommand{\FBsurA}{F_{B/A}}
\newcommand{\FAsurB}{F_{A/B}}
\newcommand{\vvFAsurB}{\vv{F}_{A/B}}
\newcommand{\vvFBsurA}{\vv{F}_{B/A}}
%%%%%%%%%%%%%%%%%%%%%%%%%%%%%%%%%%%%%%%%%%%%%%%%%%%%%%%%%%%%%
%%%% figures simples
\newcommand{\tkzRect}[4]{
  \fill[color=#1] (#2,#4) -- (-#2,#4) -- (-#2,#3) -- (#2,#3);
}
\newcommand{\tkzEllipse}[4]{
  \fill[color=#1] (0,#3) ellipse (#2 and #4);
}

%%%% Point et vecteurs
% \tikzCercle [couleur fond] (x, y) {rayon} [couleur ligne]
\NewDocumentCommand{\tikzCercle}{O{black} r() m o}{
  \IfValueTF{#4}{
    \filldraw [color = #4, fill = #1, very thick] (#2) circle (#3 pt);
  }{
    \filldraw [#1] (#2) circle (#3 pt);
  }
}
% \tikzPointLabel (x, y) [texte] (xtexte, ytexte), trace un point avec un label; * = pas de point
\NewDocumentCommand{\tikzLabel}{s r() m d()}{
  \IfBooleanTF{#1}{
    \node at (#2) {#3};
  }{
    \IfValueTF{#4}{
      \filldraw (#2) circle (2pt);
      \node at (#4) {#3};
    }{
     \filldraw (#2) circle (2pt) node[above] {#3};
    }
  }
}
% \tikzVecteur [couleur] (x, y) (lx, ly) {legende} [position legende];  * = <->
\NewDocumentCommand{\tikzVecteur}{s O{black} r() r() m O{right}}{
  \coordinate (A) at (#3);
  \coordinate (B) at (#4);
  \coordinate (AB) at ($(A)+(B)$);
  \IfBooleanTF{#1}{
    \draw[#2, <->, very thick] (A) -- (AB) node[#6] {#5};
  }{
    \draw[#2, ->, very thick] (A) -- (AB) node[#6] {#5};
  }
}

% \tikzLegende [couleur] (x, y) (lx, ly) {légende}; * = <-
% ajouter une * passe de la version gauche -> à la version droite <-
\NewDocumentCommand{\tikzLegende}{s O{black} r() r() m}{
  \coordinate (A) at (#3);
  \coordinate (B) at (#4);
  \coordinate (AB) at ($(A)+(B)$);
  \IfBooleanTF{#1}{
    \draw[#2, ->, very thick] (AB) node[right] {#5} -- (A);
  }{
    \draw[#2, ->, very thick] (A) node[left] {#5} -- (AB);
  }
}

%% Trace une petite barre de pourcentage partiellement remplie sur la ligne
\newcommand{\barrePourcentage}[1]{%
  \begin{tikzpicture}
    \fill[color=couleurSec]    (0.0,    0.0) rectangle (#1*8ex, 1.5ex);
    \fill[color=couleurSec!20] (#1*8ex, 0.0) rectangle (8.0ex,  1.5ex);
  \end{tikzpicture}
}


%%%%%%%%%%%%%%%%%%%%%%%%%%%%%%%%%%%%%%%%%%%%%%%%%%%%%%%%%%%%%
%% Trace une flèche de progression pour les plans de travail
% \flecheProgression {<nombre de boucles>} [<largeur>] [<espacement vertical>]
\NewDocumentCommand{\flecheProgression}{m O{17} O{2.4}}{%
  \strut\vspace*{8pt}
  \def\nombreBoucle{\numexpr((#1 - 1)*2)}
  \begin{center}
    \begin{tikzpicture}
      \tikzset{bluestyle/.style={line width = 20pt, rounded corners = 10mm, color = couleurSec}}
      % Premier bout pour l'alignement, les parenthèses sont nécessaires
      \draw[bluestyle] (0, {(\nombreBoucle)*#3}) -- (1, {(\nombreBoucle)*#3}); 
      % Partie centrale répétée
      \draw[bluestyle]
        \foreach \x in {0,2,...,\nombreBoucle}  {
          \ifnum \x < \nombreBoucle
            ( 1, {(\nombreBoucle - \x)  *#3}) -- (#2,  {(\nombreBoucle - \x)  *#3}) --
            (#2, {(\nombreBoucle - \x - 1)*#3}) -- ( 0,  {(\nombreBoucle - \x - 1)*#3}) --
            ( 0, {(\nombreBoucle - \x - 2)*#3}) -- (1.1, {(\nombreBoucle - \x - 2)*#3})
          \fi
        };
      % Flèche finale
      \draw[-{Triangle [width = 36pt, length = 16pt]}, bluestyle] (0.8, 0) -- (#2, 0);
    \end{tikzpicture}
  \end{center}
  %\vspace*{-{#1*118}pt} %% Trouver comment faire le calcul automatiquement...
}


%%%%%%%%%%%%%%%%%%%%%%%%%%%%%%%%%%%%%%%%%%%%%%%%%%%%%%%%%%%%%
%%%% plan de classe
%% Trace un texte centré dans un cadre (x, x+l) -- (y, y+h)
% #1 couleur cadre ; #2 Positionition x ;
% #3 largeur l ;     #4 Positionition y ;
% #5 hauteur h ;     #6 texte.
\NewDocumentCommand{\texteCadre}{O{black} r() O{2} r() O{2} m}{
  \filldraw [fill=white, draw=#1, ultra thick] (#2, #4) rectangle (#2 + #3, #4 + #5);
  \node at (#2 + #3/2, #4 + #5/2) [font=\sffamily] {\textbf{#6}};
}
%% place dans la classe
\NewDocumentCommand{\place}{r() m}{
  \texteCadre(#1)[3](0)[2] {#2}
}

%% Pour tracer une rangée d'élève avec 2 ou 3 colonnes
% \rang {<numero rangee>} {<eleves>} {<eleves>} [<eleves>]
\ExplSyntaxOn
% Position de la place horizontale
\int_new:N \l_rangPositionX_int
\NewDocumentCommand{\rang}{m >{\SplitList{,}} m >{\SplitList{,}} m >{\SplitList{,}} d[]}{
  \begin{tikzpicture}
    % Initialisation de la position horizontale
    \int_set:Nn \l_rangPositionX_int {0}
    % Première rangée
    \ProcessList{#2}{\rangImpl}
    \int_add:Nn \l_rangPositionX_int { 1 }
    % Deuxième rangée
    \ProcessList{#3}{\rangImpl}
    % Troisième rangée
    \IfValueT{#4}{
      \int_add:Nn \l_rangPositionX_int { 1 }
      \ProcessList{#4}{\rangImpl}
    }
  \end{tikzpicture}
  \bigskip
}
\NewDocumentCommand{\rangImpl}{m}{
  \int_add:Nn \l_rangPositionX_int { 3 }
  \place(\l_rangPositionX_int){#1} %
}
\ExplSyntaxOff


%%%%%%%%%%%%%%%%%%%%%%%%%%%%%%%%%%%%%%%%%%%%%%%%%%%%%%%%%%%%%
%%%% tube à essais
\newcommand{\tkzTubeEssais}[3]{
  \draw[thick] (#1,#2) -- (#1,0) arc (0:-180:#1) -- (-#1,#2);
  \draw[thick] (0,#2) ellipse (#1 and #3);
}
\newcommand{\tkzBasTubeEssais}[5]{
  \fill[color=#1] (-#2,#3) -- (#2,#3) arc (0:-180:#2);
  \tkzRect{#1}{#2}{#3 - 0.01}{#4}
  \tkzEllipse{#1!85!black}{#2}{#4}{#5}
}
\newcommand{\tkzPhaseTubeEssais}[5]{
  \tkzRect{#1}{#2}{#3}{#4}
  \tkzEllipse{#1}{#2}{#3}{#5}
  \tkzEllipse{#1!85!black}{#2}{#4}{#5}
}

%%%% tube à essai de sang
\newcommand{\tubeEssaisSolution}[1]{
  \begin{tikzpicture}
    \tkzBasTubeEssais{#1}{0.25}{0}{0.75}{0.1} % contenu du tube
    \tkzTubeEssais{0.25}{1.5}{0.1} % tube
  \end{tikzpicture}
}

\newcommand{\tubeEssaisSangCentrifuge}[3]{
  \begin{tikzpicture}
    % phases dans le tube à essai
    \tkzBasTubeEssais{rougeSombre!75!white} {0.35}{0}{#1}{0.1}
    \tkzPhaseTubeEssais{gray!10!white}      {0.35}{#1}{#2}{0.1}
    \tkzPhaseTubeEssais{jauneClair!75!white}{0.35}{#2}{#3}{0.1}
    \tkzTubeEssais{0.35}{#3 + 1}{0.1}
    % Légende
    \tkzLegende(0.4)(#3 - 0.1) [1]{Plasma}*
    \tkzLegende(0.4)(#2 - 0.08)[1]{Globules blancs}*
    \tkzLegende(0.4)(-0.1)     [1]{Globules rouges}*
  \end{tikzpicture}
}
%%%% Ce fichier sert à déclarer les titres des chapitres des différents niveaux

%% Commun
\newcommand{\methode} {Outils pratiques}

%% Seconde
%%%% Chapitre
\newcommand{\snd}{Seconde}
\newcommand{\sndCorp} {Corps purs et mélanges}
\newcommand{\sndSolu} {Solutions}
\newcommand{\sndMouv} {Mouvement et interactions}
\newcommand{\sndAtom} {Structure de l'atome}
\newcommand{\sndMole} {Des atomes à la matière}
\newcommand{\sndLumi} {Ondes lumineuses et optique}
\newcommand{\sndTran} {Transformations de la matière}
\newcommand{\sndChim} {Transformations chimiques}
\newcommand{\sndSign} {Signaux et capteurs}

%%%% en-tête correspondant
\newcommand{\teteSndAP}   {\enTete[\snd]{Accompagnement personnalisé}}
\newcommand{\teteSndMeth} {\enTete[\snd]{\chapitre{\methode}}}
\newcommand{\teteSndCorp} {\enTete[\snd]{\chapitre{\sndCorp}}[1]}
\newcommand{\teteSndSolu} {\enTete[\snd]{\chapitre{\sndSolu}}[2]}
\newcommand{\teteSndMouv} {\enTete[\snd]{\chapitre{\sndMouv}}[3]}
\newcommand{\teteSndAtom} {\enTete[\snd]{\chapitre{\sndAtom}}[4]}
\newcommand{\teteSndMole} {\enTete[\snd]{\chapitre{\sndMole}}[6]}
\newcommand{\teteSndLumi} {\enTete[\snd]{\chapitre{\sndLumi}}[3]}
\newcommand{\teteSndTran} {\enTete[\snd]{\chapitre{\sndTran}}[7]}
\newcommand{\teteSndChim} {\enTete[\snd]{\chapitre{\sndChim}}[8]}
\newcommand{\teteSndSign} {\enTete[\snd]{\chapitre{\sndSign}}[9]}


%% Première ST2S
%%%% Chapitres
\newcommand{\premStss}{Première ST2S}
\newcommand{\premStssChim} {Sécurité chimique dans l'habitat}
\newcommand{\premStssVisi} {Propagation de la lumière et vision}
\newcommand{\premStssRedo} {Antiseptique et désinfectant, oxydoréduction}
\newcommand{\premStssLumi} {Les infrarouges et leurs applications}
\newcommand{\premStssStru} {Molécules d'intérêt biologique}
\newcommand{\premStssBiom} {Biomolécules dans l’organisme}
\newcommand{\premStssRout} {Sécurité routière}
\newcommand{\premStssAlim} {Gestion des ressources naturelles et alimentation}
\newcommand{\premStssElec} {Sécurité électrique dans l'habitat}
\newcommand{\premStssPres} {Propriétés des fluides et pression sanguine}
\newcommand{\premStssSono} {Ondes sonores et audition}

%%%% en-tête
\newcommand{\tetePremStssMeth} {\enTete[\premStss]{\chapitre{\methode}}}
\newcommand{\tetePremStssChim} {\enTete[\premStss]{\chapitre{\premStssChim}}[1]}
\newcommand{\tetePremStssVisi} {\enTete[\premStss]{\chapitre{\premStssVisi}}[2]}
\newcommand{\tetePremStssRedo} {\enTete[\premStss]{\chapitre{\premStssRedo}}[3]}
\newcommand{\tetePremStssLumi} {\enTete[\premStss]{\chapitre{\premStssLumi}}[4]}
\newcommand{\tetePremStssStru} {\enTete[\premStss]{\chapitre{\premStssStru}}[5]}
\newcommand{\tetePremStssBiom} {\enTete[\premStss]{\chapitre{\premStssBiom}}[6]}
\newcommand{\tetePremStssRout} {\enTete[\premStss]{\chapitre{\premStssRout}}[7]}
\newcommand{\tetePremStssAlim} {\enTete[\premStss]{\chapitre{\premStssAlim}}[8]}
\newcommand{\tetePremStssElec} {\enTete[\premStss]{\chapitre{\premStssElec}}[9]}
\newcommand{\tetePremStssPres} {\enTete[\premStss]{\chapitre{\premStssPres}}[10]}
\newcommand{\tetePremStssSono} {\enTete[\premStss]{\chapitre{\premStssSono}}[11]}


%% Terminale ST2S
%%%% Chapitres
\newcommand{\termStss}{Terminale ST2S}
\newcommand{\termStssOrga} {Représentation des molécules organiques}
\newcommand{\termStssAlim} {Sécurité physico-chimique dans l'alimentation}
\newcommand{\termStssImag} {La physique de l'imagerie médicale}
\newcommand{\termStssBiom} {Biomolécules et alimentation}
\newcommand{\termStssMedi} {De la molécule aux médicaments}
\newcommand{\termStssEnvi} {Sécurité chimique dans l'environnement}
\newcommand{\termStssDosa} {Analyser la composition d'un milieu}
\newcommand{\termStssRout} {Sécurité routière}
\newcommand{\termStssCosm} {L'usage responsable des cosmétiques}

%%%% en-tête
\newcommand{\teteTermStssMeth} {\enTete[\termStss]{\chapitre{\methode}}}
\newcommand{\teteTermStssOrga} {\enTete[\termStss]{\chapitre{\termStssOrga}}[1]}
\newcommand{\teteTermStssRout} {\enTete[\termStss]{\chapitre{\termStssRout}}[8]}
\newcommand{\teteTermStssAlim} {\enTete[\termStss]{\chapitre{\termStssAlim}}[2]}
\newcommand{\teteTermStssEnvi} {\enTete[\termStss]{\chapitre{\termStssEnvi}}[6]}
\newcommand{\teteTermStssImag} {\enTete[\termStss]{\chapitre{\termStssImag}}[3]}
\newcommand{\teteTermStssDosa} {\enTete[\termStss]{\chapitre{\termStssDosa}}[7]}
\newcommand{\teteTermStssBiom} {\enTete[\termStss]{\chapitre{\termStssBiom}}[4]}
\newcommand{\teteTermStssMedi} {\enTete[\termStss]{\chapitre{\termStssMedi}}[5]}
\newcommand{\teteTermStssCosm} {\enTete[\termStss]{\chapitre{\termStssCosm}}[9]}

%%%%%%%%%%%%%%%%%%%%%%%%%%%%%%%%%%%%%%%%%%%%%%%%%%%%%%%%%%%%%
%%%% Réglage de chemfig
\setchemfig{
  atom sep=24pt,
  bond style={line width=1pt},
  angle increment=30
}


%%%%%%%%%%%%%%%%%%%%%%%%%%%%%%%%%%%%%%%%%%%%%%%%%%%%%%%%%%%%%
%% Pour faire des parenthèses dans les molécules 
\def\parentheseG{\llap{$\left(\strut\right.$}}
\def\parentheseD{\rlap{$\left.\strut\right)$}}

%% Pour avoir des molécules en gras dans un texte
\newcommand{\moleculesGras}{
  \renewcommand*\printatom[1]{\ensuremath{\mathbf{##1}}}
}
\newcommand{\moleculesNormale}{
  \renewcommand*\printatom[1]{\ensuremath{\mathrm{##1}}}
}
\newcommand{\chemfigHaworth}[1]{
  \chemfig[cram width=3pt, atom sep=2.25em]{ #1 }
}

%% parties colorées
\definesubmol\cetoneCouleur{(=[3,,,,couleurQuat] \textcolor{couleurQuat}{O}) -[-1,,,,couleurQuat]}
%% ramification
\definesubmol\alkyleG{(-[-5] R_1)}
\definesubmol\alkyleD{(-[-1] R_2)}

%%%% Élément récurrent, pour faciliter la lecture
\newcommand{\hydrogene}{\chemfig{H} }
\newcommand{\carbone}{\chemfig{C} }
\newcommand{\oxygene}{\chemfig{O} }
\newcommand{\dioxygene}{\chemfig{O_2} }
\newcommand{\azote}{\chemfig{N} }
\newcommand{\eau}{\chemfig{H_2O} }
\newcommand{\oxonium}{\chemfig{H_3O^+} }
\newcommand{\hydroxyde}{\chemfig{HO^{-}} }
\newcommand{\electron}{\chemfig{e^{-}} }
\newcommand{\ionHydrogene}{\chemfig{H^{+}} }
\newcommand{\bicarbonateDeSodium}{\chemfig{NaHCO_3} }
\newcommand{\azotureDeSodium}{\chemfig{NaN_3} }

%%%% État physique
\newcommand{\aq} { \ensuremath{_\text{(aq)}} }
\newcommand{\sol}{ \ensuremath{_\text{(s)}} }
\newcommand{\liq}{ \ensuremath{_\text{(l)}} }
\newcommand{\gaz}{ \ensuremath{_\text{(g)}} }

%%%%%%%%%%%%%%%%%%%%%%%%%%%%%%%%%%%%%%%%%%%%%%%%%%%%%%%%%%%%%
%%%% Pour simplifier certaines molécules
\definesubmol\cu{ -[::60] } % Carbone vers le haut (carbon up)
\definesubmol\cd{ -[::-60] } % carbone vers le bas (carbon down)
\definesubmol\cud{ -[::60] -[::-60] } % Liaison C-C ^ (carbon up down)
\definesubmol\cdu{ -[::-60] -[::60] } % liaison C-C v (carbon down up)
\definesubmol\cis{ -[::60] =[::-60] -[::-60] } % Liaison -C=C- cis
\definesubmol\trans{ -[::60] =[::-30] -[::-30] } % Liaison -C=C- trans
%% Hydrogène saturés
\definesubmol\paireH{(-[::90] H) (-[::-90] H)}
\definesubmol\paireSatH{(-[::30] H) (-[::-30] H)}
\definesubmol\saturationH{(-[::90] H) (-[::-90] H) (-[::0] H)}
%% Quelques groupes caractéristiques
\definesubmol\teteAcide{ O-[::30] (=[::60] O) -[::-60] }
\definesubmol\teteAcideDev{ - O - C (=[::90] O) - }
\definesubmol\OH { -[::60] OH }
\definesubmol\carboxyle{ (=[::-60] O) (-[::60] OH) }
\definesubmol\carbonyle{ (=[::60] O) -[::-60] }
\definesubmol\ester{ (=[:90] O) -[:-30] O}
\definesubmol\ether{ -[:30] O -[:-30]}
\definesubmol\amide{ (=[:90] O) -[:-30] N}
%% Représentation de hamworth
\definesubmol\gluHaw{
  <[-1.5, 0.7] (-[:90, 0.6] OH)
  -[::45,,,,line width = 3.4pt] (-[:-90, 0.6] OH)
  >[::45, 0.7]
}
\definesubmol\gluLeftHaw{
  <[-4.5, 0.7] (-[:-90, 0.6] OH)
  -[::-45,,,,line width = 3.4pt] (-[:90, 0.6] OH)
  >[::-45, 0.7] (-[1.5,0.7])
}


%%%%%%%%%%%%%%%%%%%%%%%%%%%%%%%%%%%%%%%%%%%%%%%%%%%%%%%%%%%%%
%% Acides gras
\definesubmol\palmitique{
  HO -[::30] !\carbonyle !\cud !\cud !\cud !\cud !\cud !\cud !\cud
}
\definesubmol\linolenique{
  HO -[::30] !\carbonyle !\cud !\cud !\cud !\trans !\trans !\trans !\cu
}
\definesubmol\oleique{
  HO -[::30] !\carbonyle !\cud !\cud !\cud !\cis !\cu !\cud !\cud !\cud
}
\definesubmol\linoleique{
  HO -[::30] !\carbonyle !\cud !\cud !\cud !\cis !\cis !\cud !\cud
}
\definesubmol\arachidonique{
  HO -[::30] !\carbonyle !\cud !\cis !\cis !\cis !\cis !\cu !\cud !\cu
}
%% Formes semi-developpées
\definesubmol\steraiqueSemiDev{
  !\teteAcideDev C_{17}H_{35}
}
\definesubmol\caproiqueSemiDev{
  !\teteAcideDev - CH_2 - CH_2 - CH_2 - CH_2 - CH_3
}
\definesubmol\oleiqueSemiDev{
  C_{17} H_{33} -C (=[1.5]O) (-[-1.5]OH)
}
\definesubmol\oleateSemiDev{
  C_{17} H_{33} -C (=[1.5]O) (-[-1.5]O^{-})
}
%% Pour l'utilisation dans les lipides
\definesubmol\tripalmitique{
  !\cdu !\cdu !\cdu !\cdu !\cdu !\cdu !\cd
}
\definesubmol\trilinolenique{
  !\cud !\cud !\cud !\cis !\cis !\cis !\cu
}
\definesubmol\trioleique{
  !\cud !\cud !\cud !\cu =[::60] !\cu !\cud !\cud !\cd !\cd !\cd
}
\definesubmol\trilinoleique{
  !\cud !\cud !\cud !\cis !\cis !\cud !\cud
}


%%%%%%%%%%%%%%%%%%%%%%%%%%%%%%%%%%%%%%%%%%%%%%%%%%%%%%%%%%%%%
%% Lipide
\definesubmol\oleine{
   (-[::150] !\cu O-[::-60] !\carbonyle !\trioleique)
   (-[::-90] -[::-60] O!\cu !\carbonyle !\trioleique)
   -[::30] O!\cu !\carbonyle !\trioleique
}
\definesubmol\palmitine{
   (-[::150] !\cu O !\cd !\carbonyle !\tripalmitique) % haut
   (-[::-90] !\cu O !\cu (=[::-60] O) !\cu !\tripalmitique) % bas
   -[::30] O!\cu !\carbonyle !\cud !\cud !\cud !\cud !\cud !\cud !\cu % centre
}
\definesubmol\phosphatidylcholine{
  -[::-30]N
    (-[::-30])(-[::-90])
  !\cud !\cu O !\cd P
    ( =[::-30]O )( -[::-90]O )
  !\cu O !\cd !\cu
    (!\cu O !\cd !\carbonyle !\trioleique)
  !\cd !\cu O !\cd (=[::-60] O) !\cud !\cud !\cud !\cud !\cud
}
\definesubmol\oleineSemiDev{
  H C                 (!\teteAcideDev C_{17} H_{33}) 
  (-[3,1.7,2,2] H_2C  (!\teteAcideDev C_{17} H_{33}))
  -[-3,1.7,2,2] H_2 C (!\teteAcideDev C_{17} H_{33})
}
\definesubmol\caproineSemiDev{
  H_2C                (!\caproiqueSemiDev)
  -[-3,1.7,2,2] H C   (!\caproiqueSemiDev)
  -[-3,1.7,2,2] H_2 C (!\caproiqueSemiDev)
}
\definesubmol\palmitineSemiDev{
  H C (!\teteAcideDev C_{15} H_{31}) 
  (-[3,1.7,2,2] H_2C (!\teteAcideDev C_{15} H_{31}))
  -[-3,1.7,2,2] H_2 C (!\teteAcideDev C_{15} H_{31})
}

%% glycérol
\definesubmol\glycerol{
  HO -[-1] 
  -[1] (-[3] OH)
  -[-1] -[1] OH
}
\definesubmol\glycerolSemiDev{
  HC (-OH)
  (-[3,,2,2] H_2C (-OH))
  -[-3,,2,2] H_2C (-OH)
}


%%%%%%%%%%%%%%%%%%%%%%%%%%%%%%%%%%%%%%%%%%%%%%%%%%%%%%%%%%%%%
%% Stérols
\definesubmol\cholesterol{
  HO-[1] *6(-- % 1er cycle
    *6(=-- % 2eme cycle
      *6(- % 3eme cycle
        *5(--- 
          (-[::-35] (!\cu) !\cd !\cd !\cud (!\cd) !\cu) % lipide
          -  -
        ) % 4eme
        - (-[::0]) ---
      ) % 3eme
      ---
    ) % 2eme
    - (-[::0]) ---
  ) % 1er
}
\definesubmol\testosterone{
  O=[1] *6(-=
    *6(---
      *6(-
        *5(--- 
          (-[::-100] OH) % alcool
        --
        ) % 4
      - (-[::0]) ---
      ) % 3
    ---
    ) % 2
  - (-[::0]) ---
  ) % 1
}
\definesubmol\cortisol{
  O=[::30] *6(
    -= *6(
      --- *6(
        - *5(
          --- (-[::-100] OH)
          (-[::-35] (=[::60] O) !\cd!\cu OH)
          -
        ) % 4
        - (-[::0]) -- (-OH) -
      ) % 3
      --
    ) % 2
    - (-[::0]) ---
  ) % 1
}
\definesubmol\progesterone{
  O=[::30] *6(
    -=  *6(
      --- *6(
        - *5(--- (=O) -)
        - (-[::0])
        ---
      ) % 3
      --
    ) % 2
    - (-[::0]) ---
  ) % 1
}

\definesubmol\estradiol{
    HO-[::30] *6(
    -= *6(
      --- *6(
        - *5(--- (-OH) -)
        - (-[::0]) ---
      ) % 3
      --
    ) % 2
    -=-=
  ) % 1
}


%%%%%%%%%%%%%%%%%%%%%%%%%%%%%%%%%%%%%%%%%%%%%%%%%%%%%%%%%%%%%
%%%% Glucides
%% Amidon
\definesubmol\amylopec{
  -[-1]O-[1]
  -[1] (-[3] CH_2 OH)
  !\cd O !\cd
  (!\cd (!\OH) !\cd (!\OH) !\cd)
}
\newcommand{\amylopectine}{
  % partie gauche
  ... !\amylopec !\amylopec
  % cycle central
  -[-1] O -[-3,1.2] CH_2O -[-3,1.2] % on agrandit l'espace vertical
  -[-1] O !\cd (
    !\cd (-[::60] OH) !\cd (-[::60] OH) !\cd
    % cycle à gauche
    (-[::120,1.25] O -[::-60,1.2]
    *6(-O- (-[,0.5]CH_2OH) -(-[6]O-[6]...) -(-OH) -(-OH)-))
    % partie droite
    !\cd
  ) !\amylopec !\amylopec -[-1] ...
}
\definesubmol\amylopecHaw{
  % début du cycle
  O -[1,0.6]
  !\gluHaw
  (-[4.5,0.7] O
  -[6] (-[3,0.5] -[5,0.75] OH)
  -[-4.5,0.7])
  % fin du cycle
  -[-1,0.6]
}
\definesubmol\amylopecGaucheHaw{
  O -[5,0.6]
  % début du cycle, perspective
  !\gluLeftHaw  (-[-5,0.6] O -[5] ...)
  % fin du cycle
  -[1.5,0.7] (-[3,0.4] -[5,0.6] OH)
  -O -[-1.5,0.7]
}
\definesubmol\amylopecCentraleHaw{
  O -[-1,0.7] -[-3,0.6]
  % début du cycle
  - O -[-1.5,0.7] ( % perspective
    !\gluLeftHaw  (-[-5,0.6] !\amylopecGaucheHaw)
  )
  % fin du cycle
  -[-1,0.6]
}
\definesubmol\amylopectineHaw{
  ...\phantom{B}-[-1] !\amylopecHaw
  !\amylopecHaw O -[-3]
  !\amylopecCentraleHaw
  !\amylopecHaw !\amylopecHaw O -[1]..
}

%% glucose
\definesubmol\glucoseHaw{
  % début du cycle
  HO -[3,0.6,2]?
  !\gluHaw
  (-[-3,0.6] OH)
  -[4.5,0.7] O 
  % fin du cycle
  -[6]? (-[3,0.5] -[5,0.75] OH)
}
\definesubmol\glucoseCycle{
  *6 (-(-OH) -(-OH) -(-OH) -O -(- -[3]OH) -) (-[-5]HO)
}
\definesubmol\glucose{
  HO -[6] (-[-4] (-[6] HO) -[-2] -[-4] HO) -[4] (-[6] HO) -[2] (-OH) -[4] (=[6] O) -[2] H
}
\definesubmol{\ose}{ -[-3] C (-H) (-[6] HO) }
\definesubmol\glucoseSemiDev{
  HO -C (-H) (
    -[3] C (-H) (-[6] HO)
    (-[3] C (-[5]H) =[1] O)
  ) % aldehyde et alcool
  !\ose !\ose !\ose
  -[-3] H
}

%% fructose
\definesubmol\fructoseHaw{
  % début du cycle
  HO -[3,0.6,2]?
  !\gluHaw 
  (-[-3,0.5] -[-1,0.75] OH)
  -[4.5,0.7] O 
  % fin du cycle
  -[6]?
}
\definesubmol\fructoseCycle{
  *6(-(-OH) -(-OH) -(-[3]OH) (-[0]-[1]OH) -O- -(-OH))
}
\definesubmol\fructose{
  HO -[6] (-[-4] (-[6] HO) -[-2] -[-4] HO) -[4] (-[6] HO) -[2] (=O) -[4] -[2] OH
}
\definesubmol\fructoseSemiDev{
  HO -C (-H) (
    -[3] C (=O)
    (-[3] C (-[3]H) (-[6]H) -OH)
  ) % cétone et alcool
  !\ose !\ose !\ose
  -[-3] H
}


%%%%%%%%%%%%%%%%%%%%%%%%%%%%%%%%%%%%%%%%%%%%%%%%%%%%%%%%%%%%%
%% Acides alpha aminés
\definesubmol\isoleucine{
  H_2N -[1] (-[3] (-[1]) -[5] -[3]) -[-1] !\carboxyle
}
\definesubmol\leucine{
  H_2N -[1] (-[3] -[5] (-[-5]) -[3]) -[-1] !\carboxyle
}
\definesubmol\methionine{
  H_2N -[1] (-[3] -[5] -[3] S -[5]) -[-1] !\carboxyle
}
\definesubmol\valine{
  H_2N -[1] (-[3] (-[1]) -[5]) -[-1] !\carboxyle
}
\definesubmol\alanineSemiDev{
  CH_3- CH (-[-3] NH_2) - C (=[1.5] O) -[-1.5] OH
}
\definesubmol\asparagineSemiDev{
  HO -[1]C (=[3] O) -[-1]CH (-[-3] NH_2) -[1]CH_2 -[-1]C (-[-3] NH_2) =[1] O
}
\definesubmol\glycineSemiDev{
  NH_2- CH_2- C (=[1.5] O) -[-1.5] OH
}
\definesubmol\alaninePoly{
  - CH (-[-3] CH_3) - C (=[3] O) -
}
\definesubmol\glycinePoly{
  - CH_2 - C (=[3] O) -
}
\definesubmol\isoleucinePoly{
  -CH (-[-3] CH (-[-5] CH_2 -[-3]CH_3) -[-1]CH_2) -C (=[3] O) -
}
\definesubmol\valinePoly{
  -CH (-[3] CH (-[5] CH_3) -[1]CH_2) -C (=[-3] O) -
}


%%%%%%%%%%%%%%%%%%%%%%%%%%%%%%%%%%%%%%%%%%%%%%%%%%%%%%%%%%%%%
%% Hormones
\definesubmol\creatinine{
  O= *5(-N (-[-3,0.5]H) -(=NH) -N (-) --)
}
\definesubmol\DOPA{
  HO -[1] *6(= (-OH) -= (--[-1] (-[-3]NH_2) -[1] COOH) -=-)
}
\definesubmol\DOPAH{
  HO -[1] *6(= (-OH) -= (--[-1] (-[-3]NH_3^+) -[1] COOH) -=-)
}
\definesubmol\prostaglandine{
  OH-[::75] *5(
    - (
      -=[::60]!\cd  (!\cd OH) !\cud !\cud !\cu 
    )
    - (-[::-65]!\cd !\cud !\cud !\cu  (=[::60]O) !\cd OH)
    - (=O)
    --
  )
}

%%%%%%%%%%%%%%%%%%%%%%%%%%%%%%%%%%%%%%%%%%%%%%%%%%%%%%%%%%%%%
%% Produit de contraste
\definesubmol\ionChelate{
  N (-[::-45, 0.9,,, draw = none] Gd^{3+}) 
      (-[::140] !\cd COO^{-}) -[::80] -[ 0] -[::-80]
    N (-[::140] !\cd COO^{-}) -[::70] -[-3] -[::-80]
    N (-[::120] !\cd COO^{-}) -[::70] -[-6] -[::-80]
    N (-[::140] !\cd ^{-}OOC) -[::70] -[ 3] -[::-80, 0.8]
}
\definesubmol\chelateAlcool{
  N (-[::-45, 0.9,,, draw = none] Gd^{3+}) 
    (-[::140] !\cd COO^{-})         -[::80] -[ 0] -[::-80]
  N (-[::140] -[0] (-[2] OH) -[-2]) -[::70] -[-3] -[::-80]
  N (-[::120] !\cd COO^{-})         -[::70] -[-6] -[::-80]
  N (-[::140] !\cd ^{-}OOC)         -[::70] -[ 3] -[::-80, 0.8]
}


%%%%%%%%%%%%%%%%%%%%%%%%%%%%%%%%%%%%%%%%%%%%%%%%%%%%%%%%%%%%%
%% Vitamines
\definesubmol\acideAscorbique{ % Vitamine C
  HO-[-1] -[1](-[3]OH) -[-1] 
  *5(
    -(-OH) =(-OH) -(=O) -O-
  )
}
\definesubmol\cholecarciferol{ % Vitamine D
  OH-[-1]
  *6( % 1er cycle
    ---(=)- ( % ramification
      = !\cd =[::60] *6(- % 2eme
        *5(
          --- (-(-[::60]) !\cd !\cud -[::60](-[::60]) !\cd) --
        ) % 3eme
        -(-[::0])----
      ) % 2eme
    ) % ramification
    --
  ) % 1er
}
\definesubmol\cret{ =[-1] -[1] }
\definesubmol\retinol{ % Vitamine A
  *6( % cycle
    --(-)= ( % chaine
      -[1] !\cret (-[3]) !\cret !\cret (-[3]) !\cret OH
    ) % chaine
    -(-[1]) (-[5])--
  ) % cycle
}


%%%%%%%%%%%%%%%%%%%%%%%%%%%%%%%%%%%%%%%%%%%%%%%%%%%%%%%%%%%%%
%% Aspirine
\definesubmol\aspirineSemiDev{
  O=[-1] C (-[1]OH) -[-3]C % carboxyle
  *6( % cycle
    =HC -[,,2,2]HC =\chembelow{C}{H} -CH =C (
      -[1] O -[-1] C (=[-3] O) -[1]CH_3 % cétone
    )
    -
  ) % cyle
}
\definesubmol\aspirine{
  *6 (
    -=- (-O -[-1] (=[-3]) -[1]) = (- (=[5] O) -[1] OH) -=
  )
}

%% Paracétamol
\definesubmol\paracetamol{
  *6(
    (-HO)-=-(
      -NH (!\cd (=[::-60]O)!\cu)
    ) % amide
    =-=
  )
}
\definesubmol\paracetamolSemiDev{
  *6(
    C (-HO) -CH =CH -C (
      -NH (-[-1]C (=[-3]O) -CH_3)
    ) % amide
    =CH -HC =[,4,2]
  )
}
\definesubmol\paracetamolDev{
  H -O -[1]C *6(
    -C(-H) =C(-H) -C (
      -N (-[3]H) (-[-1]C (=[-3]O) (-C!\saturationH)) 
    ) % amide
    =C(-H) -C(-H) =
  )
}

%% Aspartame
\definesubmol\aspartame{
  [:150]
  *6(-=-=-=) % phenyl
  -[0]-[-2] (
    -[-4] (=[6]O) -[-2]O -[-4]
  ) % ester
  -[0]NH -[-2,,1] (=[-4] O) % Amide
  -[0] (
    -[2] NH_2
  ) % amine
  -[-2] -[0] (=[2] O) -[-2] OH % acide carboxylique
}

%%%%%%%%%%%%%%%%%%%%%%%%%%%%%%%%%%%%%%%%%%%%%%%%%%%%%%%%%%%%%
%%%% Molécules odorantes
%% géraniol
\definesubmol\geraniol{
  -[-3] *6( % tete
    --- 
      (= (!\cd) (!\cu)) % pied
    -[,,,,draw=none] -[,,,,draw=none]
      (- !\cd OH) % alcool
    =
  )
}
\definesubmol\geraniolSemiDev{
  CH_3 -[-3]C *6 ( % tete
    -H_2C -[,,2,2]H_2C -CH 
      (=C (!\cd CH_3) (!\cu CH_3)) % pied
    -[,,1,1,draw=none] -[,,,,draw=none]CH
      (-CH_2 !\cd OH) % alcool
    =CH_2 
  )
}
\definesubmol\oxyphenylone{
  HO -[::30] *6(
    -=- (
      -!\cd !\cu (=[::60]O) !\cd 
    )
    =-=
  )
}
\definesubmol\vanilline{
  O=[::-30] (!\cu H)
  !\cd  *6(
    - = - (-OH)
    = (-O!\cu)
    -=
  )
}

%%%%%%%%%%%%%%%%%%%%%%%%%%%%%%%%%%%%%%%%%%%%%%%%%%%%%%%%%%%%%
%%%% Drogues
\definesubmol\THC{
  -[::-90] *6 ( % 1er
    --- *6 ( % 2eme
      - (-[::-90]) (-[::-30])
      -O- *6 ( % 3eme
        -= (!\cd !\cud !\cud)
        -= (!\cd OH) -=
      )
      --
    )
    --=
  )
}
\definesubmol\cocaineHaw{
  ? 
    <[::60,0.7] (-[::60] N -[::60])
    -[::-60,,,,line width = 3.4pt]
    >[::-30,0.7] ( % ether-phenyl
      -[::60] O -[::-60] (=[::-60]O)
      !\cu  *6(=-=-=-)
    )
  -[::130,0.7] ( % ester
    -[::-50] (=[::60]O)
    !\cd O !\cu 
  )
  -[::80, 0.9] (-[::-30, 0.85]) -[::60, 0.7] ?
}
%%%% Clés utilisées dans le tableau périodique
\pgfkeys{% définition de la famille de clefs
  /periodique/.is family, /periodique,
  defaut/.style = {
    couleur = green-50,
    nom =,
    electronegativite = 0,
    masse = 0,
    charge = 0,
    echelle = 1
  },
  nom/.store in = \periodiqueNom,
  couleur/.store in = \periodiqueCouleur,
  symbole/.store in = \periodiqueSymbole,
  hauteur case/.store in = \periodiqueHauteur,
  largeur case/.store in = \periodiqueLargeur,
  electronegativite/.store in = \periodiqueElectroneg,
  masse/.store in = \periodiqueMasse,
  charge/.store in = \periodiqueCharge,
  echelle/.store in = \periodiqueEchelle
}

%%%% Pour afficher un tableau périodique
\NewDocumentEnvironment{tableauPeriodique}{O{} +m}{%
  \pgfkeys{/periodique, defaut, #1}
  %
  \begin{tikzpicture}[scale = \periodiqueEchelle, transform shape]
    #2
}{%
  \end{tikzpicture}
}

%%%% Pour afficher une case du tableau périodique
\NewDocumentCommand{\tkzElement}{O{} D(){}}{%
  \pgfkeys{/periodique, defaut, #1}
  %
  \couleurElectronegativite{}
  \node [
    node distance  = \periodiqueHauteur and \periodiqueLargeur,
    on grid,
    minimum width  = \periodiqueLargeur,
    minimum height = \periodiqueHauteur,
    name = \periodiqueSymbole,
    fill = \pgfkeysvalueof{/periodique/couleur},
    draw = cyan-800!50!black,
    align = center,
    #2
  ] {
    % nom de l'élément
    \IfValueT{\periodiqueNom}{
      {\scriptsize \periodiqueNom}%
      \compare {\periodiqueCharge > 0}{ \\[-1pt] }{ \\[2pt] }%
    }
    % nombre atomique
    \compareT {\periodiqueCharge > 0}{\important[black]{\periodiqueCharge}\\[2pt]}%
    % symbole atomique
    {\Large \important[black]{\periodiqueSymbole}}%
    % masse atomique
    \compareT {\periodiqueMasse > 0}{\\[-2pt]%
      {\footnotesize \num{\periodiqueMasse}}%
    }%
    % électronégativité
    \compareT {\periodiqueElectroneg > 0}{%
      \compare{\periodiqueMasse > 0}{\\}{\\[-2pt]}%
      {\footnotesize $\chi = \num{\periodiqueElectroneg}$}%
    }%
  };
}

%%%% Réglage des couleurs automatiques si l'électronégativité est réglé
\newcommand{\couleurElectronegativite}{%
  \compare {\periodiqueElectroneg > 3.5}{%T
    \pgfkeyssetvalue{/periodique/couleur}{red-400}
  }{%F
    \compare {\periodiqueElectroneg > 3.0}{% T
      \pgfkeyssetvalue{/periodique/couleur}{red-300}
    }{% F
      \compare {\periodiqueElectroneg > 2.5}{% T
        \pgfkeyssetvalue{/periodique/couleur}{red-200!95!black}
      }{% F
        \compare {\periodiqueElectroneg > 2.0}{% T
          \pgfkeyssetvalue{/periodique/couleur}{orange-200}
        }{% F
          \compare {\periodiqueElectroneg > 1.5}{% T
            \pgfkeyssetvalue{/periodique/couleur}{yellow-150}
          }{% F
            \compare {\periodiqueElectroneg > 1.0}{% T
              \pgfkeyssetvalue{/periodique/couleur}{green-100}
            }{% F
              \pgfkeyssetvalue{/periodique/couleur}{\periodiqueCouleur}
            }% pas d'électronégativité défaut
          }% 0.5 < chi < 1.0
        }% 1.0 < chi < 1.5
      }% 1.5 < chi < 2.0
    }% 2.0 < chi < 2.5
  }% 2.5 < chi < 3.0
}% 3.0 < chi < 3.5

%%%% Pour faciliter l'utilisation du tableau périodique
\NewDocumentCommand{\elementH}  {O{} D(){}} {\tkzElement[symbole = H,  charge = 1,  nom = Hydrogène, #1](#2)} % masse = 1.00, 
\NewDocumentCommand{\elementHe} {O{} D(){}} {\tkzElement[symbole = He, charge = 2,  nom = Hélium, #1](#2)}    % masse = 4.00, 
\NewDocumentCommand{\elementLi} {O{} D(){}} {\tkzElement[symbole = Li, charge = 3,  nom = Lithium, #1](#2)}   % masse = 6.94, 
\NewDocumentCommand{\elementBe} {O{} D(){}} {\tkzElement[symbole = Be, charge = 4,  nom = Béryllium, #1](#2)} % masse = 9.01, 
\NewDocumentCommand{\elementB}  {O{} D(){}} {\tkzElement[symbole = B,  charge = 5,  nom = Bore, #1](#2)}      % masse = 10.8, 
\NewDocumentCommand{\elementC}  {O{} D(){}} {\tkzElement[symbole = C,  charge = 6,  nom = Carbone, #1](#2)}   % masse = 12.0, 
\NewDocumentCommand{\elementN}  {O{} D(){}} {\tkzElement[symbole = N,  charge = 7,  nom = Azote, #1](#2)}     % masse = 14.0, 
\NewDocumentCommand{\elementO}  {O{} D(){}} {\tkzElement[symbole = O,  charge = 8,  nom = Oxygène, #1](#2)}   % masse = 16.0, 
\NewDocumentCommand{\elementF}  {O{} D(){}} {\tkzElement[symbole = F,  charge = 9,  nom = Fluor, #1](#2)}     % masse = 19.0, 
\NewDocumentCommand{\elementNe} {O{} D(){}} {\tkzElement[symbole = Ne, charge = 10, nom = Néon, #1](#2)}      % masse = 20.2, 
\NewDocumentCommand{\elementNa} {O{} D(){}} {\tkzElement[symbole = Na, charge = 11, nom = Sodium, #1](#2)}    % masse = 23.0, 
\NewDocumentCommand{\elementMg} {O{} D(){}} {\tkzElement[symbole = Mg, charge = 12, nom = Magnésium, #1](#2)} % masse = 24.3, 
\NewDocumentCommand{\elementAl} {O{} D(){}} {\tkzElement[symbole = Al, charge = 13, nom = Aluminium, #1](#2)} % masse = 27.0, 
\NewDocumentCommand{\elementSi} {O{} D(){}} {\tkzElement[symbole = Si, charge = 14, nom = Silicium, #1](#2)}  % masse = 28.1, 
\NewDocumentCommand{\elementP}  {O{} D(){}} {\tkzElement[symbole = P,  charge = 15, nom = Phosphore, #1](#2)} % masse = 31.0, 
\NewDocumentCommand{\elementS}  {O{} D(){}} {\tkzElement[symbole = S,  charge = 16, nom = Soufre, #1](#2)}    % masse = 32.1, 
\NewDocumentCommand{\elementCl} {O{} D(){}} {\tkzElement[symbole = Cl, charge = 17, nom = Chlore, #1](#2)}    % masse = 35.5, 
\NewDocumentCommand{\elementAr} {O{} D(){}} {\tkzElement[symbole = Ar, charge = 18, nom = Argon, #1](#2)}     % masse = 39.9, 
\NewDocumentCommand{\elementK}  {O{} D(){}} {\tkzElement[symbole = K,  charge = 19, nom = Potassium, #1](#2)} % masse = 39.1, 
\NewDocumentCommand{\elementCa} {O{} D(){}} {\tkzElement[symbole = Ca, charge = 20, nom = Calcium, #1](#2)}   % masse = 40.0, 


%%%% quelque couleurs
\definecolor{vertHerbe}   {RGB} {124, 179,  66}
\definecolor{vertSapin}   {RGB} {  0,  95,  17}
\definecolor{vertSombre}  {RGB} { 14,  84,  60}
%
\definecolor{cyan}        {RGB} {  0, 140, 128}
\definecolor{cyanSombre}  {RGB} {  0,  98, 116}
\definecolor{bleuPale}    {RGB} { 39,  76, 167}
%
\definecolor{jauneClair}  {RGB} {218, 173,   0}
\definecolor{jauneSombre} {RGB} {213, 145,   2}
\definecolor{orangeSombre}{RGB} {174,  82,   0}
%
\definecolor{rougeClair}  {RGB} {224,  59,  54}
\definecolor{rougeSombre} {RGB} {148,  31,   0}

%%%% Couleurs réglables
\colorlet{couleurPrim}{cyan}
\colorlet{couleurSec}{bleuPale}
\colorlet{couleurTer}{vertSombre}
\colorlet{couleurQuat}{orangeSombre}


%%%% Réglages de la taille des indentations et des sauts de paragraphes
\setlength{\parskip}{0cm}
\setlength{\parindent}{0cm}
\renewcommand{\baselinestretch}{1}
% réglage du niveau (sous-section) ou s'arrête la table des matières
\setcounter{tocdepth}{2}


%%%% Réglage de la géométrie des pages
\geometry{
  a4paper, % format
  left=1.3cm, right=1.3cm, % marge horizontale
  top=2.2cm, bottom=2.1cm % marge verticale
}


%%%% Réglage de chemfig
\setchemfig{
  atom sep=24pt,
  bond style={line width=1pt},
  angle increment=30
}


%%% Apparence (couleur) des liens
\hypersetup{
  colorlinks=true,
  linkcolor=black, % lien type table des matière
  citecolor=black, % citation
  filecolor=black, 
  urlcolor=couleurPrim!10!black % lien internet
}


%%%% Réglage de tikz (flèche et caractères)
\usetikzlibrary{babel}
\tikzset{>=latex}


%%%% Réglage des en-tête
\renewcommand{\headrulewidth}{0.4pt}
\setlength{\headheight}{22.50113pt}


%%%% Réglage de dashundergaps pour avoir des points et pas de numération
\dashundergapssetup{
  gap-numbers = false,
  gap-format = dot,
  gap-widen,
  gap-extend-percent
}


%%%% Réglage de siunit
\sisetup{
  locale = FR, % français
	 group-minimum-digits = 4, % groupage des chiffres par millier
  inter-unit-product = \ensuremath { { } \cdot { } }, % point médian entre les unités,
  detect-weight, propagate-math-font = true, reset-math-version = false % pour avoir les versions grasse des typo
}
\AtBeginDocument{\RenewCommandCopy\qty\SI} % Pour "écraser" la commande \qty du package physics