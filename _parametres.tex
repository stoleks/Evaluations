%%%% Pour avoir les accents et autre caractère français
\usepackage[french]{babel}
\usepackage[T1]{fontenc}
\usepackage[utf8]{inputenc}

%%%% Paquets utilisés
%\usepackage{ghsystem} % pictogramme de sécurité
\usepackage{subcaption} % pour les légendes des figures
\usepackage[european, straightvoltages, RPvoltages]{circuitikz} % pour dessiner des circuits électrique
\usepackage{pdfpages} % pour inclure des fichiers pdf
\usepackage{geometry} % définition des marges
\usepackage[export]{adjustbox} % alignement vertical des images
%% Paquets persos
\usepackage{biomolecules} % pour dessiner des formules chimiques
\usepackage{profSciences} % mise en page et autre


%%%% Commandes prédéfinies
%%%%%%%%%%%%%%%%%%%%%%%%%%%%%%%%%%%%%%%%%%%%%%%%%%%%%%%%%%%%%
%% grandeurs récurrentes
% Physique
\newcommand{\ISS}{\text{ISS}}
\newcommand{\Terre}{\text{Terre}}
\newcommand{\inertie}{\text{inertie}}
\newcommand{\Tfus}{T_\text{f}}
\newcommand{\Teb}{T_\text{éb}}
% Chimie
\newcommand{\solute}{\text{soluté}}
\newcommand{\solution}{\text{solution}}
\newcommand{\espece}{\text{espèce}}
\newcommand{\avogadro}{\num{6,02e23}}

%% vecteurs
\newcommand{\FBsurA}{F_{B/A}}
\newcommand{\FAsurB}{F_{A/B}}
\newcommand{\vvFAsurB}{\vv{F}_{A/B}}
\newcommand{\vvFBsurA}{\vv{F}_{B/A}}

%%%% Ce fichier sert à déclarer les titres des chapitres des différents niveaux

%% Commun
\newcommand{\methode} {Outils pratiques}

%% Seconde
%%%% Chapitre
\newcommand{\snd}{Seconde}
\newcommand{\sndCorp} {Corps purs et mélanges}
\newcommand{\sndSolu} {Solutions}
\newcommand{\sndMouv} {Mouvement et interactions}
\newcommand{\sndAtom} {Structure de l'atome}
\newcommand{\sndMole} {Des atomes à la matière}
\newcommand{\sndLumi} {Ondes lumineuses et optique}
\newcommand{\sndTran} {Transformations de la matière}
\newcommand{\sndChim} {Transformations chimiques}
\newcommand{\sndSign} {Signaux et capteurs}

%%%% en-tête correspondant
\newcommand{\teteSndAP}   {\enTete[\snd]{Accompagnement personnalisé}}
\newcommand{\teteSndMeth} {\enTete[\snd]{\chapitre{\methode}}}
\newcommand{\teteSndCorp} {\enTete[\snd]{\chapitre{\sndCorp}}[1]}
\newcommand{\teteSndSolu} {\enTete[\snd]{\chapitre{\sndSolu}}[2]}
\newcommand{\teteSndMouv} {\enTete[\snd]{\chapitre{\sndMouv}}[4]}
\newcommand{\teteSndAtom} {\enTete[\snd]{\chapitre{\sndAtom}}[5]}
\newcommand{\teteSndMole} {\enTete[\snd]{\chapitre{\sndMole}}[6]}
\newcommand{\teteSndLumi} {\enTete[\snd]{\chapitre{\sndLumi}}[3]}
\newcommand{\teteSndTran} {\enTete[\snd]{\chapitre{\sndTran}}[8]}
\newcommand{\teteSndChim} {\enTete[\snd]{\chapitre{\sndChim}}[7]}
\newcommand{\teteSndSign} {\enTete[\snd]{\chapitre{\sndSign}}[9]}


%% Première ST2S
%%%% Chapitres
\newcommand{\premStss}{Première ST2S}
\newcommand{\premStssChim} {Sécurité chimique dans l'habitat}
\newcommand{\premStssVisi} {Propagation de la lumière et vision}
\newcommand{\premStssRedo} {Antiseptique et désinfectant, oxydoréduction}
\newcommand{\premStssLumi} {Les infrarouges et leurs applications}
\newcommand{\premStssOrga} {Structure des molécules organiques}
\newcommand{\premStssBiol} {Molécules d'intérêt biologique}
\newcommand{\premStssBiom} {Biomolécules dans l’organisme}
\newcommand{\premStssRout} {Sécurité routière}
\newcommand{\premStssAlim} {Gestion des ressources naturelles et alimentation}
\newcommand{\premStssElec} {Sécurité électrique dans l'habitat}
\newcommand{\premStssPres} {Propriétés des fluides et pression sanguine}
\newcommand{\premStssSono} {Ondes sonores et audition}

%%%% en-tête
\newcommand{\tetePremStssMeth} {\enTete[\premStss]{\chapitre{\methode}}}
\newcommand{\tetePremStssChim} {\enTete[\premStss]{\chapitre{\premStssChim}}[1]}
\newcommand{\tetePremStssVisi} {\enTete[\premStss]{\chapitre{\premStssVisi}}[2]}
\newcommand{\tetePremStssRedo} {\enTete[\premStss]{\chapitre{\premStssRedo}}[3]}
\newcommand{\tetePremStssLumi} {\enTete[\premStss]{\chapitre{\premStssLumi}}[4]}
\newcommand{\tetePremStssOrga} {\enTete[\premStss]{\chapitre{\premStssOrga}}[5]}
\newcommand{\tetePremStssBiol} {\enTete[\premStss]{\chapitre{\premStssBiol}}[6]}
\newcommand{\tetePremStssRout} {\enTete[\premStss]{\chapitre{\premStssRout}}[7]}
\newcommand{\tetePremStssBiom} {\enTete[\premStss]{\chapitre{\premStssBiom}}[8]}
\newcommand{\tetePremStssAlim} {\enTete[\premStss]{\chapitre{\premStssAlim}}[9]}
\newcommand{\tetePremStssElec} {\enTete[\premStss]{\chapitre{\premStssElec}}[10]}
\newcommand{\tetePremStssPres} {\enTete[\premStss]{\chapitre{\premStssPres}}[11]}
\newcommand{\tetePremStssSono} {\enTete[\premStss]{\chapitre{\premStssSono}}[12]}


%% Terminale ST2S
%%%% Chapitres
\newcommand{\termStss}{Terminale ST2S}
\newcommand{\termStssOrga} {Représentation des molécules organiques}
\newcommand{\termStssAlim} {Sécurité physico-chimique dans l'alimentation}
\newcommand{\termStssImag} {La physique de l'imagerie médicale}
\newcommand{\termStssBiom} {Biomolécules et alimentation}
\newcommand{\termStssMedi} {De la molécule aux médicaments}
\newcommand{\termStssEnvi} {Sécurité chimique dans l'environnement}
\newcommand{\termStssDosa} {Analyser la composition d'un milieu}
\newcommand{\termStssRout} {Sécurité routière}
\newcommand{\termStssCosm} {L'usage responsable des cosmétiques}

%%%% en-tête
\newcommand{\teteTermStssMeth} {\enTete[\termStss]{\chapitre{\methode}}}
\newcommand{\teteTermStssOrga} {\enTete[\termStss]{\chapitre{\termStssOrga}}[1]}
\newcommand{\teteTermStssRout} {\enTete[\termStss]{\chapitre{\termStssRout}}[8]}
\newcommand{\teteTermStssAlim} {\enTete[\termStss]{\chapitre{\termStssAlim}}[2]}
\newcommand{\teteTermStssEnvi} {\enTete[\termStss]{\chapitre{\termStssEnvi}}[6]}
\newcommand{\teteTermStssImag} {\enTete[\termStss]{\chapitre{\termStssImag}}[3]}
\newcommand{\teteTermStssDosa} {\enTete[\termStss]{\chapitre{\termStssDosa}}[7]}
\newcommand{\teteTermStssBiom} {\enTete[\termStss]{\chapitre{\termStssBiom}}[4]}
\newcommand{\teteTermStssMedi} {\enTete[\termStss]{\chapitre{\termStssMedi}}[5]}
\newcommand{\teteTermStssCosm} {\enTete[\termStss]{\chapitre{\termStssCosm}}[9]}

\newcommand{\largeurCaseTableauPeriodique}{1.5}

%%%% Pour afficher un élément dans le tableau périodique
\NewDocumentCommand{\elementTexteCharge}{m m m o}
{
  \begin{minipage}{\largeurCaseTableauPeriodique cm}
    \begin{center}
      \IfValueTF{#4}{ \vAligne{-20pt} }{ \vAligne{-34pt} } % position du nom
      {\small #3} \\[2pt] % nom de l'élément
      {\ensuremath\footnotesize \textbf{#1}} \\[6pt] % nombre atomique
      \chemfig[atom style={scale = 1.8}]{#2} % symbole atomique
      % \element{#1}{#2} % element symbol and atomic number
      \IfValueT{#4}{
        \\ {\small \qty{#4}{\g/\mole}}
      }
    \end{center}
  \end{minipage}
}

%%%% Pour afficher un élément dans le tableau périodique
\NewDocumentCommand{\elementElectroneg}{m m}
{
  \begin{minipage}{\largeurCaseTableauPeriodique cm}
    \begin{center}
      {\Large \important[black]{#1} \\[2pt]} % symbole atomique
      {\footnotesize $\chi = $\num{#2}} % électronégativité
    \end{center}
  \end{minipage}
}


%%%% Pour afficher un tableau périodique
%% #1 : largeur ; #2 : hauteur ; #3 : élements
\NewDocumentCommand{\tableauPeriodique}{O{2.6} O{2.7} m}{
\begin{tikzpicture}[font=\sffamily, scale=0.75, transform shape]
  
%% Couleur de remplissage
  \tikzstyle{jauneClair}  = [fill=yellow!30]
  \tikzstyle{jaune}       = [fill=yellow!45]
  \tikzstyle{jauneFonce}  = [fill=yellow!60]
  \tikzstyle{rougeClair}  = [fill=red!20]
  \tikzstyle{rouge}       = [fill=red!35]
  \tikzstyle{rougeFonce}  = [fill=red!50]
  \tikzstyle{orangeClair} = [fill=orange!30]
  \tikzstyle{orange}      = [fill=orange!45]
  \tikzstyle{orangeFonce} = [fill=orange!60]
  \tikzstyle{vertClair}   = [fill=vertForet!20]
  \tikzstyle{vert}        = [fill=vertForet!35]
  \tikzstyle{vertFonce}   = [fill=vertForet!50]
  
%% Type d'élément, par famille
  \tikzstyle{Alcali} = [Element, vertFonce]
  \tikzstyle{Alcalo} = [Element, vert]
  \tikzstyle{Metaux} = [Element, vertClair]
  \tikzstyle{Metoid} = [Element, orangeClair]
  \tikzstyle{NoMeta} = [Element, orange]
  \tikzstyle{Haloge} = [Element, orangeFonce]
  \tikzstyle{GazRar} = [Element, rouge]

%% Type d'élément, par électronégativité
 \tikzstyle{elec1} = [Element, vertClair]
 \tikzstyle{elec2} = [Element, vert]
 \tikzstyle{elec3} = [Element, jaune]
 \tikzstyle{elec4} = [Element, orangeClair]
 \tikzstyle{elec5} = [Element, orange]
 \tikzstyle{elec6} = [Element, orangeFonce]
 \tikzstyle{elec7} = [Element, rouge]
 \tikzstyle{elec8} = [Element, rougeFonce]
  
%% Style des éléments
  \tikzstyle{Element} = [
    draw=black, jaune,
    minimum width  = #1 cm, % Largeur de la case
    node distance  = #1 cm, % Espace entre deux case
    minimum height = #2 cm, % Hauteur de la case
  ]

%% Période, groupe et titre
  \tikzstyle{Period} = [font={\sffamily\LARGE}, node distance=2cm]
  \tikzstyle{Groupe} = [font={\sffamily\LARGE}, minimum width=2.5cm, node distance=2cm]
  \tikzstyle{Titre}  = [font={\sffamily\Huge\bfseries}]

%% Place des éléments
  #3
\end{tikzpicture}
}


%%%% Pour faciliter l'utilisation du tableau périodique
\newcommand{\elementH} {\elementTexteCharge{1} {H} {Hydrogène}[1,00]}
\newcommand{\elementHe}{\elementTexteCharge{2} {He}{Hélium}   [4,00]}
\newcommand{\elementLi}{\elementTexteCharge{3} {Li}{Lithium}  [6,94]}
\newcommand{\elementBe}{\elementTexteCharge{4} {Be}{Béryllium}[9,01]}
\newcommand{\elementB} {\elementTexteCharge{5} {B} {Bore}     [10,8]}
\newcommand{\elementC} {\elementTexteCharge{6} {C} {Carbone}  [12,0]}
\newcommand{\elementN} {\elementTexteCharge{7} {N} {Azote}    [14,0]}
\newcommand{\elementO} {\elementTexteCharge{8} {O} {Oxygène}  [16,0]}
\newcommand{\elementF} {\elementTexteCharge{9} {F} {Fluor}    [19,0]}
\newcommand{\elementNe}{\elementTexteCharge{10}{Ne}{Néon}     [20,2]}
\newcommand{\elementNa}{\elementTexteCharge{11}{Na}{Sodium}   [23,0]}
\newcommand{\elementMg}{\elementTexteCharge{12}{Mg}{Magnésium}[24,3]}
\newcommand{\elementAl}{\elementTexteCharge{13}{Al}{Aluminium}[27,0]}
\newcommand{\elementSi}{\elementTexteCharge{14}{Si}{Silicium} [28,1]}
\newcommand{\elementP} {\elementTexteCharge{15}{P} {Phosphore}[31,0]}
\newcommand{\elementS} {\elementTexteCharge{16}{S} {Soufre}   [32,1]}
\newcommand{\elementCl}{\elementTexteCharge{17}{Cl}{Chlore}   [35,5]}
\newcommand{\elementAr}{\elementTexteCharge{18}{Ar}{Argon}    [39,9]}
\newcommand{\elementK} {\elementTexteCharge{19}{K} {Potassium}[39,1]}
\newcommand{\elementCa}{\elementTexteCharge{20}{Ca}{Calcium}  [40,0]}


%%%% Réglages de la taille des indentations et des sauts de paragraphes
\setlength{\parskip}{0cm}
\setlength{\parindent}{0cm}
\renewcommand{\baselinestretch}{1}
% réglage du niveau (sous-section) ou s'arrête la table des matières
\setcounter{tocdepth}{2}


%%%% Réglage de la géométrie des pages
\geometry{
  a4paper, % format
  left=1.3cm, right=1.3cm, % marge horizontale
  top=2.2cm, bottom=2.1cm % marge verticale
}

%%%% Réglage des en-tête
\renewcommand{\headrulewidth}{0.4pt}
\setlength{\headheight}{22.50113pt}
