%%%% Pour avoir les accents et autre caractère français
\usepackage[french]{babel}
\usepackage[T1]{fontenc}
\usepackage[utf8]{inputenc}

%%%% Paquets utilisé
\usepackage{ifthen} % pour programmer avec des boucle et des conditions
%% Images/dessin
\usepackage{subcaption} % pour les légendes des figures
\usepackage{graphicx} % pour insérer des images
\usepackage[european, straightvoltages, RPvoltages]{circuitikz} % pour dessiner des circuits électrique
\usepackage{pdfpages} % pour inclure des fichiers pdf
\usepackage{wrapfig} % pour entourer les images par du texte 
\usepackage{chemfig} % pour dessiner des formules chimiques
\usepackage{fontawesome} % pour dessiner de jolies icônes
%% Mise en page
\usepackage{geometry} % définition des marges
\usepackage{dashundergaps} % pour avoir générer des textes à compléter
\usepackage{fancyhdr} % pour faire des en-tête
\usepackage[many]{tcolorbox} % pour faire de jolie boîtes colorée
\usepackage{enumitem} % pour pouvoir définir des listes personnalisées
\usepackage{hyperref} % pour insérer des liens
\usepackage{multicol} % pour avoir plusieurs colonnes côte-à-côte
\usepackage{listings} % pour insérer du code
\usepackage{marginnote} % pour insérer des notes sur le côté
%% Tableau
\usepackage{tabularray} % pour avoir de meilleurs tableaux
%% QR code
\usepackage{qrcode} % Note : il faut que le qrcode soit sur une ligne séparé...
%% Math
\usepackage{amsmath} % symboles mathématiques
\usepackage{amssymb} % symboles mathématiques en gras
\usepackage{wasysym} % pour avoir des symbole d'intégrale
\usepackage{accents} % pour les notations mathématiques avec une barre
\usepackage{physics} % pour les dérivées, les bra, les kets, etc.
\usepackage{esvect} % pour faire de jolis vecteurs
\usepackage{siunitx} % pour avoir de jolie grandeurs avec des unités
\usetikzlibrary{calc} % pour faire des opérations dans tikz


%%%% Commandes prédéfinies
%%%% 
\NewDocumentCommand{\palette}{m m}{
  \colorlet{#1}     {#2-600}
  \colorlet{#1-950} {#2-950}
  \colorlet{#1-900} {#2-900}
  \colorlet{#1-850} {#2-850}
  \colorlet{#1-800} {#2-800}
  \colorlet{#1-700} {#2-700}
  \colorlet{#1-500} {#2-500}
  \colorlet{#1-400} {#2-400}
  \colorlet{#1-300} {#2-300}
  \colorlet{#1-200} {#2-200}
  \colorlet{#1-150} {#2-150}
  \colorlet{#1-100} {#2-100}
  \colorlet{#1-50}  {#2-50}
}


%%%% Couleurs flexoki : https://stephango.com/flexoki
\definecolor{red-50}  {RGB} {255, 225, 213}
\definecolor{red-100} {RGB} {255, 202, 187}
\definecolor{red-150} {RGB} {253, 178, 162}
\definecolor{red-200} {RGB} {248, 154, 138}
\definecolor{red-300} {RGB} {232, 112, 95}
\definecolor{red-400} {RGB} {209, 77, 65}
\definecolor{red-500} {RGB} {192, 62, 53}
\definecolor{red-600} {RGB} {175, 48, 41}
\definecolor{red-700} {RGB} {148, 40, 34}
\definecolor{red-800} {RGB} {108, 32, 28}
\definecolor{red-850} {RGB} {85, 27, 24}
\definecolor{red-900} {RGB} {62, 23, 21}
\definecolor{red-950} {RGB} {38, 19, 18}

\definecolor{orange-50}  {RGB} {255, 231, 206}
\definecolor{orange-100} {RGB} {254, 211, 175}
\definecolor{orange-150} {RGB} {252, 193, 146}
\definecolor{orange-200} {RGB} {249, 174, 119}
\definecolor{orange-300} {RGB} {236, 139, 73}
\definecolor{orange-400} {RGB} {218, 112, 44}
\definecolor{orange-500} {RGB} {203, 97, 32}
\definecolor{orange-600} {RGB} {188, 82, 21}
\definecolor{orange-700} {RGB} {157, 67, 16}
\definecolor{orange-800} {RGB} {113, 50, 13}
\definecolor{orange-850} {RGB} {89, 41, 13}
\definecolor{orange-900} {RGB} {64, 32, 13}
\definecolor{orange-950} {RGB} {39, 24, 14}

\definecolor{yellow-50}  {RGB} {250, 238, 198}
\definecolor{yellow-100} {RGB} {246, 226, 160}
\definecolor{yellow-150} {RGB} {241, 214, 126}
\definecolor{yellow-200} {RGB} {236, 203, 96}
\definecolor{yellow-300} {RGB} {223, 180, 49}
\definecolor{yellow-400} {RGB} {208, 162, 21}
\definecolor{yellow-500} {RGB} {190, 146, 7}
\definecolor{yellow-600} {RGB} {173, 131, 1}
\definecolor{yellow-700} {RGB} {142, 107, 1}
\definecolor{yellow-800} {RGB} {102, 77, 1}
\definecolor{yellow-850} {RGB} {80, 61, 2}
\definecolor{yellow-900} {RGB} {58, 45, 4}
\definecolor{yellow-950} {RGB} {36, 30, 8}

\definecolor{green-50}  {RGB} {237, 238, 207}
\definecolor{green-100} {RGB} {221, 226, 178}
\definecolor{green-150} {RGB} {205, 213, 151}
\definecolor{green-200} {RGB} {190, 201, 126}
\definecolor{green-300} {RGB} {160, 175, 84}
\definecolor{green-400} {RGB} {135, 154, 57}
\definecolor{green-500} {RGB} {118, 141, 33}
\definecolor{green-600} {RGB} {102, 128, 11}
\definecolor{green-700} {RGB} {83, 105, 7}
\definecolor{green-800} {RGB} {61, 76, 7}
\definecolor{green-850} {RGB} {49, 61, 7}
\definecolor{green-900} {RGB} {37, 45, 9}
\definecolor{green-950} {RGB} {26, 30, 12}

\definecolor{cyan-50}  {RGB} {221, 241, 228}
\definecolor{cyan-100} {RGB} {191, 232, 217}
\definecolor{cyan-150} {RGB} {162, 222, 206}
\definecolor{cyan-200} {RGB} {135, 211, 195}
\definecolor{cyan-300} {RGB} {90, 189, 172}
\definecolor{cyan-400} {RGB} {58, 169, 159}
\definecolor{cyan-500} {RGB} {47, 150, 141}
\definecolor{cyan-600} {RGB} {36, 131, 123}
\definecolor{cyan-700} {RGB} {28, 108, 102}
\definecolor{cyan-800} {RGB} {22, 79, 74}
\definecolor{cyan-850} {RGB} {20, 63, 60}
\definecolor{cyan-900} {RGB} {18, 47, 44}
\definecolor{cyan-950} {RGB} {16, 31, 29}

\definecolor{blue-50}  {RGB} {225, 236, 235}
\definecolor{blue-100} {RGB} {198, 221, 232}
\definecolor{blue-150} {RGB} {171, 207, 226}
\definecolor{blue-200} {RGB} {146, 191, 219}
\definecolor{blue-300} {RGB} {102, 160, 200}
\definecolor{blue-400} {RGB} {67, 133, 190}
\definecolor{blue-500} {RGB} {49, 113, 178}
\definecolor{blue-600} {RGB} {32, 94, 166}
\definecolor{blue-700} {RGB} {26, 79, 140}
\definecolor{blue-800} {RGB} {22, 59, 102}
\definecolor{blue-850} {RGB} {19, 48, 81}
\definecolor{blue-900} {RGB} {18, 37, 59}
\definecolor{blue-950} {RGB} {16, 26, 36}

\definecolor{purple-50}  {RGB} {240, 234, 236}
\definecolor{purple-100} {RGB} {226, 217, 233}
\definecolor{purple-150} {RGB} {211, 202, 230}
\definecolor{purple-200} {RGB} {196, 185, 224}
\definecolor{purple-300} {RGB} {166, 153, 208}
\definecolor{purple-400} {RGB} {139, 126, 200}
\definecolor{purple-500} {RGB} {115, 94, 181}
\definecolor{purple-600} {RGB} {94, 64, 157}
\definecolor{purple-700} {RGB} {79, 54, 133}
\definecolor{purple-800} {RGB} {60, 42, 98}
\definecolor{purple-850} {RGB} {49, 35, 78}
\definecolor{purple-900} {RGB} {38, 28, 57}
\definecolor{purple-950} {RGB} {26, 22, 35}

\definecolor{magenta-50}  {RGB} {254, 228, 229}
\definecolor{magenta-100} {RGB} {252, 207, 218}
\definecolor{magenta-150} {RGB} {249, 185, 207}
\definecolor{magenta-200} {RGB} {244, 164, 194}
\definecolor{magenta-300} {RGB} {228, 125, 168}
\definecolor{magenta-400} {RGB} {206, 93, 151}
\definecolor{magenta-500} {RGB} {183, 69, 131}
\definecolor{magenta-600} {RGB} {160, 47, 111}
\definecolor{magenta-700} {RGB} {135, 40, 94}
\definecolor{magenta-800} {RGB} {100, 31, 70}
\definecolor{magenta-850} {RGB} {79, 27, 57}
\definecolor{magenta-900} {RGB} {57, 23, 43}
\definecolor{magenta-950} {RGB} {36, 19, 29}
%%%%%%%%%%%%%%%%%%%%%%%%%%%%%%%%%%%%%%%%%%%%%%%%%%%%%%%%%%%%%%%%%%%%%%%%%%
%% rectangle coloré
\NewDocumentCommand{\rectangle}{O{couleurPrim} m m}{%
  \shorthandoff{;}
  \tikz \node (rect) [draw, fill, color=#1,
              minimum width=#2,
              minimum height=#3] {};
  \shorthandon{;}
}

%%%%%%%%%%%%%%%%%%%%%%%%%%%%%%%%%%%%%%%%%%%%%%%%%%%%%%%%%%%%%%%%%%%%%%%%%%%
%%
\tcbset{
  boite cassable/.style = {
    breakable, enhanced jigsaw, % pour s'étendre sur plusieurs pages
  },
  %
  couleur fond/.style = {
    colback = #1, % fond blanc
    colbacktitle = #1, % fond pour le titre blanc
  },
  %
  titre sans separation/.style = {
    couleur fond = white,
    coltitle = black, % couleur du titre
    colframe = couleurSec-800, % couleur de la boite
    boxrule = #1, arc = 0.5mm, % largeur et arrondi des traits de la boite
    titlerule = 0mm, top = 0mm, % pour ne pas avoir de séparation titre/boite
    fonttitle = \bfseries\sffamily, % titre en gras et sans serif
  },
  %
  boite pleine/.style = {
    frame hidden, sharp corners, boxrule = 0mm, % pas de contours
    colback = #1, % fond
  },
  %
  titre sans boite/.style = {
    empty, % pas de boite automatique
    fonttitle = \bfseries\sffamily, coltitle = black, % paramètre du titre
    attach boxed title to top left = {yshift=-2.5mm}, % position
    boxed title style = {empty, size = small, top = 1mm, bottom = 0.5mm},
    title = #1,
  },
}

%% une simple boite vide
\newtcolorbox{boite}[1][]{
  boite cassable, colback = white, top = 4pt,
  #1
}

%%%% Boite colorée avec 
% \begin{boiteColoree}{couleur} contenu \end{boiteColoree}
\newtcolorbox{boiteColoree}[2][]{
  boite cassable, boite pleine = #2,
  #1
}

%%%% Boite pour les documents des activités, avec le format suivant
% \begin{doc}{titre}{label} contenu \end{doc}
\newcounter{documentNum}
\newtcolorbox{doc}[3][]{
  boite cassable,
  titre sans separation = 0.5mm,
  before title = {\refstepcounter{documentNum}},
  title = {Document \arabic{documentNum} -- #2\strut \label{#3}},
  #1
}

%%%% Boîtes pour les activités et TP pour une séquence en plan de travail
% Boite pour afficher la durée recommandée en bas à gauche
\newtcbox{\dureeActivite}[1][]{
  arc = 2mm, % courbes des bords
  colback = couleurTer, colframe = white, % couleurs boite
  coltext = white, % couleur texte
}
% Boite de base pour les activités ou TP, avec le format suivant 
% \begin{boiteActivite}{titre activité}{durée}{label}{compteur}{format titre}
%   contenu
% \end{boiteActivite}
\newtcolorbox{boiteActivite}[6][]{
  boite cassable,
  titre sans separation = 0.5mm,
  before title = {\refstepcounter{#5}},
  title = {#6 \arabic{section}.\arabic{#5} -- #2\strut \label{#4}},
  enlarge bottom by = 12pt,
  overlay= {
    \node at ([xshift = -36pt, yshift = -6pt] frame.south east) {
      \dureeActivite{
        {\small \important[white]{#3}}
      }
    };
  }, % affiche la durée de l'activité dans une petite boite
  remember as = #4,
  #1
}

% Boite pour les activités/TP en plan de travail.
% \begin{activite ou TP}{titre activité}[durée]{label} ; durée = 1h acti, 2h TP par défaut
%   contenu
% \end{activite ou TP}
\NewDocumentEnvironment{activite}{m O{1 h} m}{%
  \begin{boiteActivite}{#1}{#2}{#3}
    {activiteNum}{\documentaire* \hspace{-4pt} Activité}
}{
  \end{boiteActivite}
}
% Compteur pour les TP
\newcounter{TPNum} 
\NewDocumentEnvironment{TP}{m O{2 h} m}{%
  \begin{boiteActivite}{#1}{#2}{#3}
    {TPNum}{\mesure* \hspace{-4pt} TP}
}{
  \end{boiteActivite}
}
% Pour réferencer une activité
\newcommand{\reference}[1]{\arabic{section}.\ref{#1}}

% Boîte pour la tâche finale
\newtcolorbox{tacheFinale}[1][]{
  boite cassable,
  titre sans separation = 0.5mm,
  title = {Tâche finale}, % titre
  #1
}

% Boîte pour l'organisation des séances
\newcounter{seanceNum}
\newtcbox{\seance}[2][]{
  boite cassable,
  titre sans separation = 0.5mm,
  before title = {\refstepcounter{seanceNum}},
  title = {Séance \arabic{seanceNum} \hfill (#2)}, % titre,
  halign=center, valign=center, % pour centrer le contenu
  height = 0.13\textheight, % hauteur de la boite
  #1
}
% Pour afficher 3/2/1 séances dans la programmation
\NewDocumentEnvironment{programmeSeance}{O{3} D(){34}}{
  \begin{tcbraster}[
    raster columns = #1, 
    raster width center = (\linewidth - 3cm - 5cm*(3 - #1))
  ]
}{
  \end{tcbraster}
  \vspace*{#2 pt}
}


%%%% Passage important à connaître
\newtcolorbox{importants}[1][]{
  boite cassable,
  boite pleine = couleurPrim-50, % fond
  shadow = {-4pt}{0mm}{0mm}{couleurPrim}, % barre gauche
  #1
}

%%%% Boîte pour le contexte
\newtcolorbox{contexte}[1][]{
  boite cassable, 
  titre sans separation = 3pt,
  couleur fond = couleurPrim-50!50, 
  colframe = couleurPrim, sharp corners, %  contours
  title = {Contexte :}, % titre
  detach title, before upper={\vspace{2pt}\hspace{-8pt}\tcbtitle\;}, % titre "en ligne"
  #1
}

%%%% Pour les objectifs
\newenvironment{objectifs}{ %
  \begin{listeObjectifs}
} {\end{listeObjectifs}}
%
\tcolorboxenvironment{objectifs}{
  titre sans boite = {Objectifs :},
  frame code = { % tracé de la boite
    \path [draw=couleurPrim, line width = 3pt]
    (frame.west) |- ([xshift=6mm] title.north east)
    to[out=0, in=180] ([xshift=20mm] title.east) -| % définit la courbe
    (frame.east) |- (frame.south) -| cycle; % trace la boite
  },
}

%% Pour les pré-requis
\newenvironment{prerequis}{ %
  \begin{listeObjectifs}
} {\end{listeObjectifs}}
%
\tcolorboxenvironment{prerequis}{
  titre sans boite = {Prérequis :},
  frame code = { % tracé de la boite
    \path [draw=couleurPrim, line width = 3pt]
    (frame.south east) |- ([xshift=-34mm, yshift=-4mm] frame.south east)
    to[out=180, in=0] ([xshift=-47mm] frame.south east) -| % définit la courbe
    (frame.west) |- (title.north) -| cycle; % trace la boite
  },
}

%%%% Espace pour un coup de pouce
\newcounter{coupDePouceNum}
\newtcolorbox{coupDePouce}[1][]{
  boite cassable, 
  titre sans separation = 0.5mm,
  before title = {\refstepcounter{coupDePouceNum}},
  title = {
    \textcolor{couleurPrim}{\faThumbsUp}
    Coup de pouce \arabic{coupDePouceNum} :
    \flushright \vspace*{-24pt}\faSquareO
  },
  #1
}

%%%% Espace pour une appréciation
\newcommand{\appreciation}[1]{
  \pasCorrection{
    \begin{boite}
      \vspace*{-4pt}
      \important[black]{Appréciation et remarques}
      \vspace*{#1 cm} \phantom{b}
    \end{boite}
  }
}

%%%% Espace fiche TP
\newtcolorbox{boiteMateriel}[2][]{
  titre sans separation = 0.5mm,
  coltitle = white, colbacktitle = couleurPrim, % fond pour le titre blanc
  title = {\centering \large \phantom{À} #2 \phantom{g}}, % titre
  #1
}

%%%%%%%%%%%%%%%%%%%%%%%%%%%%%%%%%%%%%%%%%%%%%%%%%%%%%%%%%%%%%%%%%%%%%%%%%%
%%%% pagination et sections
\NewDocumentCommand{\titre}{O{black} m}{
  \begin{center}
    \textcolor{#1} {\textsf{\bfseries \Large #2}}
  \end{center}
}
\newcommand{\pasDePagination}{
  \thispagestyle{empty}
}
\newcommand{\feuilleBlanche}{
  \newpage
  \phantom{b}
  \pasDePagination
}
\NewDocumentCommand{\inclusActivite}{O{1} m}{
  \numeroActivite{#1}
  \input{#2}
}
    

%%%% activité ou TP
\newcounter{activiteNum}
\newcommand{\titreActi}[2]{
  \refstepcounter{activiteNum}
  \titre{#1 \arabic{section}.\arabic{activiteNum} -- #2}
}
\NewDocumentCommand{\titreTP}{s m}{
  \IfBooleanTF{#1}{ % Version *
    \titreActi{Activité expérimentale}{#2}
  }{
    \titreActi{TP}{#2}
  }
}
\NewDocumentCommand{\titreActivite}{s m O{Activité}}{
  \IfBooleanTF{#1}{ % Version * sans le numéro du chapitre
    \refstepcounter{activiteNum}
    \titre{#3 \arabic{activiteNum} -- #2}
  }{
    \titreActi{#3}{#2}
  }
}
\NewDocumentCommand{\titreEvaluation}{o m}{
  \IfNoValueTF{#1}{
    \titre{Évaluation \arabic{section} -- #2}
  }{
    \titre{Évaluation #1 -- #2}
  }
  % reinitialisation du numéro de page et d'exercices
  \reinitialiseCompteur
}
\newcounter{exerciceNum}
\newcommand{\exercice}[1]{
  \refstepcounter{exerciceNum}
  \important[black]{\large Exercice \arabic{exerciceNum} : #1}
  % reset des numéros de questions
  \setcounter{questionNum}{0}
  \setcounter{documentNum}{0}
}


%%%% chapitre, partie, section et sous-section
\newcommand{\chapitre}[1]{ % Réglage du titre du chapitre
  Chapitre \arabic{section} -- #1
}
\newcommand{\titreChapitre}[1]{ % Affichage du titre du chapitre
  \titre{Chapitre \arabic{section} -- #1}
}
\newcommand{\titrePartie}[1]{
  \vspace*{24pt}
  \refstepcounter{subsection}
  \rectangle{40pt}{1pt}
  \important[black]{\Large \Roman{subsection} -- #1}
  \rectangle{40pt}{1pt}
  \vspace*{10pt}
}
\newcounter{sousSectionNum}
\newcommand{\titreSection}[1]{
  \vspace*{16pt}
  \refstepcounter{subsubsection}
  \setcounter{sousSectionNum}{0}
  \rectangle{30pt}{4pt}
  \important[black]{\large \arabic{subsubsection} -- #1}
  \vspace*{4pt}
}
\newcommand{\titreSousSection}[1]{
  \vspace*{12pt}
  \refstepcounter{sousSectionNum}
  \important[black]{\Alph{sousSectionNum} -- #1}
  \vspace*{4pt}
}

%%%% fixe le numéro de l'activité
\newcommand{\reinitialiseCompteur}{
    % fixe les compteurs LaTeX
  \setcounter{subsection}{0}
  \setcounter{subsubsection}{0}
  \setcounter{figure}{0}
  % fixe les compteurs internes
  \setcounter{documentNum}{0}
  \setcounter{questionNum}{0}
  \setcounter{coupDePouceNum}{0}
  \setcounter{sousSectionNum}{0}
  \setcounter{seanceNum}{0}
}
\newcommand{\numeroActivite}[1]{
  \reinitialiseCompteur
  \setcounter{activiteNum}{#1 - 1}
}
% fixe le numéro de partie (#1) et le numéro de la page (#2)
\newcommand{\numeroPartieCours}[2]{
  \newpage
  \setcounter{subsection}{#1 - 1}
  \setcounter{page}{#2}
}

%%%% lignes
\newcommand{\ligne}{
  \par\noindent\rule{\textwidth}{0.4pt}
}
\NewDocumentCommand{\lignePointillee}{o}{
  \IfValueTF{#1}{
    \makebox[#1\linewidth]{\dotfill}
  }{
    \phantom{b}\hspace*{-12pt} \dotfill
  }
}


%%%%%%%%%%%%%%%%%%%%%%%%%%%%%%%%%%%%%%%%%%%%%%%%%%%%%%%%%%%%%%%%%%%%%%%%%%
%%%% Paramètre par défaut pour l'en-tête
\newcommand{\classe}{Réglez avec \textbackslash renewcommand\{\textbackslash classe\}\{Seconde\}}
\newcommand{\etablissement}{Réglez avec \textbackslash renewcommand\{\textbackslash etablissement\}\{Lycée\}}

%%%% en-tête
\newcommand{\teteGauche}[2]{
  \lhead{
    \textbf{\footnotesize #1}
    \newline
    \footnotesize #2
  }
}
\NewDocumentCommand{\teteDroite}{m o}{
  \rhead{
    \IfValueT{#2}{
      \hfill \textbf{\footnotesize #2}
    }
    \newline 
    \hfill \footnotesize #1
  }
}
%% \enTete [compteur] {titre} [numéro de section] ; * = version simplifié sans pagination
\NewDocumentCommand{\enTete}{s o m O{0}}{
  % reset des compteurs
  \newpage
  \reinitialiseCompteur
  \setcounter{section}{#4}
  
  \IfBooleanTF{#1}{ % affichage de l'entête version fiche réussite (*)
    \pasDePagination
    \phantom{b} \vspace*{-70pt}
    \titre{#3}
  }{ % affichage de l'entête version activité
    \pagestyle{fancy}
    \teteGauche{\etablissement{}}{#3} % left header
    \chead{} % central header
    \teteDroite{#2} % right header
  }
}


%%%%%%%%%%%%%%%%%%%%%%%%%%%%%%%%%%%%%%%%%%%%%%%%%%%%%%%%%%%%%%%%%%%%%%%%%%
%%%% exercice
% définit un booléen pour entrer ou sortir du mode correction
\newboolean{modeProf}
\setboolean{modeProf}{false}
\newcommand{\modeCorrection}{
  \setboolean{modeProf}{true}
  \TeacherModeOn
}

%% Affiche le numéro d'une question avec choix du compteur et de l'espacement
\NewDocumentCommand{\numeroQuestion}{s O{questionNum} O{16}}{
  \refstepcounter{#2}
  \setcounter{sousQuestionNum}{0}
  \vspace*{2pt}
  \IfBooleanTF{#1}{}{ % cas non étoilé
    \ifnum \thequestionNum > 9
      \hspace{6 pt}
    \else
      \hspace{#3 pt}
    \fi
  }
  \textcolor{couleurSec}{
    \textbf{\arabic{#2}} {\small\faMinus}
  }
}
%% Sous question sour la forme 1.2
\NewDocumentCommand{\numeroSousQuestion}{O{16}}{
  \refstepcounter{sousQuestionNum}
  \ifnum \thesousQuestionNum > 9
    \hspace{6 pt}
  \else
    \hspace{#1 pt}
  \fi
  \textcolor{couleurSec}{
    \textbf{\arabic{questionNum}.\arabic{sousQuestionNum}.}
  }
}


%% trace des lignes pointillées pour répondre aux questions
% \lignesDeReponse* complète la ligne actuelle par des pointillées
% \lignesDeReponse commence à la ligne suivante
\newcounter{ligneNum}
\NewDocumentCommand{\lignesDeReponse}{s m}{
  % Trace la fin de la ligne, ou pas
  \IfBooleanTF{#1}{ % Version *
    \espaceReponse \dotfill\phantom{bb}
    \ifnum #2 < 1
      \newline
    \fi
  }{}
  % Trace le bon nombre de lignes
  \setcounter{ligneNum}{-1}
  \loop
    \stepcounter{ligneNum}
    \ifnum \value{ligneNum} < #2
      \\[8pt] \lignePointillee
  \repeat
  \vspace*{1pt}
}


%% définit une commande pour afficher une question 
% \qestion {question} {réponse} [nombre de lignes] ; * = pas d'alinéa
\newcounter{questionNum}
\newcounter{sousQuestionNum}
\NewDocumentCommand{\question}{s +m +m O{0}}{
  \IfBooleanTF{#1}{ % étoile
    \numeroQuestion* \!#2
  }{ % 
    \numeroQuestion \!#2
  }
  % pointille ou correction
  \ifthenelse {\boolean{modeProf}} { % prof
    \begin{boiteColoree}{couleurPrim-50}
      #3
    \end{boiteColoree}
  }{ % eleve
    \lignesDeReponse{#4}
  }
}

% Affiche le contenu en mode correction
\newcommand{\correction}[1]{
  \ifthenelse {\boolean{modeProf}} { % correction
    #1
  }{}
}

% Affiche le contenu si on est pas en mode correction
\newcommand{\pasCorrection}[1]{
  \ifthenelse{\boolean{modeProf}} {}{ % pas correction
    #1
  }
}

% Point associé à une question
\newcommand{\points}[1]{
  \marginnote{#1}
}


% sous questions
\newcommand{\sousQuestion}[2]{
  \hspace{16pt}
  \textcolor{couleurSec}{\textbullet} #1
  
  \vspace*{8pt}
  \reponse{#2}
}

%%%% qcm
\newlist{QCM}{itemize}{2}
\setlist[QCM]{label = $\square$, leftmargin = 2cm}
%% #1 : question, 
%% #2 : réponses
\NewDocumentEnvironment{qcm}{+m +m}{
  \numeroQuestion #1
  \begin{QCM}
    #2
}{
  \end{QCM}
}

% À ajouter devant la bonne réponse dans un qcm
\newcommand{\reponseQCM}{
  \correction{
    \hspace*{-21pt}
    {\large\textcolor{couleurSec}{\faCheck}}
    \hspace*{-12pt}
  } % Note : trace une croix à la bonne position
}

%%%% Pour afficher les compétences
\newcommand{\competence}[1]{
  ~{\footnotesize\textit{(#1)}}
}

%%%% Espace pour indiquer nom, prénom et classe
\newcommand{\nomPrenomClasse}{
  \pasCorrection{
    \vspace*{-24pt}
    Nom : \lignePointillee[0.3]
    Prénom : \lignePointillee[0.3]
    Classe : \dotfill
  }
}
\newcommand{\nomPrenom}{
  \pasCorrection{
    \vspace*{-24pt}
    Nom : \lignePointillee[0.3]
    Prénom : \lignePointillee[0.3]
  }
}


%%%%%%%%%%%%%%%%%%%%%%%%%%%%%%%%%%%%%%%%%%%%%%%%%%%%%%%%%%%%%%%%%%%%%%%%%%
% texte à trou avec option pour régler la largeur
\NewDocumentCommand{\texteTrou}{s o +m}{
  \ifthenelse {\boolean{modeProf}}{ % prof
    \IfBooleanTF{#1}
    {#3}
    {\important[black]{#3}}
  }{ % élève
    \IfValueTF{#2}{ % Si la largeur est réglée, on utilise des lignes
      \espaceReponse
      \lignePointillee[#2]
      \hspace*{-12pt}
    }{ % Sinon on utilise dash undergap pour la version automatique
      \espaceReponse \hspace*{0.1pt}
      \gap{#3}
    }
  }
}

% texte à trou avec option pour laisser plusieurs lignes
\NewDocumentCommand{\texteTrouLignes}{O{0} +m}{
  \ifthenelse {\boolean{modeProf}} {% prof
    \important[black]{#2}
  }{% élève
    \lignesDeReponse*{#1}
  }
}

% espace vertical pour la réponse
\newcommand{\espaceReponse}{
  \phantom{$\dfrac{1}{1}$} % espace vertical
  \hspace*{-38pt} \phantom{b} % ajuste l'espace horizontal
}


%%%%%%%%%%%%%%%%%%%%%%%%%%%%%%%%%%%%%%%%%%%%%%%%%%%%%%%%%%%%%%%%%%%%%%%%%%
%%%% Pour choisir parmi deux sujets
\newboolean{sujetA}
\setboolean{sujetA}{true}
\newcommand{\sujetB}{
  \setboolean{sujetA}{false}
}
\newcommand{\sujetA}{
  \setboolean{sujetA}{true}
}

%%%% Pour faire plusieurs sujets en parallèle
\newcommand{\variationSujet}[2]{
  \hspace*{-6pt}
  \ifthenelse{\boolean{sujetA}}{#1}{#2}
  \hspace*{-6pt}
}


%%%%%%%%%%%%%%%%%%%%%%%%%%%%%%%%%%%%%%%%%%%%%%%%%%%%%%%%%%%%%%%%%%%%%%%%%%
%%%% Tableau générique avec la première ligne bleue
\NewDocumentEnvironment{tableau}{m}{
  \begin{center}
    \begin{tblr}{
      hlines,
      colspec = #1,
      row{1} = {couleurSec-100},
    }
}{
    \end{tblr}
  \end{center}
}

%%%% Tableau de competence
\newenvironment{tableauCompetences}{
  \begin{center}
    \begin{tblr}{
      colspec = {c X[l] c c c c},
      rows = {m}, hlines, vlines,
      row{1} = {c, couleurSec-100, font = \bfseries}
    }
      Comp. & Items & D & C & B & A \\
}{
    \end{tblr}
  \end{center}
}

%%%% Tableau de connaissances pour les fiches de révisions
\newenvironment{tableauConnaissances}{
  \begin{center}
  \begin{tblr}{
    colspec = {Q[t, wd=0.7\textwidth] c c c},
    rows = {m}, hlines, vlines,
    column{4} = {0.2},
    row{1} = {couleurSec-100, c}
  }
    \important{Connaissances et capacités exigibles} & \ok & \pasOk & \important{En classe} \\
}{ 
  \end{tblr}
  \end{center}
}

%%%% Tableau de mémorisation pour les fiches de mémorisation
% Question | | | Réponse
\newenvironment{tableauMemorisation}{
  \begin{center}
 \begin{tblr}{
    colspec = {
      Q[l, wd=0.3\textwidth] % Question
      X[c] X[c] % Auto-évaluation
      Q[l, wd=0.3\textwidth] % Réponse
      Q[c, wd=0.045\textwidth] Q[c, wd=0.045\textwidth] Q[c, wd=0.045\textwidth] % Répétitions
    }, rows = {m}, hlines, vlines,
    row{1,2} = {couleurSec-100, c}
  }
    \SetCell[r=2]{c} \important{Questions} & 
    \SetCell[c=2]{c} \important{Auto-évaluation} & &
    \SetCell[r=2]{c} \important{Réponses} & 
    \SetCell[r=2]{c} \important{J-7} &
    \SetCell[r=2]{c} \important{J-30} &
    \SetCell[r=2]{c} \important{J-180} \\
    %
    & % questions
    \ok & \pasOk & % auto evaluation
    & % réponses 
    & & \\ % répétitions
    %
}{ 
  \end{tblr}
  \end{center}
}


%%%% Alignement vertical dans un tableau
\newcommand{\vAligne}[1]{
  \strut \\ \vspace*{#1}
}


%%%%%%%%%%%%%%%%%%%%%%%%%%%%%%%%%%%%%%%%%%%%%%%%%%%%%%%%%%%%%%%%%%%%%%%%%%
%%%% symboles : chevron, flèche, attention, etc.
\NewDocumentCommand{\chevron}{O{couleurPrim}}{
  \textcolor{#1}{\small \faChevronRight}
}
%
\NewDocumentCommand{\fleche}{O{couleurPrim}}{
  \textcolor{#1}{\faCaretRight}
}
%
\NewDocumentCommand{\attention}{O{couleurPrim}}{
  \textcolor{#1}{\faExclamationTriangle}
}
%
\NewDocumentCommand{\flecheLongue}{O{couleurPrim}}{
  \textcolor{#1}{\faLongArrowRight}
}
%
\NewDocumentCommand{\ok}{O{couleurPrim}}{
  \textcolor{#1}{\faCheckCircle}
}
%
\NewDocumentCommand{\pasOk}{O{couleurPrim}}{
  \textcolor{#1}{\faTimesCircle}
}
%
\NewDocumentCommand{\pointCyan}{O{couleurPrim}}{
  \textcolor{#1}{\textbullet}
}

%% Commande générale pour colorer un texte avec une couleur par défaut
% L'intérêt de cette commande est d'éviter des répétion de O{couleurSec}
\NewDocumentCommand{\questionSpeciale}{O{couleurSec} m}{
  \textcolor{#1}{#2}
}
% Quelques questions spéciales avec couleur réglable et un espace horizontal
\NewDocumentCommand{\mesure}{s o}{
  \IfBooleanTF{#1}{}{ \hspace{7pt} }
  \questionSpeciale[#2]{\faFlask\hspace{1pt} \faWrench\!}
}
%
\NewDocumentCommand{\telechargement}{s o}{
  \IfBooleanTF{#1}{}{ \hspace{7pt} }
  \questionSpeciale[#2]{\faDownload\, \faMobile\;}
}
%
\NewDocumentCommand{\documentaire}{s o}{
  \IfBooleanTF{#1}{}{ \hspace{7pt} }
  \questionSpeciale[#2]{\faFileText\hspace{1pt} \faBook}
}

%% pictogramme de sécurité \picto {largeur} {pictogramme}
\newcommand{\picto}[2]{
  \image{#1}{images/pictogrammes/picto_#2}
}
% Nombre dans un cercle A REVOIR
\newcommand*\nombreCercle[1]{
  % \tikz[baseline=(char.base)]{
  %   \node [shape=circle, draw filled, inner sep=1.2pt, color=couleurSec-50] (char) {\textcolor{black}{#1};
  % }
  \important[couleurSec]{#1}
}
% Pour légender une image
\newcommand{\legende}[1]{
  \vspace*{4pt}
  \textcolor{couleurPrim}{\faArrowUp} \; #1
}

%%%%%%%%%%%%%%%%%%%%%%%%%%%%%%%%%%%%%%%%%%%%%%%%%%%%%%%%%%%%%%%%%%%%%%%%%%
%%%% points importants
\NewDocumentCommand{\important}{O{couleurSec-800} +m}{
  \!\textcolor{#1}{\textsf{\bfseries #2}}\!\!
}
%%%% Pour donner des exemples. 
% \exemple est au singulier, \exemple* est au pluriel.
\NewDocumentCommand{\exemple}{s}{
  \fleche
  \IfBooleanTF{#1}{
    \textit{Exemples :}
  }{
    \textit{Exemple :}
  }
}

%%%% Citation (#1) avec la source (#2)
\newcommand{\extrait}[2]{
  « #1 »
  
  \vspace*{-12pt}
  \begin{flushright}
    \textit{#2}
  \end{flushright}
  \vspace*{-12pt}
}

%%%% image avec la largeur réglée par rapport à celle de la ligne
\newcommand{\image}[2]{
  \includegraphics[width=#1\linewidth]{#2}
}

%%%% qr code en insert sur la droite
\NewDocumentCommand{\qrcodeCote}{o +m D(){-16pt} O{1.5cm}}{
  \IfNoValueTF{#1} {
    \begin{wrapfigure}{r}{0.1\linewidth}
      \vspace*{#3}
      \qrcode[height = #4]{#2}
    \end{wrapfigure}
  }{
    \begin{wrapfigure}[#1]{r}{0.1\linewidth}
      \vspace*{#3}
      \qrcode[height = #4]{#2}
    \end{wrapfigure}
  }
}


%%%%%%%%%%%%%%%%%%%%%%%%%%%%%%%%%%%%%%%%%%%%%%%%%%%%%%%%%%%%%%%%%%%%%%%%%%
%%%% protocole, avec plusieurs colonnes possibles
\NewDocumentEnvironment{protocole}{o +m}{
  \IfValueTF{#1}{
    \vspace*{-8pt}
    \begin{multicols}{#1}
  }{}
  \begin{itemize}[label = {\footnotesize \fleche[couleurSec]}]
    #2
}{
  \end{itemize}
  \IfValueTF{#1}{ \end{multicols} }{}
}

%%%% liste de points, avec plusieurs colonnes possibles
\NewDocumentEnvironment{listePoints}{o +m}{
  \IfValueTF{#1}{
    \vspace*{-8pt}
    \begin{multicols}{#1}
  }{}
  \begin{itemize}[label = \pointCyan]
    #2
}{
  \end{itemize}
  \IfValueTF{#1}{ \end{multicols} }{}
}

%%%% jeu de données, avec plusieurs colonnes possibles
\NewDocumentEnvironment{donnees}{o +m}{  
  \important{Données :}
  \IfValueTF{#1}{
    \vspace*{-8pt}
    \begin{multicols}{#1}
  }{}
  \begin{listeTirets}
    #2
}{
  \end{listeTirets}
  \IfValueTF{#1}{ \end{multicols} }{}
}

%%%% liste d'objectif
\newlist{listeObjectifs}{itemize}{2}
\setlist[listeObjectifs]{leftmargin=6pt, label=\fleche}

%%%% liste tirets
\newlist{listeTirets}{itemize}{2}
\setlist[listeTirets]{label = \textcolor{couleurPrim}{\small\faMinus}}

%%%% liste avec des flèches
\newlist{listeFleche}{itemize}{2}
\setlist[listeFleche]{label = \textbf{\flecheLongue}}

%%%% liste avec chiffre
\newlist{enumeration}{enumerate}{2}
\setlist[enumeration]{label = \textcolor{couleurSec}{\textbf{\arabic*.}} }

%%%% problématique 
\newcommand{\problematique}[1]{
  \hspace*{-12pt}
  \flecheLongue[couleurTer]
  \hspace*{-10pt}
  \important[couleurTer]{#1}
}


%%%%%%%%%%%%%%%%%%%%%%%%%%%%%%%%%%%%%%%%%%%%%%%%%%%%%%%%%%%%%%%%%%%%%%%%%%
%%%% Séparation de la page en trois blocs
\NewDocumentCommand{\separationTroisBlocs}{+m O{0.3} +m O{0.3} +m O{0.3}}{
  \begin{minipage}[T]{#2\linewidth}
    #1
  \end{minipage}
  ~
  \begin{minipage}[T]{#4\linewidth}
    #3
  \end{minipage}
  ~
  \begin{minipage}[T]{#6\linewidth}
    #5
  \end{minipage}
}
%%%% Separation en deux blocs
\NewDocumentCommand{\separationBlocs}{+m O{0.48} +m O{0.48}}{
  \begin{minipage}[T]{#2\linewidth}
    #1
  \end{minipage}
  \hfill
  \begin{minipage}[T]{#4\linewidth}
    #3
  \end{minipage}
}


%%%%%%%%%%%%%%%%%%%%%%%%%%%%%%%%%%%%%%%%%%%%%%%%%%%%%%%%%%%%%%%%%%%%%%%%%%
%% nombre algébrique et réaction chimique
\newcommand{\algebrique}[1]{
  \ensuremath{\overline{\mathrm{#1}}}
}
\newcommand{\reaction}{
  \;\text{\faLongArrowRight}\; % flèche courte et jolie
}

%% Pour simplifier l'écriture des formules brutes
% #1 : # carbone, #2 : # hydrogène, #3 : oxygène
\newcommand{\bruteCHO}[3]{
  \chemfig{C_{#1} H_{#2} O_{#3}}
}

%% pour les masse molaire et atomique (* -> en indice)
\NewDocumentCommand{\masseMol}{s m}{
  \IfBooleanTF{#1}{
    \ensuremath{M_{\chemfig{#2}}}
  }{
    \ensuremath{M(\chemfig{#2})}
  }
}
\NewDocumentCommand{\masseAtom}{s m}{
  \IfBooleanTF{#1}{
    \ensuremath{m_{\chemfig{#2}}}
  }{
    \ensuremath{m(\chemfig{#2})}
  }
}


%% Unités pour siunit
\DeclareSIUnit{\dioptre}{\text{$\delta$}}
\DeclareSIUnit{\dornic}{\text{\textdegree D}}
\DeclareSIUnit{\ppm}{\text{ppm}}
\DeclareSIUnit{\COeq}{\text{kgCO$_{2}$e}}
\DeclareSIUnit{\jour}{\text{jour}}
% \DeclareSIUnit{}{\text{}}


%%%% atome ou isotope 
%#1: Z, #2: A, #3: X
\makeatletter
\newcommand{\isotope}[3]{%
   \settowidth\@tempdimb{\ensuremath{\scriptstyle#1}}%
   \settowidth\@tempdimc{\ensuremath{\scriptstyle#2}}%
   \ifnum\@tempdimb>\@tempdimc%
       \setlength{\@tempdima}{\@tempdimb}%
   \else%
       \setlength{\@tempdima}{\@tempdimc}%
   \fi%
  \begingroup%
  \ensuremath{
    ^{\makebox[\@tempdima][r]{\ensuremath{\scriptstyle#1}}}
    _{\makebox[\@tempdima][r]{\ensuremath{\scriptstyle#2}}}
    \chemfig{#3}
  }%
  \endgroup%
}%
\makeatother

%% element chimique dans le tableau périodique
\makeatletter
\newcommand{\element}[2]{%
   \settowidth\@tempdimb{\ensuremath{\footnotesize #1}}%
  \begingroup%
  \ensuremath{
    _{\makebox[\@tempdimb][r]{\ensuremath{\small #1}}} 
    \chemfig[atom style={scale=1.3}]{#2}
  }%
  \endgroup%
}%
\makeatother

%% siècle
\newcommand{\siecle}[1]{
  \textsc{\romannumeral #1}\textsuperscript{e}~siècle%
}

%% texte avec une boite autour
\NewDocumentCommand{\texteEncadre}{m O{black}}{
  \textcolor{#2}{
    \frame{
      \vphantom{$\dfrac{1}{1}$} \textcolor{black}{\text{#1}}
    }
  }
}

%% case cochée
\newcommand{\caseCochee}{
  $\text{\rlap{$\checkmark$}}\square$
}


%%%%%%%%%%%%%%%%%%%%%%%%%%%%%%%%%%%%%%%%%%%%%%%%%%%%%%%%%%%%%%%%%%%%%%%%%%
%%%% Style python
\lstdefinestyle{codePython}{
  commentstyle=\color{magenta-500},
  keywordstyle=\color{green-500},
  stringstyle =\color{purple-500},
  numberstyle =\tiny\color{black!50},
  basicstyle  =\ttfamily\footnotesize,
  breakatwhitespace=false, breaklines=true, keepspaces=true,
  showspaces=false, showstringspaces=false, showtabs=false, tabsize=2,
  captionpos=b, numbers=left, numbersep=5pt,
  extendedchars=true,
  literate={é}{{\'e}}1 {è}{{\`e}}1 {à}{{\'a}}1 {°}{{\textdegree}}1 {²}{{$^2$}}1, 
}
\def\inline{\lstinline[style=codePython,language=python]}


%%%%%%%%%%%%%%%%%%%%%%%%%%%%%%%%%%%%%%%%%%%%%%%%%%%%%%%%%%%%%%%%%%%%%%%%%%
%%%% circuit tikz
\NewDocumentCommand{\fixedvlen}{O{0.5cm} m m O{}}{% [semilength]{node}{label}[extra options]
  % get the center of the standard arrow
  \coordinate (#2-Vcenter) at ($(#2-Vfrom)!0.5!(#2-Vto)$);
  % draw an arrow of a fixed size around that center and on the same line
  \draw[-Triangle, #4] ($(#2-Vcenter)!#1!(#2-Vfrom)$) -- ($(#2-Vcenter)!#1!(#2-Vto)$);
  % position the label as in the normal voltages
  \node[anchor=\ctikzgetanchor{#2}{Vlab}, #4] at (#2-Vlab) {#3};
}
%%%%%%%%%%%%%%%%%%%%%%%%%%%%%%%%%%%%%%%%%%%%%%%%%%%%%%%%%%%%%
%% grandeurs récurrentes
% Physique
\newcommand{\ISS}{\text{ISS}}
\newcommand{\Terre}{\text{Terre}}
\newcommand{\inertie}{\text{inertie}}
\newcommand{\Tfus}{T_\text{f}}
\newcommand{\Teb}{T_\text{éb}}
% Chimie
\newcommand{\solute}{\text{soluté}}
\newcommand{\solution}{\text{solution}}
\newcommand{\espece}{\text{espèce}}
\newcommand{\avogadro}{\num{6,02e23}}

%% vecteurs
\newcommand{\FBsurA}{F_{B/A}}
\newcommand{\FAsurB}{F_{A/B}}
\newcommand{\vvFAsurB}{\vv{F}_{A/B}}
\newcommand{\vvFBsurA}{\vv{F}_{B/A}}

%%%%%%%%%%%%%%%%%%%%%%%%%%%%%%%%%%%%%%%%%%%%%%%%%%%%%%%%%%%%%
%%%% figures simples
\newcommand{\tkzRect}[4]{
  \fill[color=#1] (#2,#4) -- (-#2,#4) -- (-#2,#3) -- (#2,#3);
}
\newcommand{\tkzEllipse}[4]{
  \fill[color=#1] (0,#3) ellipse (#2 and #4);
}

% \tkzCercle {x}{y} {couleur} {rayon}
\newcommand{\tkzCercle}[4]{
  \filldraw [#3] (#1, #2) circle (#4pt);
}
% \tkzCercleLigne {x}{y} {couleurFond}{couleurTrait} {rayon}
\newcommand{\tkzCercleLigne}[5]{
  \filldraw [color = #4, fill = #3, very thick] (#1, #2) circle (#5pt);
}

%%%% tube à essais
\newcommand{\tkzTubeEssais}[3]{
  \draw[thick] (#1,#2) -- (#1,0) arc (0:-180:#1) -- (-#1,#2);
  \draw[thick] (0,#2) ellipse (#1 and #3);
}
\newcommand{\tkzBasTubeEssais}[5]{
  \fill[color=#1] (-#2,#3) -- (#2,#3) arc (0:-180:#2);
  \tkzRect{#1}{#2}{#3 - 0.01}{#4}
  \tkzEllipse{#1!85!black}{#2}{#4}{#5}
}
\newcommand{\tkzPhaseTubeEssais}[5]{
  \tkzRect{#1}{#2}{#3}{#4}
  \tkzEllipse{#1}{#2}{#3}{#5}
  \tkzEllipse{#1!85!black}{#2}{#4}{#5}
}

%%%% Point et vecteurs
\newcommand{\tkzLabel}[3]{
  \node at (#1, #2) {#3};
}
\newcommand{\tkzPointLabel}[3]{
  \filldraw (#1, #2) circle (2pt) node[above] {#3};
}
% \tkzVecteur [couleur] (x) [longueur x] (y) [longueur y] {legende} [position legende] 
% ajouter une * à la fin transforme la flèche en double flèche <->
\NewDocumentCommand{\tkzVecteur}{O{black} r() O{0} r() O{0} m O{right} s}{
  \IfBooleanTF{#8}{
    \draw[#1, <->, very thick] (#2, #4) -- (#2 + #3, #4 + #5) node[#7] {#6};
  }{
    \draw[#1, ->, very thick] (#2, #4) -- (#2 + #3, #4 + #5) node[#7] {#6};
  }
}
% \tkzLegende (x) (y) [longueur fleche] {légende} 
% ajouter une * passe de la version gauche -> à la version droite <-
\NewDocumentCommand{\tkzLegende}{O{black} r() r() O{1.25} m s}{
  \IfBooleanTF{#6}{
    \draw[#1, ->, very thick] (#2 + #4, #3) node[right] {#5} -- (#2, #3);
  }{
    \draw[#1, ->, very thick] (#2, #3) node[left] {#5} -- (#2 + #4, #3);
  }
}


\newcommand{\barrePourcentage}[1]{%
  \begin{tikzpicture}
    \fill[color=couleurSec]    (0.0,    0.0) rectangle (#1*8ex, 1.5ex);
    \fill[color=couleurSec!20] (#1*8ex, 0.0) rectangle (8.0ex,  1.5ex);
  \end{tikzpicture}
}


%%%%%%%%%%%%%%%%%%%%%%%%%%%%%%%%%%%%%%%%%%%%%%%%%%%%%%%%%%%%%
%%%% plan de classe
\NewDocumentCommand{\texteCadre}{O{black} r() O{2} r() O{2} m}{
  \filldraw [fill=white, draw=#1, ultra thick] (#2, #4) rectangle (#2 + #3, #4 + #5);
  \node at (#2 + #3/2, #4 + #5/2) [font=\sffamily] {\textbf{#6}};
}

%% place dans la classe
\NewDocumentCommand{\place}{r() r() m}{
  \texteCadre(#1)[3](#2)[2] {#3}
}
\NewDocumentCommand{\places}{r()r() r[] d[] d[] d[]}{
  \place(#1)(#2) {#3}
  \IfValueT{#4}{ \place(#1 + 1*3)(#2) {#4} }
  \IfValueT{#5}{ \place(#1 + 2*3)(#2) {#5} }
  \IfValueT{#6}{ \place(#1 + 3*3)(#2) {#6} }
}

%% rangée de classe ou de TP
\NewDocumentCommand{\rangee}{m r()r() r()r()r()d() r()r()}{
  \places(0)(0 - 3*#1) [#2][#3]
  \IfValueTF{#7}{
    \places(7) (0 - 3*#1) [#4][#5][#6][#7]
    \places(20)(0 - 3*#1) [#8][#9]
  }{
    \places(8.5)(0 - 3*#1) [#4][#5][#6]
    \places(20) (0 - 3*#1) [#8][#9]
  }
}

\NewDocumentCommand{\rangeeTP}{m r[]r[]r[] r()r()r()d()}{
  \places(3)(0 - 3*#1) [#2][#3][#4]
  \IfValueTF{#8}{
    \places(14) (0 - 3*#1) [#5][#6][#7][#8]
  }{
    \places(14) (0 - 3*#1) [#5][#6][#7]
  }
}


%%%% tube à essai de sang
\newcommand{\tubeEssaisSolution}[1]{
  \begin{tikzpicture}
    \tkzBasTubeEssais{#1}{0.25}{0}{0.75}{0.1} % contenu du tube
    \tkzTubeEssais{0.25}{1.5}{0.1} % tube
  \end{tikzpicture}
}

\newcommand{\tubeEssaisSangCentrifuge}[3]{
  \begin{tikzpicture}
    % phases dans le tube à essai
    \tkzBasTubeEssais{rougeSombre!75!white} {0.35}{0}{#1}{0.1}
    \tkzPhaseTubeEssais{gray!10!white}      {0.35}{#1}{#2}{0.1}
    \tkzPhaseTubeEssais{jauneClair!75!white}{0.35}{#2}{#3}{0.1}
    \tkzTubeEssais{0.35}{#3 + 1}{0.1}
    % Légende
    \tkzLegende(0.4)(#3 - 0.1) [1]{Plasma}*
    \tkzLegende(0.4)(#2 - 0.08)[1]{Globules blancs}*
    \tkzLegende(0.4)(-0.1)     [1]{Globules rouges}*
  \end{tikzpicture}
}
%%%% Ce fichier sert à déclarer les titres des chapitres des différents niveaux

%% Commun
\newcommand{\methode} {Outils pratiques}

%% Seconde
%%%% Chapitre
\newcommand{\snd}{Seconde}
\newcommand{\sndCorp} {Corps purs et mélanges}
\newcommand{\sndSolu} {Solutions}
\newcommand{\sndMouv} {Mouvement et interactions}
\newcommand{\sndAtom} {Structure de l'atome}
\newcommand{\sndMole} {Des atomes à la matière}
\newcommand{\sndLumi} {Ondes lumineuses et optique}
\newcommand{\sndTran} {Transformations de la matière}
\newcommand{\sndChim} {Transformations chimiques}
\newcommand{\sndSign} {Signaux et capteurs}

%%%% en-tête correspondant
\newcommand{\teteSndAP}   {\enTete[\snd]{Accompagnement personnalisé}}
\newcommand{\teteSndMeth} {\enTete[\snd]{\chapitre{\methode}}}
\newcommand{\teteSndCorp} {\enTete[\snd]{\chapitre{\sndCorp}}[1]}
\newcommand{\teteSndSolu} {\enTete[\snd]{\chapitre{\sndSolu}}[2]}
\newcommand{\teteSndMouv} {\enTete[\snd]{\chapitre{\sndMouv}}[4]}
\newcommand{\teteSndAtom} {\enTete[\snd]{\chapitre{\sndAtom}}[5]}
\newcommand{\teteSndMole} {\enTete[\snd]{\chapitre{\sndMole}}[6]}
\newcommand{\teteSndLumi} {\enTete[\snd]{\chapitre{\sndLumi}}[3]}
\newcommand{\teteSndTran} {\enTete[\snd]{\chapitre{\sndTran}}[8]}
\newcommand{\teteSndChim} {\enTete[\snd]{\chapitre{\sndChim}}[7]}
\newcommand{\teteSndSign} {\enTete[\snd]{\chapitre{\sndSign}}[9]}


%% Première ST2S
%%%% Chapitres
\newcommand{\premStss}{Première ST2S}
\newcommand{\premStssChim} {Sécurité chimique dans l'habitat}
\newcommand{\premStssVisi} {Propagation de la lumière et vision}
\newcommand{\premStssRedo} {Antiseptique et désinfectant, oxydoréduction}
\newcommand{\premStssLumi} {Les infrarouges et leurs applications}
\newcommand{\premStssOrga} {Structure des molécules organiques}
\newcommand{\premStssBiol} {Molécules d'intérêt biologique}
\newcommand{\premStssBiom} {Biomolécules dans l’organisme}
\newcommand{\premStssRout} {Sécurité routière}
\newcommand{\premStssAlim} {Gestion des ressources naturelles et alimentation}
\newcommand{\premStssElec} {Sécurité électrique dans l'habitat}
\newcommand{\premStssPres} {Propriétés des fluides et pression sanguine}
\newcommand{\premStssSono} {Ondes sonores et audition}

%%%% en-tête
\newcommand{\tetePremStssMeth} {\enTete[\premStss]{\chapitre{\methode}}}
\newcommand{\tetePremStssChim} {\enTete[\premStss]{\chapitre{\premStssChim}}[1]}
\newcommand{\tetePremStssVisi} {\enTete[\premStss]{\chapitre{\premStssVisi}}[2]}
\newcommand{\tetePremStssRedo} {\enTete[\premStss]{\chapitre{\premStssRedo}}[3]}
\newcommand{\tetePremStssLumi} {\enTete[\premStss]{\chapitre{\premStssLumi}}[4]}
\newcommand{\tetePremStssOrga} {\enTete[\premStss]{\chapitre{\premStssOrga}}[5]}
\newcommand{\tetePremStssBiol} {\enTete[\premStss]{\chapitre{\premStssBiol}}[6]}
\newcommand{\tetePremStssRout} {\enTete[\premStss]{\chapitre{\premStssRout}}[7]}
\newcommand{\tetePremStssBiom} {\enTete[\premStss]{\chapitre{\premStssBiom}}[8]}
\newcommand{\tetePremStssAlim} {\enTete[\premStss]{\chapitre{\premStssAlim}}[9]}
\newcommand{\tetePremStssElec} {\enTete[\premStss]{\chapitre{\premStssElec}}[10]}
\newcommand{\tetePremStssPres} {\enTete[\premStss]{\chapitre{\premStssPres}}[11]}
\newcommand{\tetePremStssSono} {\enTete[\premStss]{\chapitre{\premStssSono}}[12]}


%% Terminale ST2S
%%%% Chapitres
\newcommand{\termStss}{Terminale ST2S}
\newcommand{\termStssOrga} {Représentation des molécules organiques}
\newcommand{\termStssAlim} {Sécurité physico-chimique dans l'alimentation}
\newcommand{\termStssImag} {La physique de l'imagerie médicale}
\newcommand{\termStssBiom} {Biomolécules et alimentation}
\newcommand{\termStssMedi} {De la molécule aux médicaments}
\newcommand{\termStssEnvi} {Sécurité chimique dans l'environnement}
\newcommand{\termStssDosa} {Analyser la composition d'un milieu}
\newcommand{\termStssRout} {Sécurité routière}
\newcommand{\termStssCosm} {L'usage responsable des cosmétiques}

%%%% en-tête
\newcommand{\teteTermStssMeth} {\enTete[\termStss]{\chapitre{\methode}}}
\newcommand{\teteTermStssOrga} {\enTete[\termStss]{\chapitre{\termStssOrga}}[1]}
\newcommand{\teteTermStssRout} {\enTete[\termStss]{\chapitre{\termStssRout}}[8]}
\newcommand{\teteTermStssAlim} {\enTete[\termStss]{\chapitre{\termStssAlim}}[2]}
\newcommand{\teteTermStssEnvi} {\enTete[\termStss]{\chapitre{\termStssEnvi}}[6]}
\newcommand{\teteTermStssImag} {\enTete[\termStss]{\chapitre{\termStssImag}}[3]}
\newcommand{\teteTermStssDosa} {\enTete[\termStss]{\chapitre{\termStssDosa}}[7]}
\newcommand{\teteTermStssBiom} {\enTete[\termStss]{\chapitre{\termStssBiom}}[4]}
\newcommand{\teteTermStssMedi} {\enTete[\termStss]{\chapitre{\termStssMedi}}[5]}
\newcommand{\teteTermStssCosm} {\enTete[\termStss]{\chapitre{\termStssCosm}}[9]}

%%%%%%%%%%%%%%%%%%%%%%%%%%%%%%%%%%%%%%%%%%%%%%%%%%%%%%%%%%%%%
%% Pour faire des parenthèses dans les molécules 
\def\parentheseG{\llap{$\left(\strut\right.$}}
\def\parentheseD{\rlap{$\left.\strut\right)$}}

%% Pour avoir des molécules en gras dans un texte
\newcommand{\moleculesGras}{
  \renewcommand*\printatom[1]{\ensuremath{\mathbf{##1}}}
}
\newcommand{\moleculesNormale}{
  \renewcommand*\printatom[1]{\ensuremath{\mathrm{##1}}}
}
\newcommand{\chemfigHaworth}[1]{
  \chemfig[cram width=3pt, atom sep=2.25em]{ #1 }
}

%% parties colorées
\definesubmol\cetoneCouleur{(=[3,,,,couleurQuat] \textcolor{couleurQuat}{O}) -[-1,,,,couleurQuat]}
%% ramification
\definesubmol\alkyleG{(-[-5] R_1)}
\definesubmol\alkyleD{(-[-1] R_2)}

%%%% Élément récurrent, pour faciliter la lecture
\newcommand{\hydrogene}{\chemfig{H} }
\newcommand{\carbone}{\chemfig{C} }
\newcommand{\oxygene}{\chemfig{O} }
\newcommand{\dioxygene}{\chemfig{O_2} }
\newcommand{\azote}{\chemfig{N} }
\newcommand{\eau}{\chemfig{H_2O} }
\newcommand{\oxonium}{\chemfig{H_3O^+} }
\newcommand{\hydroxyde}{\chemfig{HO^{-}} }
\newcommand{\azoture}{\chemfig{NaN_3} }
\newcommand{\electron}{\chemfig{e^{-}} }
\newcommand{\ionHydrogene}{\chemfig{H^{+}} }
\newcommand{\bicarbonateSodium}{\chemfig{NaHCO_3} }

%%%% État physique
\newcommand{\aq} { \ensuremath{_\text{(aq)}} }
\newcommand{\sol}{ \ensuremath{_\text{(s)}} }
\newcommand{\liq}{ \ensuremath{_\text{(l)}} }
\newcommand{\gaz}{ \ensuremath{_\text{(g)}} }

%%%%%%%%%%%%%%%%%%%%%%%%%%%%%%%%%%%%%%%%%%%%%%%%%%%%%%%%%%%%%
%%%% Pour simplifier certaines molécules
\definesubmol\cu{ -[::60] } % Carbone vers le haut (carbon up)
\definesubmol\cd{ -[::-60] } % carbone vers le bas (carbon down)
\definesubmol\cud{ -[::60] -[::-60] } % Liaison C-C ^ (carbon up down)
\definesubmol\cdu{ -[::-60] -[::60] } % liaison C-C v (carbon down up)
\definesubmol\cis{ -[::60] =[::-60] -[::-60] } % Liaison -C=C- cis
\definesubmol\trans{ -[::60] =[::-30] -[::-30] } % Liaison -C=C- trans
%% Hydrogène saturés
\definesubmol\paireH{(-[::90] H) (-[::-90] H)}
\definesubmol\paireSatH{(-[::30] H) (-[::-30] H)}
\definesubmol\saturationH{(-[::90] H) (-[::-90] H) (-[::0] H)}
%% Quelques groupes caractéristiques
\definesubmol\teteAcide{ O-[::30] (=[::60] O) -[::-60] }
\definesubmol\teteAcideDev{ - O - C (=[::90] O) - }
\definesubmol\OH { -[::60] OH }
\definesubmol\carboxyle{ (=[::-60] O) (-[::60] OH) }
\definesubmol\carbonyle{ (=[::60] O) -[::-60] }
\definesubmol\ester{ (=[:90] O) -[:-30] O}
\definesubmol\ether{ -[:30] O -[:-30]}
\definesubmol\amide{ (=[:90] O) -[:-30] N}
%% Représentation de hamworth
\definesubmol\gluHaw{
  <[-1.5, 0.7] (-[:90, 0.6] OH)
  -[::45,,,,line width = 3.4pt] (-[:-90, 0.6] OH)
  >[::45, 0.7]
}
\definesubmol\gluLeftHaw{
  <[-4.5, 0.7] (-[:-90, 0.6] OH)
  -[::-45,,,,line width = 3.4pt] (-[:90, 0.6] OH)
  >[::-45, 0.7] (-[1.5,0.7])
}


%%%%%%%%%%%%%%%%%%%%%%%%%%%%%%%%%%%%%%%%%%%%%%%%%%%%%%%%%%%%%
%% Acides gras
\definesubmol\palmitique{
  HO -[::30] !\carbonyle !\cud !\cud !\cud !\cud !\cud !\cud !\cud
}
\definesubmol\linolenique{
  HO -[::30] !\carbonyle !\cud !\cud !\cud !\trans !\trans !\trans !\cu
}
\definesubmol\oleique{
  HO -[::30] !\carbonyle !\cud !\cud !\cud !\cis !\cu !\cud !\cud !\cud
}
\definesubmol\linoleique{
  HO -[::30] !\carbonyle !\cud !\cud !\cud !\cis !\cis !\cud !\cud
}
\definesubmol\arachidonique{
  HO -[::30] !\carbonyle !\cud !\cis !\cis !\cis !\cis !\cu !\cud !\cu
}
%% Formes semi-developpées
\definesubmol\steraiqueSemiDev{
  !\teteAcideDev C_{17}H_{35}
}
\definesubmol\caproiqueSemiDev{
  !\teteAcideDev - CH_2 - CH_2 - CH_2 - CH_2 - CH_3
}
\definesubmol\oleiqueSemiDev{
  C_{17} H_{33} -C (=[1.5]O) (-[-1.5]OH)
}
\definesubmol\oleateSemiDev{
  C_{17} H_{33} -C (=[1.5]O) (-[-1.5]O^{-})
}
%% Pour l'utilisation dans les lipides
\definesubmol\tripalmitique{
  !\cdu !\cdu !\cdu !\cdu !\cdu !\cdu !\cd
}
\definesubmol\trilinolenique{
  !\cud !\cud !\cud !\cis !\cis !\cis !\cu
}
\definesubmol\trioleique{
  !\cud !\cud !\cud !\cu =[::60] !\cu !\cud !\cud !\cd !\cd !\cd
}
\definesubmol\trilinoleique{
  !\cud !\cud !\cud !\cis !\cis !\cud !\cud
}


%%%%%%%%%%%%%%%%%%%%%%%%%%%%%%%%%%%%%%%%%%%%%%%%%%%%%%%%%%%%%
%% Lipide
\definesubmol\oleine{
   (-[::150] !\cu O-[::-60] !\carbonyle !\trioleique)
   (-[::-90] -[::-60] O!\cu !\carbonyle !\trioleique)
   -[::30] O!\cu !\carbonyle !\trioleique
}
\definesubmol\palmitine{
   (-[::150] !\cu O !\cd !\carbonyle !\tripalmitique) % haut
   (-[::-90] !\cu O !\cu (=[::-60] O) !\cu !\tripalmitique) % bas
   -[::30] O!\cu !\carbonyle !\cud !\cud !\cud !\cud !\cud !\cud !\cu % centre
}
\definesubmol\phosphatidylcholine{
  -[::-30]N
    (-[::-30])(-[::-90])
  !\cud !\cu O !\cd P
    ( =[::-30]O )( -[::-90]O )
  !\cu O !\cd !\cu
    (!\cu O !\cd !\carbonyle !\trioleique)
  !\cd !\cu O !\cd (=[::-60] O) !\cud !\cud !\cud !\cud !\cud
}
\definesubmol\oleineSemiDev{
  H C                 (!\teteAcideDev C_{17} H_{33}) 
  (-[3,1.7,2,2] H_2C  (!\teteAcideDev C_{17} H_{33}))
  -[-3,1.7,2,2] H_2 C (!\teteAcideDev C_{17} H_{33})
}
\definesubmol\caproineSemiDev{
  H_2C                (!\caproiqueSemiDev)
  -[-3,1.7,2,2] H C   (!\caproiqueSemiDev)
  -[-3,1.7,2,2] H_2 C (!\caproiqueSemiDev)
}
\definesubmol\palmitineSemiDev{
  H C (!\teteAcideDev C_{15} H_{31}) 
  (-[3,1.7,2,2] H_2C (!\teteAcideDev C_{15} H_{31}))
  -[-3,1.7,2,2] H_2 C (!\teteAcideDev C_{15} H_{31})
}

%% glycérol
\definesubmol\glycerol{
  HO -[-1] 
  -[1] (-[3] OH)
  -[-1] -[1] OH
}
\definesubmol\glycerolSemiDev{
  HC (-OH)
  (-[3,,2,2] H_2C (-OH))
  -[-3,,2,2] H_2C (-OH)
}


%%%%%%%%%%%%%%%%%%%%%%%%%%%%%%%%%%%%%%%%%%%%%%%%%%%%%%%%%%%%%
%% Stérols
\definesubmol\cholesterol{
  HO-[1] *6(-- % 1er cycle
    *6(=-- % 2eme cycle
      *6(- % 3eme cycle
        *5(--- 
          (-[::-35] (!\cu) !\cd !\cd !\cud (!\cd) !\cu) % lipide
          -  -
        ) % 4eme
        - (-[::0]) ---
      ) % 3eme
      ---
    ) % 2eme
    - (-[::0]) ---
  ) % 1er
}
\definesubmol\testosterone{
  O=[1] *6(-=
    *6(---
      *6(-
        *5(--- 
          (-[::-100] OH) % alcool
        --
        ) % 4
      - (-[::0]) ---
      ) % 3
    ---
    ) % 2
  - (-[::0]) ---
  ) % 1
}
\definesubmol\cortisol{
  O=[::30] *6(
    -= *6(
      --- *6(
        - *5(
          --- (-[::-100] OH)
          (-[::-35] (=[::60] O) !\cd!\cu OH)
          -
        ) % 4
        - (-[::0]) -- (-OH) -
      ) % 3
      --
    ) % 2
    - (-[::0]) ---
  ) % 1
}
\definesubmol\progesterone{
  O=[::30] *6(
    -=  *6(
      --- *6(
        - *5(--- (=O) -)
        - (-[::0])
        ---
      ) % 3
      --
    ) % 2
    - (-[::0]) ---
  ) % 1
}

\definesubmol\estradiol{
    HO-[::30] *6(
    -= *6(
      --- *6(
        - *5(--- (-OH) -)
        - (-[::0]) ---
      ) % 3
      --
    ) % 2
    -=-=
  ) % 1
}


%%%%%%%%%%%%%%%%%%%%%%%%%%%%%%%%%%%%%%%%%%%%%%%%%%%%%%%%%%%%%
%%%% Glucides
%% Amidon
\definesubmol\amylopec{
  -[-1]O-[1]
  -[1] (-[3] CH_2 OH)
  !\cd O !\cd
  (!\cd (!\OH) !\cd (!\OH) !\cd)
}
\newcommand{\amylopectine}{
  % partie gauche
  ... !\amylopec !\amylopec
  % cycle central
  -[-1] O -[-3,1.2] CH_2O -[-3,1.2] % on agrandit l'espace vertical
  -[-1] O !\cd (
    !\cd (-[::60] OH) !\cd (-[::60] OH) !\cd
    % cycle à gauche
    (-[::120,1.25] O -[::-60,1.2]
    *6(-O- (-[,0.5]CH_2OH) -(-[6]O-[6]...) -(-OH) -(-OH)-))
    % partie droite
    !\cd
  ) !\amylopec !\amylopec -[-1] ...
}
\definesubmol\amylopecHaw{
  % début du cycle
  O -[1,0.6]
  !\gluHaw
  (-[4.5,0.7] O
  -[6] (-[3,0.5] -[5,0.75] OH)
  -[-4.5,0.7])
  % fin du cycle
  -[-1,0.6]
}
\definesubmol\amylopecGaucheHaw{
  O -[5,0.6]
  % début du cycle, perspective
  !\gluLeftHaw  (-[-5,0.6] O -[5] ...)
  % fin du cycle
  -[1.5,0.7] (-[3,0.4] -[5,0.6] OH)
  -O -[-1.5,0.7]
}
\definesubmol\amylopecCentraleHaw{
  O -[-1,0.7] -[-3,0.6]
  % début du cycle
  - O -[-1.5,0.7] ( % perspective
    !\gluLeftHaw  (-[-5,0.6] !\amylopecGaucheHaw)
  )
  % fin du cycle
  -[-1,0.6]
}
\definesubmol\amylopectineHaw{
  ...\phantom{B}-[-1] !\amylopecHaw
  !\amylopecHaw O -[-3]
  !\amylopecCentraleHaw
  !\amylopecHaw !\amylopecHaw O -[1]..
}

%% glucose
\definesubmol\glucoseHaw{
  % début du cycle
  HO -[3,0.6,2]?
  !\gluHaw
  (-[-3,0.6] OH)
  -[4.5,0.7] O 
  % fin du cycle
  -[6]? (-[3,0.5] -[5,0.75] OH)
}
\definesubmol\glucoseCycle{
  *6 (-(-OH) -(-OH) -(-OH) -O -(- -[3]OH) -) (-[-5]HO)
}
\definesubmol\glucose{
  HO -[6] (-[-4] (-[6] HO) -[-2] -[-4] HO) -[4] (-[6] HO) -[2] (-OH) -[4] (=[6] O) -[2] H
}
\definesubmol{\ose}{ -[-3] C (-H) (-[6] HO) }
\definesubmol\glucoseSemiDev{
  HO -C (-H) (
    -[3] C (-H) (-[6] HO)
    (-[3] C (-[5]H) =[1] O)
  ) % aldehyde et alcool
  !\ose !\ose !\ose
  -[-3] H
}

%% fructose
\definesubmol\fructoseHaw{
  % début du cycle
  HO -[3,0.6,2]?
  !\gluHaw 
  (-[-3,0.5] -[-1,0.75] OH)
  -[4.5,0.7] O 
  % fin du cycle
  -[6]?
}
\definesubmol\fructoseCycle{
  *6(-(-OH) -(-OH) -(-[3]OH) (-[0]-[1]OH) -O- -(-OH))
}
\definesubmol\fructose{
  HO -[6] (-[-4] (-[6] HO) -[-2] -[-4] HO) -[4] (-[6] HO) -[2] (=O) -[4] -[2] OH
}
\definesubmol\fructoseSemiDev{
  HO -C (-H) (
    -[3] C (=O)
    (-[3] C (-[3]H) (-[6]H) -OH)
  ) % cétone et alcool
  !\ose !\ose !\ose
  -[-3] H
}


%%%%%%%%%%%%%%%%%%%%%%%%%%%%%%%%%%%%%%%%%%%%%%%%%%%%%%%%%%%%%
%% Acides alpha aminés
\definesubmol\isoleucine{
  H_2N -[1] (-[3] (-[1]) -[5] -[3]) -[-1] !\carboxyle
}
\definesubmol\leucine{
  H_2N -[1] (-[3] -[5] (-[-5]) -[3]) -[-1] !\carboxyle
}
\definesubmol\methionine{
  H_2N -[1] (-[3] -[5] -[3] S -[5]) -[-1] !\carboxyle
}
\definesubmol\valine{
  H_2N -[1] (-[3] (-[1]) -[5]) -[-1] !\carboxyle
}
\definesubmol\alanineSemiDev{
  CH_3- CH (-[-3] NH_2) - C (=[1.5] O) -[-1.5] OH
}
\definesubmol\asparagineSemiDev{
  HO -[1]C (=[3] O) -[-1]CH (-[-3] NH_2) -[1]CH_2 -[-1]C (-[-3] NH_2) =[1] O
}
\definesubmol\glycineSemiDev{
  NH_2- CH_2- C (=[1.5] O) -[-1.5] OH
}
\definesubmol\alaninePoly{
  - CH (-[-3] CH_3) - C (=[3] O) -
}
\definesubmol\glycinePoly{
  - CH_2 - C (=[3] O) -
}
\definesubmol\isoleucinePoly{
  -CH (-[-3] CH (-[-5] CH_2 -[-3]CH_3) -[-1]CH_2) -C (=[3] O) -
}
\definesubmol\valinePoly{
  -CH (-[3] CH (-[5] CH_3) -[1]CH_2) -C (=[-3] O) -
}


%%%%%%%%%%%%%%%%%%%%%%%%%%%%%%%%%%%%%%%%%%%%%%%%%%%%%%%%%%%%%
%% Hormones
\definesubmol\creatinine{
  O= *5(-N (-[-3,0.5]H) -(=NH) -N (-) --)
}
\definesubmol\DOPA{
  HO -[1] *6(= (-OH) -= (--[-1] (-[-3]NH_2) -[1] COOH) -=-)
}
\definesubmol\DOPAH{
  HO -[1] *6(= (-OH) -= (--[-1] (-[-3]NH_3^+) -[1] COOH) -=-)
}
\definesubmol\prostaglandine{
  OH-[::75] *5(
    - (
      -=[::60]!\cd  (!\cd OH) !\cud !\cud !\cu 
    )
    - (-[::-65]!\cd !\cud !\cud !\cu  (=[::60]O) !\cd OH)
    - (=O)
    --
  )
}

%%%%%%%%%%%%%%%%%%%%%%%%%%%%%%%%%%%%%%%%%%%%%%%%%%%%%%%%%%%%%
%% Produit de contraste
\definesubmol\ionChelate{
  N (-[::-45, 0.9,,, draw = none] Gd^{3+}) 
      (-[::140] !\cd COO^{-}) -[::80] -[ 0] -[::-80]
    N (-[::140] !\cd COO^{-}) -[::70] -[-3] -[::-80]
    N (-[::120] !\cd COO^{-}) -[::70] -[-6] -[::-80]
    N (-[::140] !\cd ^{-}OOC) -[::70] -[ 3] -[::-80, 0.8]
}
\definesubmol\chelateAlcool{
  N (-[::-45, 0.9,,, draw = none] Gd^{3+}) 
    (-[::140] !\cd COO^{-})         -[::80] -[ 0] -[::-80]
  N (-[::140] -[0] (-[2] OH) -[-2]) -[::70] -[-3] -[::-80]
  N (-[::120] !\cd COO^{-})         -[::70] -[-6] -[::-80]
  N (-[::140] !\cd ^{-}OOC)         -[::70] -[ 3] -[::-80, 0.8]
}


%%%%%%%%%%%%%%%%%%%%%%%%%%%%%%%%%%%%%%%%%%%%%%%%%%%%%%%%%%%%%
%% Vitamines
\definesubmol\acideAscorbique{ % Vitamine C
  HO-[-1] -[1](-[3]OH) -[-1] 
  *5(
    -(-OH) =(-OH) -(=O) -O-
  )
}
\definesubmol\cholecarciferol{ % Vitamine D
  OH-[-1]
  *6( % 1er cycle
    ---(=)- ( % ramification
      = !\cd =[::60] *6(- % 2eme
        *5(
          --- (-(-[::60]) !\cd !\cud -[::60](-[::60]) !\cd) --
        ) % 3eme
        -(-[::0])----
      ) % 2eme
    ) % ramification
    --
  ) % 1er
}
\definesubmol\cret{ =[-1] -[1] }
\definesubmol\retinol{ % Vitamine A
  *6( % cycle
    --(-)= ( % chaine
      -[1] !\cret (-[3]) !\cret !\cret (-[3]) !\cret OH
    ) % chaine
    -(-[1]) (-[5])--
  ) % cycle
}


%%%%%%%%%%%%%%%%%%%%%%%%%%%%%%%%%%%%%%%%%%%%%%%%%%%%%%%%%%%%%
%% Aspirine
\definesubmol\aspirineSemiDev{
  O=[-1] C (-[1]OH) -[-3]C % carboxyle
  *6( % cycle
    =HC -[,,2,2]HC =\chembelow{C}{H} -CH =C (
      -[1] O -[-1] C (=[-3] O) -[1]CH_3 % cétone
    )
    -
  ) % cyle
}
\definesubmol\aspirine{
  *6 (
    -=- (-O -[-1] (=[-3]) -[1]) = (- (=[5] O) -[1] OH) -=
  )
}

%% Paracétamol
\definesubmol\paracetamol{
  *6(
    (-HO)-=-(
      -NH (!\cd (=[::-60]O)!\cu)
    ) % amide
    =-=
  )
}
\definesubmol\paracetamolSemiDev{
  *6(
    C (-HO) -CH =CH -C (
      -NH (-[-1]C (=[-3]O) -CH_3)
    ) % amide
    =CH -HC =[,4,2]
  )
}
\definesubmol\paracetamolDev{
  H -O -[1]C *6(
    -C(-H) =C(-H) -C (
      -N (-[3]H) (-[-1]C (=[-3]O) (-C!\saturationH)) 
    ) % amide
    =C(-H) -C(-H) =
  )
}

%% Aspartame
\definesubmol\aspartame{
  [:150]
  *6(-=-=-=) % phenyl
  -[0]-[-2] (
    -[-4] (=[6]O) -[-2]O -[-4]
  ) % ester
  -[0]NH -[-2,,1] (=[-4] O) % Amide
  -[0] (
    -[2] NH_2
  ) % amine
  -[-2] -[0] (=[2] O) -[-2] OH % acide carboxylique
}

%%%%%%%%%%%%%%%%%%%%%%%%%%%%%%%%%%%%%%%%%%%%%%%%%%%%%%%%%%%%%
%%%% Molécules odorantes
%% géraniol
\definesubmol\geraniol{
  -[-3] *6( % tete
    --- 
      (= (!\cd) (!\cu)) % pied
    -[,,,,draw=none] -[,,,,draw=none]
      (- !\cd OH) % alcool
    =
  )
}
\definesubmol\geraniolSemiDev{
  CH_3 -[-3]C *6 ( % tete
    -H_2C -[,,2,2]H_2C -CH 
      (=C (!\cd CH_3) (!\cu CH_3)) % pied
    -[,,1,1,draw=none] -[,,,,draw=none]CH
      (-CH_2 !\cd OH) % alcool
    =CH_2 
  )
}
\definesubmol\oxyphenylone{
  HO -[::30] *6(
    -=- (
      -!\cd !\cu (=[::60]O) !\cd 
    )
    =-=
  )
}
\definesubmol\vanilline{
  O=[::-30] (!\cu H)
  !\cd  *6(
    - = - (-OH)
    = (-O!\cu)
    -=
  )
}

%%%%%%%%%%%%%%%%%%%%%%%%%%%%%%%%%%%%%%%%%%%%%%%%%%%%%%%%%%%%%
%%%% Drogues
\definesubmol\THC{
  -[::-90] *6 ( % 1er
    --- *6 ( % 2eme
      - (-[::-90]) (-[::-30])
      -O- *6 ( % 3eme
        -= (!\cd !\cud !\cud)
        -= (!\cd OH) -=
      )
      --
    )
    --=
  )
}
\definesubmol\cocaineHaw{
  ? 
    <[::60,0.7] (-[::60] N -[::60])
    -[::-60,,,,line width = 3.4pt]
    >[::-30,0.7] ( % ether-phenyl
      -[::60] O -[::-60] (=[::-60]O)
      !\cu  *6(=-=-=-)
    )
  -[::130,0.7] ( % ester
    -[::-50] (=[::60]O)
    !\cd O !\cu 
  )
  -[::80, 0.9] (-[::-30, 0.85]) -[::60, 0.7] ?
}
\newcommand{\largeurCaseTableauPeriodique}{1.5}

%%%% Pour afficher un élément dans le tableau périodique
\NewDocumentCommand{\elementTexteCharge}{m m m o}
{
  \begin{minipage}{\largeurCaseTableauPeriodique cm}
    \begin{center}
      \IfValueTF{#4}{ \vAligne{-20pt} }{ \vAligne{-34pt} } % position du nom
      {\small #3} \\[2pt] % nom de l'élément
      {\ensuremath\footnotesize \textbf{#1}} \\[6pt] % nombre atomique
      \chemfig[atom style={scale = 1.8}]{#2} % symbole atomique
      % \element{#1}{#2} % element symbol and atomic number
      \IfValueT{#4}{
        \\ {\small \qty{#4}{\g/\mole}}
      }
    \end{center}
  \end{minipage}
}

%%%% Pour afficher un élément dans le tableau périodique
\NewDocumentCommand{\elementElectroneg}{m m}
{
  \begin{minipage}{\largeurCaseTableauPeriodique cm}
    \begin{center}
      {\Large \important[black]{#1} \\[2pt]} % symbole atomique
      {\footnotesize $\chi = $\num{#2}} % électronégativité
    \end{center}
  \end{minipage}
}


%%%% Pour afficher un tableau périodique
%% #1 : largeur ; #2 : hauteur ; #3 : élements
\NewDocumentCommand{\tableauPeriodique}{O{2.6} O{2.7} m}{
\begin{tikzpicture}[font=\sffamily, scale=0.75, transform shape]
  
%% Couleur de remplissage
  \tikzstyle{jauneClair}  = [fill=yellow!30]
  \tikzstyle{jaune}       = [fill=yellow!45]
  \tikzstyle{jauneFonce}  = [fill=yellow!60]
  \tikzstyle{rougeClair}  = [fill=red!20]
  \tikzstyle{rouge}       = [fill=red!35]
  \tikzstyle{rougeFonce}  = [fill=red!50]
  \tikzstyle{orangeClair} = [fill=orange!30]
  \tikzstyle{orange}      = [fill=orange!45]
  \tikzstyle{orangeFonce} = [fill=orange!60]
  \tikzstyle{vertClair}   = [fill=vertForet!20]
  \tikzstyle{vert}        = [fill=vertForet!35]
  \tikzstyle{vertFonce}   = [fill=vertForet!50]
  
%% Type d'élément, par famille
  \tikzstyle{Alcali} = [Element, vertFonce]
  \tikzstyle{Alcalo} = [Element, vert]
  \tikzstyle{Metaux} = [Element, vertClair]
  \tikzstyle{Metoid} = [Element, orangeClair]
  \tikzstyle{NoMeta} = [Element, orange]
  \tikzstyle{Haloge} = [Element, orangeFonce]
  \tikzstyle{GazRar} = [Element, rouge]

%% Type d'élément, par électronégativité
 \tikzstyle{elec1} = [Element, vertClair]
 \tikzstyle{elec2} = [Element, vert]
 \tikzstyle{elec3} = [Element, jaune]
 \tikzstyle{elec4} = [Element, orangeClair]
 \tikzstyle{elec5} = [Element, orange]
 \tikzstyle{elec6} = [Element, orangeFonce]
 \tikzstyle{elec7} = [Element, rouge]
 \tikzstyle{elec8} = [Element, rougeFonce]
  
%% Style des éléments
  \tikzstyle{Element} = [
    draw=black, jaune,
    minimum width  = #1 cm, % Largeur de la case
    node distance  = #1 cm, % Espace entre deux case
    minimum height = #2 cm, % Hauteur de la case
  ]

%% Période, groupe et titre
  \tikzstyle{Period} = [font={\sffamily\LARGE}, node distance=2cm]
  \tikzstyle{Groupe} = [font={\sffamily\LARGE}, minimum width=2.5cm, node distance=2cm]
  \tikzstyle{Titre}  = [font={\sffamily\Huge\bfseries}]

%% Place des éléments
  #3
\end{tikzpicture}
}


%%%% Pour faciliter l'utilisation du tableau périodique
\newcommand{\elementH} {\elementTexteCharge{1} {H} {Hydrogène}[1,00]}
\newcommand{\elementHe}{\elementTexteCharge{2} {He}{Hélium}   [4,00]}
\newcommand{\elementLi}{\elementTexteCharge{3} {Li}{Lithium}  [6,94]}
\newcommand{\elementBe}{\elementTexteCharge{4} {Be}{Béryllium}[9,01]}
\newcommand{\elementB} {\elementTexteCharge{5} {B} {Bore}     [10,8]}
\newcommand{\elementC} {\elementTexteCharge{6} {C} {Carbone}  [12,0]}
\newcommand{\elementN} {\elementTexteCharge{7} {N} {Azote}    [14,0]}
\newcommand{\elementO} {\elementTexteCharge{8} {O} {Oxygène}  [16,0]}
\newcommand{\elementF} {\elementTexteCharge{9} {F} {Fluor}    [19,0]}
\newcommand{\elementNe}{\elementTexteCharge{10}{Ne}{Néon}     [20,2]}
\newcommand{\elementNa}{\elementTexteCharge{11}{Na}{Sodium}   [23,0]}
\newcommand{\elementMg}{\elementTexteCharge{12}{Mg}{Magnésium}[24,3]}
\newcommand{\elementAl}{\elementTexteCharge{13}{Al}{Aluminium}[27,0]}
\newcommand{\elementSi}{\elementTexteCharge{14}{Si}{Silicium} [28,1]}
\newcommand{\elementP} {\elementTexteCharge{15}{P} {Phosphore}[31,0]}
\newcommand{\elementS} {\elementTexteCharge{16}{S} {Soufre}   [32,1]}
\newcommand{\elementCl}{\elementTexteCharge{17}{Cl}{Chlore}   [35,5]}
\newcommand{\elementAr}{\elementTexteCharge{18}{Ar}{Argon}    [39,9]}
\newcommand{\elementK} {\elementTexteCharge{19}{K} {Potassium}[39,1]}
\newcommand{\elementCa}{\elementTexteCharge{20}{Ca}{Calcium}  [40,0]}


%%%% Palettes de couleur
\palette{couleurPrim}{cyan}
\palette{couleurSec} {blue}
\palette{couleurTer} {purple}
\palette{couleurQuat}{red}


%%%% Réglages de la taille des indentations et des sauts de paragraphes
\setlength{\parskip}{0cm}
\setlength{\parindent}{0cm}
\renewcommand{\baselinestretch}{1}
% réglage du niveau (sous-section) ou s'arrête la table des matières
\setcounter{tocdepth}{2}


%%%% Réglage de la géométrie des pages
\geometry{
  a4paper, % format
  left=1.3cm, right=1.3cm, % marge horizontale
  top=2.2cm, bottom=2.1cm % marge verticale
}


%%% Apparence (couleur) des liens
\hypersetup{
  colorlinks=true,
  linkcolor=black, % lien type table des matière
  citecolor=black, % citation
  filecolor=black, 
  urlcolor=couleurPrim!10!black % lien internet
}


%%%% Réglage de tikz (flèche et caractères)
\usetikzlibrary{babel}
\tikzset{>=latex}


%%%% Réglage des en-tête
\renewcommand{\headrulewidth}{0.4pt}
\setlength{\headheight}{22.50113pt}


%%%% Réglage de dashundergaps pour avoir des points et pas de numération
\dashundergapssetup{
  gap-numbers = false,
  gap-format = dot,
  gap-widen,
  gap-extend-percent
}


%%%% Réglage de siunit
\sisetup{
  locale = FR, % français
	 group-minimum-digits = 4, % groupage des chiffres par millier
  inter-unit-product = \ensuremath { { } \cdot { } }, % point médian entre les unités,
  detect-weight, propagate-math-font = true, reset-math-version = false % pour avoir les versions grasse des typo
}
\AtBeginDocument{\RenewCommandCopy\qty\SI} % Pour "écraser" la commande \qty du package physics