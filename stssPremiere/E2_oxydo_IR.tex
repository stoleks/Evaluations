%%%% début de la page
\tetePremStssRedo

%%%%
\nomPrenomClasse

%%%%
\titreEvaluation[2]{Soigner une plaie}


%%%% Compétences évaluées
\pasCorrection{
  \sousTitre{Compétences évaluées}
  
  \begin{tableauCompetences}
    \centering APP &
    Extraire des informations d'un document.
    & & & &  \\
    %
    \centering REA &
    Réaliser un calcul en donnant le résultat en notation scientifique avec les bonnes unités.
    & & & & \\
    %
    \centering VAL &
    Comparer des grandeurs calculées avec des grandeurs données de référence.
    & & & & \\
    %
    \centering COM & 
    Communiquer clairement par écrit en faisant des phrases complètes et synthétiques.
  \end{tableauCompetences}
}

\appreciation{3}


\begin{contexte}  
  Isma s'est blessée au genou en tombant au cours d'une randonnée en montagne.
  Après plusieurs jours de marche, sa plaie s'est infectée et ses ami-es décide alors de l'emmener dans une pharmacie pour voir ce qu'ils et elles peuvent faire.
\end{contexte}
\vspace*{-8pt}


%%%% Exercice I
\titrePartie{Contrôler la température}\correction{\points{8}}

\begin{doc}{Thermomètre sans contact}{doc:E2_thermo_contact}
  La pharmacienne commence par vouloir mesurer la température d'Isma, sauf que son thermomètre sans contact est disfonctionnel.
  Au lieu d'afficher la température, il affiche la tension électrique mesurée par le capteur dans le thermomètre $U = \qty{100}{\milli\volt}$.
  
  Dans la notice, la pharmacienne lit que la température est calculée avec la relation suivante : 
  \begin{equation*}
    T = (\num{0,1}\times U + \num{28,0}) \unit{\degreeCelsius}
  \end{equation*}
  et il y a les caractéristiques techniques suivantes :
    \begin{listePoints}
      \item plage de mesure : de \qty{32,0}{\degreeCelsius} à \qty{42,0}{\degreeCelsius} ;
      \item précision : $\pm \qty{0,2}{\degreeCelsius}$ ;
      \item sensibilité du capteur IR : de \qty{8}{\micro\m} à \qty{14}{\micro\m}.
    \end{listePoints}
\end{doc}

\question{
  Calculer la température corporelle d'Isma.
}{
  On utilise la relation fournie dans la notice :
  $T = (0,1 \times 100 + 28,0)\unit{\degreeCelsius} = \qty{38,0}{\degreeCelsius}$.
  \points{1,5}
}{2}

\pasCorrection{
  \newpage
  \vspace*{-8pt}
}
\question{
  Peut-on dire avec certitude qu'Isma a de la fièvre ?
}{
  Une température corporelle normale est comprise entre \qty{36,1}{\degreeCelsius} et \qty{37,8}{\degreeCelsius}.
  Sans tenir compte de la précision du thermomètre, Isma a de la fièvre.
  Si on en tient compte, la précision étant de $\pm \qty{0,2}{\degreeCelsius}$, c'est possible que la température d'Isma soit à la limite de de \qty{37,8}{\degreeCelsius} et qu'elle n'ait pas de fièvre.
  \points{2}
}{3}


\begin{doc}{Loi de Wien}{doc:E2_loi_wien}
  La notice explique aussi que le thermomètre repose sur la loi de Wien :
  \begin{equation*}
    \lambda = \dfrac{\qty{2,9e-3}{\kelvin\m}}{T}
  \end{equation*}
  où $\lambda$ est la longueur d'onde de la lumière émise par un corps de température $T$, température exprimé en Kelvin noté \unit{\kelvin}.

  \textbf{Données :}
  \qty{1}{\micro\m} = \qty{e-6}{\m}. 
  \qq{}
  \qty{1}{\degreeCelsius} = (1 + 273)\unit{\kelvin}.
\end{doc}

\question{
  Convertir la température d'Isma en Kelvin.
}{
  On utilise la définition des Kelvin : $T = 38,0 + \qty{273}{\kelvin} = \qty{311}{\kelvin}$.
  \points{1}
}{2}

\question{
  Calculer la longueur d'onde de la lumière émise par Isma.
}{
  On utilise la loi de Wien
  \begin{equation*}
    \lambda = \dfrac{\qty{2,9e-3}{\kelvin\m}}{\qty{311}{\kelvin}} = \qty{9,3e-6}{\m} 
  \end{equation*}
  Soit une longueur d'onde $\lambda = \qty{9,3}{\micro\m}$.
  \points{1,5}
}{2}

\question{
  Le capteur IR du thermomètre est-il sensible à ces longueurs d'onde ?
}{
  Oui, car le capteur est sensible à des longueurs d'onde comprises entre $\qty{8}{\micro\m}$ et $\qty{14}{\micro\m}$.
  \points{2}
}{2}



%%%% Exercice II
\titrePartie{Désinfecter la plaie}\correction{\points{12}}

\begin{doc}{Bétadine}{doc:E2_betadine}
  La pharmacienne leur donne de la bétadine pour nettoyer la plaie d'Isma.
  Le principe actif de la bétadine est le diiode \chemfig{I_2}, de demi-équation d'oxydoréduction 
  \begin{equation*}
    \chemfig{I_2} + 2\electron = 2\chemfig{I^{-}}
  \end{equation*}
  
  \begin{encart}
    \important{Rappels :}
    
    Un \important{oxydant} est une espèce chimique capable d'\important{obtenir} un ou plusieurs \important{électrons.}

    Un \important{réducteur} est une espèce chimique capable de \important{relâcher} un ou plusieurs \important{électrons.}
  \end{encart}
\end{doc}

\question{
  Indiquer en justifiant si le diiode est un oxydant ou un réducteur.
}{
  C'est un oxydant, car le diiode peut gagner des électrons.
  \points{2}
}{2}

\pasCorrection{
  \newpage
}
\question{
  La bétadine est-elle un désinfectant ou un antiseptique ?
  Rappeler ce qui les différencie.
}{
  C'est un antiseptique, car la bétadine sert à nettoyer des tissus vivant.
  Un désinfectant sert à tuer les micro-organismes sur des objets inertes.
  \points{2}
}{3}


\begin{doc}{Action de la bétadine}{doc:E2_action_betadine}
  Le diiode contenu dans la bétadine est capable de tuer les micro-organismes.
  Le diiode pénètre très rapidement dans les bactéries, les virus et les champignons et détruit des protéines constitutives de ces micro-organismes, ce qui entraine leur mort.

  Les protéines possèdent des molécules \chemfig{SH}, dont la demi-équation d'oxydoréduction est la suivante :
  \begin{equation*}
    2 \chemfig{SH} = \chemfig{S_2} + 2\chemfig{H^+} + 2\electron
  \end{equation*}

  La formation de ponts disulfure \chemfig{S_2} dans la protéine entraine sa mort.
\end{doc}

\question{
  Indiquer en justifiant si la molécule \chemfig{SH} est un oxydant ou un réducteur.
}{
  C'est un réducteur, car c'est une entité chimique capable de perdre des électrons.
  \points{2}
}{2}

\question{
  En sommant les demi-équations d'oxydoréduction du diiode et de la molécule \chemfig{SH}, écrire la réaction d'oxydoréduction entre \chemfig{I_2} et \chemfig{SH}.
}{
  On somme terme à terme (côté gauche + côté gauche = côté droit + côté droit) :
  \begin{align*}
    & \chemfig{I_2} + 2\chemfig{e^{-}} + 2\chemfig{SH} =
    2\chemfig{I^{-}} + \chemfig{S_2} + 2\chemfig{H^+} + 2\chemfig{e^{-}} \\
    & \chemfig{I_2} + 2\chemfig{SH} =
    2\chemfig{I^{-}} + \chemfig{S_2} + 2\chemfig{H^+}
  \points{4}
  \end{align*}
}{6}

\question{
  En vous aidant de la réaction d'oxydoréduction et du document, expliquer comment le diiode tue les micro-organismes.
}{
  Le diiode vient oxyder les groupements \chemfig{SH} de la protéine pour former des ponts disulfure \chemfig{S_2}, ce qui entraine la mort des protéines.
  \points{2}
}{4}


\pasCorrection{\setcounter{sousSectionNum}{0}

%%%% Correction
\newpage
\vspace*{-36pt}
\titreSousSection{Ma correction (à faire après la correction du professeur)}

%%%% Tableau de correction élève
\begin{tblr}{
    row{1} = {couleurPrim!20}, hlines,
    colspec = {| X[-1, c] | X[2, c] | X[2, c] | X[2, c] |}
  }
  \textbf{Question} & 
  \textbf{L'erreur} &
  \textbf{Analyse de l'erreur} &
  \textbf{La correction} \\
  %
  \phantom{b} \vspace{55 pt} & & & \\
  \phantom{b} \vspace{55 pt} & & & \\
  \phantom{b} \vspace{55 pt} & & & \\
  \phantom{b} \vspace{55 pt} & & & \\
\end{tblr}


%%%% Bilan
\titreSousSection{Mon bilan après mon travail de correction}

%%%% Tableau bilan de la correction
\begin{tableau}{| X[c] | X[c] |}
  \textbf{Ce que je n'avais pas compris...} &
  \textbf{Ce que maintenant j'ai compris...} \\
  \phantom{b} \vspace{150 pt} & \\
\end{tableau}


%%%% Acquis
\titreSousSection{Mes acquis après mon travail de correction (à remplir par le professeur)}

\appreciation{2}}