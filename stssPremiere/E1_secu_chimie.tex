%%%% début de la page
\teteSndCorp
\setcounter{page}{1}

%%%%
\nomPrenomClasse

%%%%
\correction{\vspace*{-40pt}}
\titreEvaluation{Utiliser des produits ménagers en toute sécurité}


%%%% Compétences évaluées
\pasCorrection{
  \sousTitre{Compétences évaluées}
  
  \begin{tableauCompetences}
    \centering RCO &
    Connaître le vocabulaire du cours et les relations importantes.
    & & & & \\
    %
    \centering APP &
    Extraire des informations d'un document.
    & & & &  \\
    %
    \centering REA &
    Réaliser un calcul en donnant le résultat en notation scientifique avec les bonnes unités.
    Placer des solution sur une échelle pH.
    & & & & \\
    %
    \centering COM & 
    Communiquer clairement par écrit en faisant des phrases complètes.
  \end{tableauCompetences}
}

\appreciation{4}


%%%% Exercice I
\vspace*{-20pt}
\titrePartie{Détartrer une bouilloire}\correction{\points{8}}

Pour détartrer une bouilloire, on prépare une solution d'acide éthanoïque de concentration massique $c_m = \qty{150}{\g/\litre}$

\question{
  Donner la formule littérale de la relation qui donne la masse $m$ en soluté
  en fonction de la concentration massique $c_m$ et du volume de la solution $V$
}{
  \begin{equation*}
    m = c_m \times V
    \points{1}
  \end{equation*}
}{1}

\question{
  Quelle est la masse de soluté présente dans une solution d'acide éthanoïque de volume $V = \qty{0,2}{\litre}$ ?
}{
  \begin{equation*}
    m = c_m \times V
    = \qty{150}{\g/\litre} \times \qty{0,2}{\litre}
    = \qty{30,0}{\g}
    \points{1,5}
  \end{equation*}
}{3}

\question{
  Calculer la masse molaire de l'acide éthanoïque, sa formule brute étant \bruteCHO{2}{4}{2}.
  
  \textbf{Données : }
  M(H) = \qty{1,0}{\g/\mole}, M(C) = \qty{12}{\g/\mole}, M(O) = \qty{16}{\g/\mole}.
}{
  \begin{center}
    M(\bruteCHO{2}{4}{2})
    = $2\times$ M(C) + $4\times$ M(H) + $2\times$ M(O)
    = $24 + 4,0 + 32$\unit{\g/\mole}
    = \qty{60}{\g/\mole}
    \points{1,5}
  \end{center}
}{3}

\question{
  Après calcul, on trouve qu'il y a \qty{30,0}{\g} d'acide éthanoïque dans la solution.
  Calculer la quantité de matière $n$ d'acide éthanoïque dans la solution.
}{
  \begin{equation*}
    n = \dfrac{m}{M}
    = \dfrac{\qty{30,0}{\g}}{\qty{60}{\g/\mole}}
    = \qty{0,5}{\mole}
    \points{1,5}
  \end{equation*}
}{5}

\question{
  Rappeler la formule littérale de la relation qui donne la concentration \important{molaire} en fonction de la quantité de matière $n$ et du volume de la solution $V$.
}{
  \begin{equation*}
    c = \dfrac{n}{V}
    \points{1}
  \end{equation*}
}{2}

\question{
  Calculer la concentration \important{molaire} d'acide éthanoïque dans la solution dont le volume est $V = \qty{0,2}{\litre}$.
}{
  \begin{equation*}
    c = \dfrac{n}{V}
    = \dfrac{\qty{0,5}{\mole}}{\qty{0,2}{\litre}}
    = \qty{2,5}{\mole/\litre}
    \points{1,5}
  \end{equation*}
}{6}


\titrePartie{Déboucher des toilettes}\correction{\points{14,5}}

\begin{wrapfigure}[3]{r}{0.1\linewidth}
  \centering
  \vspace*{-36pt}
  \image{1}{images/picto_ronge}
\end{wrapfigure}
Pour déboucher des toilettes, Ahmed a lu qu'il fallait utiliser une solution \important{basique}.
Il dispose d'une solution d'acide éthanoïque, d'une solution d'acide chlorhydrique, d'eau et d'une solution de soude.
Un pictogramme est présent sur la bouteille de soude, il est présenté à droite.
Il mesure le pH des solutions pour savoir laquelle utiliser et trouve les résultats suivants :

\correction{\vspace*{-8pt}}
\begin{tableau}{|c |c |c |c |c |}
  Produit & Acide éthanoïque & Acide chlorhydrique & Soude & Eau \\
  pH      & \num{2,4} & \num{1,0} & \num{13,9} & \num{7,0} 
\end{tableau}

\numeroQuestion
Placer sur l'échelle de pH ci-dessous les produits utilisé par Ahmed, en indiquant où se trouve les solutions acides, basiques et neutres.
\correction{\points{2}}

\pasCorrection{\vspace*{50pt}}
\begin{center}
  \begin{tikzpicture}
    \tkzVecteur(-0.75)[15.5](0){pH}
    \foreach \i in {0,1,...,14} \draw (\i,0.1)--(\i,-0.1) node[below]{\i};
  \end{tikzpicture}
\end{center}
\pasCorrection{\vspace*{50pt}}

\question{
  Rappeler la relation entre le pH et la concentration en ion oxonium \oxonium.
}{
  \begin{equation*}
    [\oxonium] = 10^{-pH}
    \points{1}
  \end{equation*}
}{2}

\question{
  Calculer la concentration en ion oxonium dans la soude et dans l'acide chlorhydrique.
}{
  \begin{align*}  
    [\oxonium]_\text{soude} &= 10^{-13,9} = \qty{1,26e-14}{\mole/\litre} \\
    [\oxonium]_\text{acide chrlorhydrique} &= 10^{-1,0} = \qty{1,0e-1}{\mole/\litre}
    \points{2}
  \end{align*}
}{4}

\question{
  Rappeler la signification du pictogramme de sécurité présent sur la bouteille de soude.
}{
  C'est le symbole qui désigne les produits corrosifs,
  qui peuvent ronger la peau et/ou les yeux en cas de contact.\points{2}
}{3}


\question{
  Donner la définition d’une espèce chimique acide selon Br\o{}nsted.
}{
  C'est une espèce chimique susceptible de libérer un proton, c'est-à-dire un ion \chemfig{H^+}.\points{2}
}{4}

\numeroQuestion
Dans les couples acido-basiques ci-dessous, entourer d'une couleur l'espèce chimique acide du couple et d'une autre couleur la base.
\begin{enumerate}[label = (\alph*)]
  \item \chemfig{NH_4^+}   / \chemfig{NH_3}
  \item \chemfig{HCl}      / \chemfig{Cl^{-}}
  \item \chemfig{CH_3COOH} / \chemfig{CH_3COO^{-}}
  \correction{\points{1,5}}
\end{enumerate}


\question{
  Établir la réaction acido-basique entre l'ammoniac de formule brute \chemfig{NH_3} et l'acide éthanoïque \chemfig{CH_3COOH}.
}{
  \begin{equation*}  
    \chemfig{NH_3} + \chemfig{CH_3COOH} \reaction \chemfig{NH_4^+} + \chemfig{CH_3COO^{-}}
    \points{1}
  \end{equation*}
}{2}

\question{
  Établir les réactions acido-basique avec l'eau \chemfig{H_2O} et les espèces chimiques suivantes : \chemfig{CH_3COOH}, \chemfig{HCl}, \chemfig{NH_3}.

  \textbf{Données :} Les couples acide/base de l'eau sont
  \chemfig{H_2O}/\hydroxyde et \oxonium/\chemfig{H_2O}
}{
  \begin{align*}
    \chemfig{H_2O} + \chemfig{CH_3COOH} &\reaction \oxonium + \chemfig{CH_3COO^{-}} \\
    \chemfig{H_2O} + \chemfig{NH_3} &\reaction \hydroxyde + \chemfig{NH_4^+} \\
    \chemfig{H_2O} + \chemfig{HCl} &\reaction \oxonium + \chemfig{Cl^{-}}
    \points{3}
  \end{align*}
}{6}


\pasCorrection{\setcounter{sousSectionNum}{0}

%%%% Correction
\newpage
\vspace*{-36pt}
\titreSousSection{Ma correction (à faire après la correction du professeur)}

%%%% Tableau de correction élève
\begin{tblr}{
    row{1} = {couleurPrim!20}, hlines,
    colspec = {| X[-1, c] | X[2, c] | X[2, c] | X[2, c] |}
  }
  \textbf{Question} & 
  \textbf{L'erreur} &
  \textbf{Analyse de l'erreur} &
  \textbf{La correction} \\
  %
  \phantom{b} \vspace{55 pt} & & & \\
  \phantom{b} \vspace{55 pt} & & & \\
  \phantom{b} \vspace{55 pt} & & & \\
  \phantom{b} \vspace{55 pt} & & & \\
\end{tblr}


%%%% Bilan
\titreSousSection{Mon bilan après mon travail de correction}

%%%% Tableau bilan de la correction
\begin{tableau}{| X[c] | X[c] |}
  \textbf{Ce que je n'avais pas compris...} &
  \textbf{Ce que maintenant j'ai compris...} \\
  \phantom{b} \vspace{150 pt} & \\
\end{tableau}


%%%% Acquis
\titreSousSection{Mes acquis après mon travail de correction (à remplir par le professeur)}

\appreciation{2}}