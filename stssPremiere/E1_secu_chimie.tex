%%%% début de la page
\teteSndCorp
\setcounter{page}{1}

%%%%
\nomPrenomClasse

%%%%
\titreEvaluation{Utiliser des produits ménagers en toute sécurité}

%%%% Compétences évaluées
\sousTitre{Compétences évaluées}

\begin{tableauCompetences}
  \centering RCO &
  Connaître le vocabulaire du cours et les relations importantes.
  & & & & \\
  %
  \centering APP &
  Extraire des informations d'un document.
  & & & &  \\
  %
  \centering VAL &
  Comparer des valeurs calculées avec des valeurs de références pour valider un raisonnement.
  & & & & \\
  %
  \centering REA &
  Réaliser un calcul en donnant le résultat en notation scientifique avec les bonnes unités. Placer des solution sur une échelle pH.
  & & & & \\
\end{tableauCompetences}

\appreciation{4}


%%%% Exercice I
\vspace*{-20pt}
\titrePartie{Déboucher des toilettes}


Les marais salants sont de grands bassin remplis par d'eau de mer, qui est riche en sel.
Le sel est du chlorure de sodium de formule brute \chemfig{NaCl}.

%
\question{
  Indiquer en justifiant si l'eau de mer est un corps pur ou un mélange.\competence{RCO, APP}
}{
  C'est un mélange, composé d'au moins deux espèces chimique : l'eau et le sel.
}{1}

Le soleil et le vent font s'évaporer l'eau de mer, mais le sel reste au fond des bassins. 
Après plusieurs étapes d'évaporation et de remplissage, la quantité de sel contenue dans l'eau des bassins devient très importante.
La masse volumique de l'eau salée augmente avec la quantité de sel.

\question{
  Rappeler la relation mathématique entre la masse volumique de l'eau salée $\rho$, sa masse $m$ et le volume $V$ qu'elle occupe.\competence{RCO}
}{
  \begin{equation*}
    \rho = \frac{m}{V}
  \end{equation*}
}{2}


\titrePartie{}


\setcounter{sousSectionNum}{0}

%%%% Correction
\newpage
\vspace*{-36pt}
\titreSousSection{Ma correction (à faire après la correction du professeur)}

%%%% Tableau de correction élève
\begin{tblr}{
    row{1} = {couleurPrim!20}, hlines,
    colspec = {| X[-1, c] | X[2, c] | X[2, c] | X[2, c] |}
  }
  \textbf{Question} & 
  \textbf{L'erreur} &
  \textbf{Analyse de l'erreur} &
  \textbf{La correction} \\
  %
  \phantom{b} \vspace{55 pt} & & & \\
  \phantom{b} \vspace{55 pt} & & & \\
  \phantom{b} \vspace{55 pt} & & & \\
  \phantom{b} \vspace{55 pt} & & & \\
\end{tblr}


%%%% Bilan
\titreSousSection{Mon bilan après mon travail de correction}

%%%% Tableau bilan de la correction
\begin{tableau}{| X[c] | X[c] |}
  \textbf{Ce que je n'avais pas compris...} &
  \textbf{Ce que maintenant j'ai compris...} \\
  \phantom{b} \vspace{150 pt} & \\
\end{tableau}


%%%% Acquis
\titreSousSection{Mes acquis après mon travail de correction (à remplir par le professeur)}

\appreciation{2}