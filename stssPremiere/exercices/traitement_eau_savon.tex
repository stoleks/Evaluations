\exercice{Traitement des eaux }

\medskip
L'un des enjeux primordiaux du développement durable est la préservation des ressources en eau de la planète.
L'objectif de cet exercice est de mieux comprendre comment sont traitées les eaux usagées de certaines industries avant leur rejet dans le milieu naturel.

\medskip
La fabrication du savon de Marseille nécessite plusieurs étapes.
Il faut d'abords transformer des huiles végétales en savon sous l'action à chaud de soude concentrée dont la valeur du pH est comprise entre 12 et 13.
La pâte de savon obtenue est ensuite lavée plusieurs fois à l'eau salée afin d'éliminer la soude en excès.
Le savon doit alors cuire pendant dix jours à une température de \qty{120}{\degreeCelsius}.
Puis plusieurs lavages à l'eau pure permettent d'obtenir un savon sans impuretés.
La pâte de savon est alors coulée dans des moules, puis séchée pendant 48 h à l'air libre avant d'être découpée en savonnettes de tailles variées.

\medskip
Les dangers des solutions aqueuses acides sont bien connus, les substances basiques peuvent être tout aussi corrosives et, si elles ne sont pas traitées, peuvent endommager la faune, la flore et l'écosystème environnants.
Les normes de rejets dans les eaux contrôlées, tels que les cours d'eau de surface et les eaux souterraines, exigent un pH compris entre 5,5 et 8,5.

\question{
  Préciser le caractère (neutre, acide ou basique) des eaux de lavage d'une savonnerie.
  Citer une méthode rapide permettant de le vérifier expérimentalement.                            \points{1}
}{
  À 5,5 l'eau est légèrement acide, à 8,5 elle est légèrement basique, entre les deux elle est neutre. 
  On peut le vérifier avec un papier pH.
}

\question{
   Expliquer pourquoi il est nécessaire de traiter ces eaux avant leur rejet dans le milieu naturel.
   \points{2}
}{
  Pour ne pas polluer les milieux naturel en y déversant des solutions basiques, qui vont attaquer la faune et la flore de ces milieux naturels.
}

\medskip
\begin{boite}
  \begin{center}  
    Carte d'identité de l'acide chlorhydrique (source http://www.inrs.fr)
  \end{center}
  
  \separationBlocs{
    Acide chlorhydrique concentré (\oxonium, \chemfig{Cl^{-}}).
    
    \centering
    \image{0.3}{images/pictogrammes/picto_nocif}
    \image{0.3}{images/pictogrammes/picto_corrosif}
  }{
    H331 -- Toxique par inhalation
    
    H314 -- Provoque des brûlures de la peau et des lésions oculaires graves
  }
\end{boite}

\question{
  Rappeler la signification des pictogrammes de sécurités présent sur l'étiquette de l'acide chlorhydrique. \points{1}
}{
  Le pictogramme de gauche indique des solutions nocives pour la santé.
  Le pictogramme de droite indique que les solutions sont corrosives.
  \points{1}
}

% Le principe de la valorisation du dioxyde de carbone consiste à le considérer comme une matière première, que l'on capte à la sortie des fumées industrielles et que l'on exploite pour réaliser un certain nombre de produits ou d'opérations commercialement rentables. La neutralisation au gaz carbonique provenant des gaz de fumée est un procédé écologique et peu onéreux.

% \textbf{Données utiles :}
% \begin{itemize}
%   \item \qty{1}{\m\cubed} = \qty{1000}{\litre}
%   \item Volume moyen : baignoire, $V_b = \qty{0,4}{\m\cubed}$ ; 
%   piscine olympique, $Vp = \qty{2500}{\m\cubed}$ ;
%   \item Formule brute du dioxyde de carbone : \chemfig{CO_2}
%   \item Masse molaire : M(\chemfig{CO_2}) = \qty{44}{\g\per\mole}
%   \item Température de sublimation du \chemfig{CO_2} : \qty{-78,5}{\degreeCelsius} à la pression atmosphérique
% \end{itemize}

La dilution est l'une des méthodes de traitement des eaux alcalines.

% 3. Sachant que la valeur de la concentration des ions hydroxyde dans certaines eaux usées dont le pH vaut 13 est égale à \qty{1,0e-1}{\mole\per\litre}, calculer la valeur de la quantité de matière en ions hydroxyde, $n_\text{hydroxyde}$, présente dans un volume égal à un litre d'une telle eau usée.

% 4. On rappelle qu'une solution aqueuse est neutre si son pH vaut 7. On admet que l'intervalle de pH entre 6 et 8 est acceptable pour une neutralité approchée, sans danger. En admettant que le pH diminue de 1 unité de pH, dans l'intervalle compris entre 8 et 14, lorsqu'il y a dilution d'un facteur 10 d'une eau usée chargée en ions hydroxyde, prévoir le volume minimal d'eau à ajouter à un volume d'eau usée de 1L pour amener son pH de 13 à 8.

% 5. Commenter ce résultat en le comparant aux ordres de grandeurs fournis dans les données et expliquer pourquoi cette méthode n'est pas utilisée dans l'industrie.

Une autre méthode pour traiter les eaux usées consiste à les neutraliser par ajout de dioxyde de carbone ou par ajout d'acide minéraux tel que l'acide chlorhydrique.
La neutralisation au dioxyde de carbone provenant des gaz de fumées industrielles est un procédé écologique et peu onéreux.
Les couples acide/base mis en jeu dans la réaction de neutralisation de l'eau de lavage par la solution d'acide chlorhydrique sont \oxonium/\eau{} et \eau/\hydroxyde.

\question{
  Écrire l'équation acido-basique ajustée de la réaction de neutralisation de l'eau de lavage.            \points{2}
}{
  \begin{equation*}
    \oxonium + \hydroxyde \reaction 2\eau
  \end{equation*}
} 

\question{
  Expliquer pourquoi les industriels préfèrent neutraliser les eaux usagées à l'aide de dioxyde de carbone plutôt qu'à l'aide d'acide chlorhydrique.
  \points{2}
}{
  Les industriels utilisent le dioxyde de carbone, car c'est un procédé écologique et pas cher.
  Le dioxyde de carbone est aussi moins dangereux que l'acide chlorhydrique.
}