
\exercice{Détection et correction de problèmes de vision}

Vous êtes infirmièr-e scolaire et des parents inquiets viennent vous voir. Leur fille de 6 ans, Malala, se penche beaucoup sur ses cahiers de dessin et iels pensent que c'est lié à sa vision.


%%%%
\begin{doc}{Schéma de la position de la feuille par rapport à l’œil de Malala}{doc:E_schema_feuille}
  \centering
  \image{0.9}{images/detection_defaut_vision}
\end{doc}

\question{
  Indiquer quelles parties de l’œil sont schématisées par la lentille et par l'écran.
}{

}[2]

\question{
  En traçant des rayons lumineux particuliers, trouver où se forment les images $A'B'$ et $C'D'$ des objets $AB$ (distance normale d'un enfant avec son cahier) et $CD$ (distance de Malala).
}{

}

\question{
  En justifiant, indiquer si Malala peut voir nettement l'objet $AB$.
}{

}[2]

\question{
  Expliquer pourquoi Malala voit nette l'image de l'objet $CD$.
}{

}[2]

\pasCorrection{
  \newpage
  \vspace*{-24pt}
}
\question{
  Nommer le défaut de la vision dont souffre Malala.
}{

}[1]


%%%%
\begin{doc}{Le muscle ciliaire}{doc:E_muscles_ciliaires}
  Le \important{muscle ciliaire} sert à contrôler le pouvoir convergent du \important{cristallin.}
  Il permet de réaliser une accommodation de la vision.

  Quand on regarde un objet éloigné, le muscle ciliaire se relâche, ce qui étire les fibres zonulaires et applati le cristallin, qui devient moins convergent.

  Au contraire, quand on regarde un objet proche, le muscle ciliaire se contracte, ce qui détend les fibres zonulaires, autour du cristallin, ce qui lui permet de s'arrondir et de devenir plus convergent.
\end{doc}

\begin{doc}{Mesure de la correction des yeux}{doc:E_mesure_dioptrie}
  Pour mesurer la correction nécessaire pour compenser les défauts d'un œil, les ophtalmologues peuvent utiliser une suite de lentilles correctrices, jusqu'à ce que les patient-es voient net.

  Le problème avec cette méthode, c'est que les patient-es accommodent en permanence pour adapter leur vue, et compensent donc une partie des défauts de leur oeil. 

  Pour éviter ce problème et réaliser une \important{correction optique totale,} il faut réaliser une paralysie médicamenteuse des muscles ciliaires, on parle de \important{cyclopégie.}
  Cette cyclopégie entraine l'arrêt du phénomène d'accommodation et est réalisée à l'aide d'un collyre mis dans les yeux.
\end{doc}


\begin{doc}{Correction de la vision}{doc:E_correction_vision}
  \begin{wrapfigure}{r}{0.5\linewidth}
    \vspace*{-34pt}
    \begin{boite}
      Une paire de lunette avec monture. Verre :
      
      \centering
      Œil droit : \hfill \important{- 0,25 \unit{\dioptre}}
      
      Œil gauche : \hfill \important{- 0,5 \unit{\dioptre}}
    \end{boite}
  \end{wrapfigure}
  Sous vos recommandations, les parents de Malala sont allés voir une ophtalmologue qui leur a donné l'ordonnance suivante :  
\end{doc}

\question{
  Expliquer le rôle du muscle ciliaire dans le phénomène d'accommodation.
}{

}[2]

\question{
  Expliquer pourquoi la cyclopégie entraine l'arrêt du phénomène d'accommodation.
}{

}[3]

\question{
  Expliquer pourquoi la cylopégie est nécessaire pour prescrire une correction optique totale adaptée à l’œil.
}{

}[3]

\question{
  Indiquer en justifiant si les verres que porteront Malala seront convergents ou divergents.
}{

}[2]