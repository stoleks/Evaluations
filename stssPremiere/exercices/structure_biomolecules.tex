\exercice{Quelques molécules organiques}

\begin{center}
  \variationSujet {\chemfig{!\valine}} {\chemfig{!\isoleucine}}

  \legende{\variationSujet{Valine,}{Isoleucine,} un des 9 acides aminés essentiels pour les humains.}
\end{center}

\question{
  Donner la formule semi-développée de \variationSujet{la valine.}{l'isoleucine.}
}{
  \variationSujet {\chemfig{!\valineSemiDev}} {\chemfig{!\isoleucineSemiDev}}
  \points{2}
}[7]

\question{
  Donner la formule brute de \variationSujet{la valine.}{l'isoleucine.}
}{
  \variationSujet{\chemfig{C_5 H_{11} O_2 N}} {\chemfig{C_6 H_{12} O_2 N}}
  \points{1}
}[1]

\question{
  Entourer et nommer ses \important{groupes caractéristiques.}
}{
  Il y a un groupe carboxyle \chemfig{COOH} et un groupe amine \chemfig{NH_2}.
  \points{2}
}[2]

\begin{center}
  \chemfig{!\prostaglandine}

  \legende{Prostaglandine E1, un médiateur lipidique synthétisé à partir d'acide gras insaturés.}
\end{center}

\question{
  Entourer et nommer les \important{familles organiques} de la prostaglandine
}{
  Il y a deux familles alcool \chemfig{HO}, une cétone \chemfig{O=} et un acide carboxylique \chemfig{COOH}.
  \points{3}
}[3]