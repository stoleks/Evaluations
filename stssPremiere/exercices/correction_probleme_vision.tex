\exercice{Correction d'un problème de vision}

Vous êtes infirmièr-e scolaire et des parents inquiets viennent vous voir. Leur fille de 6 ans, Malala, \variationSujet{se penche}{s'éloigne} beaucoup \variationSujet{sur}{de} ses cahiers de dessin et iels pensent que c'est lié à sa vision. Après quelques tests, Vous leur recommandez d'aller voir un ophtalmologue.

%%%%
\begin{doc}{Le muscle ciliaire}
  Le \important{muscle ciliaire} sert à contrôler le pouvoir convergent du \important{cristallin.}
  Il permet de réaliser une accommodation de la vision.
  Quand on regarde un objet éloigné, le muscle ciliaire se relâche, ce qui étire les fibres zonulaires et aplati le cristallin, qui devient moins convergent.

  Au contraire, quand on regarde un objet proche, le muscle ciliaire se contracte, ce qui détend les fibres zonulaires, autour du cristallin, ce qui lui permet de s'arrondir et de devenir plus convergent.
\end{doc}

\begin{doc}{Mesure de la correction des yeux}
  Pour mesurer la correction nécessaire pour compenser les défauts d'un œil, les ophtalmologues peuvent utiliser une suite de lentilles correctrices, jusqu'à ce que les patient-es voient net.

  Le problème avec cette méthode, c'est que les patient-es accommodent en permanence pour adapter leur vue, et compensent donc une partie des défauts de leur œil. 

  Pour éviter ce problème et réaliser une \important{correction optique totale,} il faut réaliser une paralysie médicamenteuse des muscles ciliaires, on parle de \important{cyclopégie.}
  Cette cyclopégie entraine l'arrêt du phénomène d'accommodation et est réalisée à l'aide d'un collyre mis dans les yeux.
\end{doc}


\begin{doc}{Correction de la vision}
  \begin{wrapfigure}[3]{r}{0.5\linewidth}
    \vspace*{-34pt}
    \begin{boite}
      Une paire de lunette avec monture. Verre :
      
      \centering
      Œil droit : \hfill \variationSujet{\important{-}}{\important{+}} \qty{0,25}{\dioptre}
      
      Œil gauche : \hfill \variationSujet{\important{-}}{\important{+}} \qty{0,5}{\dioptre}
    \end{boite}
  \end{wrapfigure}
  Sous vos recommandations, les parents de Malala sont allés voir une ophtalmologue qui leur a donné l'ordonnance suivante :

  Une fois obtenu, les verres de correction sont plus épais \variationSujet{sur les côtés qu'au centre}{au centre que sur les côtés}.
\end{doc}

\question{
  Expliquer le rôle du muscle ciliaire dans le phénomène d'accommodation.
}{

}[2]

\question{
  Expliquer pourquoi la cyclopégie entraine l'arrêt du phénomène d'accommodation.
}{

}[3]

\question{
  Expliquer pourquoi la cylopégie est nécessaire pour prescrire une correction optique totale adaptée à l’œil.
}{

}[3]

\question{
  Indiquer en justifiant si les verres que porteront Malala seront convergents ou divergents.
}{

}[2]