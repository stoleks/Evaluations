\exercice{Structures de quelques molécules organiques}\points{12}

%%
\begin{center}
  \chemfig{H_2N !\valineSemiDevH OH}

  \legende{Valine, un des 9 acides aminés essentiels}
\end{center}

\question{
  Écrire la formule développée de la valine.
}{}[6]

\question{
  Donner la formule brute de la valine.
}{}[1]

\question{
  Entourer et nommer les groupes fonctionnels de la valine.
}{}[2]


%%
\begin{center}
  \chemfig{!\glycerol}

  \legende{Glycérol, qui compose les glycérides et les phospholipides dans les organismes vivant}
\end{center}

\question{
  Écrire la formule semi-développée du glycérol.
}{}[6]

\question{
  Donner la formule brute du glycérol.
}{}[1]

\question{
  Entourer et nommer les groupes fonctionnels du glycérol.
}{}[2]


%%
\begin{center}
  \chemfig{H_2N !\isoleucineSemiDevH OH}

  \legende{Isoleucine, un des 9 acides aminés essentiels}
\end{center}

\question{
  Écrire la formule topologique de l'isoleucine.
}{}[6]

\question{
  Donner la formule brute de l'isoleucine.
}{}[1]

\question{
  Entourer et nommer les groupes fonctionnels de l'isoleucine.
}{}[2]


%%
\begin{center}
  \chemfig{!\acidePantothenique}

  \legende{Acide pantothénique, ou vitamine B5, qui favorise la croissance et la résistance de la peau et des muqueuses.}
\end{center}

\question{
  Écrire la formule semi-développée de la vitamine B5.
}{}[6]

\question{
  Donner la formule brute de la vitamine B5.
}{}[1]

\question{
  Entourer et nommer les groupes fonctionnels de la vitamine B5.
}{}[2]


\begin{center}
  \chemfig{!\aspartame}

  \legende{Aspartame, molécule qui active les capteurs associés au goût sucré}
\end{center}

\question{
\textit{Question bonus:} entourer et nommer tous les groupes fonctionnels de l'aspartame.
}{}[5]