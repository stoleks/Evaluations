\exercice{Détartrer une bouilloire}\points{8}
\medskip

Pour détartrer une bouilloire, on prépare une solution d'acide éthanoïque de concentration massique $c_m = \qty{150}{\g/\litre}$

\question{
  Donner la formule littérale de la relation qui donne la masse $m$ en soluté
  en fonction de la concentration massique $c_m$ et du volume de la solution $V$
}{
  \begin{equation*}
    m = c_m \times V
    \points{1}
  \end{equation*}
}[1]

\question{
  Quelle est la masse de soluté présente dans une solution d'acide éthanoïque de volume $V = \qty{0,2}{\litre}$ ?
}{
  \begin{equation*}
    m = c_m \times V
    = \qty{150}{\g/\litre} \times \qty{0,2}{\litre}
    = \qty{30,0}{\g}
    \points{1,5}
  \end{equation*}
}[3]

\question{
  Calculer la masse molaire de l'acide éthanoïque, sa formule brute étant \bruteCHO{2}[4][2].
  
  \important{Données : }
  M(H) = \qty{1,0}{\g/\mole}, M(C) = \qty{12}{\g/\mole}, M(O) = \qty{16}{\g/\mole}.
}{
  \begin{center}
    M(\bruteCHO{2}[4][2])
    = $2\times$ M(C) + $4\times$ M(H) + $2\times$ M(O)
    = $24 + 4,0 + 32$\unit{\g/\mole}
    = \qty{60}{\g/\mole}
    \points{1,5}
  \end{center}
}[3]

\question{
  Après calcul, on trouve qu'il y a \qty{30,0}{\g} d'acide éthanoïque dans la solution.
  Calculer la quantité de matière $n$ d'acide éthanoïque dans la solution.
}{
  \begin{equation*}
    n = \dfrac{m}{M}
    = \dfrac{\qty{30,0}{\g}}{\qty{60}{\g/\mole}}
    = \qty{0,5}{\mole}
    \points{1,5}
  \end{equation*}
}[5]

\question{
  Rappeler la formule littérale de la relation qui donne la concentration \important{molaire} en fonction de la quantité de matière $n$ et du volume de la solution $V$.
}{
  \begin{equation*}
    c = \dfrac{n}{V}
    \points{1}
  \end{equation*}
}[2]

\question{
  Calculer la concentration \important{molaire} d'acide éthanoïque dans la solution dont le volume est $V = \qty{0,2}{\litre}$.
}{
  \begin{equation*}
    c = \dfrac{n}{V}
    = \dfrac{\qty{0,5}{\mole}}{\qty{0,2}{\litre}}
    = \qty{2,5}{\mole/\litre}
    \points{1,5}
  \end{equation*}
}[6]