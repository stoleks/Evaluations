\exercice{L'acide ascorbique }

\medskip
Les vivres embarquées sur les navires européens du XVe au XVIIIe siècle étaient essentiellement des salaisons, des légumes secs et des biscuits pour des raisons liées à la conservation des aliments. Or ceux-ci ne contenant pas de vitamine C (aussi appelée acide ascorbique), il en a résulté des cas de scorbut lors des expéditions de longues distance (plus de six semaines à plus de trois mois) en mer.

\medskip
Les marins atteints du scorbut subissaient des gingivites hypertrophiques (entraînant le gonflement des gencives et le déchaussement des dents) et des syndromes hémorragiques et cutanés (pertes des cheveux et hématomes) si bien qu'après un voyage de 3 mois (pour ceux qui y survivaient), ces derniers semblaient avoir vieilli de vingt ans. 

\medskip
En 2019, le scorbut est réapparu en France chez des personnes précaires ne se nourrissant pas suffisamment de légumes et fruits frais.

\medskip
%On considère que le taux en vitamine C par litre de sang doit être compris entre \num{5}  et \num{17}  \unit{\mg\per\litre} pour être en bonne santé.

\medskip
\textbf{Donnée : } 
% \textit{Formule brute de la vitamine C :} \chemfig{C_6H_8O_6}

\textit{Masse molaire atomique : $M(H) = \qty{1} {\g\per\mole}$ $M(C) = \qty{12} {\g\per\mole}$ $M(N) = \qty{14} {\g\per\mole}$ $M(O) = \qty{16} {\g\per\mole}$}

\textit{ $\qty{1} {g} = \qty{1 000} {mg}$   ;  $\qty{1} {L} = \qty{1 000} {\ml}$ }

\bigskip
\question{
  Quelle espèce chimique permet de lutter contre le scorbut ? Dans quels aliments la trouve-t-on ?               \points{2}
}{
  C'est la vitamine C qui permet de lutter contre le scorbut, qu'on trouve dans les fruits et légumes frais. \points{2}
}

La concentration massique en vitamine C dissout dans un jus de pomme est $C_m = \qty{9,0e-3}{\g\per\litre}$.

\question{
  Quel est le soluté ? Quelle est la solution ?    \points{2}
}{
  Le soluté est la vitamine C, la solution est le jus de pomme.\points{2}
}

\question{
  Calculer la masse de vitamine C présente dans une bouteille de volume $V = \qty{1,5} {\litre}$ de jus de pomme. \points{1,5}
}{
  \begin{equation*}
    m(\chemfig{C_6H_8O_6}) = C_m \times V = \qty{9,0e-3}{\g\per\litre} \times \qty{1,5}{\litre} = \qty{13,5e-3}{\g}
    \points{1,5}
  \end{equation*}
}

\question{
  La formule brute de la vitamine C est \chemfig{C_6H_8O_6}. Calculer la masse molaire moléculaire de la vitamine C.  \points{1,5}
}{
  \begin{equation*}
    M(\chemfig{C_6H_8O_6}) = 6M(H) + 8M(C) + 6M(O) = \qty{176}{\g\per\mole}
  \points{1,5}
  \end{equation*}
}

\question{
  Déterminer la quantité de matière en vitamine C présent dans un volume de $\qty{1,5} {\litre}$ de jus de pomme.  \points{1,5}
}{
  \begin{equation*}
    n = \dfrac{m(\chemfig{C_6H_8O_6})} {M(\chemfig{C_6H_8O_6})}
    = \dfrac{\qty{13,5e-3} {\g}} {\qty{176} {\g\per\mole}}
    = \qty{7,67e-5} {\mole}
  \points{1,5}
  \end{equation*}
}

\question{
  En déduire la concentration molaire en vitamine C dans le jus de pomme.                                \points{1,5}
}{
  \begin{equation*}
    c = \dfrac{n}{V}
    = \dfrac{\qty{7,67e-5}{\mole}} {\qty{1,5}{\litre}}
    = \qty{5,11e-5} {\mole\per\litre}
  \points{1,5}
  \end{equation*}
}

\medskip
Camélia se sent fatiguée et se voit prescrire de la vitamine C. Elle doit prendre chaque jour un comprimé contenant une masse $m = \qty{1000} {\mg}$ de vitamine C. 
Un comprimé se dissout dans un verre d'eau de volume $V = \qty{200} {\ml}$.

\question{
 On suppose que le comprimé se dissout entièrement. Déterminer la concentration massique de vitamine C dans son verre.                                          \points{1,5}
}{
 \begin{equation*}
   C_m = \dfrac{m}{V}
   = \dfrac{\qty{1000} {\mg}} {\qty{200} {\ml}}
    = \dfrac{\qty{1} {\g}} {\qty{0,2} {\litre}}
    = \qty{5,0} {\g\per\litre}
  \points{1,5}
  \end{equation*}
}

\question{
 Une fois ingérée et assimilée par l'organisme, la concentration molaire sanguine de vitamine C de Camélia atteint $\qty{3,6e-5} {\mol\per\litre}$.
  A-t-elle un taux de vitamine C suffisant pour être en bonne santé ?                                       \points{3}
}{
 Pour obtenir la concentration massique, on multiplie la concentration molaire par la masse molaire moléculaire de la vitamine C
 \begin{equation*}
   C_m = M(\chemfig{C_6H_8O_6}) \times c
   = \qty{176} {\g\per\mole} \times \qty{3,6e-5} {\mol\per\litre}
    = \qty{6,36e-3} {\g\per\litre}
    = \qty{6,36} {\mg\per\litre}
  \end{equation*}
  Un taux normal se trouve entre 5 et \qty{17}{\mg\per\litre}, donc elle est en bonne santé.
  \points{3}
}