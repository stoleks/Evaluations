\exercice{Automatismes (connaître son cours) }
\medskip

\numeroQuestion
Relier chaque grandeur à son symbole et à l'unité associée. \points{3}

\begin{center}    
  \begin{tblr}{
    width=\textwidth,
    colspec = {Q[wd=0.25\linewidth, l] X[l] r Q[wd=0.1\linewidth, c] X[l] r Q[wd=0.1\linewidth, c]},
    row{1} = {couleurPrim!20, c}
  }
  %% ALIGNER PLUS DE GRANDEUR 
    \textbf{Grandeur}      & & & \textbf{Symbole} & & & \textbf{Unité} \\
    Concentration massique & \pointCyan & \pointCyan & $n$   & \pointCyan & \pointCyan &  \unit{\mole\per\litre} \\
    Masse                  & \pointCyan & \pointCyan & $C$   & \pointCyan & \pointCyan &  \unit{\litre} \\
    Quantité de matière    & \pointCyan & \pointCyan & $C_m$ & \pointCyan & \pointCyan &  \unit{\kg} \\
    Volume                 & \pointCyan & \pointCyan & $V$   & \pointCyan & \pointCyan &  \unit{\g\per\litre} \\
    Concentration molaire  & \pointCyan & \pointCyan & $M$   & \pointCyan & \pointCyan &  \unit{\g\per\mole} \\
    Masse molaire          & \pointCyan & \pointCyan & $m$   & \pointCyan & \pointCyan &  \unit{\mole} \\
  \end{tblr}
\end{center}

\numeroQuestion
Entourer la ou les relations permettant de calculer les grandeurs de la première colonne. \points{2}
\vspace*{-12pt}

\begin{center}
  \begin{tblr}{
    colspec = {|X[l,m]| X[c,m]| X[c,m]| X[c,m]| X[c,m]|}, hlines,
    column{1} = {couleurPrim!20}
  }
    Pour calculer une quantité de matière &
    $\dfrac{m}{M}$ & $m\times M$ & $C \times V$ & $\dfrac{V}{C}$ \\
    %
    Pour calculer une concentration molaire &
    $\dfrac{m}{V}$ & $\dfrac{n}{V}$ & $m \times M$ & $\dfrac{M}{m}$ \\
    %
    Pour calculer la concentration en ion oxonium &
    $10^{-\text{pH}}$ & $10^{\text{pH}}$ & $10 - \text{pH}$ & $\text{pH}^{10}$ \\
    %
    Pour calculer une masse &
    $M \times n$ & $C_m \times V$ & $m \times M$ & $\dfrac{M}{m}$ \\
    %
  \end{tblr}
\end{center}

\numeroQuestion Donner le nom de ces deux ions \points{1}
\begin{center}  
  \oxonium : \texteTrou[0.1]{ion oxonium} \qq{}
  et \hydroxyde : \texteTrou[0.1]{ion hydroxyde}
\end{center}

\question{
  Définir un acide d'après Br\o{}nsted. \points{1}
}{
  C'est une espèce chimique qui peut libérer des ions \chemfig{H^+}.
  \points{1}
}[2]

\question{
  Rappeler le principe d'une dilution. \points{1}
}{
  Une dilution sert à diminuer la concentration en soluté dans une solution.
  Pour réaliser une dilution, il faut rajouter du solvant dans la solution.
  \points{1}
}[3]

\numeroQuestion
Indiquer sur l'échelle pH ci-dessous où se trouve les solutions acides, les solution basiques et les solutions neutres.
Puis placer sur cette échelle pH les solutions suivantes : \points{2}
\begin{tableau}{|c |c |c |c |c |c |}
  Solution & Acide éthanoïque & Acide chlorhydrique & Soude & Eau & Eau de mer \\
  pH       & \num{2,4} & \num{1,0} & \num{13,9} & \num{7,0} & \num{8,1} 
\end{tableau}

\pasCorrection{\vspace*{30pt}}
\begin{center}
  \begin{tikzpicture}
    \tikzVecteur (-0.75, 0) (15.5, 0) {pH}
    \foreach \i in {0,1,...,14} \draw (\i,0.1)--(\i,-0.1) node[below]{\i};
  \end{tikzpicture}
\end{center}