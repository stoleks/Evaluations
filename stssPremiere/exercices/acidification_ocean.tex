\exercice{L'acidification des océans (31 minutes)}

\titreSousSection{Acidification}

\medskip
Avant 1850, le pH des océans était autour de \num{8,2}.
Maintenant il est autour de \num{8,1} et des projections estiment qu'il sera autour de \num{7,9} en 2100.

\question{
 Expliquer pourquoi on parle « d'acidification des océans ».
 \points{2}
}{
  Le pH diminue et passe d'une valeur basique à une valeur neutre, donc il l'eau de mer s'acidifie.
}

On va chercher à comprendre pourquoi le pH des océans diminue et l'impact que cela aura sur la faune et la flore.


\medskip
Les émissions de dioxyde de carbone \chemfig{CO_2} liée aux activités humaines sont en hausses, ce qui augmente la quantité de \chemfig{CO_2} dans l'atmosphère.
Comme il y a plus de \chemfig{CO_2} dans l'atmosphère, la quantité de \chemfig{CO_2} qui est dissoute dans les océans augmente aussi.

Le dioxyde de carbone dissous dans l'eau, noté \chemfig{CO_2^{*}}, forme un couple acide/base avec l'ion hydrogénocarbonate : \chemfig{CO_2^{*}}/\chemfig{HCO_3^{-}}.

\question{
 Rappeler les deux couples acide/base que forme l'eau \eau.
 \points{2}
}{
  \oxonium/\eau et \eau/\hydroxyde.
}

\question{
 Écrire la réaction acido-basique entre le dioxyde de carbone dissous \chemfig{CO_2^{*}} et l'eau \eau, en faisant attention à bien ajuster l'équation.
 \points{2}
}{
 \begin{center}
   \chemfig{CO_2^{*}} + \eau \reaction \chemfig{HCO_3^{-}} + \oxonium
  \end{center}
}

\question{
 Rappeler la relation entre le pH et concentration en ion oxonium \oxonium.
 \points{1}
}{
  \begin{equation*}    
    10^{-\text{pH}} = [\oxonium]
  \end{equation*}
}

\question{
  Expliquer pourquoi cette réaction chimique entraine une diminution du pH des océans.
  \points{2}
}{
  Cette réaction produit des ions oxonium et augmente donc la concentration [\oxonium], ce qui diminue le pH.
}

% \titreSousSection{Émission de dioxyde de carbone}

% \medskip
% On va chercher à estimer la quantité de \chemfig{CO_2} qu'il faudrait dissoudre dans l'océan pour passer d'un pH de \num{8,1} à un pH de \num{7,9} en 2100.

% \question{
%   Calculer la concentration actuelle en ion oxonium dans les océans, avec un pH de \num{8,1}.
% }{}

% \question{
%   Calculer la concentration en ion oxonium quand le pH vaudra \num{7,9} en 2100.
% }{}

% \question{
%   Le volume total des océans vaut $V = \qty{1,37e21}{\litre}$, calculer la quantité de matière actuelle et en 2100 d'ion oxonium.
% }{}

% \question{
%   En déduire la quantité de matière en dioxyde de carbone \chemfig{CO_2} qu'il faudrait dissoudre dans les océans pour baisser le pH de 0,2.
% }{}

% \question{
%   Calculer la masse de \chemfig{CO_2} qu'il faudrait dissoudre.
%   \textbf{Donnée : } $M(\chemfig{CO_2}) = \qty{44,0}{\g\per\mole}$.
% }{}


\question{
  En 2022, l'humanité a émis environ \qty{47,5e12}{\kg} de \chemfig{CO_2} dans l'atmosphère. 
  Est-ce que tous le \chemfig{CO_2} émis a été dissous dans les océans ?
}{}


\titreSousSection{Conséquence sur la faune et la flore}

\medskip
Avec du calcaire, le carbonate de calcium \chemfig{CaCO_3}, dans une eau saturée en dioxyde de carbone dissous, on a la réaction chimique suivante :
\begin{equation*}
  \chemfig{CaCO_3}\sol + \chemfig{CO_2^{*}}\aq + \eau\liq
  \reaction
  \chemfig{Ca^{2+}}\aq + 2\chemfig{HCO_3^{2-}}\aq
\end{equation*}
Cette réaction chimique modélise la dissolution du calcaire dans l'eau.

Le calcaire est le matériau qui compose le squelette ou les coquilles de nombreuses espèces marines, comme les mollusques, les crustacés ou les coraux.
Les mollusques, les crustacés et les coraux forment la base de la chaîne alimentaire dans les océans.

\question{
  Est-ce que cette réaction chimique facilite la formation de squelettes ou de coquilles ?
  \points{1}
}{
  Elle ne facilite pas la formation de coquille ou de squelettes, car elle dissous leur principal composant (le calcaire).
}

\question{
  Expliquer pourquoi l'augmentation du dioxyde de carbone dissous dans l'océan entraine la mort de certaines espèces marines.
  \points{2}
}{
 L'augmentation du dioxyde de carbone rend plus difficile la formation de squelettes ou de coquilles, ce qui peut tuer les espèces quand elles grandissent.
}