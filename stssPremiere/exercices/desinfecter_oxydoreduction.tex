\exercice{Nettoyer une plaie infectée}

Pour nettoyer la plaie infectée qu'Isma s'est faite pendant une randonnée, ses ami-es vont dans une pharmacie pour demander un antiseptique.

\begin{doc}{\variationSujet{Bétadine}{Eau oxygénée}}
  La pharmacienne leur donne de \variationSujet{la bétadine}{l'eau oxygénée} pour nettoyer la plaie d'Isma.
  Le principe actif de \variationSujet{la bétadine}{l'eau oxygénée} est \variationSujet{le diiode \diiode}{la molécule \eauOxygenee}, de demi-équation d'oxydoréduction 
  \begin{equation*}
    \variationSujet
    {\diiode + 2\electron \reaction 2\iodure}
    {\eauOxygenee + 2\electron + 2\ionHydrogene \reaction 2\eau}
  \end{equation*}
\end{doc}

\question{
  Indiquer en justifiant si \variationSujet{le diiode}{l'eau oxygénée} est un oxydant ou un réducteur.
}{
  C'est un oxydant, car \variationSujet{le diiode}{l'eau oxygénée} peut gagner des électrons.
  \points{2}
}[2]

\question{
  Expliquer pourquoi \variationSujet{la bétadine}{l'eau oxygénée} est un antiseptique, et non un désinfectant.
}{
  C'est un antiseptique, car \variationSujet{la bétadine}{l'eau oxygénée} sert à nettoyer des tissus vivant.
  Un désinfectant sert à tuer les micro-organismes sur des objets inertes.
  \points{2}
}[3]


\begin{doc}{Action de \variationSujet{la bétadine}{l'eau oxygénée}}[\label{doc:E_action_betadine}]
  \variationSujet{Le diiode contenu}{L'eau oxygénée contenue} dans l'antispetique est capable de tuer les micro-organismes.
  La molécule pénètre très rapidement dans les bactéries, les virus et les champignons et détruit des protéines dans les cellules constitutives de ces micro-organismes, ce qui entraine leur mort.

  Les protéines possèdent des groupes hydrogénosulfures \chemfig{SH}, dont la demi-équation d'oxydoréduction est la suivante :
  \begin{equation*}
    2 \chemfig{SH} \reaction \chemfig{S_2} + 2\ionHydrogene + 2\electron
  \end{equation*}

  \important{La formation de ponts disulfure} \chemfig{S_2} dénature les protéines et \important{entraine la mort des cellules} qui les contiennent.
\end{doc}

\question{
  Indiquer en justifiant si le groupe hydrogénosulfure \chemfig{SH} est un oxydant ou un réducteur.
}{
  C'est un réducteur, car c'est une entité chimique capable de perdre des électrons.
  \points{2}
}[2]

\question{
  En sommant les demi-équations d'oxydoréduction \variationSujet{du diiode}{de l'eau oxygénée} et du groupe hydrogénosulfure, écrire la réaction d'oxydoréduction entre \variationSujet{\diiode}{\eauOxygenee} et \chemfig{SH}.
}{
  On somme terme à terme (côté gauche + côté gauche = côté droit + côté droit) :
  \begin{align*}   
    \variationSujet{%
    \diiode + 2\electron + 2\chemfig{SH} &\reaction
    2\iodure + \chemfig{S_2} + 2\ionHydrogene + 2\electron \\
    \diiode + 2\chemfig{SH} &\reaction
    2\iodure + \chemfig{S_2} + 2\ionHydrogene
    }{%
    \eauOxygenee + 2\ionHydrogene + 2\chemfig{SH} + 2\electron &\reaction
    2\iodure + \chemfig{S_2} + 2\ionHydrogene + 2\electron \\
    %
    \eauOxygenee + 2\chemfig{SH} &\reaction
    2\iodure + \chemfig{S_2} \\
    }
  \points{4}
  \end{align*}
}[6]

\question{
  En vous aidant des produits de la réaction d'oxydoréduction et du document~\ref{doc:E_action_betadine}, expliquer comment \variationSujet{le diiode}{l'eau oxygénée} tue les micro-organismes.
}{
  Le diiode vient oxyder les groupements \chemfig{SH} de la protéine pour former des ponts disulfure \chemfig{S_2}, ce qui entraine la mort des cellules.
  \points{2}
}[4]

\question{
  Un autre antiseptique proposé à la pharmacie est \variationSujet{l'eau oxygénée \eauOxygenee}{la bétadine \diiode}, avec la demi-équation d'oxydoréduction suivante :
  \begin{equation*}
    \variationSujet
    {\eauOxygenee + 2\electron + 2\ionHydrogene \reaction 2\eau}
    {\diiode + 2\electron \reaction 2\iodure}
  \end{equation*}
  Donner la réaction d'oxydoréduction entre \variationSujet{\eauOxygenee}{\diiode} et \chemfig{SH}.
}{
  On somme les deux demi-équations :
  \begin{align*}
    \variationSujet{%
    \eauOxygenee + 2\ionHydrogene + 2\chemfig{SH} &\reaction
    2\iodure + \chemfig{S_2} + 2\ionHydrogene \\
    %
    \eauOxygenee + 2\chemfig{SH} &\reaction
    2\iodure + \chemfig{S_2} \\
    }{%
    \diiode + 2\electron + 2\chemfig{SH} &\reaction
    2\iodure + \chemfig{S_2} + 2\ionHydrogene + 2\electron \\
    \diiode + 2\chemfig{SH} &\reaction
    2\iodure + \chemfig{S_2} + 2\ionHydrogene
    }
  \end{align*}
}[6]

\question{
  Y-a-t'il une différence d'action entre les deux antiseptiques proposés ?
}{
  Non, dans les cas l'espèce chimique oxydante vient former des pont disulfures qui vont dénaturer les protéines et tuer les cellules.
}[3]