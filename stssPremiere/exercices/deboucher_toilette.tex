\exercice{Déboucher des toilettes}\points{14}
\medskip

\begin{wrapfigure}[3]{r}{0.1\linewidth}
  \centering
  \vspace*{-24pt}
  \image{1}{images/pictogrammes/picto_corrosif}
\end{wrapfigure}
Pour déboucher des toilettes, Ahmed a lu qu'il fallait utiliser une solution \important{basique}.
Il dispose d'une solution d'acide éthanoïque, d'une solution d'acide chlorhydrique, d'eau et d'une solution de soude.
Un pictogramme est présent sur la bouteille de soude, il est présenté à droite.
Il mesure le pH des solutions pour savoir laquelle utiliser et trouve les résultats suivants :

\correction{\vspace*{-8pt}}
\begin{tableau}{|c |c |c |c |c |}
  Produit & Acide éthanoïque & Acide chlorhydrique & Soude & Eau \\
  pH      & \num{2,4} & \num{1,0} & \num{13,9} & \num{7,0} 
\end{tableau}

\numeroQuestion
Placer sur l'échelle de pH ci-dessous les produits utilisé par Ahmed, en indiquant où se trouve les solutions acides, basiques et neutres.
\correction{\points{2}}

\pasCorrection{\vspace*{50pt}}
\begin{center}
  \begin{tikzpicture}
    \tikzVecteur(-0.75, 0) (15.5, 0) {pH}
    \foreach \i in {0,1,...,14} \draw (\i,0.1)--(\i,-0.1) node[below]{\i};
  \end{tikzpicture}
\end{center}
\pasCorrection{\vspace*{50pt}}

\question{
  Rappeler la relation entre le pH et la concentration en ion oxonium \oxonium.
}{
  \begin{equation*}
    [\oxonium] = 10^{-pH}
    \points{1}
  \end{equation*}
}[2]

\question{
  Calculer la concentration en ion oxonium dans la soude et dans l'acide chlorhydrique.
}{
  \begin{align*}  
    [\oxonium]_\text{soude} &= 10^{-13,9} = \qty{1,26e-14}{\mole/\litre} \\
    [\oxonium]_\text{acide chlorhydrique} &= 10^{-1,0} = \qty{1,0e-1}{\mole/\litre}
    \points{2}
  \end{align*}
}[4]

\question{
  Rappeler la signification du pictogramme de sécurité présent sur la bouteille de soude.
}{
  C'est le symbole qui désigne les produits corrosifs,
  qui peuvent ronger la peau et/ou les yeux en cas de contact.\points{2}
}[3]


\question{
  Donner la définition d’une espèce chimique acide selon Br\o{}nsted.
}{
  C'est une espèce chimique susceptible de libérer un proton, c'est-à-dire un ion \chemfig{H^+}.\points{1,5}
}[4]

\numeroQuestion
Dans les couples acido-basiques ci-dessous, entourer d'une couleur l'espèce chimique acide du couple et d'une autre couleur la base.
\begin{enumerate}[label = (\alph*)]
  \item \chemfig{NH_4^+}   / \chemfig{NH_3}
  \item \chemfig{HCl}      / \chemfig{Cl^{-}}
  \item \chemfig{CH_3COOH} / \chemfig{CH_3COO^{-}}
  \correction{\points{1,5}}
\end{enumerate}


\question{
  Établir la réaction acido-basique entre l'ammoniac de formule brute \chemfig{NH_3} et l'acide éthanoïque \chemfig{CH_3COOH}.
}{
  \begin{equation*}  
    \chemfig{NH_3} + \chemfig{CH_3COOH} \reaction \chemfig{NH_4^+} + \chemfig{CH_3COO^{-}}
    \points{1}
  \end{equation*}
}[3]

\question{
  Établir les réactions acido-basique avec l'eau \chemfig{H_2O} et les espèces chimiques suivantes : \chemfig{CH_3COOH}, \chemfig{HCl}, \chemfig{NH_3}.

  \important{Données :} Les couples acide/base de l'eau sont
  \chemfig{H_2O}/\hydroxyde et \oxonium/\chemfig{H_2O}
}{
  \begin{align*}
    \chemfig{H_2O} + \chemfig{CH_3COOH} &\reaction \oxonium + \chemfig{CH_3COO^{-}} \\
    \chemfig{H_2O} + \chemfig{NH_3} &\reaction \hydroxyde + \chemfig{NH_4^+} \\
    \chemfig{H_2O} + \chemfig{HCl} &\reaction \oxonium + \chemfig{Cl^{-}}
    \points{3}
  \end{align*}
}[7]