\exercice{Détection et correction de problèmes de vision}\points{7}

Vous êtes infirmièr-e scolaire et des parents inquiets viennent vous voir. Leur fille de 6 ans, Malala, \variationSujet{se penche}{s'éloigne} beaucoup \variationSujet{sur}{de} ses cahiers de dessin et iels pensent que c'est lié à sa vision.


%%%%
\begin{doc}{Schéma de la position de la feuille par rapport à l’œil de Malala quand elle voit flou}[\label{doc:E_schema_feuille}]
  \centering
  \variationSujet
  {\image{0.9}{images/detection_defaut_vision_myope}}
  {\image{0.9}{images/detection_defaut_vision_hypermetrope}}
\end{doc}

\begin{doc}{Défauts de la vision}
  \begin{listePoints}
    \item \important{la myopie,} qui empêche de voir net de loin. À cause d'un défaut dans la forme de l'œil ou du cristallin, la formation des images se fait avant la rétine pour un objet lointain. Pour un objet proche, elle se forme sur la rétine.
    \item \important{l'hypermétropie,} qui empêche de voir net de près, pour des raisons inverses de la myopie, l'image est formée après la rétine pour un objet proche. Pour un objet lointain, elle se forme sur la rétine.
  \end{listePoints}
\end{doc}

\question{
  Indiquer quelles parties de l’œil sont schématisées par la lentille et par l'écran.
  \points{2}
}{

}

\question{
  En traçant des rayons lumineux particuliers sur le document 1, trouver où se forme l'image $A'B'$ de l'objet $AB$.
  \points{3}
}{

}

\question{
  Commenter la position de l'image obtenue pour expliquer pourquoi elle voit flou.
  \points{1}
}{

}

\question{
  Nommer le défaut de la vision dont souffre Malala.
  \points{1}
}{

}