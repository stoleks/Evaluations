\enTeteFiche{\premStssChim}

\begin{tableauConnaissances}
  Je sais calculer la masse molaire d'une molécule à partir de la formule brute de la molécule et des masse molaire atomique.
  & & & \\
  %
  Je connais les protocoles pour préparer une solution par dissolution ou par dilution.
  Je connais la définition du facteur de dilution.
  & & &  \\
  %
  Je connais la relation entre la quantité de matière dans un échantillon, la masse de l'échantillon et la masse molaire de l'échantillon $n = m / M$.
  Je sais calculer une masse à partir de cette relation.
  & & & \\
  %
  Je connais la relation qui définit la concentration molaire $c = n / V$ et celle qui définit la concentration massique $c_m = m / V$.
  Je connais les unités et le sens de ces grandeurs. 
  & & & \\
  %
  Je connais les pictogrammes de sécurités et leur signification.
  & & & \\
  %
  Je sais reconnaître si une solution est acide, neutre ou basique à partir de la mesure de son pH.
  & & & \\
  %
  Je connais la relation entre le pH et la concentration en ion oxonium \oxonium{}.
  & & & \\
  %
  Je connais la définition d'un acide et d'une base de Br\o{}nsted.
  & & & \\
  %
  Je sais lire la notation acide/base (où est l'acide et où est la base).
  & & & \\
  % 
  Je sais établir une réaction chimique acido-basique à partir des espèces chimiques présentes en solution et des couples acide/base associés à chaque espèce.
  & & & \\
  % 
  Je connais les deux couples acide/base de l'eau.
  Je connais la réaction d'autoprotolyse de l'eau.
  & & & \\
  %
\end{tableauConnaissances}


\basDePageFicheReussite

\questionFicheReussite{3}
