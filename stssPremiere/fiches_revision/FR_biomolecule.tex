\enTeteFiche{\premStssStru}

\begin{tableauConnaissances}
  Je sais qu'une molécule organique est composée majoritairement
  de Carbone \chemfig{C} et d'hydrogène \chemfig{H},
  avec un peu d'oxygène \chemfig{O} et d'azote \chemfig{N}.
  & & & \\
  %
  Je connais la valence du carbone, de l'hydrogène, de l'oxygène et de l'azote.
  & & & \\
  %
  Je connais les représentations des molécules organiques : brute, développée, semi-developpée et topologique.
  & & & \\
  % 
  Je connais et je sais identifier dans une molécules les fonctions
  alcool, acide carboxylique, aldéhyde, cétone, amine, amide, ester et étheroxyde.
  & & & \\
  %
  Je sais nommer des alcanes, des alcools, des acides carboxyliques et des carbonylés avec un squelette de 6 carbones ou plus.
  & & & \\
  %
  Je peux identifier les fonctions organiques présentes dans un glucide.
  & & & \\
  %
  Je sais que le glucose et le fructose existe sous forme linéaire ou cyclique.
  & & & \\
  %
  Je connais la différence entre un glucide simple et un glucide complexe.
  & & & \\
  %
  Je connais la définition d'un acide gras et d'un triglycéride,
  je peux identifier leurs fonctions organiques.
  & & & \\
  %
  Je sais distinguer un acide gras saturé et insaturé.
  & & & \\
  %
  Je connais la définition d'un acide aminé.
  & & & \\
  %
  Je sais identifier une liaison peptidique et je peux identifier les acides aminés constituant un polypeptide.
  & & & \\
  %
  Je sais qu'une protéine est un polypeptide avec une structure repliée particulière.
  & & & \\
\end{tableauConnaissances}


\basDePageFicheReussite

\questionFicheReussite{3}
