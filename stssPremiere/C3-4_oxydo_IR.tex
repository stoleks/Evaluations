\enTeteFicheReussite{3 \& 4}{Oxydoréduction et infrarouge}

\begin{tableauConnaissances}
  Je sais qu'une réaction chimique modélise une transformation macroscopique.
  & & & Activité 3.0 \\
  % 
  Je sais où se trouve les réactifs et les produits dans une réaction chimiques.
  Je sais ajuster une réaction chimique.
  & & & Activité 3.0 \\
  %
  Je sais qu'une réaction d'oxydoréduction correspond à des échanges d'électrons.
  Je sais définir un oxydant et un réducteur.
  & & & Activité 3.1 \\
  %
  Je sais écrire une réaction d'oxydoréduction à partir des demi-équations d'oxydoréduction.
  & & & Activité 3.1, 3.2 et 3.3 \\
  %
  Je connais la définition d'un antiseptique et d'un désinfectant et je connais leur différence d'utilisation.
  & & & Activité 3.2 et 3.3 \\
  %
  Je connais les précautions d'utilisations pour les antiseptiques et les désinfectants.
  & & & Activité 3.3 \\
  %
  Je sais que la lumière est une onde électromagnétique, dont les propriétés dépendent de sa longueurs d'onde.
  & & & Activité 4.1 \\
  %
  Je connais les longueurs d'onde du domaine visible, infrarouge et ultraviolet.
  & & & Activité 4.1 \\
  %
  Je connais la loi du rayonnement thermique des corps chauds. 
  Je sais que plus un corps a une température élevée, plus il émet de la lumière.
  & & & Activité 4.1 \\
  %
  Je connais la loi de Wien et je sais l'utiliser pour calculer la température d'un objet qui émet de la lumière.
  & & & Activité 4.1, 4.2 et 4.3 \\
  %
  Je sais qualitativement comment fonctionne un thermomètre sans contact.
  & & & Activité 4.2 \\
  %
  Je connais les risques associés aux infrarouges.
  & & & Activité 4.3 \\
  %
\end{tableauConnaissances}


\basDePageFicheReussite

\coursFicheReussite