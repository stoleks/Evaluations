\enTeteFiche{2}{\premStssVisi}

\begin{tableauConnaissances}
  Je sais que la lumière se propage en ligne droite dans un milieu homogène et transparent.
  & & & \\
  % 
  Je connais les principaux composants optiques de l'oeil,
  et je connais leur rôle : cornée, iris, cristallin, rétine et nerf optique.
  & & & \\
  %
  Je connais le modèle optique de l'oeil.
  Je connais les convention d'optique : centre optique O, foyer objet F et foyer image F'.
  & & & \\
  %
  Je sais distinguer une lentille convergente et divergente.
  & & & \\
  %
  Je peux tracer le trajet suivi par des rayons lumineux passant par O, F et F' pour une lentille convergente ou divergente.
  & & & \\
  %
  Je peux construire géométriquement l'image d'un objet par une lentille convergente.
  & & & \\
  %
  Je peux former une image réelle avec une lentille convergente.
  Je sais dans quel cas on a une loupe avec une image virtuelle.
  & & & \\
  %
  Je peux mesurer le grandissement d'un objet par une lentille.
  & & & \\
  %
  Je peux expliquer le principe de l'accommodation dans un oeil humain.
  Je sais pourquoi en vieillissant on devient presbyte.
  & & & \\
  %
  Je connais la définition d'un oeil myope ou hypermétrope.
  Je sais comment corriger la myopie ou l'hypermétropie avec des verre correcteur.
  & & & \\
  %
  Je connais l'expression de la vergence d'un système de deux lentilles minces accolées.
  Je sais l'utiliser pour calculer une correction de la vision.
  & & & \\
  %
\end{tableauConnaissances}


\basDePageFicheReussite

\questionFicheReussite{3}
