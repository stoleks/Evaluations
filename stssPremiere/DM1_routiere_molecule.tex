\newpage
\vAligne{-50pt}
\titre{Devoir maison de physique-chimie}

\textit{Les exercices sont indépendants.}
\medskip

\exercice{l'acide lactique (8 points)}
\medskip

\begin{wrapfigure}{r}{0.2\linewidth}
  \vspace*{-20pt}
  \centering
  \chemfig{HO -[-2] CH (-[-4] H_3C) - C (-[-2] OH) =[2] O}
\end{wrapfigure}
Les maladies mitochondriales peuvent provoquer une acidose lactique qui est une surproduction d’acide lactique pouvant entraîner une acidification du sang et des tissus générant des troubles cardiaques.  

La formule semi-développée de l’acide lactique est donnée ci-contre. 
 
\question{
  Écrire la formule brute de l'acide lactique et calculer la masse molaire $M_a$ de l'acide lactique.
}{
  Formule brute de l'acide lactique : \bruteCHO{3}{6}{3} \points{1}
  \begin{equation*}  
    M_a = 3\times M_C + 6\times M_H + 3\times M_O
    = (3\times 12 + 6\times 1,0 + 3\times 16)\unit{\g\per\mole}
    = \qty{90}{\g\per\mole}
    \points{2}
  \end{equation*}
}{0}

%%
\medskip
\textbf{Données :} 

Masses molaires atomiques 
$M_C = \qty{12}{\g\per\mole}$ ;
$M_O = \qty{16}{\g\per\mole}$ ;
$M_H = \qty{1,0}{\g\per\mole}$.
\medskip
%%

\question{
  Recopier la formule semi-développée de l’acide lactique sur la copie.
  Entourer et nommer les groupes fonctionnels présents dans cette molécule.
}{
  \begin{center}
    \begin{tikzpicture}[help lines/.style={thin,draw=black!50}]
      % chaine principale
      \large
      \node at (3,3) { \chemfig{HO -[-2] CH (-[-4] H_3C) - } };
      % alcool
      \node[draw] at (2.35, 3.77) { \chemfig{HO} };
      \node[left] at (1.8, 3.8) {\textbf{1}};
      % acide carboxylique
      \node[draw] at (4.8,3) { \chemfig{C (-[-2] OH) =[2] O} };
      \node[right] at (5.6,3) {\textbf{2}};
    \end{tikzpicture}
    \points{1}
  \end{center}

  1 : hydroxyle ;
  2 : carboxyle.
  (1/4 des points est attribué si le nom de la famille est donné à la place du nom du groupe)
  \points{2}
}{0}

\question{
  Donner la formule topologique de l'acide lactique.
}{}{0}



\bigskip
\exercice{Accident de voiture (12 points)}
\setcounter{questionNum}{0}
\medskip

\begin{wrapfigure}[12]{r}{0.5\linewidth}
  \vspace*{-26pt}
  \centering
  \image{1}{stssPremiere/freinage}
\end{wrapfigure}

Appelés sur le lieu d’un accident de la route, des policiers constatent qu’une voiture a percuté frontalement un arbre et que le conducteur, qui était seul à bord, n’est blessé que légèrement.
L’airbag qui s’est déclenché au moment du choc a très probablement sauvé la vie du chauffeur.

Interrogé par les policiers, le conducteur explique qu'il a perdu le contrôle de son véhicule et qu'il a essayé de freiner avant de percuter l'arbre.

Sa voiture a enregistré sa vitesse quelques secondes avant et après le choc.


\question{
  À quelle vitesse roulait le conducteur initialement ?
}{}{0}

\question{
  Le temps de réaction d'un humain typique est de 1 seconde.
  le conducteur a-t-il un temps de réaction normal ? 
}{}{0}

\question{
  Quelle vitesse avait la voiture au moment du choc ? Donner la valeur en \unit{\m\per\s}.
}{}{0}

\question{
  Calculer l'énergie cinétique au moment du choc.
}{}{0}

\textbf{Données :}
Masse de la voiture : $m = \qty{1000}{\kg}$

\question{
  Sachant que les os du thorax peuvent se briser si l'énergie cinétique du véhicule est supérieure à \qty{60}{\kilo\joule}, commenter la phrase « l'airbag [...] a très probablement sauvé la vie du chauffeur. ».
}{}{0}