% \enTeteFicheReussite{1}{\premStssChim}
\titre{Fiche « réussir son devoir commun »}

\titreSection{Ce que je dois savoir}
  
Pour savoir quoi réviser, je lis les points clés du chapitre évalués :
\begin{itemize}
  \item Si je pense maîtriser une notion, je coche la case \ok
  \item Si je pense que je dois la retravailler, je coche la case \pasOk
\end{itemize}


\begin{tableauConnaissances}
  Je sais la différence entre une espèce chimique et une entité chimique.
  & & & Activité 1.1 \\
  % 
  Je connais la définition d'une mole et la valeur du nombre d'Avogadro.
  & & & Activité 1.1 \\
  %
  Je sais calculer une masse molaire à partir de la formule brute d'une molécule et des masse molaire atomique.
  & & & Activité 1.1, TP 1.2 \\
  %
  Je connais la définition d'une solution.
  Je connais les protocoles pour préparer une solution par dissolution ou par dilution.
  Je connais la définition du facteur de dilution.
  & & & Activité 1.2, TP 1.1, TP 1.2, TP 1.3 \\
  %
  Je connais la relation entre la quantité de matière dans un échantillon, la masse de l'échantillon et la masse molaire de l'échantillon $n = m/M$.
  Je sais calculer une masse à partir de cette relation.
  & & & Activité 1.1, TP 1.2 \\
  %
  Je connais la relation qui définit la concentration molaire $c = n / V$.
  Je connais les unités et le sens de ces grandeurs. 
  & & & Activité 1.2, TP 1.2 \\
  %
  Je connais les pictogrammes de sécurités et leur signification.
  & & & TP 1.3 \\
  %
  Je sais reconnaître si une solution est acide, neutre ou basique à partir de la mesure de son pH.
  & & & TP 1.4 \\
  %
  Je connais la relation entre le pH et la concentration en ion oxonium.
  & & & TP 1.4 \\
  %
  Je connais la définition d'un acide et d'une base de Br\o{}nsted.
  & & & Activité 1.3 \\
  %
  Je sais lire la notation acide/base (où est l'acide et où est la base).
  & & & Activité 1.3 \\
  % 
  Je sais établir une réaction chimique acido-basique à partir des espèces chimiques présentent en solution et des couples acide/base associés à chaque espèce.
  & & & Activité 1.3 \& 1.4 \\
  % 
  Je connais les deux couples acide/base de l'eau.
  Je connais la réaction d'autoprotolyse de l'eau.
  & & & Activité 1.3 \& 1.4 \\
  %
  Je connais les caractéristiques de la tension du secteur (tension efficace et fréquence).
  & & & Activité 2.3 \\
  % 
  Je sais lire un oscillogramme pour trouver les caractéristique d'une tension alternative.
  & & & Activité 2.3
\end{tableauConnaissances}


% \basDePageFicheReussite

% \questionFicheReussite{3}
