\newpage

\begin{flushright}
  \large 
  Sujet 2024
\end{flushright}

\vspace{20pt}
\titre{
  DEVOIR COMMUN N°1 bis
}
\bigskip

\begin{boite}
  \titre{
    ÉPREUVE DE PHYSIQUE-CHIMIE \\ POUR LA SANTE \\
  }
\end{boite}
\bigskip

\begin{center}
  \Large 
  Durée de l'épreuve : 1 heure

  \bigskip

  Jeudi 21 novembre 2024
\end{center}

\vspace{20pt}
{\large
  Le sujet comporte 4 pages numérotées de 1 à 4.
  
  Dès que le sujet vous est remis, assurez-vous qu'il est complet.
  
  La page 2 est à compléter directement sur le sujet à rendre avec la copie.
  \bigskip

\begin{center}
\begin{tblr}{
    colspec = {X[c]|X[c]}, hlines,
    width = 0.5\linewidth
    %row{1} = {couleurPrim!20}
  }
  Exercice 1 & 10 points \\
  % Exercice 2 & 6 points \\
  Exercice 2 & 10 points \\
 % Exercice 3 & 12 points \\
  Exercice 3 & 7 points \\
\end{tblr} \\[8pt]

Présentation : 1 point
\end{center}
  

\vfill 
  \textit{Les exercices sont indépendants.} \bigskip

  \textit{L'usage de la calculatrice est autorisé.} \bigskip

  \textit{
  La clarté des raisonnements et la qualité de la rédaction interviendront pour une part importante dans l'appréciation des copies.
  }
}


%%%%
\newpage
\vspace*{-50pt}
\exercice{Automatismes (connaître son cours) }
\medskip

\numeroQuestion
Relier chaque grandeur à son symbole et à l'unité associée. \points{3}

\begin{center}    
  \begin{tblr}{
    width=\textwidth,
    colspec = {Q[wd=0.25\linewidth, l] X[l] r Q[wd=0.1\linewidth, c] X[l] r Q[wd=0.1\linewidth, c]},
    row{1} = {couleurPrim!20, c}
  }
  %% ALIGNER PLUS DE GRANDEUR 
    \textbf{Grandeur}      & & & \textbf{Symbole} & & & \textbf{Unité} \\
    Concentration massique & \pointCyan & \pointCyan & $n$   & \pointCyan & \pointCyan &  \unit{\mole\per\litre} \\
    Masse                  & \pointCyan & \pointCyan & $C$   & \pointCyan & \pointCyan &  \unit{\litre} \\
    Quantité de matière    & \pointCyan & \pointCyan & $C_m$ & \pointCyan & \pointCyan &  \unit{\kg} \\
    Volume                 & \pointCyan & \pointCyan & $V$   & \pointCyan & \pointCyan &  \unit{\g\per\litre} \\
    Concentration molaire  & \pointCyan & \pointCyan & $M$   & \pointCyan & \pointCyan &  \unit{\g\per\mole} \\
    Masse molaire          & \pointCyan & \pointCyan & $m$   & \pointCyan & \pointCyan &  \unit{\mole} \\
  \end{tblr}
\end{center}

\numeroQuestion
Entourer la ou les relations permettant de calculer les grandeurs de la première colonne. \points{2}
\vspace*{-12pt}

\begin{center}
  \begin{tblr}{
    colspec = {|X[l,m]| X[c,m]| X[c,m]| X[c,m]| X[c,m]|}, hlines,
    column{1} = {couleurPrim!20}
  }
    Pour calculer une quantité de matière &
    $\dfrac{m}{M}$ & $m\times M$ & $C \times V$ & $\dfrac{V}{C}$ \\
    %
    Pour calculer une concentration molaire &
    $\dfrac{m}{V}$ & $\dfrac{n}{V}$ & $m \times M$ & $\dfrac{M}{m}$ \\
    %
    Pour calculer la concentration en ion oxonium &
    $10^{-\text{pH}}$ & $10^{\text{pH}}$ & $10 - \text{pH}$ & $\text{pH}^{10}$ \\
    %
    Pour calculer une masse &
    $M \times n$ & $C_m \times V$ & $m \times M$ & $\dfrac{M}{m}$ \\
    %
  \end{tblr}
\end{center}

\numeroQuestion Donner le nom de ces deux ions \points{1}
\begin{center}  
  \oxonium : \texteTrou[0.1]{ion oxonium} \qq{}
  et \hydroxyde : \texteTrou[0.1]{ion hydroxyde}
\end{center}

\question{
  Définir un acide d'après Br\o{}nsted. \points{1}
}{
  C'est une espèce chimique qui peut libérer des ions \chemfig{H^+}.
  \points{1}
}{2}

\question{
  Rappeler le principe d'une dilution. \points{1}
}{
  Une dilution sert à diminuer la concentration en soluté dans une solution.
  Pour réaliser une dilution, il faut rajouter du solvant dans la solution.
  \points{1}
}{3}

\numeroQuestion
Indiquer sur l'échelle pH ci-dessous où se trouve les solutions acides, les solution basiques et les solutions neutres.
Puis placer sur cette échelle pH les solutions suivantes : \points{2}
\begin{tableau}{|c |c |c |c |c |c |}
  Solution & Acide éthanoïque & Acide chlorhydrique & Soude & Eau & Eau de mer \\
  pH       & \num{2,4} & \num{1,0} & \num{13,9} & \num{7,0} & \num{8,1} 
\end{tableau}

\pasCorrection{\vspace*{30pt}}
\begin{center}
  \begin{tikzpicture}
    \tkzVecteur(-0.75)[15.5](0){pH}
    \foreach \i in {0,1,...,14} \draw (\i,0.1)--(\i,-0.1) node[below]{\i};
  \end{tikzpicture}
\end{center}
\newpage


%%%%
%\bigskip
% \exercice{Charger son téléphone au Canada (14 minutes)}
% \medskip

% Asma est partie en vacances au Canada et veut charger son téléphone.
% Pour ça, Asma commence par mesurer les caractéristiques de la tension du secteur, et trouve l'oscillogramme suivant :
% \begin{center}
%   \def\ver{3.6} % longueur verticale
%   \def\hor{5.0} % longueur horizontale
%   \begin{tikzpicture}[
%     trace/.style={couleurPrim!75!black, ultra thick, samples = 100},
%     screen/.style={couleurPrim!10, thick},
%     axes/.style={couleurPrim, thick}
%   ]
%     % Fond de l'écran
%     \fill[screen] (-\hor, -\ver) rectangle (\hor, \ver);
%     % Grille et axes
%     \draw[thin, couleurPrim!30] (-\hor, -\ver) grid (\hor, \ver);
%     \draw[axes] (-\hor, 0) -- (\hor, 0); % temps
%     \draw[axes] (0, -\ver) -- (0, \ver); % 
%     % Graduation
%     \pgfmathparse{\hor-1}
%     % axe x
%     \foreach[parse = true] 
%       \i in {-\hor+0.2, -\hor+0.4,..., \hor-0.2} \draw[axes] (\i, -0.1) -- (\i, 0.1);
%     \foreach[parse = true] \i in {-\hor,...,\hor} \draw[axes] (\i,-0.2) -- (\i,0.2);
%     % axe y
%     \foreach[parse = true]
%       \i in {-\ver+0.20, -\ver+0.4,..., \ver-0.20} \draw[axes] (-0.1, \i) -- (0.1, \i);
%     \foreach[parse = true] \i in {-3,...,3} \draw[axes] (-0.2,\i) -- (0.2,\i);
%     % Tension sinusoïdale
%     \draw[trace] plot(\x, {1.60*sin((1.57*(1.21*\x + 1)) r)}); % r = radian
%     % Échelle
%     \tkzVecteur(-5) (-3)[1]  {\textbf{100 V}} [right]
%     \tkzVecteur(-5) [1] (-3)  {\textbf{5 ms}} [below right]
%   \end{tikzpicture}
% \end{center}

% \question{
%   Mesurer la tension maximale $U_\text{max}$ et en déduire la tension efficace du secteur $U$. 
% }{
%   $U_\text{max} = \qty{160}{\volt}$, donc $U = \dfrac{160}{\sqrt{2}} = \qty{113}{\volt}$.
%   \points{2}
% }{0}

% \question{
%   Mesurer la période $T$ et en déduire la fréquence $f$ de la tension du secteur. 
% }{
%   $T = \qty{16,5}{\ms}$, donc $f = \dfrac{1}{T} = \qty{50}{\hertz}$.
%   \points{2}
% }{0}

% \question{
%   Rappeler la fréquence et la tension efficace de la tension du secteur en France.
% }{
%   $U = \qty{230}{\volt}$ et $f = \qty{60}{\hertz}$.
%   \points{1}
% }{0}

% \question{
%   Asma pourra-t-elle utiliser un chargeur conçu pour être utilisé sur la tension du secteur en France au Canada ? Justifier.
% }{
%   Oui, car la fréquence est identique et la tension du secteur au Canada est plus faible que la tension du secteur en France.  \points{1}
% }{0}


%%%%
\vspace{24pt}
\exercice{L'acide ascorbique }

\medskip
Les vivres embarquées sur les navires européens du XVe au XVIIIe siècle étaient essentiellement des salaisons, des légumes secs et des biscuits pour des raisons liées à la conservation des aliments. Or ceux-ci ne contenant pas de vitamine C (aussi appelée acide ascorbique), il en a résulté des cas de scorbut lors des expéditions de longues distance (plus de six semaines à plus de trois mois) en mer.

\medskip
Les marins atteints du scorbut subissaient des gingivites hypertrophiques (entraînant le gonflement des gencives et le déchaussement des dents) et des syndromes hémorragiques et cutanés (pertes des cheveux et hématomes) si bien qu'après un voyage de 3 mois (pour ceux qui y survivaient), ces derniers semblaient avoir vieilli de vingt ans. 

\medskip
En 2019, le scorbut est réapparu en France chez des personnes précaires ne se nourrissant pas suffisamment de légumes et fruits frais.

\medskip
%On considère que le taux en vitamine C par litre de sang doit être compris entre \num{5}  et \num{17}  \unit{\mg\per\litre} pour être en bonne santé.

\medskip
\textbf{Donnée : } 
% \textit{Formule brute de la vitamine C :} \chemfig{C_6H_8O_6}

\textit{Masse molaire atomique : $M(H) = \qty{1} {\g\per\mole}$ $M(C) = \qty{12} {\g\per\mole}$ $M(N) = \qty{14} {\g\per\mole}$ $M(O) = \qty{16} {\g\per\mole}$}

\textit{ $\qty{1} {g} = \qty{1 000} {mg}$   ;  $\qty{1} {L} = \qty{1 000} {\ml}$ }

\bigskip
\question{
  Quelle espèce chimique permet de lutter contre le scorbut ? Dans quels aliments la trouve-t-on ?               \points{2}
}{
  C'est la vitamine C qui permet de lutter contre le scorbut, qu'on trouve dans les fruits et légumes frais. \points{2}
}{0}

La concentration massique en vitamine C dissout dans un jus de pomme est $C_m = \qty{9,0e-3}{\g\per\litre}$.

\question{
  Quel est le soluté ? Quelle est la solution ?    \points{2}
}{
  Le soluté est la vitamine C, la solution est le jus de pomme.\points{2}
}{0}

\question{
  Calculer la masse de vitamine C présente dans une bouteille de volume $V = \qty{1,5} {\litre}$ de jus de pomme. \points{1,5}
}{
  \begin{equation*}
    m(\chemfig{C_6H_8O_6}) = C_m \times V = \qty{9,0e-3}{\g\per\litre} \times \qty{1,5}{\litre} = \qty{13,5e-3}{\g}
    \points{1,5}
  \end{equation*}
}{0}

\question{
  La formule brute de la vitamine C est \chemfig{C_6H_8O_6}. Calculer la masse molaire moléculaire de la vitamine C.  \points{1,5}
}{
  \begin{equation*}
    M(\chemfig{C_6H_8O_6}) = 6M(H) + 8M(C) + 6M(O) = \qty{176}{\g\per\mole}
  \points{1,5}
  \end{equation*}
}{0}

\question{
  Déterminer la quantité de matière en vitamine C présent dans un volume de $\qty{1,5} {\litre}$ de jus de pomme.  \points{1,5}
}{
  \begin{equation*}
    n = \dfrac{m(\chemfig{C_6H_8O_6})} {M(\chemfig{C_6H_8O_6})}
    = \dfrac{\qty{13,5e-3} {\g}} {\qty{176} {\g\per\mole}}
    = \qty{7,67e-5} {\mole}
  \points{1,5}
  \end{equation*}
}{0}

\question{
  En déduire la concentration molaire en vitamine C dans le jus de pomme.                                \points{1,5}
}{
  \begin{equation*}
    c = \dfrac{n}{V}
    = \dfrac{\qty{7,67e-5}{\mole}} {\qty{1,5}{\litre}}
    = \qty{5,11e-5} {\mole\per\litre}
  \points{1,5}
  \end{equation*}
}{0}

\medskip
%Camélia se sent fatiguée et se voit prescrire de la vitamine C. Elle doit prendre chaque jour un comprimé contenant une masse $m = \qty{1000} {\mg}$ de vitamine C. 
%Un comprimé se dissout dans un verre d'eau de volume $V = \qty{200} {\ml}$.

%\question{
 % On suppose que le comprimé se dissout entièrement. Déterminer la concentration massique de vitamine C dans son verre.                                          \points{1,5}
%}{
 % \begin{equation*}
  %  C_m = \dfrac{m}{V}
   % = \dfrac{\qty{1000} {\mg}} {\qty{200} {\ml}}
    %= \dfrac{\qty{1} {\g}} {\qty{0,2} {\litre}}
    %= \qty{5,0} {\g\per\litre}
  %\points{1,5}
  %\end{equation*}
%}{0}

%\question{
 % Une fois ingérée et assimilée par l'organisme, la concentration molaire sanguine de vitamine C de Camélia atteint $\qty{3,6e-5} {\mol\per\litre}$.
  %A-t-elle un taux de vitamine C suffisant pour être en bonne santé ?                                       \points{3}
%}{
 % Pour obtenir la concentration massique, on multiplie la concentration molaire par la masse molaire moléculaire de la vitamine C
 % \begin{equation*}
  %  C_m = M(\chemfig{C_6H_8O_6}) \times c
   % = \qty{176} {\g\per\mole} \times \qty{3,6e-5} {\mol\per\litre}
    %= \qty{6,36e-3} {\g\per\litre}
    %= \qty{6,36} {\mg\per\litre}
  %\end{equation*}
  %Un taux normal se trouve entre 5 et \qty{17}{\mg\per\litre}, donc elle est en bonne santé.
  %\points{3}
%}{0}


%%%%
%\newpage
% \vspace{24pt}
%\exercice{L'acidification des océans (31 minutes)}


%\titreSousSection{Acidification}

%\medskip
%Avant 1850, le pH des océans était autour de \num{8,2}.
%Maintenant il est autour de \num{8,1} et des projections estiment qu'il sera autour de \num{7,9} en 2100.

%\question{
%  Expliquer pourquoi on parle « d'acidification des océans ». \points{2}
%}{
 % Le pH diminue et passe d'une valeur basique à une valeur neutre, donc il l'eau de mer s'acidifie.
  %\points{2}
%}{0}

%On va chercher à comprendre pourquoi le pH des océans diminue et l'impact que cela aura sur la faune et la flore.


%\medskip
%Les émissions de dioxyde de carbone \chemfig{CO_2} liée aux activités humaines sont en hausses, ce qui augmente la quantité de \chemfig{CO_2} dans l'atmosphère.
%Comme il y a plus de \chemfig{CO_2} dans l'atmosphère, la quantité de \chemfig{CO_2} qui est dissoute dans les océans augmente aussi.

%Le dioxyde de carbone dissous dans l'eau, noté \chemfig{CO_2^{*}}, forme un couple acide/base avec l'ion hydrogénocarbonate : \chemfig{CO_2^{*}}/\chemfig{HCO_3^{-}}.

%\question{
 % Rappeler les deux couples acide/base que forme l'eau \eau. \points{2}
%}{
 % \oxonium/\eau et \eau/\hydroxyde.
  %\points{2}
%}{0}

%\question{
 % Écrire la réaction acido-basique entre le dioxyde de carbone dissous \chemfig{CO_2^{*}} et l'eau \eau, en faisant attention à bien ajuster l'équation.             \points{2}
%}{
 % \begin{center}
  %  \chemfig{CO_2^{*}} + \eau \reaction \chemfig{HCO_3^{-}} + \oxonium
  %\points{2}
  %\end{center}
%}{0}

%\question{
%  Rappeler la relation entre le pH et concentration en ion oxonium \oxonium.                              \points{1}
%}{
 % \begin{equation*}    
  %  10^{-\text{pH}} = [\oxonium]
  %\points{1}
  %\end{equation*}
%}{0}

%\question{
 % Expliquer pourquoi cette réaction chimique entraine une diminution du pH des océans.                 \points{2}
%}{
 % Cette réaction produit des ions oxonium et augmente donc la concentration [\oxonium], ce qui diminue le pH.
  %\points{2}
%}{0}


% \titreSousSection{Émission de dioxyde de carbone}

% \medskip
% On va chercher à estimer la quantité de \chemfig{CO_2} qu'il faudrait dissoudre dans l'océan pour passer d'un pH de \num{8,1} à un pH de \num{7,9} en 2100.

% \question{
%   Calculer la concentration actuelle en ion oxonium dans les océans, avec un pH de \num{8,1}.
% }{}{0}

% \question{
%   Calculer la concentration en ion oxonium quand le pH vaudra \num{7,9} en 2100.
% }{}{0}

% \question{
%   Le volume total des océans vaut $V = \qty{1,37e21}{\litre}$, calculer la quantité de matière actuelle et en 2100 d'ion oxonium.
% }{}{0}

% \question{
%   En déduire la quantité de matière en dioxyde de carbone \chemfig{CO_2} qu'il faudrait dissoudre dans les océans pour baisser le pH de 0,2.
% }{}{0}

% \question{
%   Calculer la masse de \chemfig{CO_2} qu'il faudrait dissoudre.
%   \textbf{Donnée : } $M(\chemfig{CO_2}) = \qty{44,0}{\g\per\mole}$.
% }{}{0}


% \question{
%   En 2022, l'humanité a émis environ \qty{37,5e12}{\kg} de \chemfig{CO_2} dans l'atmosphère. 
%   Est-ce que tous le \chemfig{CO_2} émis a été dissous dans les océans ?
% }{}{0}


%\titreSousSection{Conséquence sur la faune et la flore}

%\medskip
%Avec du calcaire, le carbonate de calcium \chemfig{CaCO_3}, dans une eau saturée en dioxyde de carbone dissous, on a la réaction chimique suivante :
%\begin{center}
 % \chemfig{CaCO_3}(s) + \chemfig{CO_2^{*}}(aq) + \eau(l)
  %\reaction
  %\chemfig{Ca^{2+}}(aq) + 2\chemfig{HCO_3^{2-}}(aq)
%\end{center}
%Cette réaction chimique modélise la dissolution du calcaire dans l'eau.

%Le calcaire est le matériau qui compose le squelette ou les coquilles de nombreuses espèces marines, comme les mollusques, les crustacés ou les coraux.
%Les mollusques, les crustacés et les coraux forment la base de la chaîne alimentaire dans les océans.

%\question{
 % Est-ce que cette réaction chimique facilite la formation de squelettes ou de coquilles ?                  \points{1}
%}{
 % Elle ne facilite pas la formation de coquille ou de squelettes, car elle dissous leur principal composant (le calcaire).
  %\points{1}
%}{0}

%\question{
 % Expliquer pourquoi l'augmentation du dioxyde de carbone dissous dans l'océan entraine la mort de certaines espèces marines.                                        \points{2}
%}{
 % L'augmentation du dioxyde de carbone rend plus difficile la formation de squelettes ou de coquilles, ce qui peut tuer les espèces quand elles grandissent.
 % \points{2}
%}{0}


%%%%
% \bigskip
%\newpage
\exercice{Traitement des eaux }

\medskip
L'un des enjeux primordiaux du développement durable est la préservation des ressources en eau de la planète.
L'objectif de cet exercice est de mieux comprendre comment sont traitées les eaux usagées de certaines industries avant leur rejet dans le milieu naturel.

\medskip
La fabrication du savon de Marseille nécessite plusieurs étapes. %repose sur un procédé comportant de multiples étapes.
Il faut d'abords transformer des huiles végétales en savon sous l'action à chaud de soude concentrée dont la valeur du pH est comprise entre 12 et 13.
La pâte de savon obtenue est ensuite lavée plusieurs fois à l'eau salée afin d'éliminer la soude en excès.
Le savon doit alors cuire pendant dix jours à une température de \qty{120}{\degreeCelsius}.
Puis plusieurs lavages à l'eau pure permettent d'obtenir un savon sans impuretés.
La pâte de savon est alors coulée dans des moules, puis séchée pendant 48 h à l'air libre avant d'être découpée en savonnettes de tailles variées.

\medskip
Les dangers des solutions aqueuses acides sont bien connus, les substances basiques peuvent être tout aussi corrosives et, si elles ne sont pas traitées, peuvent endommager la faune, la flore et l'écosystème environnants.
Les normes de rejets dans les eaux contrôlées, tels que les cours d'eau de surface et les eaux souterraines, exigent un pH compris entre 5,5 et 8,5.

%\question{
 % Préciser le caractère (neutre, acide ou basique) des eaux de lavage d'une savonnerie.
  %Citer une méthode rapide permettant de le vérifier expérimentalement.                            \points{1}
%}{
 % À 5,5 l'eau est légèrement acide, à 8,5 elle est légèrement basique, entre les deux elle est neutre. 
  %On peut le vérifier avec un papier pH.
  %\points{1}
%}{0}

\question{
  Expliquer pourquoi il est nécessaire de traiter ces eaux avant leur rejet dans le milieu naturel.        \points{2}
}{
  Pour ne pas polluer les milieux naturel en y déversant des solutions basiques, qui vont attaquer la faune et la flore de ces milieux naturels.
  \points{2}
}{0}


\medskip
\begin{boite}
  \begin{center}  
    Carte d'identité de l'acide chlorhydrique (source http://www.inrs.fr)
  \end{center}
  
  \separationBlocs{
    Acide chlorhydrique concentré (\oxonium, \chemfig{Cl^{-}}).
    
    \centering
    \image{0.3}{images/pictogrammes/picto_nocif}
    \image{0.3}{images/pictogrammes/picto_corrosif}
  }{
    H331 -- Toxique par inhalation
    
    H314 -- Provoque des brûlures de la peau et des lésions oculaires graves
    
    % Les conseils de prudence P sont sélectionnés selon les critères de l'annexe 1 du règlement CE n° 1272/2008. 
    
    % 231-595-7
  }
\end{boite}

\question{
  Rappeler la signification des pictogrammes de sécurités présent sur l'étiquette de l'acide chlorhydrique. \points{1}
}{
  Le pictogramme de gauche indique des solutions nocives pour la santé.
  Le pictogramme de droite indique que les solutions sont corrosives.
  \points{1}
}{0}

% Le principe de la valorisation du dioxyde de carbone consiste à le considérer comme une matière première, que l'on capte à la sortie des fumées industrielles et que l'on exploite pour réaliser un certain nombre de produits ou d'opérations commercialement rentables. La neutralisation au gaz carbonique provenant des gaz de fumée est un procédé écologique et peu onéreux.

% \textbf{Données utiles :}
% \begin{itemize}
%   \item \qty{1}{\m\cubed} = \qty{1000}{\litre}
%   \item Volume moyen : baignoire, $V_b = \qty{0,4}{\m\cubed}$ ; 
%   piscine olympique, $Vp = \qty{2500}{\m\cubed}$ ;
%   \item Formule brute du dioxyde de carbone : \chemfig{CO_2}
%   \item Masse molaire : M(\chemfig{CO_2}) = \qty{44}{\g\per\mole}
%   \item Température de sublimation du \chemfig{CO_2} : \qty{-78,5}{\degreeCelsius} à la pression atmosphérique
% \end{itemize}

La dilution est l'une des méthodes de traitement des eaux alcalines.

% 3. Sachant que la valeur de la concentration des ions hydroxyde dans certaines eaux usées dont le pH vaut 13 est égale à \qty{1,0e-1}{\mole\per\litre}, calculer la valeur de la quantité de matière en ions hydroxyde, $n_\text{hydroxyde}$, présente dans un volume égal à un litre d'une telle eau usée.

% 4. On rappelle qu'une solution aqueuse est neutre si son pH vaut 7. On admet que l'intervalle de pH entre 6 et 8 est acceptable pour une neutralité approchée, sans danger. En admettant que le pH diminue de 1 unité de pH, dans l'intervalle compris entre 8 et 14, lorsqu'il y a dilution d'un facteur 10 d'une eau usée chargée en ions hydroxyde, prévoir le volume minimal d'eau à ajouter à un volume d'eau usée de 1L pour amener son pH de 13 à 8.

% 5. Commenter ce résultat en le comparant aux ordres de grandeurs fournis dans les données et expliquer pourquoi cette méthode n'est pas utilisée dans l'industrie.

Une autre méthode pour traiter les eaux usées consiste à les neutraliser par ajout de dioxyde de carbone ou par ajout d'acide minéraux tel que l'acide chlorhydrique.
La neutralisation au dioxyde de carbone provenant des gaz de fumées industrielles est un procédé écologique et peu onéreux.
Les couples acide/base mis en jeu dans la réaction de neutralisation de l'eau de lavage par la solution d'acide chlorhydrique sont \oxonium/\eau{} et \eau/\hydroxyde.

\question{
  Écrire l'équation acido-basique ajustée de la réaction de neutralisation de l'eau de lavage.            \points{2}
}{
  \begin{equation*}
    \oxonium + \hydroxyde \reaction 2\eau    
  \points{2}
  \end{equation*}
}{0} 

\question{
  Expliquer pourquoi les industriels préfèrent neutraliser les eaux usagées à l'aide de dioxyde de carbone plutôt qu'à l'aide d'acide chlorhydrique.               \points{2}
}{
  Les industriels utilisent le dioxyde de carbone, car c'est un procédé écologique et pas cher.
  Le dioxyde de carbone est aussi moins dangereux que l'acide chlorydrique.
  \points{2}
}{0} 