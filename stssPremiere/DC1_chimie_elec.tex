\newpage

\begin{flushright}
  \large 
  Sujet 2023
\end{flushright}

\vspace{20pt}
\titre{
  BACCALAURÉAT TECHNOLOGIQUE \\
  SCIENCES ET TECHNOLOGIES \\
  DE LA SANTÉ ET DU SOCIAL \\
}
\bigskip

\begin{boite}
  \titre{
    ÉPREUVE DE SCIENCES \\ PHYSIQUES ET CHIMIQUES \\
  }
\end{boite}
\bigskip

\begin{center}
  \Large 
  Durée de l'épreuve : 2 heures

  \bigskip

  Mercredi 13 décembre 2023
\end{center}

\vspace{20pt}
{\large
  Le sujet comporte 4 pages numérotées de 1 à 4.
  
  Dès que le sujet vous est remis, assurez-vous qu'il est complet.
  
  La page 2 est à compléter directement sur le sujet.
  \bigskip
  

\vfill 
  \textit{Les exercices sont indépendants.} \bigskip

  \textit{L'usage de la calculatrice est autorisé.} \bigskip

  \textit{
  La clarté des raisonnements et la qualité de la rédaction interviendront pour une part importante dans l'appréciation des copies.
  }
}


%%%%
\newpage
\vspace*{-50pt}
\exercice{Automatismes (connaitre son cours)}
\medskip

\numeroQuestion
Relier chaque grandeur à son symbole et à l'unité associée.

\begin{center}    
  \begin{tblr}{
    colspec = {l X[c] X[c]},
    row{1} = {couleurPrim!20, c}
  }
    \textbf{Grandeur} & \textbf{Symbole} & \textbf{Unité} \\
    Concentration massique & $n$   & \unit{\mole\per\litre} \\
    Masse                  & $C$   & \unit{\litre} \\
    Quantité de matière    & $C_m$ & \unit{\kg} \\
    Volume                 & $V$   & \unit{\g\per\litre} \\
    Concentration molaire  & $M$   & \unit{\g\per\mole} \\
    Masse molaire          & $m$   & \unit{\mole} \\
  \end{tblr}
\end{center}

\numeroQuestion
Entourer la ou les relations permettant de calculer les grandeurs de la première colonne
\vspace*{-12pt}

\begin{center}
  \begin{tblr}{
    colspec = {|X[l,m]| X[c,m]| X[c,m]| X[c,m]| X[c,m]|}, hlines,
    column{1} = {couleurPrim!20}
  }
    Pour calculer une quantité de matière &
    $\dfrac{m}{M}$ & $m\times M$ & $C \times V$ & $\dfrac{V}{C}$ \\
    %
    Pour calculer une concentration molaire &
    $\dfrac{m}{V}$ & $\dfrac{n}{V}$ & $m \times M$ & $\dfrac{M}{m}$ \\
    %
    Pour calculer la concentration en ion oxonium &
    $10^{-\text{pH}}$ & $10^{\text{pH}}$ & $10 - \text{pH}$ & $\text{pH}^{10}$ \\
    %
    Pour calculer une masse &
    $M \times n$ & $C_m \times V$ & $m \times M$ & $\dfrac{M}{m}$ \\
    %
  \end{tblr}
\end{center}

\numeroQuestion Donner le nom de ces deux ions
\begin{center}  
  \oxonium : \texteTrou[0.1]{ion oxonium} \qq{}
  et \hydroxyde : \texteTrou[0.1]{ion hydroxyde}
\end{center}

\question{
  Définir un acide d'après Br\o{}nsted.
}{}{2}

\question{
  Rappeler le principe d'une dilution.
}{}{3}

\question{
  Indiquer sur l'échelle pH ci-dessous où se trouve les solutions acides, les solution basiques et les solutions neutres.
  Puis placer sur cette échelle pH les solutions suivantes :
  \begin{tableau}{|c |c |c |c |c |c |}
    Solution & Acide éthanoïque & Acide chlorhydrique & Soude & Eau & Eau de mer \\
    pH       & \num{2,4} & \num{1,0} & \num{13,9} & \num{7,0} & \num{8,1} 
  \end{tableau}
}{}{0}

\pasCorrection{\vspace*{30pt}}
\begin{center}
  \begin{tikzpicture}
    \tkzVecteur(-0.75)[15.5](0){pH}
    \foreach \i in {0,1,...,14} \draw (\i,0.1)--(\i,-0.1) node[below]{\i};
  \end{tikzpicture}
\end{center}


%%%%
%\bigskip
\newpage
\exercice{Charger son téléphone au Canada}
\medskip

Asma est partie en vacances au Canada et veut charger son téléphone.
Pour ça, Asma commence par mesurer les caractéristiques de la tension du secteur, et trouve l'oscillogramme suivant :
\begin{center}
  \def\ver{3.6} % longueur verticale
  \def\hor{5.0} % longueur horizontale
  \begin{tikzpicture}[
    trace/.style={couleurPrim!75!black, ultra thick, samples = 100},
    screen/.style={couleurPrim!10, thick},
    axes/.style={couleurPrim, thick}
  ]
    % Fond de l'écran
    \fill[screen] (-\hor, -\ver) rectangle (\hor, \ver);
    % Grille et axes
    \draw[thin, couleurPrim!30] (-\hor, -\ver) grid (\hor, \ver);
    \draw[axes] (-\hor, 0) -- (\hor, 0); % temps
    \draw[axes] (0, -\ver) -- (0, \ver); % 
    % Graduation
    \pgfmathparse{\hor-1}
    % axe x
    \foreach[parse = true] 
      \i in {-\hor+0.2, -\hor+0.4,..., \hor-0.2} \draw[axes] (\i, -0.1) -- (\i, 0.1);
    \foreach[parse = true] \i in {-\hor,...,\hor} \draw[axes] (\i,-0.2) -- (\i,0.2);
    % axe y
    \foreach[parse = true]
      \i in {-\ver+0.20, -\ver+0.4,..., \ver-0.20} \draw[axes] (-0.1, \i) -- (0.1, \i);
    \foreach[parse = true] \i in {-3,...,3} \draw[axes] (-0.2,\i) -- (0.2,\i);
    % Tension sinusoïdale
    \draw[trace] plot(\x, {1.60*sin((1.57*(1.21*\x + 1)) r)}); % r = radian
    % Échelle
    \tkzVecteur(-5) (-3)[1]  {\textbf{100 V}} [right]
    \tkzVecteur(-5)[1] (-3)  {\textbf{5 ms}} [below right]
  \end{tikzpicture}
\end{center}

\question{
  Mesurer la tension maximale $U_\text{max}$ et en déduire la tension efficace du secteur $U$.
}{
  $U_\text{max} = \qty{170}{\volt}$, donc $U = \dfrac{120}{\sqrt{2}} = \qty{120}{\volt}$.
}{0}

\question{
  Mesurer la période $T$ et en déduire la fréquence $f$ de la tension du secteur.
}{}{0}

\question{
  Rappeler la fréquence et la tension efficace de la tension du secteur en France.
}{}{0}

\question{
  Asma pourra-t-elle utiliser un chargeur conçu pour être utilisé sur la tension du secteur en France au Canada ? Justifier.
}{}{0}


%%%%
\vspace{24pt}
\exercice{L'acide ascorbique}

\medskip
Les vivres embarquées sur les navires européens du XVe au XVIIIe siècle étaient essentiellement des salaisons, des légumes secs et des biscuits pour des raisons liées à la conservation des aliments. Or ceux-ci ne contenant pas de vitamine C (aussi appelée acide ascorbique), il en a résulté des cas de scorbut lors des expéditions de longues distance (plus de six semaines à plus de trois mois) en mer.

\medskip
Les marins atteints du scorbut subissaient des gingivites hypertrophiques (entraînant le gonflement des gencives et le déchaussement des dents) et des syndromes hémorragiques et cutanés (pertes des cheveux et hématomes) si bien qu’après un voyage de 3 mois (pour ceux qui y survivaient), ces derniers semblaient avoir vieilli de vingt ans. 

\medskip
En 2019, le scorbut est réapparu en France chez des personnes précaires ne se nourrissant pas suffisamment de légumes et fruits frais.

\medskip
On considère que le taux en vitamine C par litre de sang doit être compris entre \num{5}  et \num{17}  \unit{\mg\per\litre} pour être en bonne santé.

\medskip
\textbf{Donnée : } 
\textit{Formule brute de la vitamine C :} \chemfig{C_6H_8O_6}

\textit{Masse molaire atomique : $M(H) = \qty{1} {\g\per\mole}$ $M(C) = \qty{12} {\g\per\mole}$ $M(N) = \qty{14} {\g\per\mole}$ $M(O) = \qty{16} {\g\per\mole}$}

\textit{ $\qty{1} {g} = \qty{1 000} {mg}$   ;  $\qty{1} {L} = \qty{1 000} {\ml}$ }

\bigskip
\question{
  Quelle espèce chimique permet de lutter contre le scorbut ? Dans quels aliments la trouve-t-on ?
}{}{0}

La concentration massique en vitamine C dissout dans un jus de pomme est $C_m = \qty{9,0e-3}{\g\per\litre}$.

\question{
  Quel est le soluté ? Quelle est la solution ?
}{}{0}

\question{
  Calculer la masse de vitamine C présente dans une bouteille de volume $V = \qty{1,5} {\litre}$ de jus de pomme.
}{}{0}

\question{
  La formule brute de la vitamine C est $C_6H_8O_6$. Calculer la masse molaire moléculaire de la vitamine C. 
}{}{0}

\question{
  Déterminer la quantité de matière en vitamine C présent dans un volume de $\qty{1,5} {\litre}$ de jus de pomme.
}{}{0}

\question{
  En déduire la concentration molaire en vitamine C dans le jus de pomme.
}{}{0}

\medskip
Camélia se sent fatiguée et se voit prescrire de la vitamine C. Elle doit prendre chaque jour un comprimé contenant une masse $m= \qty{1 000} {mg}$ de vitamine C. 
Un comprimé se dissout dans un verre d’eau de volume $V= \qty{200} {\ml}$.

\question{
  On suppose que le comprimé se dissout entièrement. Déterminer la concentration massique de vitamine C dans son verre.
}{}{0}

\question{
  Une fois ingérée et assimilée par l’organisme, la concentration molaire sanguine de vitamine C de Camélia atteint $\qty{3,6e-5} {\mol\per\litre}$ . A-t-elle un taux de vitamine C suffisant pour être en bonne santé ?
}{}{0}


%%%%
\vspace{24pt}
\exercice{L'acidification des océans}


\titreSousSection{Acidification}

\medskip
Avant 1850, le pH des océans était autour de \num{8,2}.
Maintenant il est autour de \num{8,1} et des projections estiment qu'il sera autour de \num{7,9} en 2100.

\question{
  Expliquer pourquoi on parle « d'acidification des océans ».
}{
  Le pH diminue et passe d'une valeur basique à une valeur neutre, donc il l'eau de mer s'acidifie.
}{0}

On va chercher à comprendre pourquoi le pH des océans augmente et l'impact que cela aura sur la faune et la flore.


\medskip
Les émissions de dioxyde de carbone \chemfig{CO_2} liée aux activités humaines sont en hausses, ce qui augmente la quantité de \chemfig{CO_2} dans l'atmosphère.
Comme il y a plus de \chemfig{CO_2} dans l'atmosphère, la quantité de \chemfig{CO_2} qui est dissoute dans les océans augmente aussi.

Le dioxyde de carbone dissous dans l'eau, noté \chemfig{CO_2^{*}}, forme un couple acide/base avec l'ion hydrogénocarbonate : \chemfig{CO_2^{*}}/\chemfig{HCO_3^{-}}.

\question{
  Rappeler les deux couples acide/base que forme l'eau \eau.
}{
  \oxonium/\eau et \eau/\hydroxyde.
}{0}

\question{
  Écrire la réaction acido-basique entre le dioxyde de carbone dissous \chemfig{CO_2^{*}} et l'eau \eau, en faisant attention à bien ajuster l'équation.
}{
\begin{center}
  \chemfig{CO_2^{*}} + \eau \reaction \chemfig{HCO_3^{-}} + \oxonium
\end{center}
}{0}

\question{
  Rappeler la relation entre le pH et concentration en ion oxonium \oxonium.
}{
  $10^{-\text{pH}} = [\oxonium]$.
}{0}

\question{
  Expliquer pourquoi cette réaction chimique entraine une diminution du pH des océans.
}{
  Cette réaction produit des ions oxonium et augmente donc la concentration [\oxonium], ce qui diminue le pH.
}{0}


% \titreSousSection{Émission de dioxyde de carbone}

% \medskip
% On va chercher à estimer la quantité de \chemfig{CO_2} qu'il faudrait dissoudre dans l'océan pour passer d'un pH de \num{8,1} à un pH de \num{7,9} en 2100.

% \question{
%   Calculer la concentration actuelle en ion oxonium dans les océans, avec un pH de \num{8,1}.
% }{}{0}

% \question{
%   Calculer la concentration en ion oxonium quand le pH vaudra \num{7,9} en 2100.
% }{}{0}

% \question{
%   Le volume total des océans vaut $V = \qty{1,37e21}{\litre}$, calculer la quantité de matière actuelle et en 2100 d'ion oxonium.
% }{}{0}

% \question{
%   En déduire la quantité de matière en dioxyde de carbone \chemfig{CO_2} qu'il faudrait dissoudre dans les océans pour baisser le pH de 0,2.
% }{}{0}

% \question{
%   Calculer la masse de \chemfig{CO_2} qu'il faudrait dissoudre.
%   \textbf{Donnée : } $M(\chemfig{CO_2}) = \qty{44,0}{\g\per\mole}$.
% }{}{0}


% \question{
%   En 2022, l'humanité a émis environ \qty{37,5e12}{\kg} de \chemfig{CO_2} dans l'atmosphère. 
%   Est-ce que tous le \chemfig{CO_2} émis a été dissous dans les océans ?
% }{}{0}


\titreSousSection{Conséquence sur la faune et la flore}

\medskip
Avec du calcaire, le carbonate de calcium \chemfig{CaCO_3}, dans une eau saturée en dioxyde de carbone dissous, on a la réaction chimique suivante :
\begin{center}
  \chemfig{CaCO_3}(s) + \chemfig{CO_2^{*}}(aq) + \eau(l)
  \reaction
  \chemfig{Ca^{2+}}(aq) + 2\chemfig{HCO_3^{2-}}(aq)
\end{center}
Cette réaction chimique modélise la dissolution du calcaire dans l'eau.

Le calcaire est le matériau qui compose le squelette ou les coquilles de nombreuses espèces marines, comme les mollusques, les crustacés ou les coraux.
Les mollusques, les crustacés et les coraux forment la base de la chaîne alimentaire dans les océans.

\question{
  Est-ce que cette réaction chimique facilite la formation de squelettes ou de coquilles ?
}{
  Elle ne facilite pas la formation de coquille ou de squelettes, car elle dissous leur principal composant (le calcaire).
}{0}

\question{
  Expliquer pourquoi l'augmentation du dioxyde de carbone dissous dans l'océan entraine la mort de certaines espèces marines.
}{
  L'augmentation du dioxyde de carbone rend plus difficile la formation de squelettes ou de coquilles, ce qui peut tuer les espèces quand elles grandissent. 
}{0}


%%%%
\bigskip
\exercice{Traitement des eaux}

Une usine 