\enTeteFicheReussite{Chapitre 4 - Ondes lumineuses et optique}
\bigskip

\begin{tableauConnaissancesSansExercices}
  %
  Je sais que la lumière est une onde électromagnétique.
  Je sais que la lumière peut être monochromatique ou polychromatique.
  & & & Activité 1
  \\ \hline
  %
  Je connais les deux types de spectre d'émission et je sais les reconnaître.
  & & & TP 1
  \\ \hline
  %
  Je sais qu'un corps chaud produit un spectre continu.
  Les propriétés de ce spectre dépendent de la température du corps chaud.
  & & & TP 1
  \\ \hline
  %
  Je sais qu'un gaz atomique ou moléculaire excité produit un spectre de raies.
  Je sais que chaque élément chimique possède son propre spectre de raies qui permet de le reconnaître.
  & & & TP 1, activité 2
  \\ \hline
  %
  Je sais repérer une raie sur un spectre en mesurant sa longueur d'onde.
  & & & TP 1, activité 2
  \\ \hline
  %
  Je connais le vocabulaire de la réfraction et je sais lire les angles d'incidence et de réfraction à partir d'un schéma.
  & & & TP 2, activité 4
  \\ \hline
  %
  Je sais appliquer la loi de Snell-Descartes pour calculer un indice de réfraction ou un angle.
  & & & TP 2, activité 4
  \\ \hline
  %
  Je sais expliquer comment l'oeil parvient à faire l'image d'un objet.
  & & & Activité 3
  \\ \hline
  %
  Je sais expliquer comment fonctionne une lentille convergente et je connais le vocabulaire pour la décrire (foyer image, foyer objet, centre optique).
  & & & TP 3, activité 3
  \\ \hline
  %
  Je connais la formule du grandissement et je sais l'appliquer pour calculer la taille d'un objet ou d'une image.
  & & & TP 3, Activité 3
  \\ \hline
  %
  Je sais expliquer pourquoi la lumière blanche est dispersée après avoir traversé un prisme.
  & & & Activité 4
  %
\end{tableauConnaissancesSansExercices}
\bigskip 

% \basDePageFicheReussite
% \coursFicheReussite
\questionFicheReussite{2}