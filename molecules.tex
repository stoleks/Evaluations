%%%%%%%%%%%%%%%%%%%%%%%%%%%%%%%%%%%%%%%%%%%%%%%%%%%%%%%%%%%%%
%%%% Bouts de molécules fréquement utilisés
%% Hydrogène saturés
\definesubmol\paireH{(-[::90] H) (-[::-90] H)}
\definesubmol\paireSatH{(-[::30] H) (-[::-30] H)}
\definesubmol\saturationH{(-[::90] H) (-[::-90] H) (-[::0] H)}

%% Quelques groupes caractéristiques
\definesubmol\carboxyle{(=[:90] O) (-[:-30] OH)}
\definesubmol\carbonyle{(=[::60] O) -[::-60]}
\definesubmol\ester{(=[:90] O) -[:-30] O}
\definesubmol\ether{-[:30] O -[:-30]}
\definesubmol\amide{(=[:90] O) -[:-30] N}

%% parties colorées
\definesubmol\cetoneCouleur{(=[3,,,,couleurQuat] \textcolor{couleurQuat}{O}) -[-1,,,,couleurQuat]}
%% ramification
\definesubmol\alkyleG{(-[-5] R_1)}
\definesubmol\alkyleD{(-[-1] R_2)}

%%%% Élément récurrent, pour faciliter la lecture
\newcommand{\hydrogene}{\chemfig{H}}
\newcommand{\carbone}{\chemfig{C}}
\newcommand{\oxygene}{\chemfig{O}}
\newcommand{\azote}{\chemfig{N}}
\newcommand{\eau}{\chemfig{H_2O}}
\newcommand{\oxonium}{\chemfig{H_3O^+}}
\newcommand{\hydroxyde}{\chemfig{HO^{-}}}
\newcommand{\azoture}{\chemfig{NaN_3}}


%%%% Molécule courante
\definesubmol\paracetamol{
  *6((-HO)-=-(-NH (-[::-60] (=[::-60]O)-[::60]))=-=)
}


%%%% Acide gras
\definesubmol\cc{
  -[::60] -[::-60]
}
\definesubmol\teteAcide{
  O-[::30] (=[::60]O) -[::-60]
}
\definesubmol\teteAcideDev{
  - O - C (=[::90] O) -
}
\definesubmol\cis{
  -[::60] =[::-60] -[::-60]
}
\definesubmol\trans{
  -[::60] =[::-30] -[::-30]
}
\definesubmol\palmitique{
  !\carbonyle !\cc !\cc !\cc !\cc !\cc !\cc !\cc
}
\definesubmol\oleique{
  !\teteAcide !\cc !\cc !\cc !\cis -[::60] !\cc !\cc !\cc
}
\definesubmol\trioleique{
  !\carbonyle !\cc !\cc !\cc -[::60] =[::60] -[::60] !\cc !\cc -[::-60] -[::-60] -[::-60]
}
\definesubmol\alphaLinoleique{
  !\teteAcide !\cc !\cc !\cc !\trans !\trans !\trans -[::60]
}
\definesubmol\alphaLinolenique{
  !\teteAcide !\cc !\cc !\cc !\cis !\cis !\cis -[::60]
}
\definesubmol\steraique{
  !\teteAcideDev C_{17}H_{35}
}
\definesubmol\caproique{
  - O - C (=[3] O) - CH_2 - CH_2 - CH_2 - CH_2 - CH_3
}
\definesubmol\trioleine{
   (-[::150] -[::60] O-[::-60] !\trioleique)
   (-[::-90] -[::-60] O-[::60] !\trioleique)
   -[::30] O-[::60] !\trioleique
}
\definesubmol\tripalmitine{
   (-[::150] -[::60] O-[::-60] (=[::60] O) -[::-60] -[::-60] !\cc !\cc !\cc !\cc !\cc !\cc -[::60]) % gauche
   (-[::-90] -[::60] O-[::-60] (=[::-60]O) -[::60] -[::60] !\cc !\cc !\cc !\cc !\cc !\cc -[::60]) % droite
   -[::30] O-[::60] !\palmitique % bas
}
%% glycerol
\definesubmol\glycerol{HO -[-1] -[1] (-[3] OH) -[-1] -[1] OH}
