%%%%%%%%%%%%%%%%%%%%%%%%%%%%%%%%%%%%%%%%%%%%%%%%%%%%%%%%%%%%%
%%%% Bouts de molécules fréquement utilisés
%% Hydrogène saturés
\definesubmol\paireH{(-[::90] H) (-[::-90] H)}
\definesubmol\paireSatH{(-[::30] H) (-[::-30] H)}
\definesubmol\saturationH{(-[::90] H) (-[::-90] H) (-[::0] H)}

%% Quelques groupes caractéristiques
\definesubmol\carboxyle{(=[:90] O) (-[:-30] OH)}
\definesubmol\carbonyle{(=[:90] O) -[:-30]}
\definesubmol\ester{(=[:90] O) -[:-30] O}
\definesubmol\ether{-[:30] O -[:-30]}
\definesubmol\amide{(=[:90] O) -[:-30] N}

%% parties colorées
\definesubmol\cetoneCouleur{(=[3,,,,couleurQuat] \textcolor{couleurQuat}{O}) -[-1,,,,couleurQuat]}
%% ramification
\definesubmol\alkyleG{(-[-5] R_1)}
\definesubmol\alkyleD{(-[-1] R_2)}

%%%% Élément récurrent, pour faciliter la lecture
\newcommand{\hydrogene}{\chemfig{H}}
\newcommand{\carbone}{\chemfig{C}}
\newcommand{\oxygene}{\chemfig{O}}
\newcommand{\azote}{\chemfig{N}}
