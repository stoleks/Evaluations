%%%% début de la page
\newpage
\enTete{Mouvement et interactions}{2}

\nomPrenomClasse

%%%% titre
\vspace{8pt}
\chevron L'objectif de cette interrogation \textbf{non notée} est d'évaluer où vous en êtes en physique-chimie.

%%%% question
%
\exo{Indiquer l'unité qui correspond à une vitesse.}{0}
\begin{qcm}
  \item Le mètre par seconde (m/s).
  \item Le kilomètre heure (km$\cdot$h).
  \item La seconde par mètre (s/m).
\end{qcm}

%
\exo{Une vitesse instantanée est représentée par une flèche qui porte 3 informations :}{0}
\begin{qcm}
  \item la distance, la direction et la durée.
  \item le sens, l'accélération et le nombre.
  \item la direction, le sens et la valeur.
\end{qcm}

%
\exo{Une observatrice observe depuis le quai un métro roulant à une vitesse constante.
Pour cette observatrice, le mouvement apparent du train est :}{1}

%
\exo{Un observateur est assis dans le métro roulant à une vitesse constante. 
Pour cet observateur, le mouvement apparent du train est :}{1}

%
\exo{Donner la formule de la force de pesanteur, le poids $P$, que l'on ressent sur Terre.}{1}

%
\exo{Indiquer la bonne expression de la force de pesanteur universelle $F_G$ entre deux objets de masse $m_A$ et $m_B$, séparés par une distance $d$.}{0}
\begin{qcm}
  \item $F_G = G \Frac{m_A m_B}{d^2}$
  \item $F_G = G m_A m_B$
  \item $F_G = G \Frac{m_A m_B}{d}$
\end{qcm}

%
\exo{Donner l'unité d'une force.}{1}