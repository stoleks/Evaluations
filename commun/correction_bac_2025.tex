\title{Détail de la correction du baccalauréat ST2S}
\maketitle

\important{Pour tout le sujet : -0,25 s'il manque l'unité dans le résultat, avec un maximum de -1 sur toute la copie.}

\important{On ne sanctionne pas les chiffres significatifs, sauf si chaque résultat comporte tous les chiffres de la calculette : -0,25}

%%%%
\bigskip
\exercice{La fumée de cigarette}

\begin{multicols}{2}
\numeroQuestion 
\begin{itemize} 
  \item 1 si la formule est développée sans erreur,
  \item 0,25 si le squelette carbonée est ok, mais qu'il y a une erreur avec une partie semi-développée, par exemple :
  \begin{center}
    \chemfig{ O = C (!\lh H) !\lb C (!\lb H) = CH_2 }
  \end{center}
\end{itemize}

\numeroQuestion 
\begin{itemize}
  \item 0,5 si le groupe est bien entouré,
  \item 1 si groupe carbonyle,
  \item 0,5 si fonction aldéhyde.
\end{itemize}

\numeroQuestion 
\begin{itemize}
  \item 1 si désintégration $\alpha$,
  \item 0,5 si juste $\alpha$.
\end{itemize}

\numeroQuestion  
\begin{itemize}
  \item 1 si radioactivité ou désexcitation $\gamma$,
  \item 0,5 si juste $\gamma$.
\end{itemize}

\numeroQuestion 
\begin{itemize}
  \item 0,5 par coefficients corrects,
  \item 1 si les produits et les réactifs sont bien équilibré avec un \chemfig{CO_2} dans les produits,
  \item 0,25 si l'oxygène est bien ajusté, mais le reste est faux.
\end{itemize}

\numeroQuestion  
\begin{itemize}
  \item 0,25 par conversion ($T$, $P$, $V$), donc 0,75 en tout,
  \item 0,75 si le calcul est correct, même sans expression littérale,
  \item 0,5 si la formule littérale est mise seule, sans application numérique.
\end{itemize}

\numeroQuestion 0,25 pour la bonne pression. 0,25 pour la bonne température. 0,25 pour le calcul. 0,25 pour la comparaison.
\begin{itemize}
  \item 0,25 pour la bonne pression,
  \item 0,25 pour la bonne température,
  \item 0,25 pour le calcul,
  \item 0,25 pour la comparaison.
\end{itemize}

\numeroQuestion
\begin{itemize}
  \item 1 si la démarche aboutit à un résultat,
  \item 0,5 si le résultat est correct,
  \item On ne sanctionne pas une éventuelle conversion en heure.
\end{itemize}
\end{multicols}

%%%%
\newpage
\exercice{Le tabagisme passif}

\begin{multicols}{2}
\numeroQuestion
\begin{itemize}
  \item 1 si l'équation est correcte.
  \item 0,5 si la réaction est inversée,
  \item 0,25 par demi-équation acido-basique donnée, sans réaction finale,
  \item 0,25 si les deux réactifs sont bien identifiés.
\end{itemize}

\numeroQuestion
\begin{itemize}
  \item 1 si la définition est correcte.
  \item 0,5 si une phrase type « c'est un acide parce qu'il réagit avec une base » ou « c'est un acide, car il est à gauche dans le couple acide/base ».
  \item 0 dans les autres cas.
\end{itemize}

\numeroQuestion
\begin{itemize}
  \item 1 si réaction correcte, même si les états ne sont pas précisés.
  \item pas de sanction si la molécule d'eau est présente des 2 côté.
  \item 0,5 si l'eau n'est que du côté des réactifs.
  \item 0 si l'eau est à droite.
  \item pas de sanction si un signe $=$ est utilisé.
\end{itemize}

\numeroQuestion
\begin{itemize}
  \item 0,5 pour chaque étape correcte du raisonnement présentée dans le barème,
  \item 0,25 pour l'égalité des concentrations,
  \item 0,25 si la masse molaire moléculaire est correcte,
  \item 0,25 si le volume de 1 L est bien identifié.
\end{itemize}

\numeroQuestion
\begin{itemize}
  \item 0,25 si on pèse un solide,
  \item 0,25 si on utilise une fiole jaugée,
  \item 0,25 si on ajoute de l'eau (pas besoin de préciser qu'elle est distillée),
  \item 0,25 si on homogénéise,
  \item 0,25 si on complète avec de l'eau,
  \item 0,25 si on complète jusqu'au trait de jauge.
\end{itemize}

\numeroQuestion
\begin{itemize}
  \item 0,5 pour une longueur d'onde de \qty{450}{\nm},
  \item 0,5 pour la justification sur l'absorbance maximale,
  \item -0,25 s'il n'y a pas de mention explicite de l'absorbance.
\end{itemize}

\numeroQuestion
\begin{itemize}
  \item 0,25 pour le tracé des axes,
  \item 0,25 pour le nom des grandeurs, on ne sanctionne pas le manque d'unités,
  \item 0,25 pour la droite,
  \item 0,25 pour le report des points,
  \item 0,5 pour la mesure de la bonne concentration, comprise dans l'intervalle [8,2 ; 9,8] \unit{\mg\per\litre}.
\end{itemize}

\numeroQuestion
\begin{itemize}
  \item 0,5 si l'élève multiplie par 40 la concentration,
  \item 0,5 pour la référence au valeurs de l'énoncé,
  \item 0,5 si la comparaison et la conclusion liée est pertinente.
\end{itemize}
\end{multicols}