%\newpage
\pasDePagination

%%%% liste des compétences
\vspace*{-50pt}
\titre{Liste des compétences évaluées}

% Ce tableau liste l'ensemble des compétences évaluées au cours de l'année.
% Ces compétences sont liées à la méthode scientifique : chaque compétence représente une étape nécessaire pour étudier une situation nouvelle en appliquant la méthode scientifique.
% Pour chaque compétence, des exemples de capacités concrètes sont présentées.
% \bigskip

% L'objectif de cette liste est de vous aider à identifier ce qui est attendu de vous au cours de l'année.
% \bigskip

% Au début de chaque évaluation, les compétences effectivement évaluées seront présentées clairement. 
% \bigskip


\begin{tblr}{
  colspec = {|X[-2, c, m] | X[l, m] |}, hlines, row{1} = {couleurPrim!20, c}
}
  \textbf{Compétences} & \textbf{Capacités associées} \\
  %
  { Restituer ses \\ connaissances (RCO) } &
  Connaître les définitions du cours.
  Énoncer des exemples courants présentés en cours.  \\
  %
  { S'approprier \\ (APP) } &
  Énoncer un problème à résoudre (problématique).
  Extraire des informations d'un document.
  Représenter une situation avec un schéma. \\
  %
  { Analyser/Raisonner \\ (ANA/RAI) } &
  Formuler des hypothèses.
  Évaluer des ordres de grandeurs.
  Choisir un modèle ou des lois pertinentes.
  Choisir, élaborer, justifier un protocole.
  Procéder à des analogies. \\
  %
  { Réaliser \\ (REA) } &
  Utiliser un modèle.
  Calculer, représenter, collecter des données, etc.
  Mettre en \oe{}uvre un protocole expérimental en respectant les règles de sécurités. \\
  %
  { Valider \\ (VAL) } & 
  Faire preuve d'esprit critique.
  Identifier des sources d'erreur, estimer une incertitude.
  Comparer avec des valeurs de références.
  Confronter un modèle à des résultats expérimentaux.
  Proposer des améliorations de la démarche ou du modèle. \\
  %
  { Communiquer \\ (COM) } &
  Présenter de manière argumentée, synthétique et cohérente.
  Utiliser un vocabulaire ou une représentation adaptée.
  Échanger entre élèves. \\
\end{tblr}