%\newpage
\pasDePagination

%%%% liste des compétences
\titre{Liste des compétences évaluées}

Ce tableau liste l'ensemble des compétences évaluées au cours de l'année.
Ces compétences sont liées à la méthode scientifique : chaque compétence représente une étape nécessaire pour étudier une situation nouvelle en appliquant la méthode scientifique.
Pour chaque compétence, des exemples de capacités concrètes sont présentées.
\bigskip

L'objectif de cette liste est de vous aider à identifier ce qui est attendu de vous au cours de l'année.
\bigskip

Au début de chaque évaluation, les compétences effectivement évaluées seront présentées clairement. 
\bigskip

\begin{tabularx}{\linewidth}{| m{0.2\linewidth} | X |}
  \hline
  \rowcolor{gray!20} 
  \centering \textbf{Compétences} & \textbf{Capacités associées}
  \\ \hline
  %
  \centering Restituer ses connaissances (RCO) &
  -- Énoncer les définitions du cours. \newline
  -- Énoncer des exemples courants présentés en cours
  \\ \hline
  %
  \centering S'approprier (APP) &
  -- Énoncer un problème à résoudre (problématique). \newline
  -- Rechercher et organiser l'information. \newline
  -- Représenter la situation par un schéma.
  \\ \hline
  %
  \centering Analyser/ Raisonner (ANA/RAI) &
  -- Formuler des hypothèses. \newline
  -- Proposer une stratégie de résolution. \newline
  -- Évaluer des ordres de grandeurs. \newline
  -- Choisir un modèle ou des lois pertinentes. \newline
  -- Choisir, élaborer, justifier un protocole. \newline
  -- Prévoir à l'aide d'un modèle. \newline
  -- Procéder à des analogies.
  \\ \hline
  %
  \centering Réaliser (REA) &
  -- Mettre en \oe{}uvre les étapes d'une démarche. \newline
  -- Utiliser un modèle. \newline
  -- Calculer, représenter, collecter des données, etc. \newline
  -- Mettre en \oe{}uvre un protocole expérimental en respectant les règles de sécurités.
  \\ \hline
  %
  \centering Valider (VAL) & 
  -- Faire preuve d'esprit critique. \newline
  -- Identifier des sources d'erreur, estimer une incertitude .\newline
  -- Comparer avec des valeurs de références. \newline
  -- Confronter un modèle à des résultats expérimentaux. \newline
  -- Proposer des améliorations de la démarche ou du modèle.
  \\ \hline
  %
  \centering Communiquer (COM) &
  -- Présenter de manière argumentée, synthétique et cohérente. \newline
  -- Utiliser un vocabulaire ou une représentation adaptée. \newline
  -- Échanger entre élèves.
  \\ \hline
\end{tabularx}