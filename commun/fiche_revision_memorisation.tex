%
\newcommand{\travailExerciceCorrige}{
  Pour être sûr-e d'obtenir une bonne note, je m'entraîne avec les exercices corrigés du manuel indiqués dans la colonne de droite.
}
\newcommand{\questionsFicheRevision}[1]{
  Je note ici les questions qu'il me reste pour les poser au début de l'évaluation :
  \lignesDeReponse{#1}
}
\newcommand{\preparationFicheRevision}{
  Je prépare une fiche au format A4 avec toutes les notions, définitions ou grandeurs dont je pense avoir besoin pendant l'évaluation.
  \bigskip
  
  \begin{center}
    \textcolor{couleurSec}{\faThumbsUp}
    \important{La fiche préparée pourra être utilisée pendant l'évaluation pour s'aider !}
  \end{center}
}


%%%% Commande pour afficher une fiche de révision avec 
%%%% | connaissances | ok | pas ok | activités en classe |
\NewDocumentCommand{\ficheRevision}{s m +m}{
  \enTete{Fiche de révision -- #2}[0]*
  \vspace*{-6pt}

  %% Corps de la fiche
  \titreSection{Ce que je dois savoir}
  
  Pour savoir quoi réviser, je lis les points clés du chapitre évalués :
  \begin{listePoints}
    \item Si je pense maîtriser une notion, je coche la case \ok
    \item Si je pense que je dois la retravailler, je coche la case \pasOk
  \end{listePoints}
  \vspace*{4pt}
  
  \begin{tableauConnaissances}
    #3
  \end{tableauConnaissances}

  %% Bas de la fiche
  \IfBooleanTF{#1}{}{
    \titreSection{Ce qu'il me reste à faire}
  }
}

%%%% Commande pour afficher une fiche de mémorisation avec 
%%%% | questions | ok | pas ok | réponses | répétition 1 semaine | 1 mois | 6 mois |
\newcommand{\ficheMemorisation}[2]{
  \enTete{Fiche de mémorisation -- #1}[0]*
  \vspace*{-6pt}

  %% Corps de la fiche
  \titreSection{Les notions à connaître}
  
  Pour réviser, j'essaye de répondre aux question du tableau.
  Si j'ai répondu comme la réponse corrigée, je coche la case \ok.
  Sinon je coche la case \pasOk.

  Je reprends les notions que je pense maîtriser au bout de 1 semaine (J-7), 1 mois (J-30) et 6 mois (J-180).
  Si je ne les maîtrise pas, je ressors mes activités pour les retravailler !
  \vspace*{4pt}
  
  \begin{tableauMemorisation}
    #2
  \end{tableauMemorisation}
}