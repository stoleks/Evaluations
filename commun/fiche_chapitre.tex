%
\newcommand{\enTeteFiche}[2]{
  \newpage
  \setcounter{section}{0}
  \setcounter{subsection}{0}
  \setcounter{subsubsection}{0}
  \setcounter{sousSectionNum}{0}
  \pasDePagination
  
  \phantom{b}
  \vspace*{-70pt}
  % \titre{Connaissances et capacités à apprendre}
  \titre{Chapitre #1 -- #2}
  
  \vspace*{-6pt}
  \titreSection{Ce que je dois savoir}
  
  Pour savoir quoi réviser, je lis les points clés du chapitre évalués :
  \begin{itemize}
    \item Si je pense maîtriser une notion, je coche la case \ok
    \item Si je pense que je dois la retravailler, je coche la case \pasOk
  \end{itemize}
  
  %Pour travailler les notions qui ne sont pas maîtrisées, je reprend les activités associés.
  \vspace*{4pt}
}
\newcommand{\basDePageFicheReussite}{
  \titreSection{Ce qu'il me reste à faire}
}
\newcommand{\travailExerciceCorrige}{
  Pour être sûr-e d'obtenir une bonne note, je m'entraîne avec les exercices corrigés du manuel indiqués dans la colonne de droite.
}
\newcommand{\questionFicheReussite}[1]{
  Je note ici les questions qu'il me reste pour les poser au début de l'évaluation :
  \lignesDeReponse{#1}
}
\newcommand{\coursFicheReussite}{
  Je prépare une fiche au format A4 avec toutes les notions, définitions ou grandeurs dont je pense avoir besoin pendant l'évaluation.
  \bigskip
  
  \begin{center}
    \textcolor{couleurSec}{\faThumbsUp}
    \important{La fiche préparée pourra être utilisée pendant l'évaluation pour s'aider !}
  \end{center}
}
