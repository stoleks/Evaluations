\teteTermStssOrga

\begin{center}
  \chemfig{
    HO-[1] *6(-- % 1er cycle
      *6(=-- % 2eme cycle
        *6(- % 3eme cycle
          *5(--- 
            (-[:60] (-[:120]) -[:10] -[:60] -[:10] -[:60] (-[:120]) -[:10]) % lipide
          - (-[:90]) -) % 4eme
        ----) % 3eme
      ---) % 2eme
    - (-[:90]) ---) % 1er
  }
  \\[8pt]
  \important{Cholestérol}, lipide de la famille des stérol
\end{center}

\question{
  Donner le nom de la représentation de la molécule.
}{
  C'est la formule topologique.
}{1}

\question{
  Donner la formule brute du cholestérol.
}{
  \bruteCHO{27}{42}
}{1}

\question{
  Entourer et nommer le(s) groupe(s) fonctionnel(s) du cholestérol.
}{
  Alcool et alcène.
}{2}

%\vfill 
%Si on devait utiliser la nomenclature pour nommer la molécule de cholestérol, son nom serait

% \centering
% (3S, 8S, 9S, 10R, 13R, 14S, 17R)-10,13-diméthyl-17-[(2R)-6-méthylheptan-2-yl]-2,3,4,7,8,9,11,12,14,15,16,17-dodécahydro-1H-cyclopenta[a]phénanthren-3-ol