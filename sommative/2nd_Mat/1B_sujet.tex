%%%% début de la page
\newpage
\setcounter{page}{1}
\sndEnTeteUn

%%%%
\nomPrenomClasse

%%%%
\numeroActivite{2}
\vspace*{-8pt}
\titreEvaluation{Solutions}

%%%% evaluation
\sousTitre{\large Compétences évaluées}
\vspace*{6pt}

\competenceEvaluationDeux

\appreciation{120}

%%%% Exercice 2
\titrePartie{Sang et anémie}

Le sang est un mélange liquide composé de $54 \%$ de plasma, $45 \%$ de globules rouges et $1 \%$ de globules blancs.
On peut séparer ses constituants en utilisant une centrifugeuse, ce qui donnerait un mélange constitué de trois phases, comme présenté figure~\ref{fig:sang}.

%
\question{
  Indiquer en justifiant si le contenu des tubes à essais de la figure~\ref{fig:sang} est un mélange homogène ou un mélange hétérogènes.\competence{RCO, APP}
}{
  ...
}{0}


Le plasma est une solution aqueuse, qui contient des minéraux, des nutriments et les gaz lié à la respiration (dioxygène \chemfig{O_2} et dioxyde de carbone \chemfig{CO_2}).

%
\question{
  Indiquer le solvant et les solutés qui constituent le plasma.\competence{RCO}
}{
  ...
}{0}


Pour assurer son bon fonctionnement, l'organisme d'un être humain a besoin de fer \chemfig{Fe}.
On dit qu'une personne souffre d'anémie si la concentration massique en fer dans le sang est trop faible.
Le fer est transporté par une molécule dans le sang : l'hémoglobine.

\begin{figure}[!ht]
  \centering
  %
  \begin{subfigure}{0.48\linewidth}
    \tubeEssaiSangC{1.3}{1.4}{2.5}
    \label{fig:sang_normal}
  \end{subfigure}
  %
  \begin{subfigure}{0.48\linewidth}
    \tubeEssaiSangC{0.45}{0.55}{2.5}
    \label{fig:sang_anemie}
  \end{subfigure}
  %
  \caption{
    \centering
    Tube à essai contenant un échantillon de sang centrifugé : (a) d'une personne normale ; (b) d'une personne souffrant d'anémie.
  }
  \label{fig:sang}
\end{figure}

%
\question{
  En utilisant la figure~\ref{fig:sang}, indiquer en justifiant quel constituant du sang contient les molécules d'hémoglobines.\competence{APP, ANA/RAI}
}{
  ...
}{0}
\newpage


Mesurer la concentration massique en hémoglobine dans le sang permet de détecter les cas d'anémies.
On parle d'anémie si cette concentration massiques est inférieure a $1,\!2 \unit{g/L}$ pour une femme et $1,\!3 \unit{g/L}$ pour un homme.
Pour mesurer cette concentration, on peut réaliser une échelle de teinte, car c'est l'hémoglobine qui donne sa teinte rouge au sang.

\begin{figure}[!ht]
  %
  \centering
  \tubeEssaiSang{rougeClair}
  \tubeEssaiSang{rougeClair!75!white}
  \tubeEssaiSang{rougeClair!50!white}
  \tubeEssaiSang{rougeClair!25!white}
  \tubeEssaiSang{rougeClair!10!white} \\[4pt]
  \begin{subfigure}{0.6\linewidth}
    \centering
    \setlength{\extrarowheight}{4pt}
    \begin{tabular}{c | c | c | c | c | c}
      \rowcolor{gray!15} Solution & a   & b   & c   & d   & e \\ \hline
      Concentration (g/L)         & 1,4 & 1,3 & 1,2 & 1,1 & 1,0
    \end{tabular}
    \caption{}
  \end{subfigure}
  \tubeEssaiSang{rougeClair!65!white}
  %
  \caption{Schéma de l'échelle de teinte réalisée, avec les solutions étalons (a, b, c, d, e), leurs concentrations (f) et l'échantillon de sang à doser (g).}
  \label{fig:echelle_teinte}
\end{figure}

%
\question{
  Rappeler avec vos mots le principe général d'un dosage par étalonnage (que veut-on mesurer et comment fait-on).\competence{RCO, COM}
}{
  ...
}{0}

%
\question{
  Pour préparer des solutions, on peut effectuer une dilution ou une dissolution. Indiquer en justifiant laquelle des deux on effectue pour passer de la solution (a) à la solution (b).\competence{RCO, APP}
}{
  ...
}{0}
  
%
\question{
  Donner le nom de deux verreries nécessaires pour réaliser une dilution.\competence{RCO}
}{
  ...
}{0}

%
\question{
  En utilisant la figure~\ref{fig:echelle_teinte}, indiquer en justifiant la concentration en hémoglobine de l'échantillon de sang (g).\competence{APP, ANA/RAI, VAL}
}{
  ...
}{0}

%
\question{
  L'échantillon vient \variationSujet{d'une femme}{d'un homme}. Indiquer en justifiant si \variationSujet{elle}{il} souffre d'anémie ou non.\competence{APP, ANA/RAI, VAL}
}{
  ...
}{0}


%%%% Exercice 1
\newpage
\vspace*{-8pt}
\titrePartie{Conduite et alcoolémie}

Mélanie et sa femme Sihame sortent en voiture pour aller manger dehors.
Au restaurant Sihame boit un verre de \variationSujet{$200 \unit{mL}$}{$250 \unit{mL}$} d'alcool à $10 \degree$ : c'est-à-dire que $10 \%$ du volume de la boisson est de l'éthanol.

On va chercher à déterminer si Sihame pourra de nouveau conduire après le repas.

%
\question{
  Calculer le volume d'éthanol dans le verre.\competence{APP, REA}
}{
  ...
}{0}

%
\question{
  Sachant que l'éthanol a une masse volumique qui vaut $\rho_\text{éth} = 0,\!8 \unit{g/mL}$ et que $m_\text{éth} = \rho_\text{éth} \times V_\text{éth}$, calculer la masse d'éthanol bue par Sihame.\competence{APP, REA}
}{
  ...
}{0}


Le corps d'une femme adulte contient en moyenne $4,\!5 \unit{L}$ de sang. En France, \og \textit{il est interdit de conduire avec un taux d'alcool dans le sang supérieur ou égal à $0,5 \unit{g/l}$ de sang} \fg.

%
\question{
  Indiquer le nom de la grandeur utilisé en physique-chimie pour désigner le taux d'alcool dans le sang. Expliquer avec vos mots la différence entre cette grandeur et la masse volumique.\competence{RCO, COM}
}{
  ...
}{0}

%
\question{
  Rappeler la formule de la concentration massique.\competence{RCO}
}{
  ...
}{0}

%
\question{
  Calculer la concentration massique d'éthanol dans le sang de Sihame.\competence{APP, REA}
}{
  ...
}{0}

%
\question{
  Indiquer, en justifiant, si Sihame pourra conduire en sortant du restaurant.\competence{APP, VAl, ANA/RAI, COM}
}{
  ...
}{0}


En fait, quand un humain boit une boisson alcoolisé seule une petite partie de l'éthanol et absorbé par l'organisme. En moyenne seulement $12\%$ de l'éthanol passe dans le sang (si on a bu $10\unit{g}$ d'éthanol, $1,\!2\unit{g}$ passe dans le sang).

%
\question{
  Calculer de nouveau la concentration massique dans le sang de Sihame en tenant compte de cette information. Indiquer, en justifiant, si Sihame pourra conduire en sortant du restaurant.\competence{APP, REA, VAL, ANA/RAI, COM}
}{
  ...
}{0}

\setcounter{sousSectionNum}{0}

%%%% Correction
\newpage
\vspace*{-36pt}
\titreSousSection{Ma correction (à faire après la correction du professeur)}

\correctionEleve{40}


%%%% Bilan
\titreSousSection{Mon bilan après mon travail de correction}

\bilanCorrection{125}


%%%% Acquis
\titreSousSection{Mes acquis après mon travail de correction (à remplir par le professeur)}

\appreciation{3}