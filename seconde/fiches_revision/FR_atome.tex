\enTeteFiche{\sndAtom}

\begin{tableauConnaissances}
  % 
  Je sais calculer et utiliser les puissances de dix.
  Je sais utiliser la notation scientifique.
  Je sais calculer un ordre de grandeur avec les puissances de dix.
  & & & \\
  %
  Je sais que la masse d’un atome est essentiellement contenue dans celle de son noyau.
  Je sais que le noyau est très petit devant la taille de l'atome.
  & & & \\
  %
  Je connais les charges électriques des électrons, neutrons et protons.
  J'ai compris pourquoi l'atome a une charge globale neutre.
  & & & \\
  %
  Je connais les composants d'un atome et de son noyau. 
  Je sais faire la différence entre électrons, protons, neutrons et nucléons.
  & & & \\
  %
  Je peux déterminer la composition d'un atome à partir de sa notation symbolique \isotope{A}{Z}{X} et inversement.
  & & & \\
  %
  Je sais faire la différence entre les termes atome, ion, isotope.
  & & & \\
  %
  Je sais que la description d'un atome est un modèle, dont l'évolution dépend d'observations expérimentales.
  & & & \\
  %
  Je peux écrire la configuration électronique d'un atome en remplissant les sous-couches s et p dans le bon ordre.
  & & & \\
  %
  Je sais que les éléments chimiques sont rangées par colonne (famille) et par ligne (période) dans le tableau périodique.
  & & & \\
  %
  Je peux identifier la couche externe d'un atome et combien d'électrons de valence s'y trouvent.
  & & & \\
  %
  Je sais repérer la famille des gaz nobles dans le tableau périodique.
  Je sais que leur couche externe pleine les rend très stables.
  & & & \\
\end{tableauConnaissances}

\basDePageFicheReussite
\bigskip

\coursFicheReussite