%%%% début de la page
\teteSndMouv


%%%% TODO : 
% Changer les questions sur le poids -> calcul poids sur Terre et sur ISS, comparaison des deux 
% Deux forces qui s'exercent sur la spationaute : P et Finertie
% Changer question inertie -> F = m * v * v / (RT + h) est-ce égal à P = m*g


%%%%
\nomPrenomClasse

%%%%
\titreEvaluation{\sndMouv}

%%%% Compétences évaluées
\pasCorrection{
  \sousTitre{Compétences évaluées}
  
  \begin{tableauCompetences}
    APP &
    Représenter une situation par un schéma simple.
    Extraire des informations d'un document.
    & & & & \\
    %
    REA &
    Réaliser un calcul en donnant le résultat en notation scientifique avec les bonnes unités.
    Calculer des ordres de grandeurs.
    & & & & \\
    %
    VAL &
    Comparer des valeurs calculées avec des valeurs de références pour valider un raisonnement.
    & & & & \\
    %
    ANA/RAI &
    Mener un raisonnement à partir de grandeur données ou calculées.
  \end{tableauCompetences}
}

\appreciation{3}



%%%%
\titreSection{Impesanteur}

%%
\vspace*{-10pt}
\begin{doc}{Station spatiale internationale (ISS)}{doc:E3_station_ISS}
  On lit parfois que les spationautes flottent dans les stations spatiales, car la gravité terrestre n'agit plus sur les spationautes.
  
  \begin{wrapfigure}{r}{0.4\linewidth}
    \vspace*{-30pt}
    \centering
    \image{0.8}{images/spationaute_ISS}
  \end{wrapfigure}
  
  On s'intéresse à la station spatiale internationale, notée ISS, en orbite circulaire autour de la Terre à une hauteur $h$.
  L'ISS a une vitesse constante $v$.

  \important{Données :}
  \begin{listePoints}
    \item $G = \qty{6,67e-11}{\newton\m\squared \per\kg\squared}$
    \item $\MTerre = \qty{5,97e24}{\kg}$
    \item $\RTerre = \qty{6,37e6}{\m}$
    \item $h = \qty{3,70e5}{\m}$
    \item $v = \qty{7,66e3}{\m\per\s}$
  \end{listePoints}
\end{doc}

%%
\numeroQuestion
  \label{exo:schema_ISS}
  Quel est le mouvement décrit par l'ISS dans le référentiel lié au centre de la Terre ?
  Faire un schéma faisant figurer l'ISS, la Terre et la trajectoire qu'elle décrit.\competence{APP}\correction{\points{3}}

\question{
  Dans la station les spationautes ont un poids $P = m \times \gISS$.
  Calculer la valeur de $\gISS$ sachant que
  \begin{equation*}
    \gISS =  G \times \dfrac{\MTerre}{(\RTerre + h)^2}
  \end{equation*}\competence{APP, REA}
}{
  \begin{equation*}
    \gISS
    = \qty{6,67e-11}{\newton\m\squared \per\kg\squared}
    \dfrac{\qty{5,97e24}{\kg}} {(\num{6,37e6} + \qty{3,70e5}{\m})^2}
    %
    = \qty{8,77}{\newton}
    \points{1,5}
  \end{equation*}
}{0}


\pasCorrection{\newpage}
\question{
  Comparer avec l'accélération de pesanteur terrestre $g = \qty{9,81}{\newton\per\kg}$.
  Peut-on vraiment dire que la gravité terrestre n'agit plus sur les spationautes au sein de l'ISS ?\competence{VAL, ANA/RAI}
}{
  $\gISS$ est presque égal à $g$, donc la gravité agit toujours fortement.
  \points{1}
}{0}

\question{
  \label{exo:calcul_poids_ISS}
  En sachant que $\gISS = \qty{8,77}{\newton\per\kg}$, calculer le poids d'une spationaute de masse $m = \qty{65}{\kg}$ dans l'ISS.\competence{REA}
}{
  \begin{equation*}  
    P = \qty{65}{\kg} \times \qty{8,77}{\newton\per\kg} = \qty{570}{\newton}
    \points{1,5}
  \end{equation*}
}{0}


%%
\begin{doc}{Force d'inertie d'entraînement}{doc:E3_force_inertie}
  Un système dans un référentiel en rotation est soumis à une force \important{relative} qui dépend du référentiel, qu'on appelle \important{force d'inertie d'entraînement} $\vv{F}_\inertie$ ou encore « force centrifuge ».

  Cette force a pour direction la \important{droite reliant le centre du cercle et le centre du système.}
  Son sens est dirigé \important{vers l'extérieur du cercle.}
  C'est cette force qui explique pourquoi les passagers d'une voiture dans un rond-point sentent leur corps attiré vers l'extérieur du rond-point.

  \begin{encart}
    \important{Rappel :} le principe d'inertie dit que tout objet immobile est soumis à des forces dont la somme est nulle.
  \end{encart}
\end{doc}

\question{
  Expliquer avec vos mot le principe d'inertie.\competence{COM}
}{
  Il faut exercer une force sur un objet pour changer son mouvement, par défaut les objets se déplacent en ligne droite.
  \points{3}
}{0}

\question{
  Dans le référentiel lié à l'ISS, la spationaute est immobile.
  En utilisant le principe d'inertie et en justifiant clairement, donner la norme de la force d'inertie d'entraînement $F_\inertie$ qui s'exerce sur la spationaute.\competence{APP, ANA/RAI}
}{
  Comme la spationaute est immobile, la somme des forces qui s'exercent sur elle est nulle et $F_\inertie = P = \qty{8,77}{\newton}$
  \points{2}
}{0}

\numeroQuestion
  Compléter le schéma de la question~\ref{exo:schema_ISS} en représentant les forces s'exerçant sur la spationaute dans le référentiel lié à l'ISS.\competence{APP, REA}\correction{\points{1}}


\question{
  La norme de la force d'inertie d'entraînement exercée sur la spationaute est
  \begin{equation*}
    F_\inertie = m \times \dfrac{v^2}{R}
    \label{eq:force_inertie}
  \end{equation*}
  où $v$ est la vitesse du référentiel et $R$ est la distance entre le centre de rotation du référentiel et le centre du système (donc $R = \RTerre + h$ ici). 
  Cette relation est-elle cohérente avec le principe d'inertie ?
  
  \textit{
    Prendre des initiatives et les écrire, même si le raisonnement n'est pas complet.
    Tout début de réflexion sera valorisé.
  }
  \competence{APP, REA, VAL, ANA/RAI}
}{
  On calcule la valeur de $F_\inertie$ :
  \begin{equation*}
    F_\inertie
    = \qty{65}{\kg} \times \dfrac{(\qty{7,66}{\m\per\s})^2}{\num{6,37e6} + \qty{3,70e5}{\m}}
    = \qty{565}{\newton}
  \end{equation*}
  On retrouve presque la même valeur qu'à la question 4, cette norme est donc cohérente avec le principe d'inertie.
  \points{3}
}{0}

%% Formulation officielle du BAC
% Le candidat est invité à prendre des initiatives et à présenter la démarche suivie, même si elle n'a pas abouti. La démarche est évaluée et nécessite d'être correctement présentée.
\pasCorrection{
  \begin{coupDePouce}
    Utiliser les données de l'énoncé pour calculer $F_\inertie$. 
    Comparer cette valeur avec celle obtenue à la question~\ref{exo:calcul_poids_ISS} et conclure.
  \end{coupDePouce}
}



%%%%
\titreSection{Ordre de grandeur et écologie}

\begin{doc}{Activités humaines et émission de gaz à effet de serre}{doc:E3_effet_serre}
  On va chercher à estimer l'impact de notre alimentation sur le climat, en comparant avec l'impact du secteur automobile.
  Pour cela on va estimer l'ordre de grandeur des émissions de gaz à effet de serre rejetés lors de la production de nos aliments.
  
  Pour mesurer l'impact sur le climat d'un produit, on utilise le kilogramme de dioxyde de carbone équivalent, noté \unit{\COeq}.
  \important{Plus ce nombre est élevé et plus un produit a un impact important sur le dérèglement climatique.}
  
  Par exemple, produire 1 kg de viande de mouton équivaut à l'émission de 39,72 kg de \chemfig{CO_2}, soit 39,72 \unit{\COeq} (voir tableau).
  Cette émission correspond à l'émission d'une voiture qui parcours 400 km.
  
  \begin{encart}
    \important{Rappel :} l'ordre de grandeur d'un nombre est la puissance de 10 la plus proche de ce nombre. 
    Par exemple l'ordre de grandeur de 70 est 100.
    L'ordre de grandeur de 4 est 1.
  \end{encart}
\end{doc}

\question{
  Donner un ordre de grandeur du nombre de repas (déjeuner et dîner) par an.
  \textbf{Rappel :} 1 an = 365 jours\competence{REA}
}{
  Avec 2 repas par jour en moyenne, on a $2 \times 365 = 730 \sim 1000$ repas par an en ordre de grandeur.
  \points{1,5}
}{0}

\question{
  \label{exo:ordre_alimentation}
  À l'aide du graphique ci-dessous, calculer en ordre de grandeur le \unit{\COeq} annuel d'un régime à base de viande.
  On considère qu'un-e français-e mange en moyenne \qty{0,1}{kg} de viande par repas.\competence{APP, REA, ANA/RAI}
}{
  \begin{wrapfigure}{r}{0.5\linewidth}
    \centering
    \vspace*{-16pt}
    \image{1}{images/emission_CO2_alimentation}
  \end{wrapfigure}
  En ordre de grandeur, la production de $\qty{1}{\kg}$ de mouton ou de poulet émet $\qty{10}{\COeq}$.
  Par an, un-e français-e mange en moyenne $\qty{0,1}{\kg} \times 1000 = \qty{100}{\kg}$ de viande par an.
  Les émissions par an seront donc
  % Par an, les émissions d'un-e français-e sera donc en moyenne les émissions pour $1\unit{kg}$ multipliée par la masse de viande mangée, soit
  \begin{equation*}
    \qty{100}{\kg} \times \qty{10}{\COeq/\kg} = \qty{e3}{\COeq}
  \end{equation*}
  On notera qu'un régime végétarien permet de diviser par 10 ses émissions liées à l'alimentation en ordre de grandeur !
  \points{2}
}{0}

\pasCorrection{
  \begin{center}
    \vspace{-12pt}
    \image{0.6}{images/emission_CO2_alimentation}
  \end{center}
  \vspace{-12pt}
}


\question{
  En moyenne, une personne qui possède une voiture en France émet en ordre de grandeur \qty{e3}{\COeq} en roulant \textbf{par an}.
  Comparer avec l'ordre de grandeur des émissions annuelle dues à l'alimentation.\competence{VAL}
}{
  On voit qu'en ordre de grandeur, les émissions liées à la voiture et à une alimentation à base de viande sont identiques.
  \points{1}
}{0}

\question{
  En réalité, sur une année le transport représente en moyenne \qty{2,4e3}{\COeq} et l'alimentation \qty{2,0e3}{\COeq}, sur un total annuel d'émission de \qty{8,0e3}{\COeq} pour une personne vivant en France.
  
  Est-ce que le chiffre de l'alimentation est cohérent avec l'ordre de grandeur estimé question~\ref{exo:ordre_alimentation} ?\competence{APP, VAL}
}{
  En ordre de grandeur $\qty{2,0e3}{\COeq} \sim \qty{e3}{\COeq}$.
  Les estimations réalisées sont donc cohérentes avec les données mesurées.
  \points{1}
}{0}

\question{
  Calculer le pourcentage des \unit{\COeq} émis annuellement par un-e français-e moyen-ne pour se nourrir.
  \competence{APP, REA}
}{
  \begin{equation*}
    p = \dfrac{\qty{2,4e3}{\COeq}}{\qty{8,0e3}{\COeq}} = \qty{30}{\percent}
    \points{1,5}
  \end{equation*}
}{0}



%%%%
\pasCorrection{\setcounter{sousSectionNum}{0}

%%%% Correction
\newpage
\vspace*{-36pt}
\titreSousSection{Ma correction (à faire après la correction du professeur)}

%%%% Tableau de correction élève
\begin{tblr}{
    row{1} = {couleurPrim!20}, hlines,
    colspec = {| X[-1, c] | X[2, c] | X[2, c] | X[2, c] |}
  }
  \textbf{Question} & 
  \textbf{L'erreur} &
  \textbf{Analyse de l'erreur} &
  \textbf{La correction} \\
  %
  \phantom{b} \vspace{55 pt} & & & \\
  \phantom{b} \vspace{55 pt} & & & \\
  \phantom{b} \vspace{55 pt} & & & \\
  \phantom{b} \vspace{55 pt} & & & \\
\end{tblr}


%%%% Bilan
\titreSousSection{Mon bilan après mon travail de correction}

%%%% Tableau bilan de la correction
\begin{tableau}{| X[c] | X[c] |}
  \textbf{Ce que je n'avais pas compris...} &
  \textbf{Ce que maintenant j'ai compris...} \\
  \phantom{b} \vspace{150 pt} & \\
\end{tableau}


%%%% Acquis
\titreSousSection{Mes acquis après mon travail de correction (à remplir par le professeur)}

\appreciation{2}}