\teteSndCorp

\titre{Grille d'évaluation du TP 1.2 - Repression des fraudes}

\nomPrenom*
\medskip

\begin{tblr}{
  colspec = {l X[l,m] c c}, hlines, vlines,
  row{1} = {couleurSec-150},
}
  Items & Critères de réussite & Points max. & Points \\
  %
  \SetCell[r = 2]{l} Problématique 
  & Formulation sous forme d'une question. & 1 \\
  & Précision sur le fait que l'on veut mesurer une fraction massique ou une masse de sucre dissoute en fonction de la mission. & 1 \\
  %
  \SetCell[r = 4]{l} Protocole 
  & Toutes les étapes sont détaillées. & 1 \\
  & Étapes : bécher vide posé sur la balance, puis tare de la balance ; prélèvement du liquide à peser ; le liquide est versé dans le bécher ; lecture de la masse du liquide. & 3 \\
  & Les étapes sont dans le bon ordre. & 1 \\
  & Le volume (10 mL) de la pipette jaugée est précisé. & 1 \\
  %
  \SetCell[r = 2]{l} Mesure de la masse 
  & La masse du liquide est écrite clairement. & 1 \\
  & Les mesures ont été réalisées proprement. & 1 \\
  %
  \SetCell[r = 2]{l} Calcul de la masse volumique 
  & Rappel de la formule littérale ou numérique & 1 \\
  & L'unité de la masse volumique (\unit{\g/\mL}) est précisée & 0,5 \\
  %
  \SetCell[r = 4]{l} Conclusion
  & Utilisation de la masse volumique pour trouver la grandeur cherchée graphiquement. & 1 \\
  & Utilisation d'une phrase pour indiquer qu'on a utilisé le graphique pour trouver la fraction volumique ou la masse de sucre & 0,5 \\
  & Comparaison de la grandeur mesurée avec la grandeur annoncée par le constructeur & 1 \\
  & Conclusion claire sur qui a menti à partir de cette comparaison. & 1 \\
  Seconde mission & Répétition des étapes citées précédemment & 5 \\
  %
  \SetCell[c = 2]{c, couleurSec-150} \important{Total : } & & 20 \\
\end{tblr}