%%%% début de la page
\teteSndCorp
\setcounter{page}{1}

%%%%
\nomPrenomClasse

%%%%
\titreEvaluation{Corps purs et mélanges}

%%%% Compétences évaluées
\sousTitre{Compétences évaluées}

\begin{tableauCompetences}
  \centering RCO &
  Connaître le vocabulaire du cours et les relations importantes.
  & & & & \\
  %
  \centering APP &
  Extraire des informations d'un document.
  & & & &  \\
  %
  \centering VAL &
  Comparer des valeurs calculées avec des valeurs de références pour valider un raisonnement.
  & & & & \\
  %
  \centering REA &
  Réaliser un calcul en donnant le résultat en notation scientifique avec les unités.
  & & & & \\
\end{tableauCompetences}

\appreciation{4}


%%%% Exercice I
\vspace*{-20pt}
\titrePartie{Marais salant et pollution}

Les marais salants sont de grands bassin remplis par d'eau de mer, qui est riche en sel.
Le sel est du chlorure de sodium de formule brute \chemfig{NaCl}.

%
\question{
  Indiquer en justifiant si l'eau de mer est un corps pur ou un mélange.\competence{RCO, APP}
}{
  C'est un mélange, composé d'au moins deux espèces chimique : l'eau et le sel.
}{1}

Le soleil et le vent font s'évaporer l'eau de mer, mais le sel reste au fond des bassins. 
Après plusieurs étapes d'évaporation et de remplissage, la quantité de sel contenue dans l'eau des bassins devient très importante.
La masse volumique de l'eau salée augmente avec la quantité de sel.

\question{
  Rappeler la relation mathématique entre la masse volumique de l'eau salée $\rho$, sa masse $m$ et le volume $V$ qu'elle occupe.\competence{RCO}
}{
  \begin{equation*}
    \rho = \frac{m}{V}
  \end{equation*}
}{2}

Les productrices ou producteurs peuvent récolter le sel lorsque la masse volumique de l'eau salée dans un bassin est \textbf{supérieure} à $\rho_\text{récolte} = \qty{1,15}{\g/\ml}$.

\question{
  Une productrice de sel pèse \qty{50}{\ml} d'eau salée provenant d'un bassin et mesure une masse de \variationSujet{\qty{60}{\g}}{\qty{55}{\g}}.
  Calculer la masse volumique de l'eau salée dans ce bassin.\competence{REA}
}{
  $m_\text{eau salée} = \qty{60}{\g}$ et $V_\text{eau salée} = \qty{50}{\ml}$. Donc 
  \begin{equation*}
    \rho_\text{eau salée} = \frac{\qty{60}{\g}}{\qty{50}{\ml}} = \qty{1,2}{\g/\ml}
  \end{equation*}
}{2}

\question{
  Est-ce que la productrice peut récolter le sel dans ce bassin ? Justifier.\competence{VAL}
}{
  La masse volumique de l'eau du bassin est \variationSujet{supérieure}{inférieure} à celle nécessaire pour récolter, donc la productrice \variationSujet{peut}{ne peut pas} récolter le sel.
}{2}


Une ingénieure agronome réalise une inspection des marais salants en baie de somme.
Pour vérifier que des ions ne pollue pas les marais, elle prélève puis teste l'eau des bassins avec différentes espèces chimiques.
Un tableau récapitulatif des tests qu'elle peut réaliser est présenté ci-dessous

\begin{center}
  \begin{tableau}{| c | c | c |}
    Espèce utilisée & Ion recherché & Résultat d'un test positif \\
    Nitrate d'argent & \ionChlorure & Précipité blanc \\
    \SetCell[r = 3]{c} Hydroxyde de sodium & \ionCuivreII & Précipité bleu \\
    & \ionFerII & Précipité vert \\
    & \ionFerIII & Précipité rouille \\
    Chlorure de baryum & \ionSulfate & Précipité blanc \\
  \end{tableau}
\end{center}

\question{
  L'ingénieure commence par verser quelques gouttes de \variationSujet{chlorure de Baryum}{nitrate d'argent} dans un tube à essai contenant l'eau prélevée.
  Elle observe la formation d'un précipité blanc.
  Indiquer quel ion pollue le bassin, en justifiant.\competence{APP}
}{
   D'après le tableau, le \variationSujet{chlorure de baryum}{nitrate d'argent} a réagit avec les ions \variationSujet{sulfates}{chlorure} pour former un précipité blanc.
}{2}

\question{
  L'ingénieure veut réaliser des tests supplémentaires pour savoir si le bassin est aussi pollué par des ions Fer.
  Indiquer quel(s) réactif(s) elle doit utiliser et quel résultat permettrait de conclure à la présence d'ions Fer.\competence{APP}
}{
  Pour identifier la présence d'ions fer, l'ingénieure devra réaliser un test avec l'hydroxyde de sodium.
  Si elle voit apparaître un précipité vert, alors le bassin contient des ions Fer II.
  Si elle voit apparaître un précipité rouille, alors le bassin contient des ions Fer III.
  Si aucun précipité n'apparaît, le bassin n'est pas pollué.
}{4}


%%%% Exercice II
\vspace*{-20pt}
\titrePartie{Huile essentielle d'orange}

Les huiles essentielles de \variationSujet{lavande}{menthe} sont obtenues par distillation des fleurs de \variationSujet{lavandes}{menthe}.
Les huiles essentielles sont riches en molécules odorantes.
On réalise une Chromatographie sur Couche Mince (CCM) afin d'identifier quelques espèces chimiques présentes dans cette huile essentielle.
Le chromatogramme obtenue après la montée de l'éluant est présenté ci-dessous.

\begin{figure}[!ht]
  \centering
  \variationSujet{
    \image{0.21}{images/chromato_lavande}
  }{
    \image{0.2}{images/chromato_menthe}
  }
  
  A : huile essentielle de \variationSujet{lavande}{menthe},
  B : \variationSujet{linalol}{menthol},
  C : \variationSujet{acétate de linalyle}{limonène}
  
  Chromatogramme.
\end{figure}

\numeroQuestion
Indiquer sur le chromatogramme où se trouvent la ligne de dépôt, la couche mince et le front de l'éluant.\competence{RCO}

\question{
  Justifier que l'huile essentielle de \variationSujet{lavande}{menthe} est un mélange.\competence{RCO, APP}
}{
  D'après le chromatogramme, le dépôt d'huile essentielle s'est divisé en trois tâches, c'est donc un mélange.
}{2}

\question{
  En comparant les hauteurs des tâches, indiquer quelles sont les espèces chimiques présentes dans l'huile essentielle de \variationSujet{lavande}{menthe}.\competence{RCO, APP, VAL}
}{
  Sur un chromatogramme, deux composés sont identiques s'ils sont montés à la même hauteur. 
  Sur le chromatogramme, on voit que la première tâche d'huile essentielle de \variationSujet{lavande}{menthe} est à la même hauteur que le \variationSujet{l'acétate de linalyle}{limonène} et que la seconde tâche est à la même hauteur que le \variationSujet{linalol}{menthol}.
  L'huile essentielle de \variationSujet{lavande}{menthe} est donc composée de \variationSujet{l'acétate de linalyle}{limonène} et de \variationSujet{linalol}{menthol}.
}{3}

\question{
  Peut-on identifier le troisième composé présent dans l'huile essentielle avec ce chromatogramme ?\competence{APP, VAL}
}{
  Non, car on n'a pas d'espèce chimique de référence pour comparer.
}{2}


%%%% Exercice III
\titrePartie{Étalon du kilogramme}

\begin{wrapfigure}[6]{r}{0.2\linewidth}
  \centering
  \vspace*{-40pt}
  \image{0.9}{images/standard_kilogram.jpg}
\end{wrapfigure}

Le kilogramme est l'unité de base de la masse dans le système international.
L'étalon qui \textbf{a servi à définir le kilogramme} jusqu'en mai 2019 est conservé par le Bureau International des Poids et Mesures (BIPM).
Ce prototype est un cylindre constitué d'un alliage de platine et d'iridium, de volume $V_\text{étalon} = \qty{47,191}{\cm\cubed}$ et de masse volumique $\rho_\text{étalon} = \qty{21,191}{\g/\cm\cubed}$.

\question{
  Sans calcul, indiquer la masse de l'étalon.\competence{APP}
}{
  L'étalon sert à définir le kilogramme.
  Sa masse est donc de \qty{1}{\kg} par définition.
}{1}

\question{
  Le prototype est composé de \qty{0,9}{\kg} de platine et de \qty{0,1}{\kg} d'iridium.
  Calculer la fraction massique de platine et d'iridium.\competence{REA, APP}
}{
  Pour le platine : $\qty{0,9}{\kg} / \qty{1}{\kg} = \qty{90}{\percent}$.
  Pour l'iridium : $\qty{0,1}{\kg} / \qty{1}{\kg} = \qty{10}{\percent}$
}{2}

\textit{Rappel :} la fraction massique d'une espèce dans un échantillon est la masse de l'espèce divisée par la masse totale de l'échantillon.
Par exemple pour le platine :
\begin{equation*}
  p_v(\text{platine}) = \frac{m_\text{platine}}{m_\text{étalon}}
\end{equation*}

\question{
  Historiquement, un premier cylindre avait été réalisé avec \qty{11,1}{\percent} d'iridium, qui a une masse volumique plus élevée que le platine.
  Sachant que son volume était identique à l'étalon actuel, indiquer si la masse de ce cylindre valait 1 kg et expliquer pourquoi il avait été rejeté par le BIPM.\competence{APP}
}{
  L'étalon actuel a précisément une masse de \qty{1}{\kg} avec \qty{10}{\percent} d'iridium.
  Le premier cylindre comportait $\qty{11,1}{\percent} > \qty{10}{\percent}$ d'iridium.
  Comme l'iridium a une masse volumique plus élevée, sa masse valait plus de \qty{1}{\kg}, d'où le rejet par le BIPM.
}{2}


\setcounter{sousSectionNum}{0}

%%%% Correction
\newpage
\vspace*{-36pt}
\titreSousSection{Ma correction (à faire après la correction du professeur)}

%%%% Tableau de correction élève
\begin{tblr}{
    row{1} = {couleurPrim!20}, hlines,
    colspec = {| X[-1, c] | X[2, c] | X[2, c] | X[2, c] |}
  }
  \textbf{Question} & 
  \textbf{L'erreur} &
  \textbf{Analyse de l'erreur} &
  \textbf{La correction} \\
  %
  \phantom{b} \vspace{55 pt} & & & \\
  \phantom{b} \vspace{55 pt} & & & \\
  \phantom{b} \vspace{55 pt} & & & \\
  \phantom{b} \vspace{55 pt} & & & \\
\end{tblr}


%%%% Bilan
\titreSousSection{Mon bilan après mon travail de correction}

%%%% Tableau bilan de la correction
\begin{tableau}{| X[c] | X[c] |}
  \textbf{Ce que je n'avais pas compris...} &
  \textbf{Ce que maintenant j'ai compris...} \\
  \phantom{b} \vspace{150 pt} & \\
\end{tableau}


%%%% Acquis
\titreSousSection{Mes acquis après mon travail de correction (à remplir par le professeur)}

\appreciation{2}