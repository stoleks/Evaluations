%%
%%%% début de la page
\newpage
\setcounter{page}{1}
\teteSndMole

%%%%
\nomPrenomClasse

%%%%
\vspace*{-8pt}
\numeroActivite{2}
\titreEvaluation{\sndMole}
\vspace*{-10pt}

%%%% evaluation
\sousTitre{Compétences évaluées}
\vspace*{2pt}

\competenceEvaluationCinq

\appreciation{6}

%%%%
\sousTitre{QCM - cocher \textit{la ou les} bonnes réponses.}

\separationBlocs{
  \QCM{
    L'atome de \variationSujet
    {sodium \chemfig{Na} est devenu l'ion \chemfig{Na^+}}
    {de fluor \chemfig{F} est devenu l'ion \chemfig{F^{-}}}
    parce que
  }{
    \item\reponseQCM un électron lui a été arraché
    \item un électron lui a été donné
    \item il a gagné un proton
  }
  \QCM{
    L'ion \variationSujet{\chemfig{Na^+}}{\chemfig{F^{-}}}
  }{
    \item est un anion
    \item\reponseQCM est un cation
    \item\reponseQCM a une charge positive
    \item a une charge négative
  }
  \QCM{Le cortège électronique a une structure particulière}{
    \item\reponseQCM avec des couches ($1, 2, 3, \ldots$) et des sous-couches ($s, p, \ldots$)
    \item\reponseQCM les sous-couches $s$ peuvent contenir au plus 2 électrons
    \item\reponseQCM les sous-couches $p$ peuvent contenir au plus 6 électrons
  }
}{
  \QCM{La dernière colonne de la classification périodique s'appelle la famille }{
    \item\reponseQCM des gaz nobles
    \item des halogènes
    \item des alcalins
  }
  \QCM{Les entités chimiques \isotope{63}{29}{Cu}, \chemfig{Cu^+}, \chemfig{Cu^{2+}} sont toutes du Cuivre car elles ont}{
    \item le même nombre d'électrons
    \item\reponseQCM le même nombre de protons $Z$
    \item le même nombre de nucléons $A$
  }
  \QCM{Les atomes peuvent s'associer en molécule pour}{
    \item\reponseQCM adopter la configuration électronique du gaz noble le plus proche
    \item\reponseQCM respecter la règle de l'octet ou du duet
    \item\reponseQCM avoir une charge électrique totale nulle
  }
}

\newpage
\vspace*{-24pt}
\separationBlocs{
  \QCM{
Le gaz noble le plus proche \variationSujet{du Béryllium ($Z = 4$)}{de l'Oxygène ($Z = 8$)} est
}{
    \item\reponseQCM l'Hélium ($Z = 2$)
    \item le Néon ($Z = 10$)
    \item l'Argon ($Z = 18$)
  }
}{
  \QCM{
Pour gagner en stabilité, \variationSujet{le Béryllium}{l'Oxygène} pourra
}{
    \item\reponseQCM perdre 2 électrons
    \item gagner 2 électrons
    \item gagner 4 électrons
  }
}


%%%
\titreSection{Structure électronique d'un atome}
\vspace*{-8pt}

\begin{doc}{Tableau périodique}{doc:tableau_periodique}
  \hspace*{-16pt}
  \tableauPeriodique{
    %% Group 1 - IA
    \node[name=H, Nonmetal]             {\elementH};
    \node[name=Li, below of=H, Alkali]  {\elementLi};
    \node[name=Na, below of=Li, Alkali] {\elementNa};
    %% Group 2 - IIA
    \node[name=Be, right of=Li, AlkaliEarth] {\elementBe};
    \node[name=Mg, below of=Be, AlkaliEarth] {\elementMg};
    %% Group 13 - IIIA
    \node[name=B,  right of=Be, Metalloid] {\elementB};
    \node[name=Al, below of=B,  Metal]     {\elementAl};
    %% Group 14 - IVA
    \node[name=C,  right of=B, Nonmetal]  {\elementC};
    \node[name=Si, below of=C, Metalloid] {\elementSi};
    %% Group 15 - VA
    \node[name=N, right of=C, Nonmetal] {\elementN};
    \node[name=P, below of=N, Nonmetal] {\elementP};
    %% Group 16 - VIA
    \node[name=O, right of=N, Nonmetal] {\elementO};
    \node[name=S, below of=O, Nonmetal] {\elementS};
    %% Group 17 - VIIA
    \node[name=F,  right of=O, Halogen] {\elementF};
    \node[name=Cl, below of=F, Halogen] {\elementCl};
    %% Group 18 - VIIIA
    \node[name=Ne, right of=F,  NobleGas] {\elementNe};
    \node[name=Ar, below of=Ne, NobleGas] {\elementAr};
    \node[name=He, above of=Ne, NobleGas] {\elementHe};
  
    %% Period
    \node[name=Period1, left of=H,  PeriodLabel] {\normalsize{1}};
    \node[name=Period2, left of=Li, PeriodLabel] {\normalsize{2}};
    \node[name=Period3, left of=Na, PeriodLabel] {\normalsize{3}};
  }
\end{doc}

\question{
  Donner le nombre d'électrons de l'azote \chemfig{N}, du magnésium \chemfig{Mg} et de l'argon \chemfig{Ar}.\competence{APP, ANA/RAI}
}{
  Le numéro atomique de l'azote est 7, il possède donc 7 protons et 7 électrons par neutralité électrique de l'atome.
  Le magnésium possède 12 électrons et l'argon possède 18 électrons.
}{2}

\question{
  Donner la structure électronique de l'azote, du magnésium et de l'argon.\competence{REA}
}{
  $\chemfig{N} : 1s^2 \mathbf{2s^2 2p^3}$, $\chemfig{Mg} : 1s^2 2s^2 2p^6 \mathbf{3s^2}$, $\chemfig{Ar} : 1s^2 2s^2 2p^6 \mathbf{3s^2 3p^6}$
}{3}

\question{
  Entourer la couche externe de chacune des structures électronique et indiquer le nombre d'électrons de valence de chaque atome.\competence{COM, ANA/RAI}
}{
  Couche externe de l'azote : 2 avec 5 électrons de valences.
  Couche externe du magnésium : 3 avec 2 électrons de valences.
  Couche externe de l'argon : 3 avec 8 électrons de valences.
}{2}

\question{
  Parmi ces trois atomes, lequel est le plus stable ? Justifier.\competence{ANA/RAI}
}{
  L'argon, car sa couche externe est pleine : c'est un gaz noble.
  \vspace*{20pt}
}{2}


\vspace*{-20pt}
\question{
  Rappeler la règle de l'octet avec vos mots.\competence{COM}
}{
  Pour gagner en stabilité, un atome de numéro atomique > 6, peut perdre ou gagner des électrons pour atteindre la structure électronique du gaz noble \textbf{le plus proche}, avec 8 électrons sur sa couche externe.
}{3}

\question{
  D'après cette règles, quel ion pourra être formé à partir d'un atome de magnésium ? Expliquer.\competence{ANA/RAI, COM}
}{
  D'après la règle de l'octet, le magnésium va perdre 2 électrons pour atteindre la configuration électronique du néon. On aura donc l'ion $\chemfig{Mg^{2+}}$.
}{2}


%%%%
\titreSection{Stabilité d'une molécule}

\vspace*{-30pt}
\variationSujet{
\begin{doc}{L'ammoniac}{doc:ammoniac}
  L'ammoniac est un gaz irritant à température ambiante.
  La molécule d'ammoniac est composé d'hydrogène \chemfig{H} ($Z = 1$) et d'azote \chemfig{N} ($Z = 7$).
  Le schéma de Lewis de la molécule est le suivant :
  \begin{center}
    \chemfig[atom sep=30pt, atom style={scale=1.5}, line width=6pt]{
      H - \charge{-90:4pt=\|}{N} (-[3] H) - H
    }
  \end{center}
  
  \phantom{bla}
\end{doc}
}{
\begin{doc}{La phosphine}{doc:phosphine}
  La phosphine est un gaz incolore et mortellement toxique, utilisé comme pesticide. 
  La molécule de phosphine est composée d'hydrogène \chemfig{H} ($Z = 1$) et de phosphore \chemfig{P} ($Z = 15$).
  Le schéma de Lewis de la molécule est le suivant :
  \begin{center}
    \chemfig[atom sep=30pt, atom style={scale=1.5}, line width=6pt]{
      H - \charge{-90:4pt=\|}{P} (-[3] H) - H
    }
  \end{center}
  
  \phantom{bla}
\end{doc}
}
\vspace*{-12pt}

\question{
  Indiquer la formule de la molécule\variationSujet{d'ammoniac}{de phosphine}.\competence{APP}
}{
  La molécule est composée de 1 azote et de 3 hydrogène : \chemfig{NH_3}
}{1}

\question{
  Quelle règle doit respecter l'atome d'hydrogène pour gagner en stabilité ?\competence{COM}
}{
  La règle du duet : il doit gagner un électron en formant une liaison covalente.
}{2}

\question{
  Combien de liaisons covalentes a formé
  \variationSujet{l'azote}{le phosphore}
  dans la molécule
  \variationSujet{d'ammoniac}{de phosphine}
  ?
  Est-ce cohérent avec la règle de l'octet ?\competence{APP, VAL}
}{
  L'azote a formé 3 liaisons covalentes, ce qui lui a permis d'ajouter 3 électrons sur sa couche externe pour la compléter et respecter la règle de l'octet.
}{3}

\question{
  Légender le schéma de Lewis de la molécule
  \variationSujet{d'ammoniac}{de phosphine}
  du doc.~\ref{doc:ammoniac}.\competence{COM}
}{
  Il faut légender les doublets liants et le double non-liant en plus des éléments chimiques.
}{0}


%%
\setcounter{sousSectionNum}{0}

%%%% Correction
\newpage
\vspace*{-36pt}
\titreSousSection{Ma correction (à faire après la correction du professeur)}

\correctionEleve{40}


%%%% Bilan
\titreSousSection{Mon bilan après mon travail de correction}

\bilanCorrection{125}


%%%% Acquis
\titreSousSection{Mes acquis après mon travail de correction (à remplir par le professeur)}

\appreciation{3}