%%%% début de la page
\newpage
\setcounter{page}{1}
\sndEnTeteDeux

%%%%
\nomPrenomClasse

%%%%
\numeroActivite{3}
\titreEvaluation{\sndChapitreDeux}

%%%% evaluation
\sousTitre{\large Compétences évaluées}
\vspace*{6pt}

\competenceEvaluationTrois

\appreciation{90}

%%%%
\vspace*{-24pt}
\titrePartie{Impesanteur}

%%
\vspace*{-4pt}
\begin{doc}{Station spatiale internationale (ISS)}{doc:station_ISS}
  On lit parfois que les spationautes flottent dans les stations spatiales, car la gravité terrestre n'agit plus sur les spationautes.

  \begin{wrapfigure}{r}{0.4\linewidth}
    \vspace*{-32pt}
    \centering
    \image{0.9}{sommative/2nd_Mouv/spationaute_ISS}
  \end{wrapfigure}
  
  On s'intéresse à la station spatiale internationale (ou ISS), en orbite circulaire autour de la Terre à une hauteur $h$. L'ISS a une vitesse constante $v$.

  \textbf{Données :}
  \begin{listePoints}
    \item $G = 6,\!67 \times 10^{-11} \unit{N \cdot m^2 \cdot kg^{-2}}$
    \item $\MTerre = 5,\!97 \times 10^{24} \unit{kg}$
    \item $\RTerre = 6,\!37 \times 10^3 \unit{km}$
    \item $h = 3,\!70 \times 10^2 \unit{km}$
    \item $v = 7,\!66 \times 10^3 \unit{m\cdot s^{-1}}$
  \end{listePoints}
\end{doc}

%%
\vspace*{-24pt}
\question{
  \label{exo:schema_ISS}
  Quel est le mouvement décrit par l'ISS dans le référentiel lié au centre de la Terre ?
  Faire un schéma faisant figurer l'ISS, la Terre et la trajectoire qu'elle décrit.\competence{APP}
}{
}{0}

\question{
  Dans la station les spationautes ont un poids $P = m \times \gISS$.
  Calculer la valeur de $\gISS = G \times \dfrac{\MTerre}{(\RTerre + h)^2}$.\competence{APP, REA}
}{
}{0}

\question{
  Comparer avec l'accélération de pesanteur terrestre $g = 9,\!81 \unit{N\cdot kg^{-1}}$.
  Peut-on vraiment dire que la gravité terrestre n'agit plus sur les spationautes au sein de l'ISS ?\competence{VAL, ANA/RAI}
}{
}{0}

\question{
  \label{exo:calcul_poids_ISS}
  En sachant que $\gISS = 8,\!77 \unit{N\cdot kg^{-1}}$, calculer le poids d'une spationaute de masse $m = 65 \unit{kg}$ dans l'ISS.\competence{REA}
}{
}{0}

%%
\begin{doc}{Force d'inertie d'entraînement}
  Un système dans un référentiel en rotation est soumis à une force \textit{relative} (qui dépend du référentiel), qu'on appelle \textbf{force d'inertie d'entraînement} $\vv{F}_\inertie$ ou encore \og force centrifuge \fg.

  Cette force a pour direction la droite reliant le centre du cercle et le centre du système.
  Son sens est dirigé vers l'extérieur du cercle.
  C'est cette force qui explique pourquoi les passagers d'une voiture dans un rond-point sentent leur corps attiré vers l'extérieur du rond-point.
\end{doc}

\question{
  Expliquer avec vos mot le principe d'inertie.\competence{RCO, COM}
}{
}{0}

\question{
  Dans le référentiel lié à l'ISS, cette spationaute est immobile.
  En utilisant le principe d'inertie et en justifiant clairement, donner la norme de la force d'inertie d'entraînement $F_\inertie$ qui s'exerce sur la spationaute.\competence{RCO, APP, ANA/RAI}
}{
}{0}

\question{
  Compléter le schéma de la question~\ref{exo:schema_ISS} en représentant les forces s'exerçant sur la spationaute dans le référentiel lié à l'ISS.\competence{APP, REA}
}{
}{0}

\question{
  La norme de la force d'inertie d'entraînement exercée sur la spationaute est
  \begin{equation}
    F_\inertie = m \times \frac{v^2}{R}
    \label{eq:force_inertie}
  \end{equation}
  où $v$ est la vitesse du référentiel et $R$ est la distance entre le centre de rotation du référentiel et le centre du système. 
  Cette relation est-elle cohérente avec le principe d'inertie ?
  
  \textit{
    Prendre des initiatives et les écrire, même si le raisonnement n'est pas complet.
    Tout début de réflexion sera valorisé.
  }
  \competence{APP, REA, VAL, ANA/RAI}
}{
}{0}

%% Formulation officielle du BAC
% Le candidat est invité à prendre des initiatives et à présenter la démarche suivie, même si elle n'a pas abouti. La démarche est évaluée et nécessite d'être correctement présentée.

\begin{coupDePouce}
  Utiliser les données de l'énoncé pour calculer la norme de la force d'inertie d'entraînement avec la relation~(\ref{eq:force_inertie}). 
  Comparer cette norme avec celle obtenue à la question~\ref{exo:calcul_poids_ISS} et conclure.
\end{coupDePouce}


%%%%
\titrePartie{Mouvement d'un ballon}

%%
\vspace*{-8pt}
\begin{doc}{Penalty}{doc:penalty}
  Lors du match France-Angleterre de la coupe du monde masculine de football de 2022, un joueur anglais a tiré et raté un penalty.
  On s'intéresse au mouvement du ballon avant, puis pendant le tir du penalty.
\end{doc}

%%
\question{
  Avant le tir le ballon est immobile sur le sol. Lister les forces qui s'exercent sur le ballon. Schématiser le ballon et les forces qui s'exercent sur lui.\competence{RCO, APP, REA, ANA/RAI}
}{}{0}

\question{
  Les forces qui s'exercent sur le ballon se compensent-elles ?\competence{RCO, APP, ANA/RAI}
}{}{0}

%%
\vspace*{-8pt}
\begin{doc}{Chronophotographie}{doc:mouvement_ballon}
  Votre professeur préféré a réalisé une chronophotographie de la position du centre du ballon pendant le tir. \textbf{La durée entre chaque image est $\mathbf{\Delta t = 0,\!018 \unit{s}}$.} ($t_1 = 0\unit{s}$, $t_2 = \Delta t$, $t_3 = 2\times \Delta t, \ldots$)
  \begin{center}
    \image{0.7}{sommative/2nd_Mouv/penalty}
    
    \small{
      Position du ballon (en mètre)
    }
  \end{center}
\end{doc}

\question{
  Pendant le tir, quel référentiel permet d'étudier le mouvement du ballon ? (Donner un objet de référence.)\competence{RCO, APP, ANA/RAI}
}{}{0}

\question{
  \label{exo:mvt_ballon}
  D'après la chronophotographie du document~\ref{doc:mouvement_ballon}, décrire le mouvement du ballon, en justifiant.\competence{RCO, APP}
}{}{0}

\question{
  Calculer la norme des vecteurs vitesses $\vv{v}_2$ et $\vv{v}_6$.\competence{RCO, APP, REA}
}{}{0}

% \begin{coupDePouce}
%   Commencer par calculer la distance parcourue entre les points $P_1$ et $P_3$ pour $v_2$ (ou $P_5$ et $P_7$ pour $v_6$).
% \end{coupDePouce}

\question{
  Tracer sur le bas du document~\ref{doc:mouvement_ballon} les vecteurs vitesses $\vv{v}_2$ et $\vv{v}_6$ en choisissant une échelle pertinente.\competence{REA, ANA/RAI}
}{}{0}

\question{
  Pendant un penalty, en moyenne les ballons tirés ont une vitesse $v = 150\unit{km \cdot h^{-1}}$.
  Discuter du réalisme des données de l'énoncé. \textbf{Rappel :} $1 \unit{m\cdot s^{-1}} = 3,\!6 \unit{km\cdot h^{-1}}$ \competence{REA, VAL, ANA/RAI}
}{}{0}


\setcounter{sousSection}{0}

%%%% Correction
\newpage
\vspace*{-36pt}
\titreSousSection{Ma correction (à faire après la correction du professeur)}

\correctionEleve{40}


%%%% Bilan
\titreSousSection{Mon bilan après mon travail de correction}

\bilanCorrection{125}


%%%% Acquis
\titreSousSection{Mes acquis après mon travail de correction (à remplir par le professeur)}

\appreciation{80}