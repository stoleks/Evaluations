%%%% Exercice 1
\vspace*{-24pt}
\titrePartie{Marais salant et pollution}

Les marais salants sont de grands bassin remplis par de l'eau de mer, riche en sel.
Le sel est du chlorure de sodium (\chemfig{NaCl}).

%
\question{
  Indiquer si l'eau de mer est un corps pur ou un mélange.\competence{RCO, APP}
}{1}

Le soleil et le vent font s'évaporer l'eau de mer, mais le sel reste au fond des bassins. 
Après plusieurs étapes d'évaporation et de remplissage, la quantité de sel contenue dans l'eau des bassins devient très importante.
La masse volumique de l'eau salée augmente avec la quantité de sel.

\question{
  Rappeler la relation entre la masse volumique de l’eau salée $\rho_\text{eau salée}$, sa masse $m_\text{eau salée}$ et le volume $V_\text{eau salée}$ qu’elle occupe.\competence{RCO}
}{2}

Les salicultrices et saliculteurs peuvent récolter le sel lorsque la masse volumique de l’eau salée dans un bassin est supérieure à $\rho_\text{récolte} = 1,\!15 \unit{g/mL}$.

\question{
  Un saliculteur pèse $50 \unit{mL}$ d’eau salée provenant d'un bassin et mesure une masse de $55 \unit{g}$.
  Calculer la masse volumique de l'eau salée dans ce bassin.\competence{REA}
}{2}

\question{
  Est-ce que le saliculteur peut récolter le sel dans ce bassin ? Justifier.\competence{ANA/RAI, VAL}
}{2}


Une ingénieure agronome réalise une inspection des marais salants en baie de somme.
Pour vérifier que des ions ne pollue pas les marais, elle prélève puis teste l’eau des bassins avec différents réactifs.
Un tableau récapitulatif des tests qu'elle peut réaliser est présenté dans la figure~\ref{fig:tests_ions}.

\begin{figure}[!ht]
  \centering
  \setlength{\extrarowheight}{6pt}
  \begin{tabular}{| c | c | c |}
    \hline
    \rowcolor{gray!20}
    Réactif utilisé & Ion recherché & Résultat du test positif 
    \\ \hline
    %
    Nitrate d'argent &
    Chlorure (\chemfig{Cl^{-}}) &
    Précipité blanc, noircit* \\ \hline
    %
    \multirow{3}*{Hydroxyde de sodium} &
    Cuivre (\chemfig{Cu^{2+}}) &
    Précipité bleu \\ \cline{2-3}
    %
    &
    Fer II (\chemfig{Fe^{2+}}) &
    Précipité vert \\ \cline{2-3}
    %
    &
    Fer III (\chemfig{Fe^{3+}}) &
    Précipité rouille \\ \hline
    %
    Chlorure de baryum &
    Sulfate  (\chemfig{SO_4^{2-}}) &
    Précipité blanc \\ \hline
  \end{tabular}
  \caption{
    Tests caractéristiques de certains ions. * Le précipité blanc noircit à la lumière.
  }
  \label{fig:tests_ions}
\end{figure}

\question{
  L'ingénieure commence par verser quelques gouttes de chlorure de Baryum dans un tube à essai contenant l’eau prélevée.
  Elle observe la formation d’un précipité blanc.
  Indiquer quel ion pollue le bassin, en justifiant.\competence{APP}
}{2}

\question{
  L'ingénieure veut réaliser des tests supplémentaires pour savoir si le bassin est aussi pollué par des ions Fer.
  Indiquer quel(s) réactif(s) elle doit utiliser et quel résultat permettrait de conclure à la présence d'ions Fer.\competence{APP, ANA/RAI}
}{4}


%%%% Exercice 3
\vspace*{-24pt}
\titrePartie{Étalon du kilogramme}

\begin{wrapfigure}[6]{r}{0.2\linewidth}
  \centering
  \vspace*{-20pt}
  \image{0.9}{sommative/2nd_Mat/standard_kilogram.jpg}
\end{wrapfigure}

Le kilogramme est l'unité de base de la masse dans le système international.
L'étalon qui a servi à définir le kilogramme jusqu'en mai 2019 est conservé par le Bureau International des Poids et Mesures (BIPM).
Ce prototype est un cylindre constitué d'un alliage de platine et d'iridium, de volume $V_\text{étalon} = 47,\!19 \unit{cm}^3$ et de masse volumique $\rho_\text{étalon} = 21,\!19 \unit{g/cm}^3$.

\question{
  Indiquer la masse de l'étalon.\competence{APP}
}{1}

\question{
  Le prototype est composé de $0,9 \unit{kg}$ de platine et de $0,1 \unit{kg}$ d'iridium.
  Calculer la proportion massique de platine et d'iridium.\competence{REA, APP}
}{2}

\textit{Rappel :} la proportion massique d'une espèce dans un échantillon est la masse de l'espèce divisée par la masse de l'échantillon. Par exemple ici pour le platine :
\begin{equation*}
  \% = \frac{m_\text{platine}}{m_\text{étalon}} \times 100
\end{equation*}

\question{
  Historiquement, un premier cylindre avait été réalisé avec $11,1 \%$ d'iridium, qui a une masse volumique plus élevée que le platine.
  Sachant que son volume était identique à l'étalon actuel, indiquer si la masse de ce cylindre valait 1 kg et expliquer pourquoi il avait été rejeté par le BIPM.\competence{APP, ANA/RAI}
}{2}


%%%% Exercice 2
\vspace*{-24pt}
\titrePartie{Huile essentielle d'orange}

Les huiles essentielles d'orange (HEO) sont obtenues en pressant le zeste d'une orange.
Les huiles essentielles sont riches en molécules odorantes.
On réalise une Chromatographie sur Couche Mince (CCM) afin d'identifier quelques espèces chimiques présentes dans cette huile essentielle.
La figure~\ref{fig:CCM_HEO} présente le chromatogramme obtenue après la montée de l'éluant.

\begin{figure}[!ht]
  \centering
  \image{0.25}{sommative/2nd_Mat/chromato_HEO.png}
  \caption{
    HEO : huile essentielle d'orange, Lim : limonène, G : géraniol, Lin : linalol, Ci : citral.
  }
  \label{fig:CCM_HEO}
\end{figure}

\question{
  Indiquer sur la figure~\ref{fig:CCM_HEO} où se trouvent la ligne de dépôt, la couche mince et le front de l'éluant.\competence{RCO}
}{0}

\question{
  Justifier que l'huile essentielle d'orange est un mélange.\competence{RCO, APP}
}{1}

\question{
  En comparant les hauteurs des tâches, indiquer quelles sont les espèces chimiques présentes dans l'huile essentielle d'orange.\competence{RCO, APP, VAL}
}{3}

\question{
  Peut-on vraiment distinguer le géraniol et le linalol avec les résultats de cette CCM ?\competence{ANA/RAI}
}{2}