%%%% début de la page
\newpage
\setcounter{page}{1}
\enTete{Corps purs et solutions}{1}


%%%% Titre
\numeroActivite{2}
\vspace*{-20pt}
\titreEvaluation{Solutions (Correction)}


%%%% Exercice 1
\titrePartie{Conduite et alcoolémie}

Mélanie et sa femme Sihame sortent en voiture pour aller manger dehors.
Au restaurant Sihame boit un verre de $250 \unit{mL}$ d'alcool à $10 \degree$ : c'est-à-dire que $10 \%$ du volume de la boisson est de l'éthanol.

On va chercher à déterminer si Sihame pourra de nouveau conduire après le repas.

%
\exo{
  Calculer le volume d'éthanol dans le verre.\competence{APP, REA}
}{0}
\correction{
  \begin{align*}
    \VEth &= 250 \unit{mL} \times 10 \% \\
    &= 250 \times \Frac{10}{100} \unit{mL} \\
    &= 25 \unit{mL}
  \end{align*}
}

%
\exo{
  Sachant que l'éthanol a une masse volumique qui vaut $\rhoEth = 0,\!8 \unit{g/mL}$ et que $\mEth = \rhoEth \times \VEth$, calculer la masse d'éthanol bue par Sihame.\competence{APP, REA}
}{0}
\correction{
  \begin{align*}
    \mEth &= \VEth \times \rhoEth \\
    &= 25 \unit{mL} \times 0,\!8 \unit{g/ mL} \\
    &= 25 \times 0,\!8 \unit{g} \\
    &= 20 \unit{g}
  \end{align*}
}


Le corps d'une femme adulte contient en moyenne $4,\!5 \unit{L}$ de sang. En France, \og \textit{il est interdit de conduire avec un taux d'alcool dans le sang supérieur ou égal à $0,5 \unit{g/L}$ de sang} \fg.

%
\exo{
  Indiquer le nom de la grandeur utilisé en physique-chimie pour désigner le taux d'alcool dans le sang. Expliquer avec vos mots la différence entre cette grandeur et la masse volumique.\competence{RCO, COM}
}{0}
\correction{
  Le taux d'alcool représente la masse d'alcool présente dans le sang d'une personne, c'est donc une \textbf{concentration massique}.
  La concentration massique mesure la masse d'un soluté présent dans un volume de solution, alors que la masse volumique mesure à quel point un corps est dense, quelle est sa mase pour un volume donné.
}

%
\exo{
  Rappeler la formule de la concentration massique.\competence{RCO}
}{0}
\correction{
  \begin{equation*}
    c_\text{soluté} 
    = \Frac{m_\text{soluté}}{V_\text{solution}}
  \end{equation*}
}

%
\exo{
  Calculer la concentration massique d'éthanol dans le sang de Sihame.\competence{APP, REA}
}{0}
\correction{
  On veut calculer la concentration d'éthanol dans le sang, il faut donc diviser la masse calculée à la question 2 par le volume de sang :
  \begin{align*}
    \cEth &= \Frac{\mEth}{V_\text{sang}}
    = \Frac{20 \unit{g}}{4,\!5 \unit{L}}
    = 4,\!44 \unit{g/L}
  \end{align*}
}

%
\exo{
  Indiquer, en justifiant, si Sihame pourra conduire en sortant du restaurant.\competence{APP, VAl, ANA/RAI, COM}
}{0}
\correction{
  La concentration massique d'éthanol dans le sang de Sihame dépasse le taux légal ($4,\!44 \unit{g/L} > 0,\!5 \unit{g/L}$), donc Sihame ne pourra pas conduire.
}


En fait, quand un humain boit une boisson alcoolisé seule une petite partie de l'éthanol et absorbé par l'organisme. En moyenne seulement $12\%$ de l'éthanol passe dans le sang (si on a bu $10\unit{g}$ d'éthanol, $1,\!2\unit{g}$ passe dans le sang).

%
\exo{
  Calculer de nouveau la concentration massique dans le sang de Sihame en tenant compte de cette information. Indiquer, en justifiant, si Sihame pourra conduire en sortant du restaurant.\competence{APP, REA, VAL, ANA/RAI, COM}
}{0}
\correction{
  La masse d'éthanol dans le sang est égale à $12\%$ de celle que Sihame a ingéré :
  \begin{equation*}
    m_\text{éth sang}
    = 12\% \times \mEth
    = \Frac{12}{100} \times 20\unit{g}
    = 2,\!4 \unit{g}
  \end{equation*}
  et donc la concentration massique d'éthanol dans le sang est
  \begin{equation*}
    \cEth = \Frac{2,\!4 \unit{g}}{4,\!5 \unit{L}}
    = 0,\!53 \unit{g/L}
  \end{equation*}
  Comme $0,\!53 \unit{g/L} > 0,\!5 \unit{g/L}$, Sihame ne pourra toujours pas conduire.
}


%%%% Exercice 2
\titrePartie{Sang et anémie}

Le sang est un mélange liquide composé de $54 \%$ de plasma, $45 \%$ de globules rouges et $1 \%$ de globules blancs.
On peut séparer ses constituants en utilisant une centrifugeuse, ce qui donnerait un mélange constitué de trois phases, comme présenté figure~\ref{fig:sang}.

\begin{figure}[!ht]
  \centering
  %
  \begin{subfigure}{0.48\linewidth}
    \tubeEssaiSangC{1.3}{1.4}{2.5}
    \label{fig:sang_normal}
  \end{subfigure}
  %
  \begin{subfigure}{0.48\linewidth}
    \tubeEssaiSangC{0.45}{0.55}{2.5}
    \label{fig:sang_anemie}
  \end{subfigure}
  %
  \caption{
    \centering
    Tube à essai contenant un échantillon de sang centrifugé : (a) d'une personne normale ; (b) d'une personne souffrant d'anémie.
  }
  \label{fig:sang}
\end{figure}

%
\exo{
  Indiquer en justifiant si le contenu des tubes à essais de la figure~\ref{fig:sang} est un mélange homogène ou un mélange hétérogènes.\competence{RCO, APP}
}{0}
\correction{
  On peut distinguer les différents constituants des tubes à essais, ce sont donc des mélanges hétérogènes.
}


Le plasma est une solution aqueuse, qui contient des minéraux, des nutriments et les gaz lié à la respiration (dioxygène \chemfig{O_2} et dioxyde de carbone \chemfig{CO_2}).

%
\exo{
  Indiquer le solvant et les solutés qui constituent le plasma.\competence{RCO}
}{0}
\correction{
  Solvant : eau (solution aqueuse) ; Solutés : minéraux, nutriments, \chemfig{O_2}, \chemfig{CO_2}.
}


Pour assurer son bon fonctionnement, l'organisme d'un être humain a besoin de fer \chemfig{Fe}.
On dit qu'une personne souffre d'anémie si la concentration massique en fer dans le sang est trop faible.
Le fer est transporté par une molécule dans le sang : l'hémoglobine.

%
\exo{
  En utilisant la figure~\ref{fig:sang}, indiquer en justifiant quel constituant du sang contient les molécules d'hémoglobines.\competence{APP, ANA/RAI}
}{0}
\correction{
  L'hémoglobine transporte le fer et une personne en situation d'anémie manque de fer. Le seul constituant du sang dont la quantité diminue chez une personne souffrant d'anémie sont les globules rouges. Ce sont donc les globules rouges qui contiennent l'hémoglobine.
}


Mesurer la concentration massique en hémoglobine dans le sang permet de détecter les cas d'anémies.
On parle d'anémie si cette concentration massiques est inférieure a $1,\!2 \unit{g/L}$ pour une femme et $1,\!3 \unit{g/L}$ pour un homme.
Pour mesurer cette concentration, on peut réaliser une échelle de teinte, car c'est l'hémoglobine qui donne sa teinte rouge au sang.

\begin{figure}[!ht]
  %
  \centering
  \tubeEssaiSang{brightred}
  \tubeEssaiSang{brightred!75!white}
  \tubeEssaiSang{brightred!50!white}
  \tubeEssaiSang{brightred!25!white}
  \tubeEssaiSang{brightred!10!white} \\[4pt]
  \begin{subfigure}{0.6\linewidth}
    \centering
    \setlength{\extrarowheight}{4pt}
    \begin{tabular}{c | c | c | c | c | c}
      \rowcolor{gray!15} Solution & a   & b   & c   & d   & e \\ \hline
      Concentration (g/L)         & 1,4 & 1,3 & 1,2 & 1,1 & 1,0
    \end{tabular}
    \caption{}
  \end{subfigure}
  \tubeEssaiSang{brightred!65!white}
  %
  \caption{Schéma de l'échelle de teinte réalisée, avec les solutions étalons (a, b, c, d, e), leurs concentrations (f) et l'échantillon de sang à doser (g).}
  \label{fig:echelle_teinte}
\end{figure}

%
\exo{
  Rappeler avec vos mots le principe général d'un dosage par étalonnage (que veut-on mesurer et comment fait-on).\competence{RCO, COM}
}{0}
\correction{
  Un dosage par étalonnage sert à mesure la concentration massique d'un soluté pour une solution.
  Pour ça, on mesure une autre grandeur qui est proportionnelle à la concentration massique, pour différentes solutions dont on connaît la concentration.
  Par comparaison, on pourra ensuite déduire la concentration massique de la solution étudiée.
}

%
\exo{
  Pour préparer des solutions, on peut effectuer une dilution ou une dissolution. Indiquer en justifiant laquelle des deux on effectue pour passer de la solution (a) à la solution (b).\competence{RCO, APP}
}{0}
\correction{
  On effectue une dilution, car on diminue la concentration de la solution.
}
  
%
\exo{
  Nommer deux éléments de verrerie nécessaires pour effectuer une dilution.\competence{RCO}
}{0}
\correction{
  Une fiole jaugée et une pipette jaugée ou graduée.
}

%
\exo{
  En utilisant la figure~\ref{fig:echelle_teinte}, indiquer en justifiant la concentration en hémoglobine de l'échantillon de sang (g).\competence{APP, ANA/RAI, VAL}
}{0}
\correction{
  La teinte de la solution (g) est plus claire que la teinte de la solution (b) et plus foncée que la teinte de la solution (c).
  Sa concentration $c_g$ se trouve donc entre celles de la solution (b) et (c)
  \begin{equation*}
    1,\!3 \unit{g/L} > c_g > 1,\!2 \unit{g/L}
  \end{equation*}
}

%
\exo{
  L'échantillon vient d'un homme. Indiquer en justifiant si il souffre d'anémie ou non.\competence{APP, ANA/RAI, VAL}
}{0}
\correction{
  La concentration de l'échantillon $c_g$ est inférieure à $1,\!3\unit{g/L}$, l'homme souffre donc d'anémie.
}
