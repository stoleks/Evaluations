\bigskip
\titrePartie{Huile essentielle de \variationSujet{lavande}{menthe} } \correction{\points{9}}

Les huiles essentielles de \variationSujet{lavande}{menthe} sont obtenues par distillation des fleurs de \variationSujet{lavandes}{menthe}.
Les huiles essentielles sont riches en molécules odorantes.
On réalise une Chromatographie sur Couche Mince (CCM) afin d'identifier quelques espèces chimiques présentes dans cette huile essentielle.
Le chromatogramme obtenue après la montée de l'éluant est présenté ci-dessous.

\begin{figure}[!ht]
  \centering
  Chromatogramme.
  
  \variationSujet
    {\image{0.21}{images/chromato_lavande}}
    {\image{0.2}{images/chromato_menthe}}
  
  A : huile essentielle de \variationSujet{lavande}{menthe},
  B : \variationSujet{linalol}{menthol},
  C : acétate de linalyle,
  D : limonène.
\end{figure}

\numeroQuestion
Légender le chromatogramme en indiquant où se trouvent la ligne de dépôt, la couche mince et le front de l'éluant.\competence{RCO}
\correction{\points{3}}

\question{
  En analysant le chromatogramme, justifier que l'huile essentielle de \variationSujet{lavande}{menthe} est un mélange.\competence{RCO, APP}
}{
  D'après le chromatogramme, le dépôt d'huile essentielle s'est divisé en trois tâches. L'huile essentielle est donc composée d'au moins trois espèces chimiques, c'est donc un mélange.
  \points{2}
}[2]

\question{
  En comparant les hauteurs des tâches, indiquer quelles sont les espèces chimiques présentes dans l'huile essentielle de \variationSujet{lavande}{menthe}.\competence{RCO, APP, VAL}
}{
  Sur un chromatogramme, deux composés sont identiques s'ils sont montés à la même hauteur. 
  
  Sur ce chromatogramme, en partant du bas, on voit que la première tâche d'huile essentielle de \variationSujet{lavande}{menthe} est à la même hauteur que \variationSujet{l'acétate de linalyle C}{le limonène D} et que la troisième tâche est à la même hauteur que le \variationSujet{linalol}{menthol} B.
  
  L'huile essentielle de \variationSujet{lavande}{menthe} est donc composée \variationSujet{d'acétate de linalyle}{de limonène} et de \variationSujet{linalol}{menthol}.
  \points{3}
}[3]

\question{
  Justifier qu'on ne peut pas identifier le troisième composé présent dans l'huile essentielle avec ce chromatogramme.\competence{APP, VAL}
}{
  On ne peut pas l'identifier, car on n'a pas d'espèce chimique de référence pour comparer qui soit à la même hauteur.
  \points{1}
}[2]