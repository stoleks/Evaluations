%%%%
\exercice{Structure électronique d'un atome}\correction{\points{9}}

\begin{doc}{Tableau périodique}
  \hspace*{-16pt}
  \centering
  \begin{tableauPeriodique}[largeur case = 2 cm, hauteur case = 2cm, echelle = 1]
    \elementH
    \elementLi[couleur = red-200](below = of  H)
    \elementNa[couleur = red-200](below = of Li)
    %
    \elementBe[couleur = red-150](right = of Li)
    \elementMg[couleur = red-150](right = of Na)
    %
    \elementB [couleur = orange-100](right = of Be)
    \elementAl[couleur = red-150]   (right = of Mg)
    %
    \elementC (right = of  B)
    \elementSi[couleur = orange-100](right = of Al)
    %
    \elementN(right = of  C)
    \elementP(right = of Si)
    %
    \elementO(right = of N)
    \elementS(right = of P)
    %
    \elementF (right = of O)
    \elementCl(right = of S)
    %
    \elementNe[couleur = blue-150](right = of  F)
    \elementAr[couleur = blue-150](right = of Cl)
    \elementHe[couleur = blue-150](above = of Ne)
    %% Légende
    \elementB [couleur = orange-100] (above = of O, yshift = .2cm)
    \tikzVecteur(7.9, 0.75)  (1.6, 0) {Nom de l'élément chimique} [left, xshift = -1.6cm]
    \tikzVecteur(7.9, 0.29)  (1.6, 0) {Numéro atomique $Z$} [left, xshift = -1.6cm]
    \tikzVecteur(7.9, -0.28) (1.6, 0) {Symbole atomique $X$} [left, xshift = -1.6cm]
  \end{tableauPeriodique}
\end{doc}

\question{
  En vous aidant du tableau périodique, donner le nombre d'électrons de l'azote \chemfig{N}, du magnésium \chemfig{Mg} et de l'argon \chemfig{Ar}.\competence{APP, ANA/RAI}
}{
  Le numéro atomique de l'azote est 7, il possède donc 7 protons et 7 électrons par neutralité électrique de l'atome.
  Le magnésium possède 12 électrons et l'argon possède 18 électrons.
  \points{1}
}[2]


\question{
  Donner la structure électronique de l'azote, du magnésium et de l'argon.\competence{REA}
}{
  $\chemfig{N} : 1s^2 \mathbf{2s^2 2p^3}$, $\chemfig{Mg} : 1s^2 2s^2 2p^6 \mathbf{3s^2}$, $\chemfig{Ar} : 1s^2 2s^2 2p^6 \mathbf{3s^2 3p^6}$
  \points{1.5}
}[3]

\question{
  Entourer la couche externe de chacune des structures électronique et indiquer le nombre d'électrons de valence de chaque atome.\competence{COM, ANA/RAI}
}{
  Couche externe de l'azote : 2 avec 5 électrons de valences.
  Couche externe du magnésium : 3 avec 2 électrons de valences.
  Couche externe de l'argon : 3 avec 8 électrons de valences.
  \points{1.5}
}[2]

\question{
  Parmi ces trois atomes, lequel est le plus stable ? Justifier.\competence{ANA/RAI}
}{
  L'argon, car sa couche externe est pleine : c'est un gaz noble.
  \points{1}
}[2]


\question{
  Rappeler la règle de l'octet avec vos mots.\competence{COM}
}{
  Pour gagner en stabilité, un atome de numéro atomique > 6, peut perdre ou gagner des électrons pour atteindre la structure électronique du gaz noble \textbf{le plus proche}, avec 8 électrons sur sa couche externe.
  \points{2}
}[3]

\question{
  D'après la règle de l'octet, quel ion pourra être formé à partir d'un atome de magnésium ? Expliquer.\competence{ANA/RAI, COM}
}{
  D'après la règle de l'octet, le magnésium va perdre 2 électrons pour atteindre la configuration électronique du néon. On aura donc l'ion $\chemfig{Mg^{2+}}$.
  \points{1}
}[2]