%%%%
\titreSection{Structure électronique d'un atome}\correction{\points{9}}
\vspace*{-8pt}

\begin{doc}{Tableau périodique}{doc:tableau_periodique}
  \hspace*{-16pt}
  \centering
  \renewcommand{\largeurCaseTableauPeriodique}{2}
  \tableauPeriodique[2.5][2.5]{
    %% Group 1 - IA
    \node[name=H,  NoMeta]              {\elementH};
    \node[name=Li, below of=H,  Alcali] {\elementLi};
    \node[name=Na, below of=Li, Alcali] {\elementNa};
    %% Group 2 - IIA
    \node[name=Be, right of=Li, Alcalo] {\elementBe};
    \node[name=Mg, below of=Be, Alcalo] {\elementMg};
    %% Group 13 - IIIA
    \node[name=B,  right of=Be, Metoid] {\elementB};
    \node[name=Al, below of=B,  Metaux] {\elementAl};
    %% Group 14 - IVA
    \node[name=C,  right of=B,  NoMeta]  {\elementC};
    \node[name=Si, below of=C,  Metoid] {\elementSi};
    %% Group 15 - VA
    \node[name=N,  right of=C,  NoMeta] {\elementN};
    \node[name=P,  below of=N,  NoMeta] {\elementP};
    %% Group 16 - VIA
    \node[name=O,  right of=N,  NoMeta] {\elementO};
    \node[name=S,  below of=O,  NoMeta] {\elementS};
    %% Group 17 - VIIA
    \node[name=F,  right of=O,  Haloge] {\elementF};
    \node[name=Cl, below of=F,  Haloge] {\elementCl};
    %% Group 18 - VIIIA
    \node[name=Ne, right of=F,  GazRar] {\elementNe};
    \node[name=Ar, below of=Ne, GazRar] {\elementAr};
    \node[name=He, above of=Ne, GazRar] {\elementHe};
  }
\end{doc}

\question{
  Donner le nombre d'électrons de l'azote \chemfig{N}, du magnésium \chemfig{Mg} et de l'argon \chemfig{Ar}.\competence{APP, ANA/RAI}
}{
  Le numéro atomique de l'azote est 7, il possède donc 7 protons et 7 électrons par neutralité électrique de l'atome.
  Le magnésium possède 12 électrons et l'argon possède 18 électrons.
  \points{1}
}{2}


\question{
  Donner la structure électronique de l'azote, du magnésium et de l'argon.\competence{REA}
}{
  $\chemfig{N} : 1s^2 \mathbf{2s^2 2p^3}$, $\chemfig{Mg} : 1s^2 2s^2 2p^6 \mathbf{3s^2}$, $\chemfig{Ar} : 1s^2 2s^2 2p^6 \mathbf{3s^2 3p^6}$
  \points{1.5}
}{3}

\question{
  Entourer la couche externe de chacune des structures électronique et indiquer le nombre d'électrons de valence de chaque atome.\competence{COM, ANA/RAI}
}{
  Couche externe de l'azote : 2 avec 5 électrons de valences.
  Couche externe du magnésium : 3 avec 2 électrons de valences.
  Couche externe de l'argon : 3 avec 8 électrons de valences.
  \points{1.5}
}{2}

\question{
  Parmi ces trois atomes, lequel est le plus stable ? Justifier.\competence{ANA/RAI}
}{
  L'argon, car sa couche externe est pleine : c'est un gaz noble.
  \points{1}
}{2}


\question{
  Rappeler la règle de l'octet avec vos mots.\competence{COM}
}{
  Pour gagner en stabilité, un atome de numéro atomique > 6, peut perdre ou gagner des électrons pour atteindre la structure électronique du gaz noble \textbf{le plus proche}, avec 8 électrons sur sa couche externe.
  \points{2}
}{3}

\question{
  D'après la règle de l'octet, quel ion pourra être formé à partir d'un atome de magnésium ? Expliquer.\competence{ANA/RAI, COM}
}{
  D'après la règle de l'octet, le magnésium va perdre 2 électrons pour atteindre la configuration électronique du néon. On aura donc l'ion $\chemfig{Mg^{2+}}$.
  \points{1}
}{2}