%%%%
\titreSousSection{Analyse de la lumière venant du Soleil}\correction{\points{5}}

\vspace*{-12pt}
\begin{doc}{Réfraction de la lumière}{doc:E_refraction_sirius}
  \begin{wrapfigure}[7]{r}{0.4\linewidth}
    \centering
    \vspace*{-32pt}
    \image{1}{images/experience_sirius}
  \end{wrapfigure}
  Pour analyser le spectre d'émission du Soleil, on utilise un spectroscope.
  Le spectroscope contient un prisme en plexiglas qui permet de disperser la lumière.

  On cherche à mesurer l'indice de réfraction du plexiglas. Pour ça on réalise l'expérience schématisée à droite.
  
  \important{Rappels :}
  \begin{listePoints}
    \item L'indice de réfraction de l'air vaut $n_\text{air} = 1,\!0$
    \item La loi de Snell-Descartes nous dit que :
    \begin{equation*}
      n_2 = n_1 \times \dfrac{\sin(i_1)}{\sin(i_2)}
    \end{equation*}
  \end{listePoints}
\end{doc}

\question{
  Dans l'expérience du document~\ref{doc:E_refraction_sirius}, l'indice de réfraction du plexiglas est-il $n_1$ ou $n_2$ ?
  Donner la valeur de l'autre indice de réfraction.\competence{APP}
}{
  L'indice de réfraction du plexiglas est $n_2$. $n_1 = n_\text{air} = 1$.\points{1}
}[2]

\pasCorrection{\newpage}
\question{
  En vous aidant du schéma, donner la valeur des angles $i_1$ et $i_2$.\competence{APP}
}{
  $i_1 = \qty{50}{\degree}$ et $i_2 = \qty{30}{\degree}$.\points{1}
}[2]

\question{
  En utilisant les valeurs de $i_1$ et de $i_2$ et la loi de Snell-Descartes, calculer la valeur de l'indice de réfraction du plexiglas.
  \important{Données :} $\sin(\qty{30}{\degree}) = \num{0,50}$ et $\sin(\qty{50}{\degree}) = \num{0,77}$.\competence{ANA/RAI, REA}
}{
  D'après la loi de Snell-Descartes
  \begin{equation*}
    n_2 = n_1 \times \frac{\sin(i_1)}{\sin(i_2)} = \num{1,54}
  \end{equation*}\points{1,5}
}[2]

\question{
  La vitesse de la lumière est plus élevée dans le plexiglas ou dans l'air ?\competence{RCO}
}{
  $n_\text{plexiglas} > n_\text{air}$, donc $c_\text{plexiglas} < c_\text{air}$ par définition de l'indice de réfraction.\points{1,5}
}[2]