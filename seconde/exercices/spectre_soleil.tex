%%%%
\titreSousSection{Spectre d'émission du Soleil}\correction{\points{4}}

\begin{doc}{Spectre d'émission et d'absorption}{doc:E_spectre_emission}
  \begin{wrapfigure}{r}{0.55\linewidth}
    \vspace*{-24pt}
    \begin{center}
      \hspace{4pt}\image{1.05}{images/spectre_soleil} \\[-4pt]
      Spectres d'émission d'atomes et spectre du Soleil obtenu avec un spectroscope
    \end{center}
  \end{wrapfigure}
  Même si le Soleil est un corps chaud, la lumière qu'il émet n'est pas tout à fait continue.
  Son spectre comporte des \important{raies d'absorption}.
  
  Ces raies correspondent à de la lumière qui a été absorbée par des atomes présent dans l'atmosphère du Soleil.
  \important{Un atome absorbe les longueurs d'onde correspondant à ces raies d'émissions.}
  
  Si une série de raies d'absorption dans le spectre d'émission du Soleil correspond exactement au raies d'émission d'un atome, alors ça veut dire que cet atome se trouve dans l'atmosphère du Soleil.
\end{doc}

\question{
  Pour chacun des trois éléments chimique (carbone, sodium et hydrogène), indiquer s'il se trouve dans l'atmosphère du Soleil ou non. Justifier.\competence{APP, ANA/RAI, COM}
  
  \textit{
    Prendre des initiatives et les écrire, même si le raisonnement n'est pas complet.
    Tout début de réflexion sera valorisé.
  }
}{
  Si un élément chimique se trouve dans l'atmosphère du Soleil, il va absorber la lumière correspondant à \important{toutes} ses raies d'émissions. \\
  Toutes les raies d'émission du carbone ne correspondent pas à des raies d'absorption dans le spectre du Soleil, le carbone ne se trouve donc pas dans l'atmosphère du Soleil.
  Par contre, toutes les raies d'émission de l'hydrogène et du sodium correspondent à des raies d'absorption dans le spectre du Soleil : ces deux éléments se trouvent donc dans l'atmosphère du Soleil.
}[7]