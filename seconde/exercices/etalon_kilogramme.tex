\titrePartie{Étalon du kilogramme}\correction{\points{5}}

\begin{wrapfigure}[6]{r}{0.2\linewidth}
  \centering
  \vspace*{-40pt}
  \image{0.9}{images/standard_kilogram.jpg}
\end{wrapfigure}

Le kilogramme est l'unité de base de la masse dans le système international.
L'étalon qui \important{a servi à définir le kilogramme} jusqu'en mai 2019 est conservé par le Bureau International des Poids et Mesures (BIPM).
Ce prototype est un cylindre constitué d'un alliage de platine et d'iridium, de volume $V_\text{étalon} = \qty{47,191}{\cm\cubed}$ et de masse volumique $\rho_\text{étalon} = \qty{21,191}{\g/\cm\cubed}$.

\question{
  Sans calcul, indiquer la masse de l'étalon.\competence{APP}
}{
  L'étalon sert à définir le kilogramme.
  Sa masse est donc de \qty{1}{\kg} par définition.
  \points{1}
}{1}

\question{
  Le prototype est composé de \qty{0,9}{\kg} de platine et de \qty{0,1}{\kg} d'iridium.
  Calculer la fraction massique de platine et d'iridium.\competence{REA, APP}
}{
  Pour le platine : $\qty{0,9}{\kg} / \qty{1}{\kg} = \qty{90}{\percent}$.
  Pour l'iridium : $\qty{0,1}{\kg} / \qty{1}{\kg} = \qty{10}{\percent}$.
  \points{2}
}{2}

\textit{Rappel :} la fraction massique d'une espèce dans un échantillon est la masse de l'espèce divisée par la masse totale de l'échantillon.
Par exemple pour le platine :
\begin{equation*}
  p_m(\text{platine}) = \frac{m_\text{platine}}{m_\text{étalon}}
\end{equation*}

\question{
  Historiquement, un premier cylindre avait été réalisé avec \qty{11,1}{\percent} d'iridium, qui a une masse volumique plus élevée que le platine.
  Sachant que son volume était identique à l'étalon actuel, indiquer si la masse de ce cylindre valait 1 kg et expliquer pourquoi il avait été rejeté par le BIPM.\competence{APP}
}{
  L'étalon actuel a précisément une masse de \qty{1}{\kg} avec \qty{10}{\percent} d'iridium.
  Le premier cylindre comportait $\qty{11,1}{\percent} > \qty{10}{\percent}$ d'iridium.
  Comme l'iridium a une masse volumique plus élevée, sa masse valait plus de \qty{1}{\kg}, d'où le rejet par le BIPM.
  \points{2}
}{2}