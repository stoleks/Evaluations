\titrePartie{Sang et anémie}
\pasCorrection{\points{12}}
\vspace*{-12pt}

% #1 : hauteur globule rouge, # 2 : hauteur globule blanc, #3 : hauteur plasma
\newcommand{\sangCentrifugeuse}[3]{
  \begin{tikzpicture}
    % phases dans le tube à essai
    \tikzPhaseBasTubeEssais[red-400]    (0.35) (#1)
    \tikzPhaseTubeEssais   [white]      (0.35) (#1)(#2)
    \tikzPhaseTubeEssais   [yellow-300] (0.35) (#2)(#3)
    \tikzTubeEssais (0.35)(#3 + 1)
    % Légende
    \tikzLegende*(0.4, #3 - 0.1)  (1, 0) {Plasma}
    \tikzLegende*(0.4, #2 - 0.08) (1, 0) {Globules blancs}
    \tikzLegende*(0.4, -0.1)      (1, 0) {Globules rouges}
  \end{tikzpicture}
}

\begin{doc}{Composition du sang et anémie}{doc:E2_composition_sang}
  Le sang est un mélange liquide composé de \qty{54}{\percent} de plasma, \qty{45}{\percent} de globules rouges et \qty{1}{\percent} de globules blancs.
  
  Le plasma est une solution aqueuse, qui contient des minéraux,
  des nutriments et les gaz liés à la respiration :
  dioxygène \chemfig{O_2} et dioxyde de carbone \chemfig{CO_2}.
  
  Pour assurer son bon fonctionnement, l'organisme d'un être humain a besoin de fer \chemfig{Fe}.
  On dit qu'une personne souffre d'anémie si la concentration massique en fer dans le sang est trop faible.
  Le fer est transporté par une molécule dans le sang : l'hémoglobine.

  On peut séparer les constituants du sang en utilisant une centrifugeuse, ce qui donne un mélange constitué de trois phases.
  
  \vspace*{4pt}
  \separationBlocs{
    \centering \hspace{40pt}
    \sangCentrifugeuse{0.8}{0.9}{1.75}
    
    \legende{Échantillon de sang d'une personne normale.}
  }[0.48]{
    \centering \hspace{40pt}
    \sangCentrifugeuse{0.4}{0.5}{1.75}
    
    \legende{Échantillon d'une personne souffrant d'anémie.}
  }[0.52]
\end{doc}

%
\question{
  Indiquer en justifiant si le contenu des tubes à essais du document~\ref{doc:E2_composition_sang} 
  est un mélange homogène ou un mélange hétérogènes.\competence{RCO, APP}
}{
  C'est un mélange hétérogène, car on peut distinguer les constituants du mélange.
  \points{1}
}

\question{
  Indiquer le solvant et les solutés qui constituent le plasma.\competence{RCO, APP}
}{
  Solvant : eau. Solutés : minéraux, nutriments, dioxygène et dioxyde de carbone.
  \points{2}
}

\question{
  Déduire des deux échantillons de sang le composant du sang qui contient les molécules d'hémoglobines.\competence{APP}
}{
  Ce sont les globules rouges, car c'est le seul composant dont la quantité diminue pour une personne en situation d'anémie.
  \points{2}
}


%
\pasCorrection{
  \newpage
  \vspace*{-40pt}
}
\begin{doc}{Dosage de l'hémoglobine}{doc:E2_dosage_hemoglobine}
  Mesurer la concentration massique en hémoglobine dans le sang permet de détecter les cas d'anémies.
  On parle d'anémie si cette concentration massiques est inférieure a
  \qty{1,2}{\g\per\litre} pour une femme et \qty{1,3}{\g\per\litre} pour un homme.
  Pour mesurer cette concentration, on peut réaliser une échelle de teinte,
  car c'est l'hémoglobine qui donne sa teinte rouge au sang.
  
  \vspace*{4pt}
  \separationBlocs{
    \begin{center}
      \begin{tblr}{c| c| c| c| c| c}
        Solution &
        \tubeEssaisSolution{red-600} &
        \tubeEssaisSolution{red-400} &
        \tubeEssaisSolution{red-200} &
        \tubeEssaisSolution{red-100} &
        \tubeEssaisSolution{red-50}  \\
        %
        & 1 & 2 & 3 & 4 & 5 \\ \hline
        %
        Concentration \unit{\g/\litre} & \num{1,4} & \num{1,3} & \num{1,2} & \num{1,1} & \num{1,0} \\ \hline
      \end{tblr} \\[4pt]

      \legende{
        Schéma de l'échelle de teinte réalisée, avec les solutions étalons et leurs concentrations.
      }
    \end{center}
  }[0.6]{
    \begin{center}
      \variationSujet{
        \tubeEssaisSolution{red-300}
      }{
        \tubeEssaisSolution{red-150}
      }
      
      \legende{Échantillon de sang à doser.}
    \end{center}
  }[0.4]
\end{doc}

%
\question{
  Rappeler avec vos mots le principe général d'un dosage par étalonnage (que veut-on mesurer et comment fait-on).\competence{RCO, COM}
}{
  On veut mesurer la concentration massique d'un soluté responsable de la coloration de la solution.
  Pour ça on réalise une échelle de teinte : on prépare des solutions dont on connait la concentration en soluté.
  On compare ensuite les teintes des différentes solutions pour encadrer la valeur de la concentration massique.
  \points{3}
}

\question{
  Pour préparer des solutions, on peut effectuer une dilution ou une dissolution.
  Indiquer en justifiant laquelle des deux on effectue pour passer de la solution 2 à la solution 3.\competence{RCO}
}{
  On réalise une dilution, car la concentration diminue.
  \points{1}
}
  
\question{
  Donner le nom de deux verreries nécessaires pour réaliser une dilution.\competence{RCO}
}{
  Pipette jaugée et fiole jaugée.
  \points{1}
}

\question{
  En utilisant le document~\ref{doc:E2_dosage_hemoglobine}, indiquer en justifiant la concentration en hémoglobine de l'échantillon de sang.\competence{APP, VAL}
}{
  L'échantillon a une teinte plus foncé que la solution 3, mais plus claire que celle de la solution 2. Donc sa concentration se trouve entre \num{1,3} et \qty{1,2}{\g\per\litre}.
  \points{1}
}

\question{
  L'échantillon vient \variationSujet{d'une femme}{d'un homme}. Indiquer en justifiant \variationSujet{si elle}{s'il} souffre d'anémie ou non.\competence{VAL}
}{
  Comme la concentration est plus faible que \qty{1,2}{\g\per\litre}, elle souffre d'anémie.
  \points{1}
}