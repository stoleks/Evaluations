\exercice{Penalty au footbal} \correction{\points{13}}

\begin{doc}{Forces de frottements}{doc:A_forces_frottement}
  Un objet en mouvement dans un fluide comme l'air subit des forces de frottements.
  Les forces de frottements $\vv{f}$ sont opposées au vecteur vitesse $\vv{v}$ de l'objet.
  
  Pour un objet en mouvement dans l'air, on peut calculer la valeur des frottements de l'air sur l'objet, en Newton, à l'aide de la relation suivante : 
  \begin{equation*}
    f = \dfrac{1}{2} \times \rho \times C_x \times S \times v^2
  \end{equation*}
  Avec :
  \begin{listePoints}[2]
    \item $\rho = \qty{1,23}{\kg\per\cubic\m}$ la densité de l'air ;
    \item $C_x$ le coefficient de traînée, sans unité ;
    \item $v$ la vitesse de l'objet en \unit{\m\per\s} ;
    \item $S$ la surface de l'objet en \unit{\m\squared}.
  \end{listePoints}
\end{doc}

\begin{doc}{Ballon de footbal}{doc:A_penaly_foot}
  Un ballon de football est une sphère de \qty{70}{\cm} de circonférence et pesant \qty{450}{\g}, avec une pression interne de \qty{1,5}{\bar}.  
  On cherche à étudier les forces qui s'exercent sur un ballon de football pendant un penalty.
  
  \begin{donnees}
    \item Circonférence d'une sphère de rayon $r$ : $c = 2\pi r$.
    \item Surface d'une sphère de rayon $r$ : $S = 4\pi r^2$.
    \item $C_x =$ \variationSujet{\num{0,20}}{\num{0,15}} pour un ballon de foot.
    \item $v =$ \variationSujet{\num{50}}{\num{45}}\unit{\m\per\s} pendant un penalty.
    \item $g = \qty{9.81}{\newton\per\kg}$
  \end{donnees}
\end{doc}

\question{
  Indiquer quel est le système étudié et donner un référentiel approprié pour étudier son mouvement.\competence{APP}
}{
  On étudie le mouvement du ballon de football, dans le référentiel terrestre.\points{1}
}

\question{
  Citer la ou les forces qui s’exercent sur le ballon puis calculer leurs valeurs.\competence{APP, REA}
}{
  On a le poids $\vv{P}$ et la réaction du support $\vv{R}$ qui s'exerce sur le ballon.
  Pour le poids on a 
  \begin{equation*}
    P = m\times g = \qty{0,450}{\kg} \times \qty{9,81}{\newton\per\kg} = \qty{4,41}{\newton}.
  \end{equation*}
  Par définition de la réaction du support $R = P = \qty{4,41}{\newton}$. \points{2}
}

\question{
  Représenter ces forces sur un schéma propre et lisible.\competence{REA}
}{\points{2}}

\question{
  Les forces se compensent-elles ? Justifier à l'aide du principe d'inertie.\competence{ANA/RAI, VAL}
}{
  Oui, car le ballon est immobile, donc les forces doivent se compenser.\points{1}
}

\question{
  Pour tirer un penalty, un joueur frappe dans le ballon posé au sol.
  Décrire les forces s'exerçant sur le ballon lorsque celui-ci est en l’air.\competence{APP}
}{
  Il y a toujours le poids et il y a les frottements de l'air.
  \points{1}
}

\question{
  Représenter ces forces sur un autre schéma.\competence{REA}
}{
  \points{2}
}

\question{
  Peut-on utiliser le principe d’inertie pour décrire le mouvement du ballon de football lorsqu’il est en l’air ?\competence{VAL, ANA/RAI}
}{
  Non, car les forces ne se compensent pas, donc le ballon n'a pas un mouvement rectiligne uniforme.\points{1}
}

\question{
  Calculer la valeur des forces de frottements s’exerçant sur le ballon.
  \competence{APP, ANA/RAI}
  
  \textit{
    Prendre des initiatives et les écrire, même si le raisonnement n'est pas complet.
    Tout début de réflexion sera valorisé.
  }
}{
  Pour calculer l'intensité des forces de frottements, il faut calculer la surface du ballon, et donc connaître le rayon du ballon. Le rayon se calcule à partir de la circonférence $c$ :
  \begin{align*}
    2\pi r &= c \\
    r &= \dfrac{c}{2 \pi}
  \end{align*}
  et donc 
  \begin{equation*}
    S = 4\pi r^2 
    = 4\pi \left(\dfrac{c}{2\pi}\right)^2
    = \dfrac{c^2}{\pi}
    = \dfrac{(\qty{0,7}{\m})^2}{3.1415}
    = \qty{0.16}{\m\squared}
  \end{equation*}
  Finalement on peut calculer la valeur de $f$ :
  \begin{equation*}
    f = \dfrac{1}{2}\rho C_x S v^2
    = \dfrac{1}{2} \times \qty{1,23}{\kg\per\cubic\m}
    \times \;\; \variationSujet{\num{0,20}}{\num{0,15}}
    \;\; \times \qty{0.16}{\m\squared}
    \times \left(\;\;\variationSujet{50}{45}\;\; \unit{\m\per\s}\right)^2
    = \;\;\variationSujet{\num{49,2}}{\num{33,2}}\;\; \unit{\newton}
  \end{equation*}
  \points{3}
}
