\exercice{Penalty au footbal}

\begin{doc}{Forces de frottements}{doc:A_forces_frottement}
  Un objet en mouvement dans un fluide comme l'air subit des forces de frottements.
  Les forces de frottements $\vv{f}$ sont opposées au vecteur vitesse $\vv{v}$ de l'objet.
  
  Pour un objet en mouvement dans l'air, on peut calculer la valeur des frottements de l'air sur l'objet, en Newton, à l'aide de la relation suivante : 
  \begin{equation*}
    f = \dfrac{1}{2} \times \rho \times C_x \times v^2 
  \end{equation*}
  Avec :
  \begin{listePoints}[2]
    \item $\rho = \qty{1,23}{\kg\per\cubic\m}$ la densité de l'air ;
    \item $C_x$ le coefficient de traînée, sans unité ;
    \item $v$ la vitesse de l'objet en \unit{\m\per\s} ;
    \item $S$ la surface de l'objet en \unit{\m\squared}.
  \end{listePoints}
\end{doc}

\begin{doc}{Ballon de footbal}{doc:A_penaly_foot}
  Un ballon de football est une sphère de \qty{70}{\cm} de circonférence et pesant \qty{450}{\g}, avec une pression interne de \qty{1,5}{\bar}.  
  On cherche à étudier les forces qui s'exercent sur un ballon de football pendant un penalty.
  
  \begin{donnees}
    \item Circonférence d'une sphère de rayon $r$ : $c = 2\pi r$.
    \item Surface d'une sphère de rayon $r$ : $S = 4\pi r^2$.
    \item $C_x =$ \variationSujet{\num{0,20}}{\num{0,15}} pour un ballon de foot.
    \item $v =$ \variationSujet{\num{50}}{\num{45}}\unit{\m\per\s} pendant un penalty.
    \item $g = \qty{9.81}{\newton\per\kg}$
  \end{donnees}
\end{doc}

\question{
  Indiquer quel est le système étudié et donner un référentiel approprié pour étudier son mouvement.\competence{APP}
}{}

\question{
  Citer la ou les forces qui s’exercent sur le ballon puis calculer leurs valeurs.\competence{APP, REA}
}{}

\question{
  Représenter cette ou ces forces sur un schéma propre et lisible.\competence{REA}
}{}

\question{
  Les forces se compensent-elles ? Justifier à l'aide du principe d'inertie.\competence{ANA/RAI, VAL}
}{}

\question{
  Pour tirer un penalty, un joueur frappe dans le ballon posé au sol.
  Décrire les forces s'exerçant sur le ballon lorsque celui-ci est en l’air.\competence{APP}
}{}

\question{
  Représenter ces forces sur un autre schéma.\competence{REA}
}{}

\question{
  Calculer la valeur des forces de frottements s’exerçant sur le ballon.
  \competence{APP, ANA/RAI}
  
  \textit{
    Prendre des initiatives et les écrire, même si le raisonnement n'est pas complet.
    Tout début de réflexion sera valorisé.
  }
}{}


\question{
  Peut-on utiliser le principe d’inertie pour décrire le mouvement du ballon de football lorsqu’il est en l’air ?\competence{VAL, ANA/RAI}
}{}