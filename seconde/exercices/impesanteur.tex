\exercice{Impesanteur} \correction{\points{16}}

%%
\begin{doc}{Station spatiale internationale (ISS)}
  \begin{wrapfigure}{r}{0.4\linewidth}
    \vspace*{-16pt}
    \centering
    \image{0.8}{images/spationaute_ISS}
    
    \legende{Deux spationautes dans l'ISS}
  \end{wrapfigure}
  On lit parfois que les spationautes flottent dans les stations spatiales, car la gravité terrestre n'agit plus sur les spationautes, ce qui est faux.
  On s'intéresse à la station spatiale internationale, notée ISS, en orbite circulaire autour de la Terre à une distance $d$ du centre de la Terre.
  L'ISS a une vitesse constante $v$.

  \important{Données :}
  \begin{listePoints}
    \item $G = \qty{6,67e-11}{\newton\m\squared \per\kg\squared}$
    \item $M_\Terre = \qty{5,97e24}{\kg}$
    \item $d = \qty{6.784e6}{\m}$
    % \item $R_\Terre = \qty{6,37e6}{\m}$
    % \item $h = \qty{4,14e5}{\m}$
    \item $v = \qty{7,66e3}{\m\per\s}$
    \item Masse de la spationaute $m = \variationSujet{\qty{65,0}{\kg}}{\qty{70}{\kg}}$
  \end{listePoints}
\end{doc}

%%
\question{
  Quel est le mouvement de l'ISS dans le référentiel lié au centre de la Terre ?\competence{APP}
}{
  L'ISS a une trajectoire circulaire avec une vitesse constante, c'est donc un mouvement circulaire uniforme. \points{1}
}

\numeroQuestion \label{exo:schema_ISS}
  Faire un schéma propre et lisible faisant figurer l'ISS, la Terre et la trajectoire décrite par l'ISS.\competence{REA}\correction{\points{2}}

\question{
  Dans la station les spationautes ont un poids $P_\ISS = m \times g_\ISS$.
  Calculer la valeur de $g_\ISS$ sachant que
  \begin{equation*}
    g_\ISS =  \dfrac{G \times M_\Terre}{d^2}%{(R_\Terre + h)^2}
  \end{equation*}\competence{APP, REA}
}{
  \begin{equation*}
    g_\ISS
    = \qty{6,67e-11}{\newton\m\squared \per\kg\squared}
    \dfrac{\qty{5,97e24}{\kg}} {(\num{6,37e6} + \qty{4,14e5}{\m})^2}
    %
    = \qty{8,65}{\newton\per\kg}
    \points{1,5}
  \end{equation*}
}

% \pasCorrection{\newpage}
\variationSujet{\pasCorrection{\pagebreak}}{}

\question{
  Calculer le poids d'une spationaute dans l'ISS, sachant que $g_\ISS = \qty{8,65}{\newton\per\kg}$.\competence{REA}
}{
  \begin{equation*}  
    P_\ISS = \variationSujet{\qty{65}{\kg}}{\qty{70}{\kg}} \times \qty{8,65}{\newton\per\kg} = \variationSujet{\qty{562}{\newton}}{\qty{606}{\newton}}
    \points{1,5}
  \end{equation*}
}

\question{
  Calculer le poids de la même spationaute sur Terre avec $g_{Terre} = \qty{9,81}{\newton\per\kg}$.
  En comparant ces deux forces, indiquer si on peut vraiment dire que la gravité terrestre n'agit plus sur les spationautes au sein de l'ISS.\competence{VAL, ANA/RAI}
}{
  \begin{equation*}  
    P_\Terre = \variationSujet{\qty{65}{\kg}}{\qty{70}{\kg}} \times \qty{9.81}{\newton\per\kg} = \variationSujet{\qty{638}{\newton}}{\qty{688}{\newton}}
    \points{1,5}
  \end{equation*}
  On voit que $P_\ISS$ n'est pas beaucoup plus petit que $P_\Terre$ (\qty{10}{\percent} plus faible), donc la gravité terrestre attire toujours fortement les spationautes.
}

%%
\begin{doc}{Force centrifuge}%d'inertie d'entraînement}
  Un système dans un référentiel en rotation est soumis à une force \important{relative} qui dépend du référentiel, qu'on appelle \important{force centrifuge} $\vv{F}_{centrifuge}$

  Cette force a pour direction la \important{droite reliant le centre du cercle et le centre du système.}
  Son sens est dirigé \important{vers l'extérieur du cercle.}
  C'est cette force qui explique pourquoi les passagers d'une voiture dans un rond-point sentent leur corps projeté vers l'extérieur du rond-point.

  \begin{importants}
    \important{Rappel :} le principe d'inertie dit que tout objet immobile ou avec un mouvement rectiligne uniforme est soumis à des forces dont la somme est nulle.
  \end{importants}
\end{doc}

\question{
  Expliquer avec vos mot le principe d'inertie.\competence{COM}
}{
  Il faut exercer une force sur un objet pour changer son mouvement, par défaut les objets se déplacent en ligne droite.
  \points{3}
}

\question{
  Dans le référentiel lié à l'ISS, la spationaute est immobile.
  En utilisant le principe d'inertie et en justifiant clairement, donner la relation entre $\vv{F}_{centrifuge}$ et $\vv{P}_\ISS$.\competence{APP, ANA/RAI}
}{
  Comme la spationaute est immobile, la somme des forces qui s'exercent sur elle est nulle et $F_{centrifuge} = P_\ISS = \qty{562}{\newton}$
  \points{2}
}

\numeroQuestion
  Compléter le schéma de la question~\ref{exo:schema_ISS} en représentant les forces s'exerçant sur la spationaute dans le référentiel lié à l'ISS.\competence{APP, REA}\correction{\points{1}}


\question{
  La valeur de la force d'inertie d'entraînement exercée sur la spationaute est
  \begin{equation*}
    F_{centrifuge} = m \times \dfrac{v^2}{d}%{R_\Terre + h}
  \end{equation*}
  où $v$ est la vitesse du référentiel tournant. Vérifier le principe d'inertie en calculant la vitesse de l'ISS et en comparant ce résultat avec les données de l'énoncé.  \competence{APP, REA, VAL, ANA/RAI}
  
  \textit{
    Prendre des initiatives et les écrire, même si le raisonnement n'est pas complet. Tout début de réflexion sera valorisé.
  }
}{
  D'après le principe d'inertie, $F_{centrifuge} = P_\ISS$, donc
  \begin{align*}
    \dfrac{m\times v^2}{d} &= m \times g_\ISS \\
    m \times v^2 &= m \times g_\ISS \times d \\
    v^2 &= g_\ISS \times d
  \end{align*}
  Finalement
  \begin{align*}
    v &= \sqrt{g_\ISS \times d} \\
    &= \sqrt{\qty{8.65}{\m\per\s\squared} \times \qty{6,784e6}{\m}} \\
    &= \qty{7.66}{\m\per\s}
  \end{align*}
  On retrouve la même valeur que celle fournie dans l'énoncé, cette valeur est donc cohérente.
  \points{3}
}

\pasCorrection{
  \begin{coupDePouce}
    Utiliser le principe d'inertie sur la spationaute pour en déduire que $P = F_{centrifuge}$.
  \end{coupDePouce}

  \begin{coupDePouce}
    Isoler la vitesse $v$ dans la relation obtenue. Rappel : si $v^2 = a$, alors $v = \sqrt{a}$.
  \end{coupDePouce}

  \begin{coupDePouce}
    Comparer la valeur de la vitesse trouvée avec celle de l'énoncé et conclure.
  \end{coupDePouce}
}