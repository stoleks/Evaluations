%%%%
\titreSection{Stabilité d'une molécule}\correction{\points{4}}

\vspace*{-8pt}
\begin{doc}{\variationSujet{L'ammoniac}{La phosphine}}{doc:ammoniac}
  \variationSujet{
    L'ammoniac est un gaz irritant à température ambiante.
    La molécule d'ammoniac est composé d'hydrogène \chemfig{H} ($Z = 1$) et d'azote \chemfig{N} ($Z = 7$).
  }{
    La phosphine est un gaz incolore et mortellement toxique, utilisé comme pesticide. 
    La molécule de phosphine est composée d'hydrogène \chemfig{H} ($Z = 1$) et de phosphore \chemfig{P} ($Z = 15$).
  }
  Le schéma de Lewis de la molécule est le suivant :
  \begin{center}
    \variationSujet{
      \chemfig[atom sep=30pt, atom style={scale=1.5}, line width=6pt]{
        H - \charge{-90:4pt=\|}{N} (-[3] H) - H
      }
    }{
      \chemfig[atom sep=30pt, atom style={scale=1.5}, line width=6pt]{
        H - \charge{-90:4pt=\|}{P} (-[3] H) - H
      }
    }
    \vAligne{16pt}
  \end{center}
\end{doc}
\vspace*{-8pt}

\question{
  Indiquer la formule brute de la molécule\variationSujet{d'ammoniac}{de phosphine}.\competence{APP}
}{
  La molécule est composée de 1 \variationSujet{azote}{phosphore} et de 3 hydrogènes :
  \variationSujet{\chemfig{NH_3}}{\chemfig{PH_3}}
  \points{1}
}{1}

\question{
  Quelle règle doit respecter l'atome d'hydrogène pour gagner en stabilité ?
  Justifier que cette règle est respectée pour la molécule du document~\ref{doc:ammoniac}\competence{COM}
}{
  La règle du duet : il doit gagner un électron en formant une liaison covalente.
  \points{1}
}{3}

% \question{
%   Combien de liaisons covalentes a formé
%   \variationSujet{l'azote}{le phosphore}
%   dans la molécule
%   \variationSujet{d'ammoniac}{de phosphine}
%   ?
%   Est-ce cohérent avec la règle de l'octet ?\competence{APP, VAL}
% }{
%   L'azote a formé 3 liaisons covalentes, ce qui lui a permis d'ajouter 3 électrons sur sa couche externe pour la compléter et respecter la règle de l'octet.
% }{3}

\question{
  Légender chaque partie du schéma de Lewis de la molécule du document~\ref{doc:ammoniac}.\competence{COM}
}{
  Il faut légender les doublets liants et le double non-liant en plus des éléments chimiques.
  \points{2}
}{0}