\titrePartie{Marais salant et pollution}\correction{\points{11}}

Les marais salants sont de grands bassin remplis par d'eau de mer, qui est riche en sel.
Le sel est du chlorure de sodium de formule brute \chemfig{NaCl}.

%
\question{
  Indiquer en justifiant si l'eau de mer est un corps pur ou un mélange.\competence{RCO, APP}
}{
  C'est un mélange, composé d'au moins deux espèces chimique : l'eau et le sel.
  \points{2}
}{1}

Le soleil et le vent font s'évaporer l'eau de mer, mais le sel reste au fond des bassins. 
Après plusieurs étapes d'évaporation et de remplissage, la quantité de sel contenue dans l'eau des bassins devient très importante.
La masse volumique de l'eau salée augmente avec la quantité de sel.

\question{
  Rappeler la relation mathématique entre la masse volumique de l'eau salée $\rho$, sa masse $m$ et le volume $V$ qu'elle occupe.\competence{RCO}
}{
  \begin{equation*}
    \rho = \frac{m}{V}
    \points{1}
  \end{equation*}
}[2]

Les productrices ou producteurs peuvent récolter le sel lorsque la masse volumique de l'eau salée dans un bassin est \important{supérieure} à $\rho_\text{récolte} = \qty{1,15}{\g/\ml}$.

\question{
  Une productrice de sel pèse \qty{50}{\ml} d'eau salée provenant d'un bassin et mesure une masse de \variationSujet{\qty{60}{\g}}{\qty{55}{\g}}.
  Calculer la masse volumique de l'eau salée dans ce bassin.\competence{REA}
}{
  $m_\text{eau salée} =$ \variationSujet{\qty{60}{\g}}{\qty{55}{\g}} et $V_\text{eau salée} = \qty{50}{\ml}$. Donc 
  \begin{equation*}
    \rho_\text{eau salée}
    = \frac{\variationSujet {\qty{60}{\g}} {\qty{55}{\g}}}{\qty{50}{\ml}} 
    =\; \variationSujet {\qty{1,2}{\g/\ml}} {\qty{1,1}{\g/\ml}}
    \points{2}
  \end{equation*}
}[2]

\question{
  Est-ce que la productrice peut récolter le sel dans ce bassin ? Justifier.\competence{VAL}
}{
  La masse volumique de l'eau du bassin est \variationSujet{supérieure}{inférieure} à celle nécessaire pour récolter, donc la productrice \variationSujet{peut}{ne peut pas} récolter le sel.
  \points{2}
}[2]


Une ingénieure agronome réalise une inspection des marais salants en baie de somme.
Pour vérifier que des ions ne pollue pas les marais, elle prélève puis teste l'eau des bassins avec différentes espèces chimiques.
Un tableau récapitulatif des tests qu'elle peut réaliser est présenté ci-dessous

\begin{center}
  \begin{tableau}{| c | c | c |}
    Espèce utilisée & Ion recherché & Résultat d'un test positif \\
    Nitrate d'argent & \chlorure & Précipité blanc \\
    \SetCell[r = 3]{c} Hydroxyde de sodium & \ionCuivreII & Précipité bleu \\
    & \ionFerII & Précipité vert \\
    & \ionFerIII & Précipité rouille \\
    Chlorure de baryum & \sulfate & Précipité blanc \\
  \end{tableau}
\end{center}

\question{
  L'ingénieure commence par verser quelques gouttes de \variationSujet{chlorure de Baryum}{nitrate d'argent} dans un tube à essai contenant l'eau prélevée.
  Elle observe la formation d'un précipité blanc.
  Indiquer quel ion pollue le bassin, en justifiant.\competence{APP}
}{
   D'après le tableau, le \variationSujet{chlorure de baryum}{nitrate d'argent} a réagit avec les ions \variationSujet{sulfates}{chlorure} pour former un précipité blanc.
   \points{2}
}[2]

\question{
  L'ingénieure veut réaliser des tests supplémentaires pour savoir si le bassin est aussi pollué par des ions Fer.
  Indiquer quel(s) réactif(s) elle doit utiliser et quel résultat permettrait de conclure à la présence d'ions Fer.\competence{APP}
}{
  Pour identifier la présence d'ions fer, l'ingénieure devra réaliser un test avec l'hydroxyde de sodium.
  Si elle voit apparaître un précipité vert, alors le bassin contient des ions Fer II.
  Si elle voit apparaître un précipité rouille, alors le bassin contient des ions Fer III.
  Si aucun précipité n'apparaît, le bassin n'est pas pollué.
  \points{2}
}[4]