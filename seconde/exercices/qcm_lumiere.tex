%%%%
\important{QCM - cocher \textit{la ou les} bonnes réponses.}\correction{\points{7}}

\begin{qcm}{La lumière est une onde électromagnétique}
  \item\reponseQCM qui se propage en ligne droite dans un même milieu.
  \item\reponseQCM qui se propage avec une vitesse $c = \qty{3,00e8}{\m\per\s}$.
  \item qui est forcément monochromatique.
\end{qcm}
%
\begin{qcm}{Le domaine visible du spectre électromagnétique se trouve}
  \item entre \qty{400}{\m} et \qty{700}{\m}.
  \item entre \qty{500}{\micro\m} et \qty{600}{\micro\m}.
  \item\reponseQCM entre \qty{400}{\nano\m} et \qty{700}{\nano\m}.
\end{qcm}
%
\begin{qcm}{Le spectre d'émission d'un corps chaud est}
  \item un spectre de raies, avec quelques longueurs d'ondes précises.
  \item\reponseQCM un spectre continu.
\end{qcm}
%
\begin{qcm}{Plus un corps chaud a une température élevée, plus son spectre d'émission}
  \item contient des grandes longueurs d'onde (vers le rouge).
  \item\reponseQCM contient des petites longueurs d'onde (vers le bleu).
  \item s'élargit en petite et grande longueurs d'onde.
\end{qcm}
%
\begin{qcm}{Quand la lumière passe d'un milieu transparent à un autre}
  \item\reponseQCM sa trajectoire change, c'est le phénomène de réfraction.
  \item sa trajectoire ne change pas.
  \item\reponseQCM elle peut être réfléchie en fonction de son angle d'incidence.
\end{qcm}