%%%%
\titreSousSection{Aller observer le Soleil}\correction{\points{5}}

\vspace*{-12pt}
\begin{doc}{La sonde Parker}{doc:E_parker}
  La sonde solaire Parker a été lancé par l'agence spatiale américaine, la NASA, le 12 août 2018.
  Cette sonde doit aller observer la couronne solaire du Soleil. 
  La communication entre la sonde et la Terre se font par émission d'ondes électromagnétiques.
  
  La vitesse de la sonde était de $v =$ \variationSujet{\num{1,1e5}}{\num{1,2e5}} \unit{\m\per\s} lors de son envoi dans l'espace.
  Le Soleil se trouve à une distance $d = \qty{1,50e11}{\m}$ de la Terre.
\end{doc}

\question{
  Calculer le temps en seconde que mettrait la sonde pour atteindre le Soleil, si elle allait en ligne droite.\competence{APP, REA}
}{
  Le temps mis par la sonde est la distance parcourue divisée par la vitesse de la sonde :
  \begin{equation*}
    t = \frac{d}{v} = \qty{1,6e6}{\s}
  \end{equation*}\points{2}
  \vspace*{-12pt}
}[2]

\question{
  Calculer le temps en seconde que met la lumière émise par le Soleil pour atteindre la Terre.\competence{RCO, REA}
}{
  La distance est la même, mais cette fois la vitesse est celle de la lumière $c$, soit un temps $t = d / c = \qty{500}{\s}$\points{2}
}[2]

\question{
  Si la sonde se trouvait à la surface du Soleil, au bout de combien de temps recevrait-on l'onde électromagnétique émise par la sonde ?\competence{RCO, APP, ANA/RAI}
}{
  L'onde se déplace à la vitesse de la lumière et mettra donc \qty{500}{\s} pour arriver sur Terre, soit $\sim 8$ minutes.\points{1}
}[3]