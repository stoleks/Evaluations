%%%%
\titreSection{Atome et élément chimique}\correction{\points{8.5}}

\question{
  \variationSujet
  {Dans les batteries on trouve du silicium \isotope{28}{14}{Si}.}
  {Le magnésium \isotope{25}{12}{Mg} est un réactif très présent en chimie.}
  Donner la composition du noyau de cet atome.\competence{APP, REA}
}{
  L'atome de
  \variationSujet{
     silicium \isotope{28}{14}{Si} possède 14 protons et $28 - 14 = 14$
  }{
    magnésium \isotope{25}{12}{Mg} possède 12 protons et $25- 12 = 13$
  }
  neutrons.
  \points{1}
}{2}

\question{
  Indiquer en justifiant le nombre d'électrons que possède l'atome
  \variationSujet{de silicium.}{de magnésium.}\competence{APP}
}{
  L'atome est neutre électriquement, sa charge électrique totale est nulle.
  Les protons ont une charge positive, les électrons une charge négative.
  L'atome doit donc avoir autant de protons que d'électrons, soit 14 électrons.
  \points{1}
}{2}


Certains élément chimiques peuvent exister sous plusieurs formes appelée isotope, comme par exemple
\variationSujet
{le carbone : \isotope{13}{6}{C}, \isotope{14}{6}{C}.}
{l'oxygène : \isotope{17}{8}{O}, \isotope{18}{8}{O}.}


\question{
  Le troisième isotope stable
  \variationSujet
  {du carbone possède 6 protons et 6 neutrons.}
  {de l'oxygène possède 8 protons et 8 neutrons.}
  Calculer son nombre de nucléons et écrire sa représentation symbolique \isotope{A}{Z}{X}.\competence{REA}
}{
  \variationSujet{
    L'atome de carbone a $6 + 6 = 12$ nucléons, on le note \isotope{12}{6}{C}.
  }{
    L'atome d'oxygène a $8 + 8 = 16$ nucléons, on le note \isotope{16}{8}{O}.
  }
  \points{1}
}{2}

\question{
  Le cuivre \isotope{63}{29}{Cu} peut devenir l'ion \chemfig{Cu^{2+}}.
  Donner le nombre de protons, neutrons et électrons de l'ion \chemfig{Cu^{2+}}. Justifier.\competence{APP, ANA/RAI}
}{
  L'ion \chemfig{Cu^{2+}} possède deux charges positives : 2 électrons ont donc été arrachés à l'atome de cuivre \isotope{63}{29}{Cu}.
  Cet atome possède 29 proton, $63 - 29 = 34$ neutrons et comme l'atome est neutre, il possède 29 électrons.
  L'ion \chemfig{Cu^{2+}} a donc 29 protons, 34 neutrons et $29 - 2 = 27$ électrons.
  \points{2}
}{3}


%%%%
% \medskip
\pasCorrection{\newpage}
\important{QCM - cocher \textit{la ou les} bonnes réponses.}\correction{\points{3.5}}

\vspace*{-18pt}
\begin{multicols}{2}
  \begin{qcm}{
    L'atome de \variationSujet
    {sodium \chemfig{Na} est devenu l'ion \chemfig{Na^+}}
    {de fluor \chemfig{F} est devenu l'ion \chemfig{F^{-}}}
    parce que
  }
    \item\reponseQCM un électron lui a été arraché
    \item un électron lui a été donné
    \item il a gagné un proton
  \end{qcm}
  %
  \begin{qcm}{
    L'ion \variationSujet{\chemfig{Na^+}}{\chemfig{F^{-}}}
  }
    \item est un anion
    \item\reponseQCM est un cation
    \item\reponseQCM a une charge positive
  \end{qcm}
  %
  \begin{qcm}{Le cortège électronique a une structure particulière}
    \item\reponseQCM avec des couches ($1, 2, 3, \ldots$) et des sous-couches ($s, p, \ldots$)
    \item\reponseQCM les sous-couches $s$ peuvent contenir au plus 2 électrons
    \item\reponseQCM les sous-couches $p$ peuvent contenir au plus 6 électrons
  \end{qcm}
  %
  \begin{qcm}{La dernière colonne de la classification périodique s'appelle la famille}
    \item\reponseQCM des gaz nobles
    \item des halogènes
  \end{qcm}
  %
  \begin{qcm}{
    Les entités chimiques \isotope{63}{29}{Cu}, \chemfig{Cu^+}, \chemfig{Cu^{2+}} sont toutes du Cuivre car elles ont
  }
    \item le même nombre d'électrons
    \item\reponseQCM le même nombre de protons $Z$
    \item le même nombre de nucléons $A$
  \end{qcm}
  %
  \begin{qcm}{
    Le gaz noble le plus proche \variationSujet{du Béryllium ($Z = 4$)}{de l'Oxygène ($Z = 8$)} est
  }
    \item\reponseQCM l'Hélium ($Z = 2$)
    \item le Néon ($Z = 10$)
    \item l'Argon ($Z = 18$)
  \end{qcm}
  %
  \begin{qcm}{
    Pour se stabiliser \variationSujet{le Béryllium}{l'Oxygène} pourra
  }
    \item\reponseQCM perdre 2 électrons
    \item gagner 2 électrons
    % \item gagner 4 électrons
  \end{qcm}
  %
  % \begin{qcm}{
  %   Combien de liaison covalentes peut former l'atome de carbone \isotope{12}{6}{C} pour former une molécule ?
  % }
  %    \item\reponseQCM 4, car il lui manque 4 électrons pour compléter sa couche externe
  %    \item 2, pour avoir 8 électrons au total
  %    \item aucune, il est déjà très stable
  %  \end{qcm}
\end{multicols}