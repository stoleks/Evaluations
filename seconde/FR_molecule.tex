\enTeteFiche{\sndMole}

\begin{tableauConnaissances}
  Je peux écrire la configuration électronique d'un atome en remplissant les sous-couches s et p dans le bon ordre.
  & & & \\
  %
  Je sais que les éléments chimiques sont rangées par colonne (famille) et par ligne (période) dans le tableau périodique.
  & & & \\
  %
  Je peux identifier la couche externe d'un atome et combien d'électrons de valence s'y trouvent.
  & & & \\
  %
  Je sais repérer la famille des gaz nobles dans le tableau périodique.
  Je sais que leur couche externe pleine les rend très stables.
  & & & \\
  %
  Je connais la règle du duet et de l'octet.
  Je peux les appliquer pour trouver quel ion stable peut être formé à partir d'un atome.
  & & & \\
  %
  Je sais que les molécules sont composées d'atomes.
  Je sais que la stabilité d'une molécule est due aux électrons de valence partagés entre les atomes.
  & & & \\
  %
  Je peux analyser un schéma de Lewis pour expliquer la stabilité d'une molécule.
  & & & \\
  %
  Je sais qu'une espèce chimique est constitué d'un très grand nombre d'entités chimiques.
  Je sais ce que représente une mole.
  & & & \\
  %
\end{tableauConnaissances}

\basDePageFicheReussite
\bigskip

\coursFicheReussite