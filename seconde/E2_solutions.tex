%%%% début de la page
\teteSndSolu

%%%%
\nomPrenomClasse

%%%%
\titreEvaluation{\sndSolu}

%%%% Compétences évaluées
\pasCorrection{
  \vspace*{-12pt}
  \sousTitre{Compétences évaluées}
  
  \begin{tableauCompetences}
    RCO &
    Connaître le vocabulaire du cours et les relations importantes.
    & & & & \\
    %
    VAL &
    Comparer des valeurs calculées avec des valeurs de références pour valider un raisonnement.
    & & & & \\
    %
    REA &
    Réaliser un calcul en donnant le résultat en notation scientifique avec les bonnes unités.
    & & & & \\
    %
    COM & 
    Répondre avec des phrases complètes et compréhensibles sans les questions. 
    Rédiger de manière synthétique et argumentée.
  \end{tableauCompetences}
}

\appreciation{3}


%%%%
\vspace*{-20pt}
\titrePartie{Sang et anémie}
\pasCorrection{\points{12}}
\vspace*{-12pt}

\begin{doc}{Composition du sang et anémie}{doc:E2_composition_sang}
  Le sang est un mélange liquide composé de \qty{54}{\percent} de plasma, \qty{45}{\percent} de globules rouges et \qty{1}{\percent} de globules blancs.
  
  Le plasma est une solution aqueuse, qui contient des minéraux,
  des nutriments et les gaz liés à la respiration :
  dioxygène \chemfig{O_2} et dioxyde de carbone \chemfig{CO_2}.
  
  Pour assurer son bon fonctionnement, l'organisme d'un être humain a besoin de fer \chemfig{Fe}.
  On dit qu'une personne souffre d'anémie si la concentration massique en fer dans le sang est trop faible.
  Le fer est transporté par une molécule dans le sang : l'hémoglobine.

  On peut séparer les constituants du sang en utilisant une centrifugeuse, ce qui donne un mélange constitué de trois phases.
  
  \vspace*{4pt}
  \separationBlocs{
    \centering \hspace{40pt}
    \tubeEssaisSangCentrifuge{0.8}{0.9}{1.75}
    
    Échantillon de sang d'une personne normale.
  }[0.48]{
    \centering \hspace{40pt}
    \tubeEssaisSangCentrifuge{0.4}{0.5}{1.75}
    
    Échantillon d'une personne souffrant d'anémie.
  }[0.52]
\end{doc}

%
\question{
  Indiquer en justifiant si le contenu des tubes à essais du document~\ref{doc:E2_composition_sang} 
  est un mélange homogène ou un mélange hétérogènes.\competence{RCO, APP}
}{
  C'est un mélange hétérogène, car on peut distinguer les constituants du mélange.
  \points{1}
}{0}

\question{
  Indiquer le solvant et les solutés qui constituent le plasma.\competence{RCO, APP}
}{
  Solvant : eau. Solutés : minéraux, nutriments, dioxygène et dioxyde de carbone.
  \points{2}
}{0}

\question{
  Déduire des deux échantillons de sang le composant du sang qui contient les molécules d'hémoglobines.\competence{APP}
}{
  Ce sont les globules rouges, car c'est le seul composant dont la quantité diminue pour une personne en situation d'anémie.
  \points{2}
}{0}


%
\pasCorrection{
  \newpage
  \vspace*{-40pt}
}
\begin{doc}{Dosage de l'hémoglobine}{doc:E2_dosage_hemoglobine}
  Mesurer la concentration massique en hémoglobine dans le sang permet de détecter les cas d'anémies.
  On parle d'anémie si cette concentration massiques est inférieure a
  \qty{1,2}{\g\per\litre} pour une femme et \qty{1,3}{\g\per\litre} pour un homme.
  Pour mesurer cette concentration, on peut réaliser une échelle de teinte,
  car c'est l'hémoglobine qui donne sa teinte rouge au sang.
  
  \vspace*{4pt}
  \separationBlocs{
    \begin{center}
      \begin{tblr}{c| c| c| c| c| c}
        Solution &
        {\tubeEssaisSolution{rougeClair}          \\ 1} &
        {\tubeEssaisSolution{rougeClair!70!white} \\ 2} &
        {\tubeEssaisSolution{rougeClair!40!white} \\ 3} &
        {\tubeEssaisSolution{rougeClair!15!white} \\ 4} &
        {\tubeEssaisSolution{rougeClair!5!white}  \\ 5} \\ \hline
        %
        Concentration \unit{\g\per\litre} & 1,4 & 1,3 & 1,2 & 1,1 & 1,0 \\ \hline
      \end{tblr} \\[4pt]
      
      Schéma de l'échelle de teinte réalisée, avec les solutions étalons et leurs concentrations.
    \end{center}
  }[0.6]{
    \begin{center}
      \variationSujet{
        \tubeEssaisSolution{rougeClair!55!white}
      }{
        \tubeEssaisSolution{rougeClair!30!white}
      }
  
      Échantillon de sang à doser.
    \end{center}
  }[0.4]
\end{doc}

%
\question{
  Rappeler avec vos mots le principe général d'un dosage par étalonnage (que veut-on mesurer et comment fait-on).\competence{RCO, COM}
}{
  On veut mesurer la concentration massique d'un soluté responsable de la coloration de la solution.
  Pour ça on réalise une échelle de teinte : on prépare des solutions dont on connait la concentration en soluté.
  On compare ensuite les teintes des différentes solutions pour encadrer la valeur de la concentration massique.
  \points{3}
}{0}

\question{
  Pour préparer des solutions, on peut effectuer une dilution ou une dissolution.
  Indiquer en justifiant laquelle des deux on effectue pour passer de la solution 2 à la solution 3.\competence{RCO}
}{
  On réalise une dilution, car la concentration diminue.
  \points{1}
}{0}
  
\question{
  Donner le nom de deux verreries nécessaires pour réaliser une dilution.\competence{RCO}
}{
  Pipette jaugée et fiole jaugée.
  \points{1}
}{0}

\question{
  En utilisant le document~\ref{doc:E2_dosage_hemoglobine}, indiquer en justifiant la concentration en hémoglobine de l'échantillon de sang.\competence{APP, VAL}
}{
  L'échantillon a une teinte plus foncé que la solution 3, mais plus claire que celle de la solution 2. Donc sa concentration se trouve entre \num{1,3} et \qty{1,2}{\g\per\litre}.
  \points{1}
}{0}

\question{
  L'échantillon vient \variationSujet{d'une femme}{d'un homme}. Indiquer en justifiant \variationSujet{si elle}{s'il} souffre d'anémie ou non.\competence{VAL}
}{
  Comme la concentration est plus faible que \qty{1,2}{\g\per\litre}, elle souffre d'anémie.
  \points{1}
}{0}


%%%%
\titrePartie{Conduite et alcoolémie}\pasCorrection{\points{11,5}}

\variationSujet{Mélanie et sa femme Sihame}{Maxime et son mari Nassim} sortent en voiture pour aller manger dehors.
Au restaurant \variationSujet{Sihame}{Nassim} boit un verre de \variationSujet{\qty{500}{\ml}}{\qty{250}{\ml}}
d'alcool à \qty{10}{\degree} : c'est-à-dire que \qty{10}{\percent} du volume de la boisson est de l'éthanol.

On va chercher à déterminer si \variationSujet{Sihame}{Nassim} pourra de nouveau conduire après le repas.

\question{
  Calculer le volume d'éthanol dans le verre.\competence{APP, REA}
}{
  On multiplie le volume du verre par la proportion d'éthanol :
  $\qty{250}{\ml} \times 10/100 = \qty{25}{\ml}$.
  \points{1,5}
}{0}

%
L'éthanol a une masse volumique qui vaut $\rho_\text{éth} = \qty{0,8}{\g\per\ml}$.
Pour un volume $V_\text{éth}$ d'éthanol, on peut calculer d'éthanol avec cette relation
$m_\text{éth} = \rho_\text{éth} \times V_\text{éth}$.

\question{
  Calculer la masse d'éthanol bue par \variationSujet{Sihame}{Nassim}.\competence{APP, REA}
}{
  On utilise la relation fournie :
  $m_\text{éth} = \qty{0,8}{\g\per\ml} \times \qty{25}{\ml} = \qty{20}{\g}$
  \points{1,5}
}{0}

%
Le corps \variationSujet{d'une femme}{d'un homme} adulte contient en moyenne \variationSujet{\qty{4,5}{\litre}}{\qty{5,5}{\litre}} de sang.
En France, \og \textit{il est interdit de conduire avec un taux d'alcool dans le sang supérieur ou égal
à \qty{0,5}{\g\per\litre} de sang} \fg.

\question{
  En physique-chimie on parle de concentration massique plutôt que de taux d'alcool.
  Expliquer avec vos mots la différence entre cette grandeur et la masse volumique.\competence{RCO, COM}
}{
  La concentration massique mesure la quantité de soluté dans une solution,
  alors que la masse volumique mesure à quelle point un échantillon est dense.
  \points{2}
}{0}

%
\question{
  Rappeler la formule mathématique de la concentration massique.\competence{RCO}
}{
  $c_m = \dfrac{m}{V}$
  \points{1}
}{0}

\question{
  Calculer la concentration massique d'éthanol dans le sang de \variationSujet{Sihame}{Nassim}.\competence{APP, REA}
}{
  $c_m = \dfrac{\qty{20}{\g}}{\qty{4,5}{\litre}} = \qty{4,4}{\g\per\litre}$.
  \points{1,5}
}{0}

\question{
  Indiquer, en justifiant, si \variationSujet{Sihame}{Nassim} pourra conduire en sortant du restaurant.\competence{APP, VAl}
}{
  Non, car la concentration massique d'alcool est supérieure à \qty{0.5}{\g\per\litre}, il est donc interdit de conduire.
  \points{1}
}{0}

%
En fait, quand une personne boit une boisson alcoolisée,
seule une petite partie de l'éthanol et absorbé par l'organisme.
En moyenne seulement $12\%$ de l'éthanol passe dans le sang.
Si on a bu \qty{10}{\g} d'éthanol, \qty{1,2}{\g} passe dans le sang.

\question{
  Calculer de nouveau la concentration massique dans le sang de \variationSujet{Sihame}{Nassim} en tenant compte de cette information. Indiquer, en justifiant, si \variationSujet{Sihame}{Nassim} pourra conduire en sortant du restaurant.\competence{APP, REA, VAL}
}{
  La masse d'éthanol qui passe dans le sang est
  $m = \qty{20}{\g} \times \dfrac{12}{100} = \qty{2,4}{\g}$.

  Et donc la concentration massique d'alcool dans le sang vaut $c_m = \qty{2,4}{\g} / \qty{4,5}{\litre} = \qty{0,53}{\g\per\litre}$.
  Il est donc toujours interdit de conduire.
  \points{3}
}{0}