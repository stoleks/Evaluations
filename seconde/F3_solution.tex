%%%% début de la page
\newpage
\enTete{Corps purs et solutions}{1}

%%%%
\nomPrenomClasse

%%%% evaluation
\vspace*{6pt}
\begin{tableauCompetences}
  \centering RCO --
  Restituer ses connaissances.
  & & & & &
\end{tableauCompetences}


%%%% questions
%
\exo{Un ensemble d'entités chimiques identiques est : }{0}
\begin{qcm}
  \item une espèce chimique.
  \item un corps pur.
  \item un mélange.
\end{qcm}

%
\exo{Définir un mélange.}{1}

%
\vspace*{-12pt}
\exo{Deux liquides sont dits miscibles si ils}{0}
\begin{qcm}
  \item forment un mélange homogène.
  \item forment un mélange hétérogène.
\end{qcm}

%
\exo{Citer deux liquides \textbf{non-miscibles}.}{1}

%
\vspace*{-12pt}
\exo{Donner la formule de la masse volumique, en précisant le sens des symboles utilisés.}{3}

%
\vspace*{-12pt}
\exo{Une solution est :}{0}
\begin{qcm}
  \item Un mélange homogène.
  \item Un mélange hétérogène.
\end{qcm}

%
\exo{Donner le nom d'une solution dont le solvant est l'eau.}{1}

%
\vspace*{-12pt}
\exo{Donner la formule de la concentration massique, en précisant le sens des symboles utilisés.}{3}