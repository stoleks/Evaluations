\enTeteFicheReussite{Chapitre 2 - \sndChapitreDeux}
\bigskip

\begin{tableauConnaissances}
  % 
  Je sais que le mouvement dépend du référentiel choisi et je connais le modèle du point matériel.
  & & & Activité 1, 2 \& 3, cours p. 1-4 & 11, 13 p. 210 \\
  %
  Je sais décrire le mouvement d'un système (trajectoire + évolution de son vecteur vitesse). Je sais reconnaitre un mouvement rectiligne uniforme.
  & & &  Activité 1, 2 \& 3, cours p. 1-4 & 11, 14 p.210-211 \\
  %
  Je sais calculer et tracer un vecteur vitesse à partir du vecteur déplacement et de l'écart de temps entre deux positions. 
  & & & Activité 3 \& 7, cours p. 1-4 & 14, 16 p. 211, 11 p. 243 \\
  %
  Je sais qu'une force s'exprime en newton (N) et je sais représenter une force en terme de vecteurs, en faisant attention à son point d'application.
  & & & Activité 4 \& 5, cours p. 6-8 & 15, 16 p. 228, 10 p. 243 \\
  %
  Je connais les caractéristiques des forces suivantes : poids, réaction du support et forces de frottements.
  & & & Activité 4 \& 5, cours p. 6-8 & 11, 14, 17 p. 227-228 \\
  %
  Je sais utiliser le principe d'inertie et sa contraposée pour relier le mouvement d'un système au forces qui s'exercent sur lui.
  & & & Activité 5, 6, 7 \& 8, cours p. 9-10 & 10, 11 p.243 \\
  %
\end{tableauConnaissances}

\basDePageFicheReussite{2}