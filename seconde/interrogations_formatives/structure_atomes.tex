%%%% début de la page
\teteSndAtom
\vspace*{-8pt}
\nomPrenom
\titreEvaluation{Structure d'un atome}

\pasCorrection{
  \important{Compétences évaluées}
  \vspace*{-8pt}
  
  \begin{tableauCompetences}
    APP & Savoir lire l'écriture symbolique d'un atome. \\
    REA & Calculer le nombre \variationSujet{de neutrons}{d'électrons} d'un atome. 
  \end{tableauCompetences}
}

\question{
  Donner le nom et la charge les trois particules qui constituent un atome (cortège électronique et noyau). 
}{
  Protons (charge positive), neutrons (charge nulle), électrons (charge négative).
}[3]

\question{
  Donner le nombre de protons, de neutrons et \variationSujet{de nucléons}{d'électrons} dans l'atome de \variationSujet{\isotope{18}{9}{F}}{\isotope{14}{7}{N}}.
}{
  \variationSujet{18}{14} nucléons, \variationSujet{9}{7} protons, \variationSujet{9 neutrons.}{7 électrons.}
}[3]

\question{
  La charge d'un électron est $-e = \qty{-1.6e-19}{\ampere\s}$. Donner son ordre de grandeur.
}{
  C'est \qty{e-19}{\ampere\s}.
}[1]

\question{
  Estimer l'ordre de grandeur du nombre d'électrons qui parcourent un fil ayant un courant électrique de \qty{1}{\ampere} pendant \qty{1}{\s}.
}{
  On a une charge totale $Q = \qty{1}{\ampere} \times \qty{1}{\s}$ dans le fil pendant \qty{1}{\s}. Pour trouver le nombre d'électrons, il suffit de diviser par l'ordre de grandeur de la charge d'un électron
  \begin{equation*}
    N = \dfrac{\qty{1}{\ampere\s}}{\qty{e-19}{\ampere\s}} = \num{e-19}.
  \end{equation*}
}[2]