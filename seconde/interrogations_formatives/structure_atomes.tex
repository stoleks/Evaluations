%%%% début de la page
\teteSndAtom
\setcounter{page}{1}
% \vspace*{-4pt}
\nomPrenom
\titreEvaluation{Structure d'un atome}

\pasCorrection{
  \important{Compétences évaluées}
  \vspace*{-8pt}
  
  \begin{tableauCompetences}
    APP & Savoir lire l'écriture symbolique d'un atome. \\
    REA & Calculer le nombre de neutrons ou d'électrons d'un atome ou d'un ion.
  \end{tableauCompetences}
}

\question{
  Donner le nom et la charge les trois particules qui constituent un atome (cortège électronique et noyau). 
}{
  Protons (charge positive), neutrons (charge nulle), électrons (charge négative).
}[3]

\question{
  Donner le nombre de protons, de neutrons et \variationSujet{de nucléons}{d'électrons} dans l'atome de \variationSujet{fluor \isotope{F}{9}{18}}{d'oxygène \isotope{O}{8}{16}}.
}{
  \variationSujet{18}{14} nucléons, \variationSujet{9}{7} protons, \variationSujet{9 neutrons.}{7 électrons.}
}[3]

\question{
  Donner le nombre d'électrons de l'ion \variationSujet{fluorure \chemfig{F^{-}}}{oxygène \chemfig{O^{2-}}}.
}{}[1]

\question{
  En justifiant clairement, donner le nombre de d'électrons du \variationSujet{magnésium \chemfig{Mg^{2+}}, le magnésium}{sodium \chemfig{Na^{+}}, le sodium} ayant pour symbole \variationSujet{\isotope{Mg}{12}{24}}{\isotope{Na}{11}{23}}
}{}[3]

% \question{
%   La charge d'un électron est $e = \qty{1.6e-19}{\ampere\s}$. Donner son ordre de grandeur.
% }{
%   C'est \qty{e-19}{\ampere\s}.
% }[1]

% \question{
%   La charge totale $Q$ dans un fil électrique est simplement le nombre d'électron $N$ multiplié par la charge d'un électron $e$, soit $Q = N\times e$.
%   Estimer l'ordre de grandeur du nombre d'électrons qui parcourent un fil ayant une charge électrique $Q = \qty{1}{\ampere\s}$
% }{
%   On a une charge totale $Q = \qty{1}{\ampere} \times \qty{1}{\s}$ dans le fil pendant \qty{1}{\s}. Pour trouver le nombre d'électrons, il suffit de diviser par l'ordre de grandeur de la charge d'un électron
%   \begin{equation*}
%     N = \dfrac{\qty{1}{\ampere\s}}{\qty{e-19}{\ampere\s}} = \num{e-19}.
%   \end{equation*}
% }[2]