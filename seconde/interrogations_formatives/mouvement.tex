%%%% début de la page
\teteSndMouv

%%%%
\nomPrenom

%%%% evaluation
\vspace*{6pt}
\begin{tableauCompetences}
  \centering RCO 
  & Restituer ses connaissances.
  & & & &
\end{tableauCompetences}
\vspace*{8pt}

%%%% questions
%
\begin{qcm}{Indiquer l'unité qui correspond à une vitesse.}
  \item Le mètre par seconde (m/s).
  \item Le kilomètre heure (km$\cdot$h).
\end{qcm}

%
\begin{qcm}{Une vitesse est représenté par un vecteur qui porte 4 informations :}
  \item la position, la distance, la direction et la durée.
  \item le point, le sens, l'accélération et le nombre.
  \item le point d'application, la direction, le sens et la norme.
\end{qcm}

%
\question{
  Une observatrice voit passer un métro roulant à une vitesse constante.
  Pour elle le mouvement du métro est :
}{}{1}

%
\question{
  Un observateur est assis dans le métro roulant à une vitesse constante. 
  Pour lui le mouvement du métro est :
}{}{1}

%
\question{
  Donner la formule du vecteur vitesse $\vv{v}_2$ d’un système au point $P_2$ entre les instants $t_1$ et $t_3$ :
}{}{1}