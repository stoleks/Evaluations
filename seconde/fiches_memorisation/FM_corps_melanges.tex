\ficheMemorisation{\sndCorp}{
  Quelle est la différence entre une espèce chimique et une entité chimique ?
  & & &
  Une espèce chimique est constituée de plein de répétitions d'une entité chimique \\
  % 
  Quelle est la différence entre un corps pur et un mélange ?
  & & & 
  Un corps pur est constitué d'une espèce chimique, alors qu'une mélange est constitué de plusieurs espèce chimique. \\
  %
  Quelle est la relation entre la masse volumique d'un objet, sa masse et son volume ?
  & & & 
  $\rho = m/V$, avec la masse volumique en g/L, si la masse est en g et le volume en L. \\
  %
  Comment peut-on identifier un échantillon dont on peut mesurer le volume et la masse ?
  & & &
  On peut mesurer sa masse volumique et la comparer avec des masses volumiques de références, car la masse volumique est unique pour un corps pur. \\
  %
  Quel est le matériel nécessaire pour réaliser une chromatographie sur couche mince ?
  & & & 
  Il faut une couche mince qui sert de support, une cuve à CCM, un éluant et des échantillons à identifier. \\
  %
  Comment savoir si un échantillon est un corps pur à partir d'un chromatogramme ?
  & & & 
  Si le dépôt de l'échantillon s'est séparé en plusieurs tâches, alors c'est un mélange. \\
}