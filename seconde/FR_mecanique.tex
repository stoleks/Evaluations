\enTeteFiche{3}{\sndMouv}
\bigskip

\begin{tableauConnaissances}
  % 
  Je sais que le mouvement dépend du référentiel choisi pour observer le mouvement. 
  & & & TP 3.1 et activité 3.1 \\
  %
  Je sais décrire le mouvement d'un système (trajectoire + évolution de son vecteur vitesse). Je sais reconnaître un mouvement rectiligne uniforme.
  & & &  Activité 3.1 \\
  %
  Je connais le modèle du point matériel.
  & & & Activité 3.1 \\
  % 
  Je sais calculer et tracer un vecteur vitesse à partir du vecteur déplacement et de l'écart de temps entre deux positions. 
  & & & TP 3.2 \\
  %
  Je sais qu'une force s'exprime en newton noté N et je sais représenter une force en terme de vecteurs, en faisant attention à son point d'application.
  & & & Activité 3.2, 3.3, 3.4, 3.5, 3.6 et 3.7 \\
  %
  Je sais quand deux forces se compensent.
  & & & Activité 3.3, 3.4 et 3.7 \\
  %
  Je connais les caractéristiques des forces suivantes : poids, réaction du support et forces de frottements.
  & & & Activité 3.2 et 3.6 \\
  %
  Je sais utiliser le principe d'inertie pour décrire les forces qui s'exercent sur un objet dont je connais le mouvement.
  & & & Activité 3.3, 3.4 et 3.7 \\
  %
  Je sais utiliser la contraposée du principe d'inertie pour décrire le mouvement d'un objet à partir des forces qui s'exercent sur l'objet.
  & & & Activité 3.3, 3.4 et 3.7 \\
  %
  Je sais sais vérifier l'homogénéité d'une relation entre deux grandeurs.
  & & & Activité 0.3 \\
  %
  Je sais calculer et utiliser des ordres de grandeurs.
  & & & Activité 0.3 \\
  %
\end{tableauConnaissances}

\basDePageFicheReussite

\coursFicheReussite