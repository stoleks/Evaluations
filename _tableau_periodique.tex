\newcommand{\largeurCaseTableauPeriodique}{1.5}

%%%% Pour afficher un élément dans le tableau périodique
\NewDocumentCommand{\elementTexteCharge}{m m m o}
{
  \begin{minipage}{\largeurCaseTableauPeriodique cm}
    \begin{center}
      \IfValueTF{#4}{ \vAligne{-20pt} }{ \vAligne{-34pt} } % position du nom
      {\small #3} \\[2pt] % nom de l'élément
      {\ensuremath\footnotesize \textbf{#1}} \\[6pt] % nombre atomique
      \chemfig[atom style={scale = 1.8}]{#2} % symbole atomique
      % \element{#1}{#2} % element symbol and atomic number
      \IfValueT{#4}{
        \\ {\small \qty{#4}{\g/\mole}}
      }
    \end{center}
  \end{minipage}
}

%%%% Pour afficher un élément dans le tableau périodique
\NewDocumentCommand{\elementElectroneg}{m m}
{
  \begin{minipage}{\largeurCaseTableauPeriodique cm}
    \begin{center}
      {\Large \important[black]{#1} \\[2pt]} % symbole atomique
      {\footnotesize $\chi = $\num{#2}} % électronégativité
    \end{center}
  \end{minipage}
}


%%%% Pour afficher un tableau périodique
%% #1 : largeur ; #2 : hauteur ; #3 : élements
\NewDocumentCommand{\tableauPeriodique}{O{2.6} O{2.7} m}{
\begin{tikzpicture}[font=\sffamily, scale=0.75, transform shape]
  
%% Couleur de remplissage
  \tikzstyle{jauneClair}  = [fill=yellow!30]
  \tikzstyle{jaune}       = [fill=yellow!45]
  \tikzstyle{jauneFonce}  = [fill=yellow!60]
  \tikzstyle{rougeClair}  = [fill=red!20]
  \tikzstyle{rouge}       = [fill=red!35]
  \tikzstyle{rougeFonce}  = [fill=red!50]
  \tikzstyle{orangeClair} = [fill=orange!30]
  \tikzstyle{orange}      = [fill=orange!45]
  \tikzstyle{orangeFonce} = [fill=orange!60]
  \tikzstyle{vertClair}   = [fill=vertSapin!20]
  \tikzstyle{vert}        = [fill=vertSapin!35]
  \tikzstyle{vertFonce}   = [fill=vertSapin!50]
  
%% Type d'élément, par famille
  \tikzstyle{Alcali} = [Element, vertFonce]
  \tikzstyle{Alcalo} = [Element, vert]
  \tikzstyle{Metaux} = [Element, vertClair]
  \tikzstyle{Metoid} = [Element, orangeClair]
  \tikzstyle{NoMeta} = [Element, orange]
  \tikzstyle{Haloge} = [Element, orangeFonce]
  \tikzstyle{GazRar} = [Element, rouge]

%% Type d'élément, par électronégativité
 \tikzstyle{elec1} = [Element, vertClair]
 \tikzstyle{elec2} = [Element, vert]
 \tikzstyle{elec3} = [Element, jaune]
 \tikzstyle{elec4} = [Element, orangeClair]
 \tikzstyle{elec5} = [Element, orange]
 \tikzstyle{elec6} = [Element, orangeFonce]
 \tikzstyle{elec7} = [Element, rouge]
 \tikzstyle{elec8} = [Element, rougeFonce]
  
%% Style des éléments
  \tikzstyle{Element} = [
    draw=black, jaune,
    minimum width  = #1 cm, % Largeur de la case
    node distance  = #1 cm, % Espace entre deux case
    minimum height = #2 cm, % Hauteur de la case
  ]

%% Période, groupe et titre
  \tikzstyle{Period} = [font={\sffamily\LARGE}, node distance=2cm]
  \tikzstyle{Groupe} = [font={\sffamily\LARGE}, minimum width=2.5cm, node distance=2cm]
  \tikzstyle{Titre}  = [font={\sffamily\Huge\bfseries}]

%% Place des éléments
  #3
\end{tikzpicture}
}


%%%% Pour faciliter l'utilisation du tableau périodique
\newcommand{\elementH} {\elementTexteCharge{1} {H} {Hydrogène}[1,00]}
\newcommand{\elementHe}{\elementTexteCharge{2} {He}{Hélium}   [4,00]}
\newcommand{\elementLi}{\elementTexteCharge{3} {Li}{Lithium}  [6,94]}
\newcommand{\elementBe}{\elementTexteCharge{4} {Be}{Béryllium}[9,01]}
\newcommand{\elementB} {\elementTexteCharge{5} {B} {Bore}     [10,8]}
\newcommand{\elementC} {\elementTexteCharge{6} {C} {Carbone}  [12,0]}
\newcommand{\elementN} {\elementTexteCharge{7} {N} {Azote}    [14,0]}
\newcommand{\elementO} {\elementTexteCharge{8} {O} {Oxygène}  [16,0]}
\newcommand{\elementF} {\elementTexteCharge{9} {F} {Fluor}    [19,0]}
\newcommand{\elementNe}{\elementTexteCharge{10}{Ne}{Néon}     [20,2]}
\newcommand{\elementNa}{\elementTexteCharge{11}{Na}{Sodium}   [23,0]}
\newcommand{\elementMg}{\elementTexteCharge{12}{Mg}{Magnésium}[24,3]}
\newcommand{\elementAl}{\elementTexteCharge{13}{Al}{Aluminium}[27,0]}
\newcommand{\elementSi}{\elementTexteCharge{14}{Si}{Silicium} [28,1]}
\newcommand{\elementP} {\elementTexteCharge{15}{P} {Phosphore}[31,0]}
\newcommand{\elementS} {\elementTexteCharge{16}{S} {Soufre}   [32,1]}
\newcommand{\elementCl}{\elementTexteCharge{17}{Cl}{Chlore}   [35,5]}
\newcommand{\elementAr}{\elementTexteCharge{18}{Ar}{Argon}    [39,9]}
\newcommand{\elementK} {\elementTexteCharge{19}{K} {Potassium}[39,1]}
\newcommand{\elementCa}{\elementTexteCharge{20}{Ca}{Calcium}  [40,0]}