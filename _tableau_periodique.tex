%%%% Clés utilisées dans le tableau périodique
\pgfkeys{% définition de la famille de clefs
  /periodique/.is family, /periodique,
  defaut/.style = {
    couleur = green-50,
    nom =,
    electronegativite = 0,
    masse = 0,
    charge = 0,
    echelle = 1
  },
  nom/.store in = \periodiqueNom,
  couleur/.store in = \periodiqueCouleur,
  symbole/.store in = \periodiqueSymbole,
  hauteur case/.store in = \periodiqueHauteur,
  largeur case/.store in = \periodiqueLargeur,
  electronegativite/.store in = \periodiqueElectroneg,
  masse/.store in = \periodiqueMasse,
  charge/.store in = \periodiqueCharge,
  echelle/.store in = \periodiqueEchelle
}

%%%% Pour afficher un tableau périodique
\NewDocumentEnvironment{tableauPeriodique}{O{} +m}{%
  \pgfkeys{/periodique, defaut, #1}
  %
  \begin{tikzpicture}[scale = \periodiqueEchelle, transform shape]
    #2
}{%
  \end{tikzpicture}
}

%%%% Pour afficher une case du tableau périodique
\NewDocumentCommand{\tkzElement}{O{} D(){}}{%
  \pgfkeys{/periodique, defaut, #1}
  %
  \couleurElectronegativite{}
  \node [
    node distance  = \periodiqueHauteur and \periodiqueLargeur,
    on grid,
    minimum width  = \periodiqueLargeur,
    minimum height = \periodiqueHauteur,
    name = \periodiqueSymbole,
    fill = \pgfkeysvalueof{/periodique/couleur},
    draw = cyan-800!50!black,
    align = center,
    #2
  ] {
    % nom de l'élément
    \IfValueT{\periodiqueNom}{
      {\scriptsize \periodiqueNom}%
      \compare {\periodiqueCharge > 0}{ \\[-1pt] }{ \\[2pt] }%
    }
    % nombre atomique
    \compareT {\periodiqueCharge > 0}{\important[black]{\periodiqueCharge}\\[2pt]}%
    % symbole atomique
    {\Large \important[black]{\periodiqueSymbole}}%
    % masse atomique
    \compareT {\periodiqueMasse > 0}{\\[-2pt]%
      {\footnotesize \num{\periodiqueMasse}}%
    }%
    % électronégativité
    \compareT {\periodiqueElectroneg > 0}{%
      \compare{\periodiqueMasse > 0}{\\}{\\[-2pt]}%
      {\footnotesize $\chi = \num{\periodiqueElectroneg}$}%
    }%
  };
}

%%%% Réglage des couleurs automatiques si l'électronégativité est réglé
\newcommand{\couleurElectronegativite}{%
  \compare {\periodiqueElectroneg > 3.5}{%T
    \pgfkeyssetvalue{/periodique/couleur}{red-400}
  }{%F
    \compare {\periodiqueElectroneg > 3.0}{% T
      \pgfkeyssetvalue{/periodique/couleur}{red-300}
    }{% F
      \compare {\periodiqueElectroneg > 2.5}{% T
        \pgfkeyssetvalue{/periodique/couleur}{red-200!95!black}
      }{% F
        \compare {\periodiqueElectroneg > 2.0}{% T
          \pgfkeyssetvalue{/periodique/couleur}{orange-200}
        }{% F
          \compare {\periodiqueElectroneg > 1.5}{% T
            \pgfkeyssetvalue{/periodique/couleur}{yellow-150}
          }{% F
            \compare {\periodiqueElectroneg > 1.0}{% T
              \pgfkeyssetvalue{/periodique/couleur}{green-100}
            }{% F
              \pgfkeyssetvalue{/periodique/couleur}{\periodiqueCouleur}
            }% pas d'électronégativité défaut
          }% 0.5 < chi < 1.0
        }% 1.0 < chi < 1.5
      }% 1.5 < chi < 2.0
    }% 2.0 < chi < 2.5
  }% 2.5 < chi < 3.0
}% 3.0 < chi < 3.5

%%%% Pour faciliter l'utilisation du tableau périodique
\NewDocumentCommand{\elementH}  {O{} D(){}} {\tkzElement[symbole = H,  charge = 1,  nom = Hydrogène, #1](#2)} % masse = 1.00, 
\NewDocumentCommand{\elementHe} {O{} D(){}} {\tkzElement[symbole = He, charge = 2,  nom = Hélium, #1](#2)}    % masse = 4.00, 
\NewDocumentCommand{\elementLi} {O{} D(){}} {\tkzElement[symbole = Li, charge = 3,  nom = Lithium, #1](#2)}   % masse = 6.94, 
\NewDocumentCommand{\elementBe} {O{} D(){}} {\tkzElement[symbole = Be, charge = 4,  nom = Béryllium, #1](#2)} % masse = 9.01, 
\NewDocumentCommand{\elementB}  {O{} D(){}} {\tkzElement[symbole = B,  charge = 5,  nom = Bore, #1](#2)}      % masse = 10.8, 
\NewDocumentCommand{\elementC}  {O{} D(){}} {\tkzElement[symbole = C,  charge = 6,  nom = Carbone, #1](#2)}   % masse = 12.0, 
\NewDocumentCommand{\elementN}  {O{} D(){}} {\tkzElement[symbole = N,  charge = 7,  nom = Azote, #1](#2)}     % masse = 14.0, 
\NewDocumentCommand{\elementO}  {O{} D(){}} {\tkzElement[symbole = O,  charge = 8,  nom = Oxygène, #1](#2)}   % masse = 16.0, 
\NewDocumentCommand{\elementF}  {O{} D(){}} {\tkzElement[symbole = F,  charge = 9,  nom = Fluor, #1](#2)}     % masse = 19.0, 
\NewDocumentCommand{\elementNe} {O{} D(){}} {\tkzElement[symbole = Ne, charge = 10, nom = Néon, #1](#2)}      % masse = 20.2, 
\NewDocumentCommand{\elementNa} {O{} D(){}} {\tkzElement[symbole = Na, charge = 11, nom = Sodium, #1](#2)}    % masse = 23.0, 
\NewDocumentCommand{\elementMg} {O{} D(){}} {\tkzElement[symbole = Mg, charge = 12, nom = Magnésium, #1](#2)} % masse = 24.3, 
\NewDocumentCommand{\elementAl} {O{} D(){}} {\tkzElement[symbole = Al, charge = 13, nom = Aluminium, #1](#2)} % masse = 27.0, 
\NewDocumentCommand{\elementSi} {O{} D(){}} {\tkzElement[symbole = Si, charge = 14, nom = Silicium, #1](#2)}  % masse = 28.1, 
\NewDocumentCommand{\elementP}  {O{} D(){}} {\tkzElement[symbole = P,  charge = 15, nom = Phosphore, #1](#2)} % masse = 31.0, 
\NewDocumentCommand{\elementS}  {O{} D(){}} {\tkzElement[symbole = S,  charge = 16, nom = Soufre, #1](#2)}    % masse = 32.1, 
\NewDocumentCommand{\elementCl} {O{} D(){}} {\tkzElement[symbole = Cl, charge = 17, nom = Chlore, #1](#2)}    % masse = 35.5, 
\NewDocumentCommand{\elementAr} {O{} D(){}} {\tkzElement[symbole = Ar, charge = 18, nom = Argon, #1](#2)}     % masse = 39.9, 
\NewDocumentCommand{\elementK}  {O{} D(){}} {\tkzElement[symbole = K,  charge = 19, nom = Potassium, #1](#2)} % masse = 39.1, 
\NewDocumentCommand{\elementCa} {O{} D(){}} {\tkzElement[symbole = Ca, charge = 20, nom = Calcium, #1](#2)}   % masse = 40.0, 